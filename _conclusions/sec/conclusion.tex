\section{Conclusion}
\label{ch-conclusions:sec:conclusion}

As identified in Chapter~\ref{ch-introduction} of this thesis the aim of the presented work was to:

\textit{[...] make a contribution in control of Battery Energy Storage Systems (BESS) that can aid Distribution Network Operators (DNOs) in improving the operation and reliance of their Low-Voltage (LV) networks.}

To achieve this aim, four objectives were identified from the literature review in Chapter~\ref{ch-literature} and met in the four contribution chapters of this thesis (i.e. Chapter~\ref{ch1} to Chapter~\ref{ch4}).
Within these chapters, key network parameters were identified to assess the impact of BESS control methods on the underlying power delivery network when adjusting BESS phase powers whilst conforming to a half-hourly schedule.
When alleviating this scheduling constraint, it was shown how a developed dynamic BESS control can achieve greater reduction of power peaks.
Then coordinated control of multiple BESS were assessed regarding the desynchronisation of control instructions.
Since this analysis showed the sensitivity of control methods on their implementation (even when operating in a desynchronised environment) a truly communication less control algorithm was developed next.
Through developing these algorithms and testing them by simulating several LV distribution network models, aspects involving network operation, system deployability, information propagation and telecommunication restrictions were studied.
Therefore, a technical contribution that can aid DNOs in improving the operation and reliance of LV networks has been achieved and thus the overarching aim was met.

The limitations of the presented research do however ask for future work to be conducted in order to ready the algorithms and methods for implementation in industry.
Beside the restrictions outlined in Section~\ref{ch-conclusions:sec:research-limits} due to time constraints and the targeted research focus, future work will comprise further improvement of network and BESS models, mathematical formulation of proof of stability and convergence.
More specifically, DNOs should address the issue of ownership and possible means of incentivising customers to provide network support with their home-connected assets.
Also, safety, security and ethics associated with emerging control methods and data handling should also be considered, not only by academics, but also by industry, their customers and legislation makers.

Overall, in the context of power systems, the conducted research focuses on providing improvements at the fringes of the electricity network; that is at the LV distribution level.
With the international aim of transitioning towards a low-carbon economy, national issues are mostly expected to occur in the transmission and interconnection of electricity grids.
This issue is particularly apparent when planning to generate bulk power at remote locations (for example offshore) without having installed the required power lines.
Equally, with higher reliance on renewable energy resources, national power systems must become more flexible to not only cater for the volatility in demand but also for the expected intermittency in supply.
Whilst energy storage on a national scale would address the difficulty of matching modern supply and demand, the complexity of a project, the size and the associated cost make deployment of such a system a significant logistic challenge.
By focusing on smaller scaled projects first (i.e. BESS in the LV networks) the first step is provided towards this larger goal.
After all, it is easier to develop, test and eventually deploy control algorithms on such a smaller scale than it is for larger scaled projects.
Also, at the end of the day, it is the DNO that needs to provide the final physical link between demand and supply and their power delivery networks are expected to cater for the immediate increase in volatile demand.
Therefore, targeting research at this fringe of the electricity network may provide a small contribution to the overall operation of the electricity network, but this research still provides support for DNOs when catering for the future load scenarios.
