\section{Conclusion}
\label{ch-conclusions:sec:conclusion}

As identified in Chapter~\ref{ch-introduction} of this thesis the aim of the presented work was to:

\textit{[...] make a contribution in control of Battery Energy Storage Systems (BESS) that can aid Distribution Network Operators (DNOs) in improving the operation and reliance of their Low-Voltage (LV) networks.}

To achieve this aim, four objectives were identified from the literature review in Chapter~\ref{ch-literature} and met in the four contribution chapters of this thesis (i.e. Chapter~\ref{ch1} to Chapter~\ref{ch4}).
Within these chapters, key network parameters were identified to assess the impact of BESS control methods on the underlying power delivery network i.e. when adjusting BESS phasors whilst maintaining a half-hourly schedule.
When alleviating this scheduling constraint, it was shown how a developed dynamic BESS control can achieve greater peak power reduction.
Then coordinated control of multiple BESS were analysed in regards to the desynchronisation of control instructions.
Since this analysis showed the sensitivity of control methods on their control parameters (even when operating in a desynchronised environment) a truly communication less control algorithm was developed next.
Through developing these algorithms and testing them with simulations of several LV distribution network models, aspects involving network operation, system deployability, information propagation and telecommunication restrictions were addressed.
Therefore, a technical contribution that can aid DNOs in improving the operation and reliance of their networks has been achieved and thus the overarching aim was met.

In the context of power systems however, the conducted research focuses on providing improvements at the fringes of the electricity network (i.e. at the LV distribution level).
With the international aim of transitioning towards a low-carbon economy, national issues are mostly expected to occur in the transmission and interconnection of electricity grids.
This issue is particularly apparent when planning to generate bulk power at remote locations (e.g. offshore) without having installed the required power lines.
Equally, with higher reliance on renewable energy resources, national power systems must become more flexible to not only cater for the volatility in demand but also for the expected intermittency in supply.
Whilst energy storage on a national scale would address the difficulty of aligning modern supply and demand, project complexity, size and cost make deployment of such a system a significant logistic challenge.
By focusing on smaller scaled projects (i.e. BESS in the LV networks) the first step is provided since control algorithms can easily be developed, tested and eventually ported into comparable but larger scaled projects.
Also, at the end of the day, it is the DNO that needs to provide the final physical link between demand and supply and their power delivery networks are expected to cater for the immediate increase in volatile demand.
Therefore, targeting research at this fringe of the electricity network may provide a small contribution to the overall operation of the electricity grid, but this research still provides a significant support to DNOs when catering for the future load scenarios.
