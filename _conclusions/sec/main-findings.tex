\section{Overview of Main Findings}
\label{ch-conclusion:main-findings}

This thesis' problem statement, which has been presented in Chapter~\ref{ch-introduction}, can be summarised as follows:

\begin{itemize}
	\item
	The aim of this thesis is to investigate how BESS in the LV network should be controlled in order to achieve best possible network support, including the reduction of peak load, voltage deviations and phase balance.
	\item 
	To assess the impact of BESS on the LV network's topology, simulations are being run to compare on-line and off-line control performance.
	\item
	Given that BESS can operate flexibly but have a limited energy resource and often have a predetermined half-hourly schedule, the research studies whether sub-half-hourly corrections can improve the performance of LV networks by incorporating load forecasts.
\end{itemize}

The reviewed literature in Chapter~\ref{ch-literature} as well as the findings from Chapter~\ref{ch1} emphasise the need for improved methods of control for energy storage in the LV network.
In Chapter~\ref{ch1}, a set of key network parameters were introduced to highlight the breadth of possible network improvement functions.
Using the LV connected BESS, its impact on each of these key network parameters was assessed by optimising each parameter through its corresponding cost function.
The same BESS would traditionally have been operated with a half-hourly schedule that dictates the device's active powers.
Using this operation as a benchmark, sub-half-hourly phasor adjustments were proposed to tune the BESS operation to achieve optimal impact for any given key network parameter, yet without violating the higher resolution power constraints.
As shown in several resulting time-series plots in Section~\ref{ch1:sec:results-and-discussion}, and as summarised in Table~\ref{ch1:tab:cost-table}, optimising BESS operation for certain key network parameters has two resulting impacts.
Firstly, the targeted key network parameter and the associated network operation are improved, and secondly, other key network parameters are also impacted.
The second impact however need not be a positive impact since, e.g. reducing voltage deviation can lead to a significantly worse power factor due to the injection of reactive power.
Nonetheless, showing that sub-half-hourly phasor adjustments can result in improved network operation formed the basis for the subsequent chapter, Chapter~\ref{ch2}, where the half-hourly active power constraint is  eliminated.

\hl{Link to next chapter}

\hl{Explain what Ch 4 did and highlight the main findings}

\hl{Link to next chapter}

\hl{Explain what Ch 5 did and highlight the main findings}

\hl{Link to next chapter}

\hl{Explain what Ch 6 did and highlight the main findings}

\hl{Make link back to all 4 chapters}


The research that has been presented in this thesis is therefore beneficial to both industry and the academic research community.
All objectives that were set out in the problem statement have been met and the aforementioned key contributions have been made.


