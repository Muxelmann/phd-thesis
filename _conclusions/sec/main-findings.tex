\section{Overview of Main Findings}
\label{ch-conclusions:sec:main-findings}

The objective of this thesis, which has been presented in Chapter~\ref{ch-introduction}, is summarised as follows:

\begin{itemize}
	\item
	It was to investigate how BESS in the LV network should be controlled in order to achieve best possible network support, including the reduction of peak load, voltage deviations and phase unbalance.
	\item 
	To assess the impact of BESS on the operation of the LV network, simulations were run to compare on-line and off-line control performance.
	\item
	Given that BESS can operate flexibly but had a predetermined half-hourly schedule based on half-hourly load forecasts, the presented research studied how sub-half-hourly corrections can improve the performance of LV networks without exceeding the limited energy resource of the BESS.
	\item
	Additionally the aim was extended to assess the effects of, and to develop algorithms for, desynchronised and communication less BESS control.
\end{itemize}

The reviewed literature in Chapter~\ref{ch-literature} as well as the findings in Chapter~\ref{ch1} emphasised the need for improving methods of control for energy storage in the LV network.
Chapter~\ref{ch1} explores how optimising sub-half-hourly BESS operation on a specific metric impacts other metrics, that define network performance
More specifically, using the LV connected BESS, its impact on each of key network parameters was assessed by optimising each parameter through its corresponding cost function.
The same BESS would have been operated traditionally with a half-hourly schedule that dictates the active powers of the device.
Using this operation as a benchmark, sub-half-hourly phase adjustments were proposed to tune the BESS operation to achieve optimal impact for any given key network parameter without violating the power constraints.
As shown in several resulting time-series plots in Section~\ref{ch1:sec:results-and-discussion} that were summarised in Table~\ref{ch1:tab:cost-table}, optimising BESS operation for certain key network parameters had two resulting impacts:

\begin{enumerate}
	\item The minimisation of a cost function, derived from a specific key network parameter, results in BESS operation  improving the associated network operation. For example, when minimising the cost that was linked to distribution losses, then a mean reduction in losses of 5.0kWh was achieved instead of a 1.2kWh reduction which would have been the result for traditional BESS scheduling.
	\item The minimisation of a cost function, derived from a specific key network parameter, also results in BESS operation that impacts other parameters, indirectly associated to the same network operation. For example, when minimising the cost that was linked to BESS voltage deviation, then the worst voltage deviation, the worst line loadings and the network's neutral currents were also reduced, but voltage deviation, line loadings and power factor at substation level were worsened.
\end{enumerate}

The analysis of simulations in Chapter~\ref{ch1} also showed that whilst the half-hourly average BESS schedule remained unaffected, there are positive and negative relationships between combinations of metrics being selected for optimisation and network performance assessment.
A statistically significant positive impact (i.e. where $p<0.05$) was proven for only certain pairs of network parameters (for example and as stated earlier, maximum bus voltage deviation, phase unbalance and neutral power when minimising voltage deviation at ESMU level).
Nonetheless, showing that sub-half-hourly phase power adjustments can result in improved network operation formed the basis for the next chapter, Chapter~\ref{ch2}, where the half-hourly active power constraints were eliminated.

Chapter~\ref{ch2} relaxed the constraint on half-hourly average schedule of a BESS and applied model-predictive approach for sub-half-hourly adjustment.
More specifically, it presented an approach in combining both on-line and off-line energy storage control to dynamically reduce both daily and minutely load peaks.
An average peak load reduction of 5kW was achieved for the best algorithm configuration without reaching a surplus or shortage of stored energy since a half-hourly BESS schedule (similar to Chapter~\ref{ch1}) was followed.
Unlike the preceding chapter however, the BESS control in Chapter~\ref{ch2} had operational flexibility within a certain tolerance band of 10\% around its predetermined half-hourly schedule.
Combined with a predictor to estimate the sub-half-hourly power volatility, results were achieved that noticeably reduced load peaks in comparison to the traditional forecast driven control.
In fact, as shown in Figure~\ref{ch2:fig:peak-diff-pdf}, the mean peak load reduction increased from 1.7kW to around 5kW for different prediction mechanisms.
These findings from Chapter~\ref{ch1} and Chapter~\ref{ch2} thus form the contribution to knowledge regarding \ref{objective-1} and \ref{objective-2}, respectively.

Chapter~\ref{ch3} explores the impact of communication latency on the coordination of charging for multiple EVs, which demonstrated that the stability of asynchronous coordination is less sensitive to the choice of control parameter values.
For example, they simply used all network information when computing BESS control instructions without considering possible latency issues.
Therefore, Chapter~\ref{ch3} developed the new smart-charging algorithm and used a novel MAS implementation that operated in an intentionally desynchronised manner.
This desynchronisation was to assess the algorithm performance when the previously assumed communication infrastructure becomes less reliable.
Since uncoordinated EV charging is expected to put the most significant load on the LV network, any algorithm failure (like failure to coordinate this charging) would become noticeable.
And indeed, the results in Chapter~\ref{ch3} showed that the converging behaviour of the algorithm became less sensitive to its control parameters in the desynchronised environment, when compared to the traditional synchronised algorithm execution.
For example, when extreme control parameter values were chosen, an oscillating behaviour was observed for the synchronised case which lead to the repetitive allocation of a 200kW charging spike.
However, this oscillating behaviour disappeared in the desynchronised case which meant that the algorithm converged on a global level.
In this desynchronised case the algorithm's performance and convergence became less sensitive to the choice of control parameter values.
This fact became particularly apparent when the overall performance of avoiding charging peaks between the synchronised case (i.e. Figure~\ref{ch3:fig:all-sync}) and the fully desynchronised case (i.e. Figure~\ref{ch3:fig:all-async-irregular}) was compared.
Chapter~\ref{ch3} therefore achieved \ref{objective-3} by developing a robust smart-charging algorithm that is thoroughly assessed in regards to possible communication desynchronisation.

Lastly, Chapter~\ref{ch4} proposes an improvement on an iterative and uncoordinated BESS algorithm for voltage support and the improved algorithm achieves more uniform voltage support.
More specifically, in Chapter~\ref{ch4} a communication-less control method for distributed BESS was developed to reduce peak load, voltage deviation and unequal asset utilisation.
This communication-less control was achieved by using individualised control parameters in a modified AIMD algorithm.
Dynamic loads (i.e. uncoordinated EVs) were simulated on the feeder to maximise the stress on the LV network that the developed control algorithm aimed to mitigate.
The results showed that for different EV uptake levels, BESS could always yield improvements for both AIMD and the proposed AIMD+ control methods.
However, as seen in Figure~\ref{ch4:fig:storage-aimd}, only the latter method did compensate uniformly across the LV network since it took into account the network specific voltage characteristics like the voltage drop along the feeder.
Therefore, these findings formed the contribution to knowledge regarding \ref{objective-4}.

The research over these four chapters has shown that energy storage algorithms can be improved by merging on-line and off-line control at high and low temporal resolution.
Additionally, the research has shown that desynchronised control instructions can yield significantly different operation of otherwise synchronised control algorithms.
However, this issue can be avoided when mitigating the need for communication technology altogether.
In each chapter, this thesis comprehensively tested the presented control algorithms on real demand data, allowing it to encapsulate varying demand behaviour and characteristics at both high and low temporal resolutions.
All findings were generated from the available datasets and were therefore subject to its properties of comprehensively capturing typical demand behaviours.

In summary, it is concluded that individual control algorithms presented in this thesis provide an improvement on the network operation using different aspects of control, but there is no algorithm presented that combines all aspects.
However, results showed that focused control can be tuned to achieve a significantly higher positive impact on a narrow set of key parameters, which is why the chapters that implement such methods did also present the means of implementing their control in regards to the available data (thus achieving the subsequent network improvements which were derived from data driven simulations).
Nonetheless, all objectives that were set out in the problem statement of this thesis have been met by making the aforementioned key contributions.
One can therefore conclude that the research presented in this thesis is beneficial to both industry and the academic research community.
These contributions to knowledge, possible research limitations and future work discussing these benefits, are outlined in the subsequent sections.
