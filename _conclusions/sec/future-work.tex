\section{Future Work}
\label{ch-conclusion:future-work}

Based on the lessons learnt, the experience gained and the assumptions made whilst conducting the presented research, potential steps for future work are presented here, in the subsequent subsections:

\subsection{Energy storage model}

Although the energy storage model that was used throughout this research is based on similar assumptions as those models that are used in other state of the art literature, there are still certain assumptions regarding this model that need not be disregarded in future research.
The nonlinearity regarding the battery's charging behaviour for instance can be included since more battery characteristics have been discovered over the recent past.
Characteristics regarding constant-current constant-voltage charging paradigms, temperature and battery condition have become better understood.
Especially with the dissemination of EV information and their charging curves, previously proprietary data has started to become ubiquitous and available to the public domain.
In future work, such data could not only aid the simulation battery control algorithms, but could also take into account the battery conditions itself to maximise its live.

\subsection{Network model and power flow simulation}

Future work will also incorporate the algorithms into a more accurate and standardised network model.
Power flow simulations on OpenDSS, PowerSim, GridLab-D or similar tools are frequently used throughout literature to create accurate network assessments.
Having learnt the standardised network model structure that is specified by the IEEE in the recently published European test feeder will allow any future work to utilise the updated collection of network models in a compliant manner.
Such compliancy would also allow the assessment of network failures, harmonic studies and islanded operation, which would previously have been impossible.
Furthermore, with improved and updated network information, simulation tools would also give more accurate results regarding, e.g. the location and scaling of LV assets.
Finally, by cooperating with research institutes, i.e. EPRI and NERL in the United States, it has now become possible to accelerate the simulation of network models by parallelising the power flow simulations.
Therefore extending the number of trials would also improve the finding's certainty and lead to more conclusive results.

\subsection{Different MPC}

So far, the selection of MPC was based on both literature recommendations and the scope of the presented research.
However, computer science driven areas of research including, e.g. artificial intelligence (AI), are becoming increasingly popular and have also begun penetrating the control of power network assets.
With the ever increasing abilities of computers, such complex control approaches are likely to outperform traditional deterministic and probabilistic methods.
Future work is therefore likely to also include AI aspects to assess their performance and to also understand how AI can safely be implemented.

\subsection{Governance}

At the present time there is also a considerable discussion regarding the changes in governance required for DNOs to own and operate energy storage devices, as well as whether DNOs will have access to customer's smart-meter data.
These changes are likely to be important, in order for the findings and methods that are presented in this thesis to fit into the DNO's business strategy and capability.

\subsection{Data acquisition and estimation}

The data that was used for simulations to trigger certain control instructions may not be available in reality.
Although substation monitoring and voltage measurements at the energy storage were a certain prerequisite for most of the presented research, other network parameters, e.g. customer voltages, can only be obtained at significant financial cost or at low resolution.
Being able to include a data acquisition mechanisms that are comparable to reality (e.g. data acquisition through the deployment of smart-meters) would strengthen the validity of the resulting findings.
In cases where data acquisition is not possible either due to the lack of measuring equipment or the legal barriers that uphold customer privacy, endpoint data needs to be estimated instead.
Developing a reliable method of estimating the endpoint data, e.g. customer voltages, from a limited number of measuring points is still an open research objective.
Future work will therefore develop a method that uses BESS and substation data to maximise the certainty of endpoint data, which in turn can be used to improve future BESS control methods.







