\section{Contribution to Knowledge}
\label{ch-conclusions:sec:knowledge-contribution}

In Chapter~\ref{ch-introduction} the background was presented, forming the basis of the present research for DNO owned storage devices in the LV network.
Then, in Section~\ref{ch-introduction:sec:problem-statement}, possible gaps in this literature are identified before summarising the relevant literature in Chapter~\ref{ch-literature}.
Upon those gaps, the problem statement of this thesis was formed as how BESS control can aid DNOs at maintaining their network operation despite the ongoing transition towards a low-carbon economy.
The main contribution presented in this thesis is thus the development of control methods for BESS in the LV network that improve network operation (for example power flow, voltage deviation and phase unbalance) whilst taking into account telecommunication limitations (for example possible desynchronisation or total absence of a communication infrastructure) in later chapters.
All chapters that are presented in this thesis make contributions within these identified gaps and in accordance to the outlined research objectives, these contributions are summarised as follows:

\begin{itemize}
	\item
	In accordance with \ref{objective-1}, a closed-loop phase power adjustment method was presented to control DNO owned BESS to maximise its beneficial impact on key network parameters of the LV network.
	Findings in Chapter~\ref{ch1} showed how network issues (for example voltage deviation, neutral currents, phase unbalance and losses) can be reduced individually when adjusting BESS operation at a sub-half-hourly resolution, even though the BESS was constrained by an active power schedule at half-hourly time scale.
	Knowing how improving one key network parameter impacts different network operation is of relevance to industry since their limited network observability constrains their assessment of any control actions.
	Being able to infer and possibly predict the overall network impact from a limited set of measurements would allow DNOs to better control BESS and thus improve operation of their LV networks.
	Nonetheless, this schedule based constraint of the proposed adjustment method still imposed limitations to the otherwise flexible BESS operation, but results still showed the benefits that could be achieved despite this constraint.
	\item
	Then, in Chapter~\ref{ch2}, a method to dynamically correct BESS schedules was developed in order to control DNO owned BESS and to maximise its capabilities at reducing both daily demand peaks (i.e. at half-hourly resolution) whilst also mitigating volatile load peaks (i.e. at sub-half-hourly resolution).
	This method is thus in accordance with \ref{objective-2} and findings in  Chapter~\ref{ch2} showed how the control method outperformed traditional BESS control and how the probability of reducing peak load could be noticeably increased.
	Having developed such a method that functions despite the underlying errors in the load forecasts allows DNOs to reduce the occurrence of power peaks and thus prevent possible tripping of protection equipment, for example.
	Therefore, DNOs will be able to leverage the proposed dynamic control method instead of only relying on the accuracy of the underlying load forecast when aiming to ensure an undisrupted supply of electricity.
	However, to achieve this improvement with the proposed control method, the implicit assumption of an ICT infrastructure could limit algorithm deployability when distributed across multiple devices (especially when considering a distributed BESS).
	\item
	Therefore, a smart-charging algorithm for distributed control of an EV fleet was developed and simulated on a standardised Multi-Agent System (MAS).
	In order to meet \ref{objective-3}, this MAS was desynchronised and the algorithm's performance was noted and compared with its synchronised counterpart.
	Findings in Chapter~\ref{ch3} showed that the algorithm became less dependent on the underlying control parameters when executed in a desynchronised environment, yet the overall performance of the algorithm remained intact.
	The difference in performance thus demonstrated the danger of assuming that distributed algorithms will function in any environment.
	Both academic and industry based research made such an assumption and complying with this assumption would have lead to the implementation of regulating features that significantly limit the algorithm's performance.
	However, mitigating the need for ICT altogether would not only circumvent the pitfall of potential desynchronisation, but it would also lower deployment requirements and possible system cost.
	\item
	Consequently and in accordance with \ref{objective-4}, a communication-less control method for distributed BESS was developed to assess its ability at reducing the negative impact from the charging of co-located EVs.
	Findings in Chapter~\ref{ch4} showed that the developed AIMD+ algorithm did not only reduce load peaks and voltage deviation, but that it also harmonised the asset utilisation across the entire feeder.
	The achievement of this improvement was due to the individual algorithm tuning which was based on the properties of the underlying power delivery network.
	Whilst academic and industry based research projects typically use historic demand data to derive set-points for their control algorithms, system models allow better tuning when the traditional demand assumptions no longer hold.
	This was done in Chapter~\ref{ch4} which is why a stochastic EV model was developed to simulate a future load scenarios.
	Such an approach in combination with the proposed AIMD+ algorithm would therefore benefit the planning and performance assessment of future power distribution networks.
	But, without any communication infrastructure the proposed algorithm was unable to address overarching network issues that lie outside the scope of the LV networks.
	However, these issues were already hinted at and considered in Chapter~\ref{ch1}.
\end{itemize}







