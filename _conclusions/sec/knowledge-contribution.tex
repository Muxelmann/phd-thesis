\section{Knowledge Contribution}
\label{ch-conclusion:knowledge-contribution}

In Chapter~\ref{ch-literature}, the literature is reviewed that surrounds the current and present control methods of DNO owned storage devices on the LV network.
This literature review supports the thesis problem statement in Section~\ref{ch-introduction:sec:problem-statement} since it concluded with the identified gaps in literature in which further investigation and research was deemed necessary and beneficial for both the industry and academic research community.
All chapters that are presented in this thesis make contributions within these identified gaps, and these contributions are summarised here:

\begin{itemize}
	\item
	In accordance with \ref{objective-1}, a closed-loop phasor adjustment method is presented to control DNO owned BESS to maximise its beneficial impact on key network parameters of the LV network.
	Findings in Chapter~\ref{ch1} show how issues including e.g. voltage deviation, neutral currents, phase unbalance and losses can be individually reduced when adjusting BESS operation at a sub-half-hourly resolution, even when the device is constrained by an active power schedule at half-hourly time scale.
	However, this constraint still imposes limitations to the otherwise flexible BESS operation, but it also shows the benefits that can be achieved despite this constraint.
	\item
	In accordance with \ref{objective-2}, a dynamic schedule correcting BESS control method is presented to control DNO owned BESS to maximise its capabilities at reducing both daily demand peaks, i.e. at half-hourly resolution, whilst also mitigating volatile load peaks, i.e. at sub-half-hourly resolution.
	Findings in Chapter~\ref{ch2} show how the control method outperforms traditional BESS control and how the probability of reducing peak load can be noticeably increased.
	However, to achieve this improvement, the implicit assumption of an ICT infrastructure could limited algorithm deployability when distributed across multiple devices.
	\item
	In accordance with \ref{objective-3}, a smart-charging algorithm for distributed control of an EV fleet was developed and deployed on a standardised MAS, which was desynchronised to assess the algorithm's performance.
	Findings in Chapter~\ref{ch3} show that the execution of the algorithm becomes less dependent on the underlying control parameters executed in a desynchronised environment, yet the overall performance of the algorithm remains intact.
	However, mitigating the need for ICT altogether would not only circumvent the issue of potential desynchronisation, but it would also lower deployment requirements and system cost.
	\item
	In accordance with \ref{objective-4}, a communication-less control method for distributed BESS was developed to assess its ability at reducing the negative impact from the charging of co-located EVs.
	Findings in Chapter~\ref{ch4} show that the developed AIMD+ algorithm does not only reduce peak loads or voltage deviation, but it also equalises the asset utilisation across the entire feeder.
	However, without any communication infrastructure, the performance of the proposed algorithm may be unable to address issues like e.g. phase unbalance, that have already been assessed in Chapter~\ref{ch1}.
\end{itemize}







