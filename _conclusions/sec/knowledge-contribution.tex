\section{Contribution to Knowledge}
\label{ch-conclusions:sec:knowledge-contribution}

The main contribution presented in this thesis is the development of control methods for BESS in the LV network that improve network operation (for example power flow, voltage deviation and phase unbalance) whilst taking into account telecommunication limitations (for example possible desynchronisation or total absence of a communication infrastructure) in later chapters.
With this knowledge, when developing BESS control systems, on-line and off-line control methods can be combined to complement each other whilst keeping in mind limitations regarding their ICT implementation.
All in all, the chapters presented in this thesis made contributions within these identified gaps and in accordance to the outlined research objectives, these contributions are summarised as follows:

\begin{itemize}
	\item
	In accordance with \ref{objective-1}, Chapter~\ref{ch1} has proposed a control algorithm to adjust the active and reactive BESS operation whilst maintaining given BESS schedule and achieving improved network operation.
	Findings in Chapter~\ref{ch1} showed how network issues (for example voltage deviation, neutral currents, phase unbalance and losses) can be reduced individually when adjusting BESS operation at a sub-half-hourly resolution, even though the BESS was constrained by an active power schedule at half-hourly time scale.
	Knowing how improving one key network parameter impacts different network operation is of relevance to industry since their limited network observability constrains their assessment of any control actions.
	Being able to infer and possibly predict the overall network impact from a limited set of measurements allows DNOs to better control BESS and thus improve operation of their LV networks.
	\item
	Then, in Chapter~\ref{ch2}, a method to dynamically correct BESS schedules was developed in order to control DNO owned BESS and to maximise its capabilities at reducing both daily demand peaks whilst also mitigating volatile load peaks.
	This method is thus in accordance with \ref{objective-2} and findings in Chapter~\ref{ch2} showed how the control method outperformed traditional BESS control and how the probability of reducing peak load could be noticeably increased.
	Having developed such a method that functions despite the underlying errors in the load forecasts allows DNOs to reduce the occurrence of power peaks and thus prevent possible tripping of protection equipment, for example.
	Therefore, DNOs will be able to leverage the proposed dynamic control method instead of only relying on the accuracy of the underlying load forecast when aiming to ensure an undisrupted supply of electricity.
	\item
	Regarding \ref{objective-3}, in Chapter~\ref{ch3}, a smart-charging algorithm for distributed control of an EV fleet was developed and simulated on a standardised Multi-Agent System (MAS) in both a synchronised and desynchronised environment to study the performance of the algorithm.
	Findings showed that the algorithm became less dependent on the underlying control parameters when executed in a desynchronised environment, yet the overall performance of the algorithm remained intact.
	This difference in performance highlights the danger of assuming that distributed algorithms will function in any environment.
	With this knowledge, both academic and industry based research can to take into account the effect of latency on an algorithm when designing their own BESS control systems.
	\item
	Lastly, in accordance with \ref{objective-4}, a communication-less control method for distributed BESS was developed in Chapter~\ref{ch4} to assess the ability at reducing the negative impact from the charging of co-located EVs.
	Findings showed that the developed AIMD+ algorithm did not only reduce load peaks and voltage deviation, but that it also harmonised the asset utilisation across the entire feeder.
	The achievement of this improvement was due to the individual algorithm tuning which was based on the properties of the underlying power delivery network.
	Whilst academic and industry based research projects typically use historic demand data to derive set-points for their control algorithms, system models allow better tuning when the traditional demand assumptions no longer hold.
	This was done in Chapter~\ref{ch4} which is why a stochastic EV model was developed to simulate a future load scenarios.
	Such an approach in combination with the proposed AIMD+ algorithm would therefore benefit the planning and performance assessment of future power distribution networks.
\end{itemize}







