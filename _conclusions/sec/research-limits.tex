\section{Research Limits and Future Work}
\label{ch-conclusions:sec:research-limits-and-future-work}

Several assumptions have been made throughout this thesis, and these assumptions imposed limitations on the presented research that may be addressed in future work.
These limitations are discussed in the subsequent subsections.

\subsection{Energy storage model}

The developed model to capture the dynamics of the energy storage system was based on the system that was deployed during the NTVV project with SSEN.
The simplicity of the energy storage model will limit the scope of the presented work when simulating the different storage control algorithms, especially when preparing for the deployment of an actual BESS.
For instance, this model does not take into account any non-linear charging or discharging behaviours, nor does it consider operating temperature and heat radiation that would impact its dynamic efficiency.
When anticipating to deploy the proposed BESS control algorithms that are presented in this thesis to an actual BESS, then these limitations will need to be included.

Although the energy storage model that was used throughout this research is based on similar assumptions as those models that are used in other state of the art literature, there are still certain assumptions regarding this model that need not be disregarded in future research.
The nonlinearity regarding the battery's charging behaviour for instance can be included since more battery characteristics have been discovered over the recent past.
Characteristics regarding constant-current constant-voltage charging paradigms, temperature and battery condition have become better understood.
Especially with the dissemination of EV information and their charging curves, previously proprietary data has started to become ubiquitous and available to the public domain.
In future work, such data could not only aid the simulation battery control algorithms, but could also take into account the battery conditions itself to maximise its lifetime.
However, since the research aim of this thesis was on the development of BESS control methods and not on the development of better BESS models, this endeavour lies outside the research scope.

\subsection{Electric vehicle charging}

In Chapter~\ref{ch4} of this thesis a stochastic EV model was developed to simulate the predicted increase in electricity demand due to the uptake of EVs.
This model assumed that vehicles are charged at home and begin their charging process immediately after being connected.
This so called ``dumb-charging'' is mitigated by the control strategies that are proposed in the same chapter.
However, recent smart-charging strategies promise to reduce the negative impacts from traditional EV charging.
Therefore, the inclusion of smart charging is seen as a future work, since the implementation, validation and extension of available smart-charging schemes currently lies beyond the scope of this thesis.

\subsection{Data}

All findings throughout the presented work were generated from historic demand data.
This data has both limitations in temporal resolution and data length, which therefore limit the data's ability to capture all possible load scenarios at the necessary accuracy to guarantee correct algorithm operation.
Since the entire work that is presented in this thesis relies on carrying out data driven simulations, using datasets with varying characteristics of demand profiles is necessary to assure the results cover a wide range of possibilities.
Also, since historic demand data does not capture the expected change in energy consumption, accompanying demand models had to be developed that also rely on several fundamental assumptions.
For instance, the stochastic EV charging model in Chapter~\ref{ch4} assumed that driving behaviour will not change, apart from the location where vehicles are being refuelled or recharged.
Being unable to make exact predictions about future loads lead to the utilisation of such models since development of preventive BESS control algorithms would not have been possible.
However, constructing models and stating the corresponding assumptions that validate their utilisation was deemed sufficient for the development and testing of BESS control methods.
After all, the scope of this thesis was on the development of BESS control methods and not on development of load prediction mechanisms.

Furthermore, the data that was used in simulations to trigger certain control instructions may not be measurable in reality.
Substation monitoring and voltage measurements at the energy storage were a certain prerequisite for most of the presented research, but obtaining other network parameters in reality (e.g. customer voltages) could only be done at significant financial cost or at unusably low resolution.
Being able to include data acquisition mechanisms that are comparable to reality (e.g. data acquisition through the deployment of smart-meters) would strengthen the validity of the resulting findings.
In cases where data acquisition is not possible either due to the lack of measuring equipment or the legal barriers that uphold customer privacy, endpoint data needs to be estimated instead.
Developing a reliable method of estimating the endpoint data (e.g. customer voltages) from a limited number of measuring points is still an open research challenge.
Future work will therefore develop a method that uses BESS and substation data to maximise the certainty of endpoint data, which in turn can be used to improve future BESS control methods.

\subsection{Network models and Power Flow Simulations}

The dominant network model that was used throughout the research was a European LV model published by the IEEE.
This so called ``LV Test Case'' complemented some of the in-house network models that were provided by SSEN - particularly since reproducibility of the findings could be assured.
Different LV networks do however have their own network characteristics and their models can only be included in the research if their results are averaged.
Also, this averaging is only valid if the number of network models is of sufficient size.
Beside this shortage of network models, their accuracy cannot be guaranteed since old assets' electrical characteristics might have changed (e.g. due to degradation or misinformation).
A higher number of detailed and accurate network models would thus allow the proposed algorithms to be tested on a larger variety of networks, which would in turn improve certainty that the proposed algorithms will function correctly if deployed.

Future work will incorporate the algorithms into a more accurate and standardised network model.
Power flow simulations on OpenDSS, PowerSim, GridLab-D or similar tools are frequently used throughout literature to create accurate network assessments.
Having learnt the standardised network model structure that is specified by the IEEE will allow any future work to utilise the updated collection of network models in an IEEE compliant manner.
Such compliancy would not only allow a better comparative assessment of network power flow solutions, but also of network failures, harmonic studies and islanded operation - previously this would not have easily been possible.
Furthermore, with improved and updated network information, simulation tools would also give more accurate results (e.g. regarding the location and scaling of LV assets).
Finally, by cooperating with research institutes (i.e. the Electric Power Research Institute (EPRI) and the National Energy Research Laboratory (NERL) in the United States) it has now become possible to accelerate the simulation of network models by parallelising the power flow simulations and porting the execution to different programming languages.
Therefore extending the number of trials would also improve the finding's certainty and lead to further conclusive results.
However, for the scope of the conducted research, the number and variety of network models was deemed sufficient since the industrial partner (i.e. SSEN) assured their accuracy and ability to qualify as ``typical UK distribution networks''.

\subsection{Different Control Methods}

So far, the selection of control methods was based on both literature recommendations and the scope of the presented research.
But computer science driven areas of research (e.g. artificial intelligence - AI) are becoming increasingly popular in the area of power delivery and have also begun penetrating the control of power network assets.
With the ever increasing abilities of computers, such complex control approaches are likely to outperform traditional deterministic and probabilistic methods.
Future work is therefore likely to also include AI aspects to assess their performance and to also understand how AI can safely be implemented.
Combining such AI driven control with the discovered issue of desynchronising message propagation would open a new inter-disciplinary research category that focuses on power delivery, control systems and telecommunication issues.
However, these areas lie outside the scope of the work that was presented in this thesis.

\subsection{Communication and Cyber Security}

Although Chapter~\ref{ch3} addressed the common assumption that control instructions propagate in a synchronised manner, other communication issues beside message desynchronisation do still remain an open research challenge.
For instance a sudden loss of communication may result in a cluster of control devices to operate in an islanded manner (i.e. from a ICT point of view).
Being able to cope with not just desynchronised messages, but with a complete loss of messages (or falsification of messages) has not been included in the presented work although the correctness and continuous receipt of messages is a common assumption throughout the area of smart grid research.
Furthermore, assessing the resilience of the underlying communication systems to cyber attacks (e.g. denial of service or man in the middle attacks) will become of vital important as digitisation and connectivity between control assets becomes the norm.
After all, assuring that both the ICT and by extension the power systems are robust in withstanding such attacks is essential to prevent possible service disruptions.
However, although these hot research topics are of high personal interest to the author of this thesis, they do lie outside the scope of the presented work.
