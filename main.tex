\documentclass[12pt,a4paper,twoside,openright]{ociamthesis}  % default square logo 
%\documentclass[12pt,beltcrest]{ociamthesis} % use old belt crest logo
%\documentclass[12pt,shieldcrest]{ociamthesis} % use older shield crest logo
\usepackage[utf8]{inputenc}

%enhanced support of Computer Modern fonts
\usepackage{lmodern}
%additional American Mathematical Society mathematical typesetting capabilities
\usepackage{amsmath}
%American Mathematical Society mathematical theorems
\usepackage{amsthm}
%American Mathematical Society mathematical symbols
\usepackage{amssymb}
% For double line spacing
\usepackage{setspace}
% For opening quote
\usepackage{epigraph}
% For coloured tables
\usepackage[table]{xcolor}
% For rotated table
\usepackage{rotating}
% For nomenclature
\usepackage[intoc]{nomencl}
% For flowchart
\usepackage{tikz}
\usetikzlibrary{shapes,arrows,automata,positioning,fit,calc,backgrounds,chains, decorations.pathreplacing}
% For coloring
\usepackage{color,soul}
% Inline enumeration
\usepackage[inline]{enumitem}

% What was additionally imported by MDPI style
\usepackage{graphicx}
\usepackage{subfig}
\usepackage{multirow}
\usepackage{enumitem}
\usepackage[linesnumbered,lined,boxed,commentsnumbered]{algorithm2e}
\usepackage{footmisc}
\usepackage{etoolbox}

\usepackage{letltxmacro}
\AtBeginDocument{
  \LetLtxMacro\Oldincludegraphics\includegraphics
%  \renewcommand{\includegraphics}[2][]{\Oldincludegraphics[width=0.9\textwidth,#1]{#2}}
  \renewcommand{\includegraphics}[2][]{%
	\ifstrempty{#1}{%
	  \Oldincludegraphics[width=0.95\textwidth]{#2}%
	}{%
	  \Oldincludegraphics[#1]{#2}%
	}%
  }
%  \renewcommand{\includegraphics}\@ifnextchar[{\@specificincludegraphics}{\@defaultincludegraphics}
%  \cmd\@specificincludegraphics[#1]#2{{\Oldincludegraphics[#1]{#2}}
%  \cmd\@defaultincludegraphics#1{{\Oldincludegraphics[width=0.9\textwidth]{#1}}
}
%\newcommand{\includefig}[1]{\includegraphics[width=0.9\textwidth]{#1}}

\setlength\epigraphrule{0pt}
\renewcommand{\epigraphflush}{center}
\setlength{\epigraphwidth}{0.95\textwidth}

% Change default figure positioning
\makeatletter
  \providecommand*\setfloatlocations[2]{\@namedef{fps@#1}{#2}}
\makeatother
\setfloatlocations{figure}{htbp}
\setfloatlocations{table}{htbp}

% New footnote style
\renewcommand{\thefootnote}{\fnsymbol{footnote}}

%% This code creates the groups
% -----------------------------------------
\usepackage{ifthen}
\renewcommand\nomgroup[1]{%
  \ifthenelse{\equal{#1}{G}}{\item[\textbf{Acronyms}]}{%
  \ifthenelse{\equal{#1}{H}}{\item[\textbf{Definition variables and dimensionality}]}{%
  \ifthenelse{\equal{#1}{I}}{\item[\textbf{Symbols used in Chapter 3}]}{%
  \ifthenelse{\equal{#1}{J}}{\item[\textbf{Symbols used in Chapter 4}]}{%
  \ifthenelse{\equal{#1}{K}}{\item[\textbf{Symbols used in Chapter 5}]}{%
  \ifthenelse{\equal{#1}{L}}{\item[\textbf{Symbols used in Chapter 6}]}{%
  {}}}}}}}}
% -----------------------------------------

% Generate the glossary and nomenclature
%\makeglossaries
\makenomenclature

%input macros (i.e. write your own macros file called mymacros.tex 
%and uncomment the next line)
%\include{mymacros}

\title{%
Schedule Adjustments and Control\\[1ex]
of Battery Based Energy Storage in\\[1ex]
Low-Voltage Distribution Networks}   %note \\[1ex] is a line break in the title

\author{Maximilian J. Zangs}             %your name
%\college{School of Built Environment}    %your college
\college{School of Systems Engineering}  %your college

%\renewcommand{\submittedtext}{change the default text here if needed}
\degree{Doctor of Philosophy in Electronic Engineering}     %the degree
\degreedate{2018}         %the degree date


%end the preamble and start the document
\begin{document}

%this baselineskip gives sufficient line spacing for an examiner to easily
%markup the thesis with comments
\baselineskip=18pt plus1pt

%set the number of sectioning levels that get number and appear in the contents
\setcounter{secnumdepth}{3}
\setcounter{tocdepth}{3}


\maketitle                  % create a title page from the preamble info
\doublespacing
\begin{romanpages}          % start roman page numbering
%\setlength{\topmargin}{-.5in}

\epigraph{\textit{It is not a dream. It is a simple feat of scientific electrical engineering. Electric power can drive the world's machinery without the need of coal, oil or gas. Although perhaps humanity is not yet sufficiently advanced to be willingly lead by the inventor's keen searching sense. Perhaps it is better in this present world of ours where a revolutionary idea may be hampered in its adolescence. All this that was great in the past was ridiculed, condemned, combatted, suppressed only to emerge all the more triumphantly from the struggle. [...] Our duty is to lay the foundation for those who are to come and to point the way, yes humanity will advance with giant strides. We are whirling through endless space with an inconceivable speed, all around everything is spinning, everything is moving, everywhere there is energy.}}{--- Nicola Tesla}


\chapter*{Abstract}

\addcontentsline{toc}{chapter}{Abstract}


%enter text for the abstract below

          % include the abstract
\chapter*{Declaration}
\addcontentsline{toc}{chapter}{Declaration}

I confirm that this is my own work and the use of all material from other sources has been properly and fully acknowledged.

\vspace{4cm}

\noindent\rule{20em}{0.4pt}

Maximilian J. Zangs       % declaration of authenticity
\chapter*{Dedication}
\addcontentsline{ded}{chapter}{Dedication}
        % include a dedication.tex file
\chapter*{Acknowledgements}
\addcontentsline{toc}{chapter}{Acknowledgements}

I would like to begin by thanking my two PhD supervisors and tutors Dr Ben Potter and Professor William Holderbaum for their continuing support, help and guidance throughout my PhD.
In addition, I would like to thank the \textit{University of Reading} and sponsors of the \textit{NTVV Project} for providing the funding for my research.
Also, I would like to thank the \textit{ITNG} members of the former \textit{School of Systems Engineering} for providing and maintaining the vital computational resources, and I would like to thank our industrial project partner, \textit{Scottish and Southern Electricity Networks}, for providing thei network models and datasets; both were were vital necessities to conduct and complete my research.

I would like to thank members of the \textit{School of the Built Environment}, the \textit{School of Biological Sciences} and the \textit{School of Psychology and Clinical Language Sciences} for taking their time to share, exchange and discuss ideas and for challenging my thinking and expertise throughout the journey towards the PhD.
In particular I would like to express special thanks to individuals in the \textit{Energy Research Laboratory}: Dr Timur Yunusov, Ana Rodriguez-Arguelles and Peter Adams; individuals from the \textit{SUSPORTS Project}: Dr Ian Harrison and Stefano Pietrosanti; an individual in the \textit{Brain Embodiment Laboratory}: Asad Malik; and individuals at the \textit{TSBE Centre} and the \textit{School of Construction Management and Engineering}: Alice Gunn and Vicky Papaioannou.

There are many current and past colleagues, researchers and friends who have not only technically aided my research, but have also reminded me of life outside academia.
Through their willingness to listen, motivate, support and (sometimes) distract me, they have made my time as a PhD student considerably more enjoyable; so thank you.

%To name a few: Thomas Chung, Alan Halford, Nicholas Holley, Joshua Eadie, Mary Pothitou, Saadia Ansari, Dr Laura Daniels, Deborah Ritzmann, Dr Dan Williams, Dr Stephen Haben, Dr Christopher Shackleton and Dr Charles Moorey.

Last but not least, I would like to express my greatest gratitude and thanks to my family for all of their support and understanding throughout my PhD and time at university; it is much appreciated and I could not have done it without you.   % include an acknowledgements.tex file

\addcontentsline{toc}{chapter}{Table of Contents}
\tableofcontents            % generate and include a table of contents
\addcontentsline{toc}{chapter}{Table of Figures}
\listoffigures              % generate and include a list of figures
\chapter*{Abbreviations}
\addcontentsline{toc}{chapter}{Abbreviations}


\begin{table*}[hbt]
  \begin{tabular}{l l}
    AR & Auto-Regressive\\
    ARMAX & Auto-Regressive Moving-Average Exogenous\\
    ARX & Auto-Regressive Exogenous\\
    BES & Battery Energy Storage\\
    BESS & Battery Energy Storage Solution\\
    CAES & Compressed Air Energy Storage\\
    DNO & Distribution Network Operator\\
    DOD & Depth of Discharge\\
    DSM & Demand Side Management\\
    DSR & Demand Side Response\\
    EPRI & Electric Power Research Institute\\
    ESMU & Energy Storage Management Unit\\
    ESS & Energy Storage Solution\\
    EV & Electric Vehicle\\
    FCES & Fuel Cell driven Energy Storage\\
    FIPA & Foundation for Intelligent Physical Agents\\
    JADE & Java Agent Development Environment\\
    LCT & Low Carb on Technology\\
    NARX & Nonlinear Auto-Regressive Exogenous\\
    NTVV & New Thames Valley Vision\\
    PV & Photo Voltaic\\
    SOC & State of Charge\\
    SQP & Sequential Quadratic Programming\\
    TES & Thermal Energy Storage\\
    VPP & Virtual Power Plan
  \end{tabular}
\end{table*}

\singlespacing

\printnomenclature[1in]

\doublespacing
\end{romanpages}            % end roman page numbering

%now include the files of latex for each of the chapters etc
\chapter{Introduction}
\label{ch-introduction}

The aim of the work that is presented in this thesis is to make a contribution that improves grid operation and reliance when using Battery Energy Storage Systems (BESS) in the UK Low-Voltage (LV) power distribution networks by adjusting sub-half-hourly operation and communication regimes of the BESS.
In this context, grid operation performance is considering mainly peak power flow, but also includes voltage deviation, phase imbalance, distribution losses, and the magnitude of neutral currents.

Predicted increase in electricity demand and demand volatility will effect the performance of the UK distribution networks.
Due to the network's design and topology those changes in demand are predicted to cause issues including voltage deviation, asset overloads, equipment damage and, in the worst case, service disruptions.
As discussed in this thesis, BESS is a suitable alternative to traditional network reinforcements.
However, successfully combining fast system response capabilities (i.e. at sub-half-hourly resolution) with traditional operation schedules (i.e. at half-hourly resolution) to yield the best impact on network performance (e.g. peak-reduction) is still an open research question.
Also, with the proliferation of household-connected BESS and Electric Vehicles (EVs), algorithms to coordinate their operation are also an ongoing research topic.

Before presenting the research objectives that were executed to achieve the research aim, background and motivation for the conducted research are presented.
Next, on the basis of the identified challenges and opportunities for battery energy storage in the electricity distribution network, the problem statement and all research objectives are outlined.
At the end of this chapter, all contributions and publications are stated, and the structure of the rest of this thesis is presented.

\section{Background and motivation}
\label{ch-introduction:sec:background}

Today's society and economy are highly dependent on the continuous availability of energy, or more specifically: electric energy.
In the UK, demand for electricity has increased over the past decades, and this trend is expected to continue into the future \cite{HMGovernment2009}.
This demand increase is only accelerated since a major focus of UK energy policies has been put on transitioning towards a low carbon economy \cite{RoyalAcademyofEngineering2010}.
Particularly the decarbonisation of heat and transport sectors are two areas of significant strategic focus and Low Carbon Technology (LCT) such as Photo-Voltaic (PV) installations, electric vehicles and heat pumps are expected to contribute significantly to this transition.

However, as adaptation of these LCTs increases and they start to penetrate power distribution networks, stress on these networks will continue to increase even further, which may result in additional service disruptions.
Furthermore, the uptake of LCTs is not expected to progress evenly throughout the entire power network, and instead clusters of early adopters are predicted to form, leading to certain Low-Voltage (LV) networks exceeding their operational constraints even at relatively low national rate of LCT adaption \cite{Poghosyan2014}.
The scale of this energy transition becomes becomes particularly apparent when referring to the UK's future energy scenarios that compare the predicted future load scenarios.

\begin{figure}\centering
	\includegraphics{_introduction/fig/electricity-demand-forecast}
	\caption{Annual residential demand for electricity from FES2016 \cite{FES2016}}
	\label{ch-introduction:fig:electricity-demand-forecast}
\end{figure}

Figure \ref{ch-introduction:fig:electricity-demand-forecast} shows the predicted increase in demand for electric energy when following the UK's 2020 and 2050 goals in reducing green-house emissions.
According to this projection, the annual energy demand will increase by more than 40TWh by the year 2040, if the ``Gone Green'' approach is implemented.
This trend is expected despite increasing device efficiencies, since the shift from oil and gas to electricity, i.e. the electrification, offset these gains.

\begin{figure}\centering
	\includegraphics{_introduction/fig/electricity-demand-change-forecast}
	\caption{``Gone Green'' power demand comparison to 2013/14 by type (excluding losses) from FES2015 \cite{FES2015}}
	\label{ch-introduction:fig:electricity-demand-change-forecast}
\end{figure}

\nomenclature[G]{CREST}{Centre for Renewable Energy Systems Technology}

When breaking down the change in demand for electricity, as done in Figure \ref{ch-introduction:fig:electricity-demand-change-forecast}, one can observe how industry sectors are expected to decrease their energy consumption.
Yet residential and commercial sectors are expected to increase their demand and outweigh the industry's energy savings.
Their negative impact on the distribution network is only amplified, since loads in the residential and commercial sectors are typically situated at the network edge, i.e. in the LV distribution network.
This part of the network is its weakest part, since its assets were designed to caters for small powers between 315kVA to 500kVA \cite{EDS08-0115}.
A study based on the findings from \textit{Electricity North West} in \cite{ElectricityNorthWestLtd2014} emphasises the issues that result from residential increase in demand for electricity, of e.g. voltage deviation due to an uptake of LCTs.
As the number of PV installations is expected to grow, the voltage deviation magnitude and frequency is also going to increase \cite{Woyte2006}.
%The voltage deviation and corresponding power profiles are shown in Figure \ref{ch-introduction:fig:lct-impact}, which was produced by simulating several loads on the IEEE LV Test Case power distribution feeder.
%Loads were modelled at high resolution, using the \textit{Centre for Renewable Energy Systems Technology} (CREST) dwelling model \cite{Richardson2010a} and a subset of loads was adjusted using a normal and Rayleigh distribution for solar irradiance and home-charging EVs, respectively.
%The findings in Figure \ref{ch-introduction:fig:lct-impact} show how uncontrolled home-charging of EVs significantly reduces voltages in the network, and although solar injection does lead to voltage rises along the feeder, unbalanced injection increases voltage deviation even further.
%
%\begin{figure}\centering
	\subfloat[]{
		\includegraphics{_introduction/fig/lct-impact-without}
		\label{ch-introduction:subfig:lct-impact-without}
	}
	\vspace{1mm}
	\subfloat[]{
		\includegraphics{_introduction/fig/lct-impact-with}
		\label{ch-introduction:subfig:lct-impact-with}
	}
	\caption{Study that compares impact on network voltages due to an uptake in LTCs on the IEEE LV test case: (a) does not contain any PV generation or EV loads; in (b) 20\% of all customers have PV generation with a peak generation of 3.5kW \cite{MongooseEnergy2015} and 20\% of all customers own EVs with Mode-2 home-charging capabilities at 7.4kW \cite{SustainableEnergyAuthorityofIreland2015}.}
	\label{ch-introduction:fig:lct-impact}
\end{figure}

Such a voltage drop behaviour was achieved with a relatively low rate of LCT adaptation in the residential environment, therefore strict regulation is in place to assure continuous operation without violating any operating constraints.
Otherwise additional voltage deviation, unbalanced network operation or potential asset overloads could be the result.

Traditional network planning approaches to follow these regulation were used to circumvent constraint violations.
These approaches follow the commonly used practice of aggregating a large number of customers and designing the power delivery network to cater for their largest probable demand, i.e. the After Diversity Maximum Demand (ADMD) method \cite{Richardson2010a}.
This ADMD method has remained the same for many years and uses historical load analysis and standard growth assumptions that are both no longer valid in this unprecedented LCT uptake scenario \cite{Yunusov2016}.
To make things worse, LV networks in the UK are generally unmonitored once installed.
Distribution Network Operators (DNOs) have become aware of this issue and are developing updated planning strategies involving ``smart'' and ``flexible'' electricity grids \cite{Fang2012}.
However, in situ equipment that will become subject to the same adaptation of LCT needs to be managed actively via innovation in the use of existing and new technologies; otherwise both frequency of service disruptions and customer minutes lost will increase alongside the proliferation of LCTs \cite{Ault2008a}.


\subsection{Topology and challenges of the UK low-voltage power distribution network}
\label{ch-introduction:subsec:topology-of-lv-network}

Today's UK electricity network has grown over the past century and is based on an interconnected high-voltage gird.
The largest part of this grid is also known as the transmission network, which connects centralised power stations to distribution networks.
Those distribution networks supply electricity to all loads across the mainland of the UK, including industrial, urban and rural customers\footnote[1]{Some small and remote UK islands like the Shetland islands are not connected to this national grid and have their separate electricity infrastructure. Therefore they are not considered as part of this thesis since the study of this kind of network lies outside the research scope.}.

The entire structure of the network is a three-phase Alternating Current (AC) system since this allowed easy voltage level conversion with the use of transformers, i.e. without the need of power electronics.
In the UK, the highest voltage level for generation and transmission is 400kV.
Such a high voltage requires a relatively small current to transmit the generated bulk power, which in turn reduces conduction losses and maximises the efficiency of the high-voltage network.
Regional supply points step-down this high voltage to 132kV\footnote[1]{In some cases regional supply points provide 127kV instead of 132kV.} to deliver power to Distribution Network Operators (DNOs).
From the primary level of the distribution network and onwards, this so called medium-voltage is stepped down to 33kV, then 11kV and finally 400V, in order to cater for heavy industry, medium clients and household sized customers, respectively.

\nomenclature[G]{DER}{Distributed Energy Resource}
\nomenclature[G]{ADMD}{After Diversity Maximum Demand}
\nomenclature[G]{OLTC}{On-Line Tap-Changer}

Primary substations in the UK are equipped with regulation equipment, e.g. On-Line Tap-Changers (OLTC), to increase or decrease the voltage on the secondary transformer side depending on the current level of demand.
Secondary transformers do not have such regulating equipment and instead apply a constant voltage conversion ratio which is set according to the network's typical demand.
This aim of this network regulation is to keep distribution level voltages within their statutory operating bands, i.e. 230V +10\% -6\% for the LV network as specified by the Electricity Supply Quality and Continuity Regulation (ESQRC) \cite{HealthandSafetyExecutive2002} and Engineering Recommendation G59 \cite{EnergyNetworksAssociation2013}.
In the UK, all households are connected to one of the three phases of the 230V distribution network.
To achieve a balanced network, each customer's phase allocation is chosen at random.
Also, throughout the majority of the power network's development period, customers were light consumptive loads, which meant that a reasonably predictable power flew from higher voltage levels towards the lower voltage levels.
Therefore, traditional network planning approaches to circumvent constraint violations, follow the commonly used practice of aggregating a large number of customers and designing the power delivery network to cater for their largest probable demand, i.e. the After Diversity Maximum Demand (ADMD) method \cite{Richardson2010a}.
However, this ADMD method has remained the same for many years and uses historical load analysis and standard growth assumptions, which are both no longer valid in this unprecedented LCT uptake scenario \cite{Yunusov2016}.

Firstly, because the injection of power from Distributed Energy Resources (DERs), e.g. rooftop solar PV, can reverse the power flow, and secondly, because large and volatile electrified products, e.g. home-charging Electric Vehicles (EVs), are predicted to significantly increase demand at peak times.
Such technologies have significant impact on the voltage stability \cite{Petinrin2016}, and combined with the expected increased in phase unbalance, traditional network management methods may no longer be able to effectively mitigate their negative impact.
This means that in situ equipment needs to be managed actively via innovation in the use of existing and new technologies; otherwise not only the frequency of constraint violations will increase, but also the frequency of service disruptions and customer minutes lost is expected to rise alongside the proliferation of LCTs \cite{Ault2008a}.
One such innovative technology, which is the main focus of the presented research, is the installation and management of battery storage \cite{Chen2009}, which is reviewed in the following section.


\subsection{Solutions to mitigate impact of LCT}
\label{ch-introduction:subsec:solutions-to-mitigate-impact-of-lct}

Two solutions exist, allowing DNOs to support LV network's operation: 
\begin{enumerate*}
	\item reinforcement of in situ network assets;
	\item deployment of network support equipment.
\end{enumerate*}
Whilst network reinforcement would certainly address immediate issues of current network capacity constraints, this approach is also the more expensive and disruptive option.
More specifically, customer will need to deal with outages during periods of asset upgrades (e.g. transformer upgrade and line re-conductoring after secondary transformers' tap settings have been adjusted).
Therefore, alternatives to defer or avoid network reinforcements have been sought and assessed \cite{Harrison2007, Zangs2016a, VanderKlauw2016d, Greenwood2017}.
Most promising alternatives are to install flexible and controllable Distributed Energy Resources (DERs), or more specifically: Battery Energy Storage Solutions (BESS) \cite{Wade2010}.
BESS has not only seen significant advancements in technology, but also received increasing attention in both academic studies and industry trials \cite{Palizban2016}.

Installing BESS on a strategic location in the LV network brings several advantages to DNOs' control over the network's performance.
Roles for BESS are addressed in the subsequent section, i.e. Section \ref{ch-literature:sec:role-of-energy-storage-a-survey}.
However, a few examples of potential benefits from BESS include the regulation of voltages to stay within statutory operating bands \cite{Yang2014}, shaving peak loads to relieve stress from the installed network assets \cite{Bennett2015}, and reducing phase unbalance to increase network efficiency \cite{Wang2015b} .
Whilst the questions regarding locating and scaling of BESS have mostly been addressed, BESS control can be split into two complementing yet unmarried approaches:

\begin{enumerate}
	\item ``off-line'' control, using load forecasts and BESS schedules \cite{Cecati2011, Chaouachi2013, Palma-Behnke2013, Khodaei2014}, and
	\item ``on-line'' control, using Set-Points Control (SPC), Model Predictive Control (MPC) or similar dynamic control methods \cite{Salinas2013, Huang2013, Huang2014a, Sun2014a}.
\end{enumerate}

Furthermore, with the anticipated uptake of household BESS, mechanisms to control several storage systems also need to be considered.
For instance, several industry leaders propose to store solar energy in order to support charging of EVs \cite{Baumann2017}.
Without rooftop PV installations, batteries need work in a cooperative manner to not impose additional strain onto the network.
The full review of storage control strategies to achieve both off-line and on-line, as well as centralised/individual and distributed battery control is presented in Section \ref{ch-literature:sec:control-of-energy-storage}.


\subsection{Smart control}
\label{ch-introduction:subsec:smart-control}

\nomenclature[G]{IoT}{Internet of Things}

As already mentioned in Section \ref{ch-introduction:subsec:solutions-to-mitigate-impact-of-lct}, off-line and on-line control strategies exist to manage BESS.
This traditional control often dealt with the dispatch of a single energy entity, but due to the distributed nature of the expected LCT uptake, methods to result in cooperative behaviour needed to be proposed.
With the penetration of smart meters and communication-enabled devices in the so called ``Internet of Things`` (IoT), power systems have the potential of becoming interlinked networks of smart devices, too.
So called ``smart control'' mechanisms complement the traditional off-line and on-line control strategies and are of great research interest to enable the uptake of distributed LCTs.

For example, ``smart charging'' is a key term that summarises EV charging mechanisms where the limited network capacity causes multiple EVs to share the available resource amongst themselves \cite{Sortomme2011, Vaya2012, Garcia-Villalobos2014}.
Intelligently limiting the EVs' maximum charging rates prevents them from adding unnecessary stress onto the network at the cost of longer charging times.
A similar key term is the so called ``smart grid'' where distributed resources communicate and cooperate in order to e.g. shed load using Demand Side Response (DSR) or maintain micrigrid operation in fault situations \cite{Samadi2012, Liu2014, Liang2014}.
The fundamental problem that is addressed in the research field of smart control is to find an optimal balance between network benefits and customer cost.
Also, the fundamental requirement for the successful realisation of smart control is the reliable exchange of information amongst the partaking smart entities.

Hence, smart control does not only require robust control mechanisms, but also a robust communications infrastructure.
Literature, which is reviewed in Chapter \ref{ch-literature}, shows that all control mechanisms dealing with the coordination of distributed energy resources either explicitly or implicitly assume a robust communication infrastructure.
For instance, this requirement is assumed when messages are received without delay and immediately after dispatched or when a single control instruction result in the synchronised reaction of all entities.
In reality however, the strength of the communication link may vary with e.g. weather or current network traffic, and fixed message delays and exact device synchronisation can no longer be guaranteed.
Therefore, not only smart control algorithms, but also their sensitivity to the strength of the underlying communications infrastructure is of interest.

\subsection{Challenges to control BESS}
\label{ch-introduction:subsec:motivation}

From the extensive catalogue of possible roles for energy storage in the electricity grid that was presented in Section \ref{ch-literature:sec:role-of-energy-storage-a-survey}, the focus of the research in this thesis is put on aiding DNOs to manage and operate their power distribution networks.
More specifically, battery energy storage is the main focus since is has the potential to defer or even mitigate costly network reinforcements.
Modern battery technology allows the storage of electrical energy in ever-decreasing form factory, whilst power electronics technology becomes more efficient at integrating batteries into power networks.

As shown in the literature review in Chapter \ref{ch-literature}, methods of controlling BESS to optimise power flow have been of great research interest.
However, the impact on particular key parameters of the three-phase networks still need to be investigated.
Subsequently, the challenge of applying real-time corrections to BESS schedules in order to decrease peak demand whilst obeying to technical and operational constraints is also a remaining research question.
Also, since the expected uptake of distributed LCTs through proliferation of household storage solutions (e.g. to counteract the impact of EVs) requires sophisticated coordination mechanisms, two additional research challenges have been identified.
The first challenge focuses on improving cooperating device behaviour despite communication disturbances (i.e. through message desynchronisation), and the second builds upon the findings from key network improvements to construct a functioning BESS control mechanism despite the absence of a telecommunications infrastructure.







\section{Problem Statement and Research Aim}
\label{ch-introduction:sec:problem-statement}

\section{Contributions and publications}
\label{ch-introduction:sec:contributions}



\section{Publications}
\label{ch-introduction:sec:publications}

\begin{itemize}
	\item M. J. Zangs, P. Adams, T. Yunusov, W. Holderbaum, and B. Potter, ``Distributed Energy Storage Control for Dynamic Load Impact Mitigation,'' Energies, vol. 9, no. 8, p. 647, Aug. 2016. doi: 10.3390/en9080647
	\item M. J. Zangs, T. Yunusov, W. Holderbaum and B. Potter, ``On-line adjustment of battery schedules for supporting LV distribution network operation," 2016 International Energy and Sustainability Conference (IESC), Cologne, 2016, pp. 1-6. doi: 10.1109/IESC.2016.7569485
\end{itemize}

\section{Thesis structure}
\label{ch-introduction:sec:thesis-structure}


\chapter{Literature Review}
\label{ch-literature}

\section{Overview}
\label{ch-review:sec:overview}


Technology advancements and increasing popularity of renewable energy sources, combined with government incentives to support their uptake, also lead to a significant rise in Distributed Energy Resources (DERs).
Yet to allow DERs to be installed without significant negative impact on the local MV or LV networks, required functions that large scale power levelling systems could not provide. More specifically, fast response to counteract highly volatile loads or unpredictable and distributed DERs; e.g. home PV installations \cite{Jewell1987}. 


\section{Role of energy storage - a survey}
\label{ch-literature:sec:role-of-energy-storage-a-survey}

\begin{figure}\centering
	\includegraphics[width=\textwidth]{_literature/fig/storage-financial-benefits}
	\caption{Energy storage applications and corresponding value for various discharge durations \cite{Deloitte2016}}
	\label{ch-literature:fig:storage-financial-benefits}
\end{figure}

The idea of using energy storage in the electricity grid has been discussed for quite some time, and its important role in future energy systems has already been identified in the 70s \cite{Kalhammer1979}.
As the name suggests electrical energy storage systems have the ability to consume, store and release electrical energy by converting it into a different form of energy.
Depending on the rate at which energy can be consumed and released (i.e. the system's power rating) as well as the amount of energy that can be stored $\text{(i.e. system's capacity)}$ different functions can be provided.
A Canadian and US based study for the Department Of Energy (DOE) showed that (when correctly exploited) these functions can yield direct financial benefits of \$157.56 billion on a national level over an estimated 10 year system lifecycle \cite{Eyer2010a}.
Figure~\ref{ch-literature:fig:storage-financial-benefits} shows these benefits in relation to the storage system's typical discharge period, and links them to their associated functions, too.
Here, Time Of Use (TOU) energy cost management yields the largest economic profit, yet from a historical point of view, bulk energy storage has played the most important role in the energy system.

Nowadays, storage can also tap into emerging revenue streams and perform additional functions.
As identified in several review articles \cite{Chen2009, Katsanevakis2017, Guney2017}, the key roles and applications of energy storage systems, regardless of profitability in the current market situation, can be identified as follows:

\begin{itemize}
\item
\textbf{Energy shifting} \hlrem{- }\hladd{(}arbitrage\hladd{)}: This function uses the difference in energy price to yield revenue.
More specifically, as energy pricing is expected to become more dynamic and responsive to current energy demand and generation, storage is controlled to charge when energy prices are low and discharge when energy prices are high \cite{Chen2009, Leou2012}.
Such dynamic pricing schemes are expected to emerge due to significant changes in demand at morning and evening peaks \cite{Koohi-Kamali2013}.
However, for small storage may not be sufficient to justify energy shifting in LV networks.
\hladd{(Time-of-use energy charges): A hurdle to DSM through flexible tariffs or TOU tariffs is the reason that consumers would have to adjust their energy consumption based on external price signals, which many are do not want to do.
Energy storage could however decouple the consumer from these tariffs and allow them to continue with their normal lifestyle }\cite{Khani2014}\hladd{.
Additionally, when exploiting the energy price difference, storage could even supply arbitrage functions to some customers and reduce their electricity bill }\cite{Nair2010a}\hladd{.
For customers with local generation (for example through a PV installation) their bill can be reduced even further.
This would be done by storing the generated energy until a period of high energy prices arises.
At this time energy storage could release the energy to maximise self-consumption }\cite{Luthander2016}\hladd{.}
\hladd{(Renewables integration): Unlike traditional energy sources, renewables have are highly volatile and have limited availability.
Since their availability (for example for PV installations) may not align with periods of high demand (i.e. during morning and evening) arbitrage functions may be provided to maximise the use of renewable generation - i.e. renewables ``shifting'' }\cite{Zakeri2015}\hladd{.
Furthermore, by discharging energy storage during times of low renewable generation (for example due to cloud cover or varying wind speeds }\cite{Jewell1987}\hladd{) a continuous supply of energy can be assured - i.e. renewables ``smoothing''.
And lastly, if a renewable resource was committed for longer periods of time, yet the associated energy forecasts overestimated its generation capacity, storage can supply the gap to avoid balancing charges - i.e. renewables ``firming'' }\cite{Chakraborty2012}\hladd{.}
\item
\textbf{Supply capacity}: In order to meet future energy demand, energy suppliers commit their resources in advance.
Doing so allows them to plan for their operation and solve the economic dispatch problem.
With increasing demand, the supply volume will have to increase, too.
However, it is predicted that energy storage can defer or even avoid investments in power plants, assuming they are sized accrodingly (i.e. several 100MW)\cite{Dobie1998}.
Bulk energy storage was the first choice to support supply capacity.
One example is pumped hydro-electric energy storage, which has seen a global growth of 127GW since 1979 \cite{Rehman2015, Barbour2015, Barbour2016}.
\item
\textbf{Ancillary services}: These services are of interest to transmission and distribution system operators since they support the operation of their networks.
For example, load following and frequency regulation are two complementing applications of that address the imbalance between demand and supply \cite{Bevrani2011}.
In case of a severe imbalance that resulted in network outage, black start is also a function that can be supplied by energy storage \cite{Cole1995, Kashem2007}.
Since modern energy storage systems can absorb and inject both active and reactive power, they can also provide voltage support \cite{Kulkarni2005}.
\item
\textbf{Grid stability}: To make the grid more resilient to network faults (\hlrem{e.g.}\hladd{for example} short-circuit or loss of a large generator), or to overcome scheduled network outages, energy storage can be used as an intermittent energy source \cite{Kundur1993}.
To provide optimal operation conditions for energy generators, storage can support rotor angle stability and voltage stability by injecting active and reactive power at the point of common coupling \cite{Chakraborty2012, Kolluri2002}.
Furthermore, sub-synchronous resonance and harmonic interference can also be reduced \cite{Wang1994}.
This coupling resonance can occur between electrical and mechanical systems and can damage the mechanical structure due to repetitive stresses and strains.
\item
\textbf{Upgrade deferral}: As already stated in Section~\ref{ch-introduction:subsec:solutions-to-mitigate-impact-of-lct}, both transmission and distribution systems would have to be upgraded unless energy storage could provide network-support functions.
By deferring network upgrades, network assets will be used more efficiently, and customer supply disruptions will be avoided \cite{Sayer2007, Eyer2010a}.
Furthermore, in areas where the expected load has already been met and growth has levelled out, deployed energy storage is flexible enough to provide alternative functions (unlike other network assets) \cite{Huff2013}.
\hladd{Equally, high congestion at substations of heavily loaded transmission or distribution lines can be tackled by co-located energy storage units }\cite{Saez-de-Ibarra2013a, Kulkarni2005}\hladd{.
This can be achieved by, for example, shaving peak load or relaxing the energy requirements from distributed generation }\cite{Reihani2016, Gerards2016d}\hladd{.}
\item
\textbf{Transmission charges}: In scenarios where generators are charged to use transmission systems (due to the capacity limitations of the transmission system), energy storage could take advantage of the price structure to maximise the profit from the generated energy \cite{Sayer2007, Leou2012}.
% TODO: Remove item
\item
\hlrem{\textbf{Congestion relief}: High congestion at substations of heavily loaded transmission or distribution lines can be tackled by co-located energy storage units [XX].
This can be achieved by e.g. shaving peak load or relaxing the energy requirements from distributed generation [XX].}
\item
\textbf{Service reliability}: In areas where a strong grid connection is needed to assure \hlrem{e.g.}\hladd{for example} industry operations, an ``uninterruptible power supply'' may be required.
Traditionally, these power supplies were diesel backup generators, but modern energy storage technology can provide similar services at lower cost \cite{Schoenung2001} (particularly when including alternative revenue streams).
% TODO: Remove item
\item
\hlrem{\textbf{Power quality}: Sub-cycle and harmonic distortions can severely deteriorate power quality, since they have unwanted effects on connected equipment (similar to the issue of sub-synchronous resonance at the generation side).
Energy storage with modern power electronics could be capable of providing power filtering functions that suppress those distortions [XX].
This feature could be of particular interest to LV networks in the UK, since customers are arbitrarily connected to a single phase of a three-phase network.
Therefore, the discrepancy of power quality between the phases is even larger, yet available energy storage resources could even address this issue [XX] (especially when considering household connected units).}
% TODO: Remove item
\item\hlrem{\textbf{Time-of-use energy charges}: A hurdle to DSM through flexible tariffs or TOU tariffs is the reason that consumers would have to adjust their energy consumption based on external price signals, which many are do not want to do.
Energy storage could however decouple the consumer from these tariffs and allow them to continue with their normal lifestyle [XX].
Additionally, when exploiting the energy price difference, storage could even supply arbitrage functions to some customers and reduce their electricity bill [XX].
For customers with local generation (e.g. through a PV installation) their bill can be reduced even further.
This would be done by storing the generated energy until a period of high energy prices arises.
At this time energy storage could release the energy to maximise self-consumption [XX].}
\item
\textbf{Demand charges}: Larger customers (i.e. industrial and commercial loads) are not only charged for their total energy demand, but also for their largest continuous power demand \cite{Oudalov2007, Mackey2013}.
Therefore, a factory that may use a relatively small amount of energy over a comparatively short amount of time, is billed accordingly.
After all, the infrastructure to deliver the required power needs to be installed and maintained.
In this scenario, energy storage could reduce the intermittent power demand without significantly increasing the total energy demand, and therefore reduce demand charges for larger customers \cite{Aghaei2013}.
% TODO: Remove item
\item
\hlrem{\textbf{Renewables integration}: Unlike traditional energy sources, renewables have are highly volatile and have limited availability.
Since their availability (e.g. for PV installations) may not align with periods of high demand (i.e. during morning and evening) arbitrage functions may be provided to maximise the use of renewable generation - i.e. renewables ``shifting'' [XX].
Furthermore, by discharging energy storage during times of low renewable generation (e.g. due to cloud cover or varying wind speeds [XX]) a continuous supply of energy can be assured - i.e. renewables ``smoothing''.
And lastly, if a renewable resource was committed for longer periods of time, yet the associated energy forecasts overestimated its generation capacity, storage can supply the gap to avoid balancing charges - i.e. renewables ``firming'' [XX].}
\item
\hladd{\textbf{Power quality}: In addition to the above-mentioned benefits, power electronics provided with the BESS may also be used beneficially.
Sub-cycle and harmonic distortions for instance can severely deteriorate power quality, since they have unwanted effects on connected equipment (similar to the issue of sub-synchronous resonance at the generation side).
Energy storage with modern power electronics could be capable of providing power filtering functions that suppress those distortions }\cite{Putrus2007}\hladd{.
This feature could be of particular interest to LV networks in the UK, since customers are arbitrarily connected to a single phase of a three-phase network.
Therefore, the discrepancy of power quality between the phases is even larger, yet available energy storage resources could even address this issue }\cite{Miret2009}\hladd{ (especially when considering household connected units).}
\end{itemize}

This extensive list of possible applications for energy storage systems emphasises the potential for energy storage solutions in the future energy market.
However, as also stated by Taylor et al. in \cite{Taylor2016}: ``\textit{The market for use [of electrical energy storage] is motivated by the need to increase the efficiency of the grid by the integration of RES}''.
For this very reason, upgrade deferral, congestion relief, ancillary services (i.e. voltage support) and renewable integration are the key challenges that are of interest to DNOs.
This finding is also supported by the motivation of research projects and field trials that were conducted with energy storage solutions in the LV distribution networks.
These research projects are reviewed in the subsequent section, Section~\ref{ch-literature:sec:energy-storage}.










\section{Energy storage research for LV application}
\label{ch-literature:sec:energy-storage}

The challenge for DNOs to manage their distribution networks is caused by the DER's and LCT's difficult predictability, their volatile nature, and the weakness of the network into which they are deployed.
If left unmanaged, voltage fluctuations caused by e.g. PV systems \cite{Woyte2006, Bravo2015} or capacity shortages due to additional loads like EVs \cite{Mohd2008a, Koureoumpezis2010} will threaten the power system's stability.
Improved network management methods that are summarised under the term ``smart grid'' have thus become increasingly popular to counteract the negative impact from DERs and LCTs \cite{Panteli2015}.
When however deferring the reinforcement or retrofitting of network assets to construct such a smart grid, deployment of BESS can provide a significant contribution to the integration of DERs and LCTs.
For instance, Grillo et al. in \cite{Grillo2012} showed how probabilistic price driven storage control successfully supports renewable integration.
Their simulated and validated BESS model provides arbitrage functions through generation shifting and was able to achieve a daily gain of more than \texteuro130.
However, such an immediate financial benefit can only be achieved when their dynamic pricing is implemented and the repetitive discharge to 20\% does not shorten the BESS lifetime.
Focusing on grid support instead, Rowe et al. in \cite{Rowe2014a} showed how a BESS schedule can maximise the peak reduction capability in order to free system resources.
Since their BESS schedules were based on sometimes unreliable demand forecasts, filtering operations varying the forecast's peak magnitude, peak width and peak shape were implemented.
This filtering maximised the resulting peak reduction performance to a median peak reduction between 5kW to 7kW - instead of 0kW if no forecast filtering was implemented.
Similarly Hosseina et al. in \cite{Hosseina2016a} also used residential load forecasts to schedule BESS operation in order to level demand by shaving peak load.
Their BESS was however installed in the medium voltage distribution network since their redox flow-battery was significantly larger than the lithium-ion battery that was used by Rowe et al - i.e. 34MWh instead of 25kWh.
Apart from the difference in scale, both pieces of research used residential load forecasts only, whilst Li et al. in \cite{Li2016} solved a stochastic unit commitment and economic dispatch problem to maximise renewable integration.
Unlike traditionally scheduled BESS operation, Li et al. also simulated real-time operation but assumed perfect demand and pricing knowledge at the time of operation.
As a result, they achieved a financial gain of more than \$34000, but did not revile the BESS impact on the underlying power distribution network.

Over the past decade electricity supplier branches from the Big Six (i.e. the UK's six major energy suppliers) begun trialling of BESS across their distribution network to better their understanding and potential contribution, since the BESS benefits had only been estimated and not thoroughly studied.
Showcase examples from some DNOs include:
Scottish and Southern Electricity Networks (SSEN) in \cite{NTVV2016} where BESSs were deployed in Bracknell distribution networks to uphold voltage stability and power quality;
EDF Energy Networks in \cite{Wade2010} where BESS was installed in the 11kV distribution network for power flow management and to validate and improve system models;
UK Power Networks (UKPN) in \cite{Lyons2015a} where BESS was installed to shave load peaks and level supply volatility from an adjacent wind farm;
E.ON UK in \cite{EON2017} where a 5MWh BESS was colocated with a combined heat and power plant to stabilise its energy supply;
and Scottish Power in \cite{ScottishPower2016} where 1MWh of distributed batteries were installed in households to support grid operation through flexible tariffs.
Nonetheless, from lessons learnt and aiming to meet statutory and physical restrictions under the future load changes, voltage control and the power flow problem have been identified as the two key challenges for DNOs \cite{Ferreira2013a, Shi2015}.

\subsection{Voltage control}
\label{ch-literature:subsec:voltage-control}

LV distribution networks in the UK operate at 230V Phase to Neutral (P2N) or 400V Phase to Phase (P2P) and have a statutory tolerance band of +10\% and -6\%.
Due to the varying load on the network, these voltages can deviate significantly.
Although todays deviation may not exceed the high-voltage or low-voltage thresholds, conduction losses and imperfect network conditions result in a lower overall system efficiency.
Traditionally, OLTC are used to raise and lower voltages across the entire LV distribution network in order to counteract voltage deviations \cite{Sun2009}.
However, such a hierarchical voltage control with OLTCs has its limited applicability, especially in cases where the voltage deviation significantly differs for several branches of a feeding network \cite{Zangs2016}.
More specifically, if voltages diverge along different branches or different phases of a feeder due to asymmetric loads, then the adjustment of transformer taps will lead to high or low voltage violations regardless of the tap change direction.
Installing a BESS at a strategic location, i.e. closer to the regions where voltage deviation takes place, and controlling the device to best suit the network's requirements is generally applicable and the commercially more viable alternative \cite{Liserre2010}.

As stated by Wade et al. \cite{Wade2009}, allocation of the BESS's limited storage capacity so it can solve the voltage problem most effectively is still a sophisticated challenge.
Nonetheless, by installing a 200kWh unit that is rated at 600kW in a project that was carried out with \textit{EDF Energy}, they showed the potential of BESS in a network to provide targeted voltage support \cite{Wade2010}.
Their results for a 0.4MWh BESS achieved a reduced voltage variation by 2.4\% which resulted in a complete elimination of any ``out-of-limit'' voltage events, and a 70.96\% reduction of all network events (including e.g. power events) over the annual simulation period.
A demonstration project in Germany that was titled ``More Microgrids'' used four 180kWh batteries and demonstrated how both voltage stability as well as grid independence could be improved \cite{Overbeeke2010}.
In this ``More Microgrids'' project a collection of holiday homes were fitted with a distributed PV system that is capable of generating a peak power of 315kW, and BESS was used to maximise the utility from this generation.
However, due to the relatively small size of the network, due to the different behavioural patterns of holiday home occupants, and due to the different means of connecting customers to the German three-phase network, voltage deviation and phase unbalance issues were not as big as a concern as they are for UK distribution feeders.
An equally sized project entitled ``GROWDERS'' also used multiple BESS in the LV network, but instead of focusing at grid independent network operation, they mainly contributed to frequency and thermal constraints as well as voltage stability \cite{GROWDERS2011}.

BESS that are sized between 100kWh to 200kWh (as those in the aforementioned projects \cite{Wade2010, Wade2009, Overbeeke2010, GROWDERS2011}) can easily address network issues, especially even when operating in a grid independent or ``islanded'' mode.
Results from these early field trials show how BESS store the excess renewable power for usage during later times.
But once capacity limits were reached, neither high or low voltage events could be omitted.
An oversized BESS would be less likely to meet its operating limits, but the associated cost makes this oversizing unfeasible.
Their findings therefore indicate that not only the sizing, but also the BESS control method is of significant importance.
Nonetheless, continuous voltage violations that require strong voltage support have not yet been encountered in any of these projects, and instead occasional violations accumulating to e.g. less than an average of 3.4 minutes per day are the norm \cite{Sugihara2013}.
Also, the majority of low-voltage events on the UK distribution networks, i.e. when voltage levels fall below 216.2V, are caused by anomalous network events or failures of the measurement equipment \cite{UKPowerNetworks2014a}.
Therefore, the complementing task of choosing correct control methods to optimally manage the network's power flow is also important.

\subsection{Power flow management}
\label{ch-literature:subsec:power-flow-management}

Interest in BESS control for power flow management has grown since improved measurement equipments in LV substations is more reliable and precise than traditional smart meter readings, but also since excessive power flow is the main cause for operational issues which do eventually lead to system overloads and outages\footnote{
In fact, according to the UK energy regulator \textit{OFGEM}, on average 45\% of all customers experienced service disruptions in the period 2015-16 \cite{Ofgem2017}.
Whilst unanticipated outages due to severe winter weather did lead to \pounds39 million worth of damages, network upgrades to prevent outages and repairs after outages had happened, did however contributed the larger amount of customer interruptions and customer minutes lost \cite{Ofgem2014}.
} \cite{Putrus2009, Pillai2010}.
To prevent future power flow from exceeding the system's capacity, BESS has been proposed to function as an instantaneous reserve \cite{Kunisch1986a, Kunisch1986}.
Resulting methods like BESS droop control use local voltage and frequency measurements to infer the latest loading and stress on the network to issue corresponding BESS control instructions \cite{Engler2005a}.
The initial simulations in \cite{Engler2005a} showed how droop control can effectively remove reactive power demand and thus free the corresponding resource. 
Conventional droop control was designed to inject reactive power into high voltage transmission lines to counteract voltage drops and inject active power to counteract phase shift \cite{Tayab2017}.
This control mechanism works since the impedance of high voltage transmission networks is more inductive than resistive.
LV distribution networks on the other hand are more resistive in nature.
Droop control for the LV applications is therefore founded on the assumption that network frequency will drop as demand begins to exceed supply, and that voltages along the distribution feeder drop more significantly when load is increased.
Yet as already stated in Section~\ref{ch-introduction:subsec:topology-of-lv-network}, reversed power flow can raise voltage levels, which makes such droop control methods less reliable and potentially unsuitable for LV network support.
This problem was also encountered by Riffonneau et al. in \cite{Riffonneau2011}, where they control BESS to solve an optimal power flow problem for grid connected PV systems.
Ultimately, they were able to achieve a 13\% reduction of electricity bills by implementing a rule-based dynamic programming optimiser, and they reduced peak power by successfully integrating PV.
However, they do not consider reactive power within their power management method, although it could free additional network resources and yield benefits to the distribution network.
The reason behind excluding it form their study was due to the potential conflicts that may arise with the proposed voltage control method, which heavily relies on voltage measurements.
Using BESS to reallocate PV generation for maximised self-consumption \cite{SaniHassan2017} or to achieve ``peak-shaving'' behaviour \cite{Bennett2015, DePaola2016} has seen continued interest in the field of BESS power flow management.

\begin{figure}\centering
	\subfloat[]{
		\includegraphics[width=0.415\textwidth]{_literature/fig/esu}
		\label{ch-literature:subfig:esmu-esu}
	}
	\subfloat[]{
		\includegraphics[width=0.585\textwidth]{_literature/fig/peu}
		\label{ch-literature:subfig:esmu-peu}
	}\\
	\subfloat[]{
		\includegraphics[width=0.7\textwidth]{_literature/fig/esmu-1}
		\label{ch-literature:subfig:esmu-esmu-1}
	}\\
	\subfloat[]{
		\includegraphics[width=0.7\textwidth]{_literature/fig/esmu-2}
		\label{ch-literature:subfig:esmu-esmu-2}
	}
	\caption{Energy Storage Management Unit overview: (a) 12.5kWh Energy Storage Unit, (b) Power Electronics Unit, (c) deployed 12.5kWh system, (d) deployed 25kWh system - pictures are taken from the NTVV close down report \cite{NTVV9.8a}}
	\label{ch-literature:fig:esmu-esu}
\end{figure}

Aiming to address both voltage and power flow problems, \textit{Scottish and Southern Electricity Networks} (SSEN) became the first UK network operator to trial street-level BESS deployment in the LV network, and they installed 500kWh worth of storage in Bracknell, UK \cite{SSEN2016}.
This capacity was achieved by 25 Energy Storage Management Units (ESMUs), like those pictured in Figure~\ref{ch-literature:fig:esmu}.
Each ESMU had cascadable 12.5kWh Energy Storage Units (ESUs), and the ESUs were connected to the distribution network via a three-phase 36kW Power Electronic Unit (PEU) to both manage the batteries and perform filtering operations.
The aim of this so called \textit{New Thames Valley Vision} (NTVV) project was to understand potential benefits, practicalities and costs of installing street-level BESS.
In the beginning, the main problem of finding an optimal deployment location for the ESMUs, to achieve their best possible impact on system voltages had to be addressed.
Yunusov et al. and Rowe et al. worked in collaboration with \textit{SSEN}, and they assessed different BESS locations in several networks \cite{Yunusov2016, Rowe2014, Rowe2014a}.
They found that a location 4/7 to 2/3 down the feeder yields the best overall impact on voltage levels.
However, their findings also show that this location can vary significantly when not only focusing on voltage support; i.e. proximity to the feeding substation was of greater importance when reducing the system's overloads or distribution losses.
Also, the chosen control system had significant impact on the BESS performance, which is why more emphasis has been put on BESS control instead of locating or constructing BESS.
Therefore, a review of BESS control methods including those that are implemented in the NTVV project are presented in the next section, Section~\ref{ch-literature:sec:control-of-energy-storage}.










\section{Control of energy storage and its applications}
\label{ch-literature:sec:control-of-energy-storage}

Installing BESS at a strategic location in the LV network brings several advantages to DNOs' control over the network's performance.
Regulating voltages to stay within statutory operating bands \cite{Yang2014}, shaving peak load to relieve stress from the installed network assets \cite{Bennett2015}, or reducing phase unbalance to increase network efficiency \cite{Wang2015b} are only a few examples of recent research in this field.
Whilst the questions regarding locating and scaling of BESS have mostly been addressed, BESS control still remains an open question and can be split into two complementing yet unmarried approaches:

\begin{enumerate}
	\item ``off-line'' control, using load forecasts and BESS schedules; and
	\item ``on-line'' control, using Set-Points Control (SPC), Model Predictive Control (MPC) or similar dynamic control methods.
\end{enumerate}

These two control approaches have evolved from two different fields of active network management.
Nonetheless, both approaches hold significant benefits to the operational performance of power distribution networks and neither of the two can be neglected.
Therefore, Section \ref{ch-literature:subsec:off-line-and-on-line-control} addresses and discusses the two control approaches and their missing link.

The current form of the NTVV project focuses on controlling a single BESS in the LV distribution network.
However, the uptake of household connected BESS will increase the number of distributed systems, which need to be managed cooperatively.
Therefore, Section \ref{ch-literature:subsec:centralised-and-distributed-control} reviews and discusses different control approaches for distributed BESS, since the control of multiple single-phase storage units in a three-phase network is inherently more challenging that controlling a single three-phase device.

\subsection{Off-line and on-line control}
\label{ch-literature:subsec:off-line-and-on-line-control}

Off-line control uses historic data to predict future load patterns, which are used to schedule BESS operation accordingly.
Early approaches, e.g. by Oudalov et al. \cite{Oudalov2007}, who used dynamic programming to generate BESS schedules, had relatively high forecast errors due to the inherent difficulty of predicting future loads.
These errors ultimately limit the ability of any given BESS schedule to e.g. effectively reduce peaks.
This is why recent research either includes uncertainty, like the work done by Baker et al. \cite{Baker2017} where uncertainty of wind power was taken into account when scheduling and sizing BESS.
Other work frequently re-evaluates BESS schedules to control and adjust its schedules after completing individual decision epochs \cite{Wang2014a}.
Nonetheless, load forecasting remain a key component for BESS scheduling despite those load forecasts (and the resulting BESS schedules) being imperfect.
This fact was emphasised by Rowe et al. in \cite{Rowe2014a}, and they developed a filtering mechanism for scheduling algorithms to reduce peak load in LV networks in spite the presence of forecast errors.
They also highlight the fact that most day-ahead load forecast only predict at a temporal resolution down to half-hourly periods.
The reason behind this choice was pointed out by Haben et al. in \cite{Poghosyan2014, Haben2014}, as they argue that forecasts at half-hourly resolution yield the best compromise between high accuracy and high temporal resolution.
Therefore, half-hourly forecasts have become the standard for generating any resource commitment and resource operation schedules.
However, sub-half-hourly load volatility imposes the biggest stress on the network and it is this volatility that cannot be addressed when relying on half-hourly forecast alone.
Therefore, on-line control has been considered as an alternative to off-line control.

One flavour of on-line control is the Set-Point Control (SPC), which is a robust technique that can immediately respond to network changes.
Since this kind of control runs the risk of reaching shortage or surplus of BESS stored energy, modifications like hysteresis control and ramp-rate control were proposed \cite{Gybel2012, Blaabjerg2006, Malesani1990, Such2012}.
In \cite{Such2012}, Such and Hill showed how a ramp controlled BESS could smoothen the volatile power from wind generation.
Furthermore, their work shows how reverse power flow can be completely omitted through the use of on-line BESS control.
However, this kind of on-line control is less effective in addressing daily demand peaks, since pure SPC can only react to present network demand and does not respond to general trends or upcoming load events.

In order to address these shortcomings SPC has been extended by using short-term load predictions through the implementation of Model Predictive Control (MPC).
Some MPC examples include Auto-Regressive (AR) models \cite{Li2009, Nie2011}, fuzzy logic models \cite{Sannomiya2001, Chen2013a}, genetic algorithms \cite{Xia2015a, Liu2015} or Artificial Neural Networks (ANN) \cite{Kalogirou2014, Quan2014, Lee2014, Pezeshki2014, Vaz2016, Reihani2016, Xiao2017}.
The increasing complexity of MPC yields a better prediction performance.
For instance, Reihani et al. in \cite{Reihani2016} use the most recent 20 minutes of load information with a complex-valued ANN to predict the next 20 minutes of minutely load variations.
Since their raw forecasts were more erratic than the actual load profile, a Kalman filter was implemented to smoothen the MPC's output, yet this step introduced significant discrepancies between the actual and the forecasted load.
Therefore, they increased MPC complexity even further by taking into account parallel time-series, i.e. they considered the same 20 minutes from previous days in the prediction mechanism.
This addition produced significantly better results and they shaved daily peaks by around 300kW.
Implementing such increasingly complex MPC to support on-line control is therefore a promising research trend, however the computational burden to deliver real-time solutions makes implementation of such systems not yet feasible.

Finding a way of combining both scheduled BESS operation, which is executed at half-hourly resolution, with a responsive correction mechanism is therefore an open research question that has not yet been addressed.
Objectives 1 and 2, as outlined in Section \ref{ch-introduction:sec:problem-statement}, and their corresponding chapters, respectively Chapter \ref{ch1} and Chapter \ref{ch2}, address this question in two ways.
First, by assessing the ability to apply scheduled BESS operation onto a three-phase network in a sub-half-hourly manner, yet without modifying its underlying half-hourly schedule.
Then, the scheduling constraint is lifted by allowing an operational tolerance in order.

\subsection{Centralised and distributed control}
\label{ch-literature:subsec:centralised-and-distributed-control}

Existing literature addresses the usage of energy storage units in low-voltage distribution networks to assure voltage security \cite{Sugihara2013, Toledo2013, Marra2013, Mokhtari2013, Atia2016}.
An approach used by, e.g., Mokhtari et al. in \cite{Mokhtari2013} relies on bus voltage and network load measurements to prevent system overloads.
Yet, these kinds of storage control systems do require communication infrastructures to relay the network information and control instructions.
This requirement has also been addressed in the comprehensive review on storage allocation and application methods by Hatziargyriou et al. \cite{Hatziargyriou2015}.
In the presented work, a control algorithm is proposed that removes the need for such an inter-BESS communication, since it only uses local voltage measurements to infer the network operation.
Yet, to prevent conflicting device behaviour, the underlying coordination mechanism is of particular importance. Assuring convergence, the AIMD algorithm is perfectly suited for such coordinated control.

Originally, AIMD algorithms were applied to congestion management in communications networks using the TCP protocol \cite{Chiu1989}, to maximise utilisation while ensuring a fair allocation of data throughput amongst a number of competing users \cite{Wirth2014}.
AIMD-type algorithms have previously been applied to power sharing scenarios in low voltage distribution networks, where the limited resource is the availability of power from the substation's transformer.

For instance, such an algorithm was first proposed for EV charging by St{\"{u}}dli et al. \cite{Studli2012}, requiring a one-way communications infrastructure to broadcast a ``capacity event'' \cite{Studli2014, Studli2014a}. Later, their work was further developed to include vehicle-to-grid applications with reactive power support \cite {Studli2015}. The battery control algorithm proposed in this paper builds upon the algorithm used by Mareels et al. \cite{Mareels2014}, where EV charging was organised by including bidirectional power flow and the use of a reference voltage profile derived from network models. Similar to the work by Xia et al. \cite{Xia2014}, who utilised local voltage measurements to adjust the charging rate, only voltage measurements at the batteries' connection sites were used in this work to control the batteries' operations.

Multi-Agent Systems (MAS) have also been used in several studies to yield voltage support \cite{Baran2007}


\section{Summary of gaps in literature}
\label{ch-literature:sec:literature-gaps}

In this chapter, Chapter~\ref{ch-literature}, the current and future roles for energy storage have been laid out.
When focusing on BESS applications that support DNO owned networks, i.e. to enable the integration of LCTs and DERs within the LV distribution network without the need for network reinforcements, two key functions have emerged:
\begin{enumerate*}
	\item limiting voltage deviation to within statutory regulations, and
	\item avoid thermal constraints by solving the power flow problem.
\end{enumerate*}
Since DNOs had little experience with using BESS in their LV networks, several research projects and field trials were undertaken over the past decade.
So far this research has already focused on sizing, locating and operating BESS.
From the presented literature BESS control methods can be split into two  categories that still remain unlinked: off-line control (\hlrem{e.g.}\hladd{for example} scheduled or forecast driven control) and on-line control (\hlrem{e.g.}\hladd{for example} SPC or MPC).
Whereas off-line control takes into account daily load trends (i.e. at half-hourly resolution) it cannot compensate for load volatility due to DERs and LTCs (i.e. at sub-half-hourly resolution).
On-line control methods on the other hand are designed to react quickly when system changes occur (i.e. at sub-half-hourly resolution), but they cannot efficiently include daily or weekly load patterns (i.e. at half-hourly resolution) due to the increase in model complexity.
On the basis of the gaps in literature, as highlighted in the literature review, Chapter~\ref{ch-literature}, and the problem statement of this thesis which is stated in Section~\ref{ch-introduction:sec:problem-statement} research \ref{objective-1} and \ref{objective-2} were derived.
Furthermore, as the number of DERs increases throughout the grid methods to manage them need to become more sophisticated, too.
However, all developed algorithms to control DERs either explicitly or implicitly assume synchronisation amongst all controlled entities which need not be the case in reality.
Assessing how information desynchronisation impacts the performance of a distributed algorithm is still an open research question that is addressed by \ref{objective-3}.
\ref{objective-4} then aims to extend a distributed control algorithm by developing a method that no longer depends on communication systems.
To summarise, the problems that arises from the identified gaps in literature are:

\begin{itemize}
	\item how to assign a BESS power profile that is pre-scheduled at half-hourly resolution to the three-phase network that operates at sub-half-hourly resolution in order to yield the largest positive and targeted impact on the underlying network performance parameters (\ref{objective-1}),
	\item how to adjust a half-hourly BESS schedule (derived from a realistic but erroneous load forecast) based on sub-half-hourly load variations to minimise daily peak demands at both temporal resolutions (\ref{objective-2}),
	\item how large the impact will be on the performance of a scheduling and control algorithm when information exchange or message passing amongst the distributed control entities becomes desynchronised (\ref{objective-3}), and
	\item how multiple BESS can be coordinated in a communication-less environment to circumvent the need for ICT whilst contributing to voltage stability and thermal constraints without allocating their energy resources unevenly (\ref{objective-4}).
\end{itemize}

Despite some of the literature including aspects of the proposed research, none of them answer the research questions that are identified above.
The novelty of the research in this thesis consists of combining on-line and off-line control, as well as to assess and extend the control of distributed BESS.
All contributions, corresponding publications and draft papers, as outlined in Section~\ref{ch-introduction:sec:contributions} and Section~\ref{ch-introduction:sec:publications}, reflect upon the novelty of the presented research against the objectives as well as their aim and gaps in literature upon which they are founded.


\section{Summary and Gaps in Litearture}
\label{ch-review:sec:summary}

\chapter{Multiple Batteries}
\label{ch1:multiple_batteries}

\section{Introduction}
\label{ch1:sec:introduction}

The adoption of electric vehicles (EV) is seen as a potential solution to the decarbonisation of future transport networks, offsetting emissions from conventional internal combustion engine vehicles. The current rate of EV uptake is anticipated to increase with improved driving range, reduced cost of purchase and greater emphasis on leading an environmentally-friendly lifestyle \cite{Shah2015}. It is predicted that by 2030, there will be three million plug-in hybrid electric vehicles (PHEV) and EVs sold in Great Britain and Northern Ireland \cite{DBER2008}, and it is expected that by 2020, every tenth car in the United Kingdom will be electrically powered \cite{Ecolan2013}. It is anticipated that the majority of PHEV/EV will be charged at~home, putting additional stress on the existing local low voltage distribution network, which must then cater for the increased demand in energy \cite{Clement-Nyns2010, PieltainFernandez2011}. Uncontrolled charging of multiple PHEV/EV can raise the daily peak power demand, which leads to: increased transmission line losses, higher voltage drops, equipment overload, damage and failure \cite{Hadley2009, Putrus2009, Pillai2010, Zhou2014}. Accommodating the increased demand and mitigation of such failures is a major area of research interest, with the focus mainly placed on the coordinating and support of home charging.

Demand Side Management (DSM) strategies for Distributed Energy Resources (DER), aim to alleviate the impacts of PHEV/EV home-charging and are a favoured solution. Mohsenian-Rad et al. in \cite{Mohsenian-Rad2010} developed a distributed DSM algorithm that implicitly controls the operation of loads, based~on game theory and the network operator's ability to dynamically adjust energy prices. Focusing on financial incentive-driven DSM strategies, in \cite{Deilami2011}, a Time-Of-Use (TOU) tariff and real-time load management strategy was proposed, where disruptive charging is avoided by allocating higher prices to times of peak demand. Financial incentives have also become a drive towards optimising the operation of Battery Energy Storage Solutions (BESS) and Distributed Generation (DG) when including PHEV/EV into the problem formulation \cite{Masoum2015}.

Research focused on grid support has been driven by the need to deliver long-term savings and to avoid the immediate costs and disruption of network reinforcements and upgrades. This~area of research proposes the implementation of alternative solutions to support the adoption of low carbon technologies, such as EVs, heat pumps and the electrification of consumer products. To reduce the resulting increased peak demand, Mohsenian-Rad et al. developed an approach of direct interaction between grid and consumer to achieve valley-filling, by means of dynamic game theory \cite{Mohsenian-Rad2010}. In \cite{Karfopoulos2013}, a Multi-Agent System (MAS) was used to manage flexible loads for the minimisation of cost in a dynamic game. The use of aggregators has been proposed to allow the participation of a number of small providers to participate in network support, such as grid frequency response \cite{Wu2012a, Samadi2012, Xu2015b}. Yet~without the availability of power demand forecasts, real-time control needs to be implemented.

Real-time DSM can either be implemented in a centralised or distributed control approach. In~the~former, a central controller relays control signals to its aggregated DERs, whereas the latter allows each DER to control itself. A common form of controlling DERs in this mode of operation is set-point control \cite{Leadbetter2012685}. Using set-point control on multiple identically-configured DERs would yield optimal operation conditions if each DER's control parameters (e.g., bus voltage) were~shared. In a system without sharing network information, DER control algorithms have to be improved to prevent, for example, devices located furthest from the substation from being used more frequently than others.

This paper therefore presents an individualised BESS control algorithm that lets distributed batteries respond to fluctuations in real-time local bus voltage readings. The proposed algorithm is based on the robust Additive Increase Multiplicative Decrease (AIMD) type algorithm, yet implements a set-point adjustment based on the location of the controlled BESS. It will be shown how these home-connected batteries can mitigate the impact of additional loads (i.e., EV uptake), whilst assuring that all BESS are cycled equally.

The key contribution of this work can be summarised as a novel distributed battery storage algorithm for mitigating the negative impact of dynamic load uptake on the low-voltage network. This~algorithm uses an individualised set-point control to regulate bi-directional battery power flow~and, for convergence, extends the traditional AIMD algorithm. As a result, the developed battery control method reduces voltage deviation, over-currents and the inequality of battery usage. Reducing this usage inequality leads to a homogeneous usage of all of the distributed batteries and,~hence, prevents unequal degradation rates and unfair device utilisation.

The remainder of this paper is organised as follows: Section \ref{ch1:sec:related-work} gives some background to related work on AIMD algorithms on which this research is based. Section \ref{ch1:sec:system-modelling} outlines the EV, network and storage models used in the research. Additionally, it explains the assumptions that accommodate and justify these models. Section \ref{ch1:sec:storage-control} elaborates on the proposed AIMD control algorithm (AIMD+). Next, Section \ref{ch1:sec:scenarios-and-comparison-metrics} details the implementation and scenarios used for a set of test cases. For later comparison, this section also outlines a set of comparison metrics. Section \ref{ch1:sec:results-and-discussion} presents and discusses the results, followed by the conclusion in Section \ref{ch1:sec:conclusion}.
\section{Related Work}
\label{ch-review:sec:related-work}

\subsection{Storage Projects}

\subsection{Multi-Agent System Based Control}

\subsubsection{Centralised Architecture}

\subsubsection{Decentralised Architecture}


%% Discuss types of energy storage and how they are proposed to be used
%
%% Types: Hydro, compressed air, flywheel, battery etc
%% Application: Grid support (peak shaving), DG uptake (flexibility), economic (arbitrage & market)
%
%Large scale energy storage solutions like pumped storage hydro-electric have been in used for a long time to provide load levelling and energy reserve functions. In 1982, Wicks et.al. \cite{Wicks1982} proposed a linear programming approach to automatically control such an energy storage solution. This first step was 
%
%The idea of using batteries to support network operation is not new. In 1982, DelMonaco et.al.\cite{DelMonaco1982} highlighted the basic considerations when planning the operation and maintenance of grid connected batteries. They were amongst the first to publicly identify the role of energy storage beside distributed renewable generation, and the associated issues of bidirectional power flow and islanding of isolated branches. 
%
%Storage is great when combined with DER
%\cite{Papathanassiou2006}
%
\section{System Modelling}
\label{ch1:sec:system-modelling}

In this section, the underlying assumptions to validate the research are addressed. Next, a model to describe EV charging behaviour is explained. This is followed by a model of the BESS. Finally, the network models used to simulate the power distribution networks are explained.

\subsection{Assumptions}

For this work, several underlying assumption were made to obtain the models:
\begin{enumerate}
\item The uptake of EVs is assumed to increase and, hence, to have a significant impact on the normal operation of the low voltage distribution network. This assumption is based on a well-established prediction that the majority of EV charging will take place at home \cite{Munkhammar2015a}.
\item The transition from internal combustion engine-powered vehicles to EVs is assumed to not impact the users' driving behaviour. Similar to \cite{Dallinger2012}, this assumption allows the utilisation of recent vehicle mobility data \cite{MiD2008} to generate leaving, driving and arriving probabilities, from which the EV charging demand can be determined.
\item The transition to low carbon technologies will increase the variability of electricity demand, and therefore, grid-supporting devices, such as BESS, are anticipated to play a more important role \cite{FES2015}. Hence, alongside a high uptake of EVs, an increased adoption of distributed BESS devices is assumed.
\item It is assumed that BESS solutions, or more specifically battery energy storage solutions, start the simulations at 50\% SOC and are not 100\% efficient at storing and releasing electrical energy, as in \cite{Rowe2014a}. Additionally, its utilisation will degrade the energy storage capability and performance over time, as shown in \cite{Laresgoiti2015}. Therefore, the requirements for equal and fair storage usage is of high importance.
\item It is assumed that the load profiles provided by the IEEE Power and Energy Society (PES) are sufficient as base load profiles for all simulations.
\end{enumerate}

\subsection{Electric Vehicle Charging Behaviour}

From publicly-available car mobility data \cite{Dallinger2012, MiD2008} an empirical model was developed to capture the underlying driving behaviour. The raw data, $n_{r}(t)$, represents the probabilities of starting a trip during a 15-min period of a weekday. Three continuous normal distribution functions, each defined as:

\begin{equation}
\begin{split}
&\hat{n}_x(\beta_x,\mu_x,\sigma_x,t) := \beta_x\frac{1}{\sigma_x\sqrt{2\pi}} \exp\left[-\frac{\left(^t/_{24}-\mu_x\right)^2}{2\sigma_x^2}\right]\\
&\text{where } t \in [0, 24]  \text{ and } \beta_x \in \mathbb{R}  \text{ and } \mu_x \in \mathbb{R}  \text{ and } \sigma_x \in \mathbb{R}
\end{split}
\label{ch4:equ:normal-distribution}
\end{equation}


were used to represent vehicles leaving in the morning, $\hat{n}_{m}(t)$, lunch time, $\hat{n}_{l}(t)$, and in the evening, $\hat{n}_{e}(t)$. The aggregate probability of these three functions was optimised using a Generalised Reduced Gradient (GRG) algorithm to fit the original data. In order to represent a symmetric commuting behaviour, i.e., vehicles departing in the morning and returning during the evening, an equality amongst the three probabilities was defined as follows:

\begin{equation}
0 = \int_{0}^{24}\left[\hat{n}_m(t) + \hat{n}_l(t) - \hat{n}_e(t)\right]dt
\end{equation}


The resulting parameters from the GRG fitting of the three distribution functions are tabulated in Table \ref{ch1:tab:starting-a-trip-probability}. Additionally, the resulting departure probabilities, as well as the reference data $n_r(t)$ are shown in Figure \ref{ch1:fig:aggregated-ev-power}.

\singlespacing
\begin{table}[htb]\centering
\small
\begin{tabular}{cccc}
\hline
\textbf{Equation} \boldmath{$\hat{n}_x(t)$} & \boldmath{$\mu_x$} \textbf{(Mean)} & \boldmath{$\sigma_x$} \textbf{(SD)} & \boldmath{$\beta_x$} \textbf{(Weight)} \\
\hline
$\hat{n}_m(t)$ & 0.3049 & 0.0488 & 0.00206 \\
$\hat{n}_l(t)$ & 0.4666 & 0.0829 & 0.00314 \\
$\hat{n}_e(t)$ & 0.7042 & 0.0970 & 0.00521\\
\hline
\end{tabular}
\caption{Parameters for normal distributions.}
\label{ch1:tab:starting-a-trip-probability}
\end{table}
\doublespacing
\begin{figure}[htb]\centering
 	\includegraphics{_chapter1/fig/input/starting-a-trip-probability}
	\caption{The probability of starting a trip at a particular time during a weekday, extrapolated into three normal distributions (RMS error: $9.482\%$).}
	\label{ch1:fig:starting-a-trip-probability}
\end{figure}


Statistical data capturing the probability distribution of a trip being of a certain distance were also extracted from the dataset. This was done for both the weekdays $w_{wd}(d)$ and weekends $w_{we}(d)$. The Weibull function was chosen to be fitted against the extracted probability distributions and is defined as:

\begin{equation}
\hat{w}_x(d) :=
	\begin{cases}
		\frac{k_x}{\gamma_x}\left(\frac{d}{\gamma_x}\right)^{k_x-1}\exp\left[-\left(\frac{d}{\gamma_x}\right)^{k_x}\right] &\text{if } d \geq 0 \\
		0 &\text{if } d < 0
	\end{cases}
	\label{ch4:equ:weibull-distribution}
\end{equation}


Performing the curve fitting using the GRG optimisation algorithm, a weekday trip distance distribution, $\hat{w}_{wd}(d)$, and a weekend trip distribution, $\hat{w}_{we}(d)$, could be estimated. The computed function parameters for these two estimated distribution functions are tabulated in Table \ref{ch1:tab:trip-distance-probailility}. Their resulting probability distributions are plotted for comparison against the real data, $w_{wd}(d)$ and $w_{we}(d)$, in Figure \ref{ch1:fig:trip-distance-probability}.

\singlespacing
\begin{table}[htb]\centering
\small 
\begin{tabular}{ccc}
\hline
\textbf{Equation} \boldmath{$\hat{w}_x(d)$} & \boldmath{$\gamma_x$} \textbf{(Scale)} & \boldmath{$k_x$} \textbf{(Shape)} \\
\hline 
$\hat{w}_{wd}(t)$ & 15.462 & 0.6182 \\
$\hat{w}_{we}(t)$ & 38.406 & 0.4653\\
\hline
\end{tabular}
\caption{Parameters for Weibull distributions.}
\label{ch1:tab:trip-distance-probailility}
\end{table}
\doublespacing
\begin{figure}[htb]\centering
	\includegraphics{_chapter1/fig/input/distance-probability}
 \caption{The probability of a trip being of a particular distance during a weekday, extrapolated into a Weibull distribution (RMS error: $3.791\%$).}
 \label{ch1:fig:trip-distance-probability}
\end{figure}

 
In addition to these probabilities, an average driving speed of 56 kmh (35 mph) and an average driving energy efficiency of 0.1305 kWh/kmh (0.21 kWh/mph) are taken from \cite{UKGovernmentDigitalService2013}. Using the predicted driving distance and average driving speed with the driving energy efficiency, it is possible to estimate an EV's energy demand upon arrival. Starting to charge from this arrival time until the energy demand has been met allows the generation of an estimated charging profile of a single EV. To do this, a maximum charging power of the U.K.'s average household circuit rating (i.e., 7.4 kW) and an immediate disconnection of the EV upon charge completion were assumed \cite{EVHomeCharging}.

Generating several of those charging profiles and aggregating them produces an estimated charging demand for an entire fleet of EVs. To provide an example, charge demand profiles for 50 EVs were generated, aggregated and plotted in Figure \ref{ch1:fig:aggregated-ev-power}. This plot shows the expected magnitude and variability in energy demand that is required to charge several EVs at consumers' homes based on the vehicles' daily usage.

This model's EV charging behaviour has been implemented to reflect EV demand if applied today without widespread smart charging infrastructure. It does therefore reflect the worst case scenario. Future smart-charging schemes would mitigate the currently present collective EV charging spike, yet the implementation and validation of available smart-charging schemes lies beyond the scope of this paper. This model's data were used to feed additional demand into the power network models, which are outlined in the next section.

\begin{figure}[htb]\centering
 \includegraphics{_chapter1/fig/input/aggregated-ev-power}
 \caption{Excerpt from the aggregated 50 EVs; charging powers that were each generated from the empirical models.}
 \label{ch1:fig:aggregated-ev-power}
\end{figure}


\subsection{Battery Modelling}

For this work, a well-established model that has been used in previous publications by this research group was used \cite{Rowe2013, Rowe2014, Rowe2014a}. This model consists of a battery with a self-discharge loss that is dependent on the current battery's State Of Charge (SOC) and an energy conversion loss to represent the energy lost when charging or discharging this battery. 

\singlespacing
\begin{table}[htb]\centering
\small
\begin{tabular}{cl}
\hline
{\bf Parameter} & {\bf Description} \\
\hline
$P_{bat}(t)$ & Battery power at time $t$\\
$SOC(t)$ & Battery state of charge at time $t$\\
$\delta_{SOC}(t)$ & Change in SOC during time period $\tau$\\
$\mu$ & Self-discharge loss factor\\
$\eta$ & Energy conversion efficiency\\
$SOC_{min}$ & Minimum rated SOC for limited battery operation\\
$SOC_{max}$ & Maximum rated SOC for limited battery operation\\
$C$ & Battery capacity\\
$P_{max}$ & Power rating of battery\\
\hline
\end{tabular}
\caption{Table of the notation used in this section.}
\label{ch1:tab:notation-ev-model}
\end{table}
\doublespacing



When an ideal battery charges or discharges, the change in SOC is related by the battery power, $P_{bat}$. When sampling battery operation at a regular period, $\tau$, then the energy transferred into the battery can be described as $P_{bat}(t)\tau$. The change in SOC for this ideal battery, $\delta_{SOC}$, is therefore defined as:

\begin{equation}
\delta_{SOC}(t) := \frac{P_{bat}(t)\tau}{C} = \text{SOC}(t) - \text{SOC}(t-\tau)
\end{equation}


The self-discharge loss is added to this ideal battery model to represent the continual loss of energy in the battery typical of chemical energy storage. This self-discharge loss, $\delta_{SOC,self\text{-}discharge}$, is proportional to the current SOC and is determined using the self-discharge loss factor, $\mu$:

\begin{equation}
	\delta_{SOC,self\text{-}discharge}(t) := \mu SOC(t)
\end{equation}


Additionally, to represent the losses in the power electronics and energy conversion process, an energy conversion loss, $\delta_{SOC,conversion}$, is defined. This loss is proportional to the rate at which the battery's SOC changes, by using the energy conversion efficiency, $\hat{\eta}$ as follows:

\begin{equation}
	\delta_{SOC,conversion}(t) := \hat{\eta}\delta_{SOC}(t)
\end{equation}


Here, the conversion losses in the power electronics are reflected as an asymmetric efficiency, which depends on the direction of the flow of energy. This is done by charging the battery at a lower power when consuming energy and discharging it more quickly when releasing energy. Mathematically, this can be represented as:

\begin{equation}
\hat{\eta} =
	\begin{cases}
		\eta &\text{if } \delta_{SOC}(t) \geq 0 \\
		\frac{1}{\eta} &\text{if } \delta_{SOC}(t) < 0
	\end{cases}
\end{equation}


When substituting the self-discharge loss and conversion losses, respectively $\delta_{SOC,self\text{-}discharge}$ and $\delta_{SOC,conversion}$, into the SOC evolution equation, the full battery model can be summarised as follows:

\begin{equation}
\begin{split}
	\text{SOC}(t) :&= \delta_{SOC}(t-\tau) - \delta_{SOC,self\text{-}discharge}(t-\tau) - \delta_{SOC,conversion}(t)\\
	&= (1-\mu)\delta_{SOC}(t-\tau) - \hat{\eta}\delta_{SOC}(t)	
\end{split}
\end{equation}


In addition, both the SOC and the $P_{bat}$ are constrained due to the device's maximum and minimum energy storage capabilities, respectively $SOC_{max}$ and $SOC_{min}$, and maximum charge and discharge rate, $P_{max}$. These limitations are captured in Equations (\ref{equ-SoC-range}) and (\ref{equ-charge-discharge-range}), respectively.


\begin{equation}
\text{SOC}_{min} \leq \text{SOC}(t) \leq \text{SOC}_{max}
\label{equ-SoC-range}
\end{equation}
\begin{equation}
\left|P_{bat}(t)\right| \leq \text{P}_{max}
\label{equ-charge-discharge-range}
\end{equation}

\subsection{Network Models}

To simulate the low-voltage energy distribution networks, the Open Distribution System Simulator (OpenDSS) developed by the Electronic Power Research Institute (EPRI) was used. It requires element-based network models, including line, load and transformer information, and generates realistic power flow results.

\begin{figure}[htb]\centering
 \subfloat[]{%
	\includegraphics[width=0.48\textwidth]{_chapter1/fig/input/EULVFeeder.png}
 	\label{ch1:fig:feeders:EULVFeeder}%
 	}
 \subfloat[]{%
	\includegraphics[width=0.48\textwidth]{_chapter1/fig/input/Deepdale_FD01}
 \label{ch1:fig:feeders:Deepdale_FD01}%
 }
 \caption{Sample Open Distribution System Simulator (OpenDSS) power flow plots of the used power networks. Consumers are indicated as red crosses and 11/0.416-kV substations are marked with a green square. (\textbf{a}) IEEE Power and Energy Society (PES) EU Low Voltage Test Feeder plot; (\textbf{b}) Scottish and Southern Energy Power Distribution (SSE-PD) Common Information Model (CIM) (UK) feeder plot.}
 \label{ch1:fig:feeders}
\end{figure}


Simulations were conducted using the IEEE's European Low Voltage Test Feeder \cite{EULVFeeder2015} and six detailed U.K. feeder models, that are based on real power distribution networks and provided by Scottish and Southern Energy Power Distribution (SSE-PD). The SSE-PD circuit models were provided as Common Information Models (CIM) during the collaboration on the New Thames Valley Vision Project Project (NTVV) \cite{NTVV2016}. An example of the IEEE EU LV Test feeder and a U.K. feeder provided by SSE-PD are shown in Figure \ref{ch1:fig:feeders:EULVFeeder} and Figure \ref{ch1:fig:feeders:Deepdale_FD01}, respectively. A summary of these model's parameters is given in the Table \ref{ch1:tab:model-parameters}.

\singlespacing
\begin{table}[htb]\centering
\small
\begin{tabular}{lccccccc}%{r p{10cm} }
\hline
\multirow{2}{*}{\textbf{Parameter}} & \textbf{IEEE EU LV} & \multicolumn{6}{c}{\multirow{2}{*}{\textbf{SSE-PD LV Feeders}}}\\
 & \textbf{Test Feeder} & \\
\hline
Network number & 1 $^{\dagger}$ & 2 $^{\dagger}$ & 3 & 4 & 5 & 6 & 7\\
\hline
No. of loads & 55 & 56 & 53 & 91 & 59 & 88 & 37\\
\hline
Median load (VA) & 227 & 227 & 231 & 241 & 224 & 237 & 237\\
\hline
Max. load (kVA)& 16.8 & 16.8 & 16.8 & 19.5 & 16.8 & 19.5 & 16.8\\
\hline
Load connection & Single-phase & \multicolumn{6}{c}{Single-phase}\\
\hline
Median network load (kVA)& 24.4 & 24.9 & 23.9 & 41.9 & 25.6 & 38.9 & 16.3\\
\hline
Max. network load (kVA)& 72.6 & 72.7 & 72.2 & 92.9 & 73.5 & 89.6 & 60.5\\ 
\hline	
\multirow{2}{*}{Feeder line model} & Three-phase & \multicolumn{6}{c}{Three-phase}\\
 & implicit-neutral & \multicolumn{6}{c}{explicit-neutral}\\
 \hline
\end{tabular}
\caption{Network model parameters.}
\label{ch1:tab:model-parameters}
\begin{tabular}{ccc}
\multicolumn{1}{c}{\footnotesize $^{\dagger}$ These networks are shown in Figure \ref{ch1:fig:feeders}.}
\end{tabular}
\end{table}
\doublespacing


Throughout this paper, all excerpt and time series results were extracted from experiments with the IEEE EU LV Test feeder (i.e., Network No. 1). All concluding results are based on an aggregation of all networks to include network diversity in the analysis.

The model-derived EV data and IEEE EU LV Test feeder consumer demand profiles were used in all simulations. The resultant demand profiles represent the total daily electricity demand of households with EVs. These profiles were sampled at $\tau = 1\text{ min}$. The OpenDSS simulation environment was controlled using MATLAB, achieved through OpenDSS's Common Object Model (COM) interface and accessible using Microsoft's ActiveX server bridge.
\section{Storage Control}
\label{ch1:sec:storage-control}

In this section, the control of the energy storage system is explained. Firstly, the additive increase multiplicative decrease algorithm is presented, and its decision mechanism is explained in full. Then,~the voltage referencing, used for AIMD+, is outlined.

\subsection{Additive Increase Multiplicative Decrease}

The proposed distributed battery storage control is shown in Algorithm \ref{alg-aimd}. The parameter $\alpha$~denotes the size of the power's additive increase step, and $\beta$ denotes the size of the multiplicative decrease step. It is worth mentioning that $\alpha$ linearly increases and $\beta$ exponentially decreases, both~charging and discharging powers, where discharging power is represented as a negative power~flow, i.e., energy released by the battery. The constants $V_{max}$ and $V_{thr}$ are the maximum historic voltage value and the set-point threshold used to regulate the total demand. In the case when the total demand is too high, the local voltages will fall below $V_{thr}$, and the batteries reduce their charging power and start discharging. This behaviour reduces total demand on the feeder. At~simulation~start, $V_{max}$ is set to the nominal voltage of the substation transformer, i.e., 240 V, and~$V_{thr}$ is set to a fraction of $V_{max}$, which was found by solving a balanced power flow analysis. The variable $V(t)$ is the battery's local bus voltage, and $P_{max}$ denotes the maximum charging/discharging power of the battery. The~charging and discharging power of the batteries is increased in proportion to the available headroom on the~network, which is inferred from the local voltage measurement $V(t)$, to~avoid any sudden overloading of the substation transformer.

\singlespacing
\begin{algorithm}[H]
 \caption{Compute battery power.}
 \label{ch1:alg:aimd}
 \begin{algorithmic}[1]
 \State $R(t) = (V(t) - V_{thr})/(V_{max} - V_{thr})$ \Comment{Defines the rate for the current voltage reading}\vspace{5pt}
\If {$V(t) \geq V_{thr}$} \Comment{Given the voltage levels are nominal...}\vspace{5pt}
 \If {$SOC < SOC_{max}$} \Comment{...and the battery is not fully charged...}\vspace{5pt}
 \State $P(t) = P(t-\tau) + \alpha P_{max} R(t)$ \Comment{...increase the charging power}\vspace{5pt}
 \Else \Comment{If the battery has fully charged...}\vspace{5pt}
 \State $P(t) = 0$ \Comment{...shut off}\vspace{5pt}
 \EndIf\vspace{5pt}
 \If {$P(t) < 0$} \Comment{If the battery has been discharging...}\vspace{5pt}
 \State $P(t) = \beta P(t-\tau)$ \Comment{...reduce the discharging power by $\beta$}\vspace{5pt}
 \EndIf \vspace{5pt}
 \Else \Comment{If voltage levels are not nominal...}\vspace{5pt}
 \If {$SOC > SOC_{min}$} \Comment{...and battery is charged sufficiently...}\vspace{5pt}
 \State $P(t) = P(t-\tau) + \alpha P_{max} R(t)$ \Comment{...increase discharge power}\vspace{5pt}
 \Else \Comment{If the battery is not sufficiently charged...}\vspace{5pt}
 \State $P(t) = 0$ \Comment{...shut off}\vspace{5pt}
 \EndIf\vspace{5pt}
 \If {$P(t) > 0$} \Comment{If the battery has been charging...}\vspace{5pt}
 \State $P(t) = \beta P(t-\tau)$ \Comment{...reduce the charging power by $\beta$}\vspace{5pt}
 \EndIf\vspace{5pt}
 \EndIf\vspace{5pt}
 \State $P(t) = \textbf{signum}(P(t)) \times \textbf{min}\{|P(t)|,P_{max}\}$ \Comment{Limit the power to battery specifications}
 \end{algorithmic}
\end{algorithm}
\doublespacing


The algorithm itself, as shown in Algorithm \ref{ch1:alg:aimd}, contains two decision levels. The first determines whether the network is over- or under-loaded by comparing the local bus voltage, $V(t)$, to the battery's set-point threshold, $V_{thr}$. In the event that the network is not under high load, the battery's SOC is compared to its operation limit to check whether the battery can charge, i.e., $SOC$ \textless~$SOC_{max}$. If there is enough charging capacity left, then the battery's charging power is linearly increased following Line~4. If the battery was previously discharging, the related discharging power is exponentially reduced (Line~9) to reflect the multiplicative decrease.

The second decision level is entered when the network is under load. Here, the discharging power is linearly increased if the battery has enough energy stored, i.e., $SOC$ \textgreater~$SOC_{min}$ (Line 13). Additionally, if the battery was previously charging, then its charging power is multiplicatively reduced (Line~18). The direction of the charging/discharging power adjustment is determined by the first decision level, as well as the threshold proximity ratio $R(t)$. As the battery's bus voltage, $V(t)$, approaches the threshold~voltage, $V_{thr}$, this ratio tends to zero and, hence, stops the battery operation. Therefore, oscillatory hunting is effectively mitigated. The last step of the algorithm (Line 21) assures that the battery charge/discharge power is within its device rating.

\subsection{Reference Voltage Profile}

When using a fixed voltage threshold, the difference in the location and load of each customer results in the over-utilisation of batteries located at the feeder end. Similar to Papaioannou et al. \cite{Papaioannou2015}, yet for the control of BESS instead of EV charging, a reference voltage profile is proposed, which is produced by performing a power flow analysis of the network under maximum demand. An example of a fixed threshold and reference voltage profile is shown in Figure \ref{ch1:fig:ref-voltage-plot}.

\begin{figure}[htb]\centering
 \includegraphics{_chapter1/fig/input/ref-voltage-plot}
 \caption{A plot showing the difference between the fixed voltage threshold (AIMD) and the reference voltage profile (AIMD+).}
 \label{ch1:fig:ref-voltage-plot}
\end{figure}


In the AIMD+, consumers located at the head of the feeder are allocated a higher voltage threshold, while those towards the end of the feeder have similar voltage thresholds to that of the fixed threshold. This replicates the expected voltage drop along the length of the feeder, hence resulting in a more equal utilisation of battery storage units that are located at those distances. The voltage threshold is set in such a way as to limit the maximum voltage drop to 3\% at the end of the feeder.

\section{Scenarios and Comparison Metrics}
\label{ch1:sec:scenarios-and-comparison-metrics}

In this section, several scenarios are explained that were used to test the performance of the battery control algorithm. Following that is the definition of three comparison metrics. These metrics quantify the improvements caused by the different algorithms in comparison to the worst case scenario.

\subsection{Test Cases and Scenarios}
\label{ch1:subsec:test-cases-and-scenarios}

In all simulations, the EVs plug-in on arrival and charge at their nominal charging rate until fully~charged. The BESS devices were chosen to have a capacity of 7 kWh with a maximum power rating of 2 kW (battery specifications are based on the Tesla Powerwall \cite{Powerwall2015}).

Four excerpt cases were defined with different levels of EV and storage uptakes, these are as follows:
\begin{enumerate}[label=\textbf{\Alph*}, leftmargin=2.2em, labelsep=5.5mm]
\item A baseline scenario, where only household demand is used.
\item A worst case scenario, in which EV uptake is 100\% and no BESS is used.
\item An AIMD scenario, in which EV uptake is 100\% and each household has a battery energy storage~device. Here, each battery was controlled using the AIMD algorithm using a fixed voltage~threshold.
\item An AIMD+ scenario, in which EV uptake is 100\%, and each household has a battery energy storage device. Here, each battery was controlled using the AIMD+ algorithm using the optimised reference voltage profile.
\end{enumerate}

A storage uptake of 100\% was adopted to represent the worst case scenario. In addition to the four defined scenarios, a full set of simulations was performed with EV and storage uptake combinations of 0\% to 100\% in steps of 10\%.

\subsection{Performance Metric Definition}
\label{ch1:subsec:performance-metric-definition}

To obtain comparable performance metrics, three parameters are defined. These parameters capture the improvements in voltage violation mitigation, line overload reduction and the equality of battery usage. All excerpt performance metrics were calculated based on simulations from the IEEE EU LV Test feeder for reproducibility.

\subsubsection{Parameter for Voltage Improvement}

The first parameters are $\zeta^{*}_\textbf{C}$ and $\zeta^{*}_\textbf{D}$ for, respectively, Cases {C} and {D}, and calculate the magnitude of the voltage level improvement by comparing two voltage frequency distributions. More specifically, they find the difference between these probability distributions and compute a weighted sum. Here, the weighting, $\delta^{*}(v)$, emphasises the voltage level improvements that deviate further from the nominal substation voltage $V_{ss}$. If the resulting weighted sum is negative, then the obtained voltage frequency distribution was improved in comparison to the associated worst case scenario. In contrast, a positive number would indicate a worse outcome. The performance metric $\zeta^{*}_\textbf{C}$ is defined as follows.

\begin{equation}
 \zeta^*_\textbf{C} := \sum_{v = V_{min}}^{V_{max}} \delta^{*}(v) \left[P_\textbf{B}(v) - P_\textbf{C}(v)\right]
 \label{ch1:equ:voltage-metric}
\end{equation}


Here, $V_{min}$ is the lowest recorded voltage, and $V_{max}$ is the highest recorded voltage. $P_\textbf{B}(v)$ is the voltage probability distribution of the worst case scenario (Case {B}), and $P_\textbf{C}(v)$ is the voltage probability distributions of Case {C} (i.e., the case with maximum EV and AIMD storage uptake). Similarly, the~parameter $\zeta_\textbf{D}^*$ therefore compares Case {D}, i.e., the AIMD+ case, with Case {B}.

The aforementioned factor, $\delta^{*}(v)$, scales down the summation in Equation (\ref{ch1:equ:voltage-metric}) for voltages within the nominal operating band, where no voltage violations take place. Voltage violations on the other hand are scaled up to increase their impact on the summation. This scaling was produced using a linear function, with its minimum at $V_{ss}$, that is defined as:

\begin{equation}
 \delta^{*}(v) := 
 \begin{cases} 
 \frac{V_{ss} - v}{V_{ss} - V_{low}} & \text{if } v \leq V_{ss} \\
 \frac{v - V_{ss}}{V_{high} - V_{ss}} & \text{otherwise}
 \end{cases}
 \label{ch1:equ:voltage-scaling-factor}
\end{equation}


$V_{low}$ and $V_{high}$ are defined as the lower and upper limits of the nominal operation voltage band, respectively. In general, the proposed voltage comparison parameter, $\zeta^*$, shows an improvement in voltage distribution when it is negative, whereas a positive value implies a voltage distribution with more voltage violations.

\subsubsection{Parameter for Line Overload Reduction}

Similar to measuring the voltage level improvements, all line utilisation probability distributions between the storage and worst case scenarios were compared. This follows a similar equation to before, but uses a different scaling factor, as described in Equation (\ref{ch1:equ:voltage-scaling-factor}):

\begin{equation}
 \zeta_\textbf{C}^{**} := \sum_{c = 0}^{C_{max}} \delta^{**}(c) \left[P_\textbf{C}(c) - P_\textbf{B}(c)\right]
 \label{ch1:equ:utilisation-metric}
\end{equation}


Here, $C_{max}$ is the highest line utilisation. $P_\textbf{B}(c)$ and $P_\textbf{C}(c)$ present the line utilisation probability distributions for Cases {B} and {C}, respectively, and $\delta^{**}(c)$ is the associated scaling factor. Since the relationship between line current and ohmic losses is quadratic, this scaling factor is defined as an exponential function that amplifies the impact of line currents beyond the line's nominal rating.

\begin{equation}
 \delta^{**}(c) = 
 \begin{cases} 
 \left(\frac{c}{1-C_{min}}\right)^2 & \text{if } c \geq C_{min} \\
 0 & \text{otherwise}
 \end{cases}
 \label{ch1:equ:utilisation-scaling-factor}
\end{equation}


The capacity scale modifier, $C_{min}$, defines from where the scaling should start and has been set to $0.5$ for this work as only line utilisation above $0.5$ p.u. was considered. Therefore, a reduction in line overloads would give a negative $\zeta^{**}$, whereas a positive value implies a higher line utilisation, i.e.,~worse results.

\subsubsection{Parameter for the Improvement of Battery Cycling}
\label{ch1:subsubsec:parameter-for-the-improvement-of-battery-cycling}

The final metric, $\zeta^{***}$, gives an indication of the inequality of battery cycling (one battery cycle is defined as a full discharge and charge of the battery at maximum operating power, i.e., $P_{max}$) across all battery units. It does this by computing the the ratio between the peak and mean battery cycling. This Peak-to-Average Ratio (PAR) of batteries' cycling is defined in the following equation.

\begin{equation}
 \zeta_\textbf{C}^{***} := \frac{\max \left|C_\textbf{C}\right|}{B^{-1} \sum_{b=1}^B{\left|c_\textbf{C}^b\right|}}
 \label{ch1:equ:peak-to-average}
\end{equation}

Here, $B$ is the number of batteries, and $c_\textbf{C}^b$ is the total cycling of battery $b$ during Scenario {C}. $C_\textbf{C}$ is a vector of $\mathbb{R}_{\geq 0}^{B}$ that contains all batteries' cycling values, i.e., $c_\textbf{C}^b \in C_\textbf{C}$. Equally, the battery cycling for Scenario {D} would be captured by $\zeta_\textbf{D}^{***}$. In the unlikely event of an equal cycling of all batteries, $\zeta^{***}$~will have a value of one. Yet, as batteries are operated differently, the value of $\zeta^{***}$ is likely to be greater than one. Therefore, a resulting PAR closer to one implies a more equal and therefore fairer utilisation of the deployed batteries.
\input{_chapter1/sec/results-and-discussion.tex}
\section{Conclusion}
\label{ch-conclusions:sec:conclusion}

As identified in Chapter~\ref{ch-introduction} of this thesis the aim of the presented work was to:

\textit{[...] make a contribution in control of Battery Energy Storage Systems (BESS) that can aid Distribution Network Operators (DNOs) in improving the operation and reliance of their Low-Voltage (LV) networks.}

To achieve this aim, four objectives were identified from the literature review in Chapter~\ref{ch-literature} and met in the four contribution chapters of this thesis (i.e. Chapter~\ref{ch1} to Chapter~\ref{ch4}).
Within these chapters, key network parameters were identified to assess the impact of BESS control methods on the underlying power delivery network when adjusting BESS phase powers whilst conforming to a half-hourly schedule.
When alleviating this scheduling constraint, it was shown how a developed dynamic BESS control can achieve greater reduction of power peaks.
Then coordinated control of multiple BESS was assessed regarding the desynchronisation of control instructions.
Since this analysis showed the sensitivity of control methods on their implementation (even when operating in a desynchronised environment) a truly communication less control algorithm was developed next.
Through developing these algorithms and testing them by simulating several LV distribution network models, aspects involving network operation, system deployability, information propagation and telecommunication restrictions were studied.
Therefore, a technical contribution that can aid DNOs in improving the operation and reliance of LV networks has been achieved and thus the overarching aim was met.

\hl{The limitations of the presented research do, however, ask for future work to be conducted in order to ready the algorithms and methods for implementation in industry.
Beside the restrictions outlined in Section~}\ref{ch-conclusions:sec:research-limits}\hl{ due to time constraints and the targeted research focus, future work will comprise further improvement of network and BESS models, mathematical formulation of proof of stability and convergence.
Furthermore, DNOs should address the issue of ownership and possible means of incentivising customers to provide network support with their home-connected assets.
Also, safety, security and ethics associated with emerging control methods and data handling should also be considered, not only by academics, but also by industry, their customers and legislation makers.}

Overall, in the context of power systems, the conducted research focused on providing improvements at the fringes of the electricity network; that is at the LV distribution level.
With the international aim of transitioning towards a low-carbon economy, national issues are mostly expected to occur in the transmission and interconnection of electricity grids.
This issue is particularly apparent when planning the generation of bulk power at remote locations (for example offshore) without having installed the required power lines.
Equally, with higher reliance on renewable energy resources, national power systems must become more flexible to not only cater for the volatility in demand but also for the expected intermittency in supply.
\hl{Whilst energy storage on a national scale would address the difficulty of matching modern supply and demand, the complexity of a project, the size and the associated cost make deployment of such a system a significant logistic challenge.
By focusing on smaller scaled projects first (i.e. BESS in the LV networks) the first step is provided towards this larger goal.
After all, it is easier to develop, test and eventually deploy control algorithms on such a smaller scale than it is for larger scaled projects.
Also, at the end of the day, it is the DNO that needs to provide the final physical link between demand and supply and their power delivery networks are expected to cater for the immediate increase in volatile demand.
Therefore, targeting research at this fringe of the electricity network provides a small contribution to the overall operation of the electricity network, but this research still provides support for DNOs when catering for the future load scenarios.}


\chapter{Real-Time Adjustment of Battery Operation using MPC Guided Schedule Deviation}
\label{ch2}

\section{Overview}
\label{ch2:sec:overview}

%Outages are still very frequent in the UK. 
%According to the UK energy regulator \textit{OFGEM}, on average 45\% of all customers experienced service disruptions in the period 2015-16 \cite{Ofgem2017}.
%Whilst unanticipated outages due to severe winter weather lead to \pounds39 million worth of damages \cite{Ofgem2014}, network upgrades and repairs however contributed the larger amount of customer interruptions and customer minutes lost.
%Such planned outages are intentions to strengthen networks and mitigate system overloads, due to increasing demand for electricity.
%This demand increase is only accelerated since major focus of UK energy policies has been put on transitioning towards a low carbon economy \cite{HMGovernment2009, RoyalAcademyofEngineering2010}.
%Particularly the decarbonisation of heat and transport sectors are two areas of significant strategic focus and Low Carbon Technology (LCT) such as photovoltaic installations, electric vehicles and heat pumps are expected to contribute significantly to this transition.
%
%However, as adaptation of these LCTs increases and they start to penetrate power distribution networks, stress on these networks will continue to increase even further, which may result in additional service disruptions.
%Furthermore, the uptake of LCTs is not expected to progress evenly throughout the entire power network, and instead clusters of early adopters are predicted to form, leading to certain Low-Voltage (LV) networks to exceed their operational constraints even at relatively low national rate of LCT adaption \cite{Poghosyan2014}.
%Traditional network planning approaches to circumvent constraint violations, follow the commonly used practice of aggregating a large number of customers and designing the power delivery network to cater for their largest probable demand, i.e. the After Diversity Maximum Demand (ADMD) method \cite{Richardson2010a}.
%This ADMD method has remained the same for many years and uses historical load analysis and standard growth assumptions that are both no longer valid in this unprecedented LCT uptake scenario \cite{Yunusov2016}.
%To make things worse, LV networks in the UK are generally unmonitored once installed.
%Distribution Network Operators (DNOs) have become aware of this issue and are developing updated planning strategies involving ``smart'' and ``flexible'' electricity grids.
%However, in situ equipment that will become subject to the same adaptation of LCT needs to be managed actively via innovation in the use of existing and new technologies; otherwise both frequency of service disruptions and customer minutes lost will increase alongside the proliferation of LCTs \cite{Ault2008a}.
%
%Two solutions exist, allowing DNOs to support LV network's operation: 
%\begin{enumerate*}
%	\item reinforcement of in situ network assets;
%	\item deployment of network support equipment.
%\end{enumerate*}
%Whilst network reinforcement would certainly address immediate issues of current network capacity constraints, it is also the more expensive and disruptive option.
%More specifically, customer will need to deal with outages during periods of asset upgrades (e.g. transformer upgrade and line re-conductoring after secondary transformers' tap settings have been adjusted).
%Therefore, alternatives to defer or avoid network reinforcements have been sought and assessed \cite{Harrison2007, Zangs2016a, VanderKlauw2016d, Greenwood2017}.
%Most promising alternatives are to install flexible and controllable Distributed Energy Resources (DERs), or more specifically: Battery Energy Storage Solutions (ESMU) \cite{Wade2010}.
%ESMU has not only seen significant advancements in technology, but also received increasing attention in both academic studies and industry trials \cite{Palizban2016}.
%
%Installing ESMU on a strategic location in the LV network brings several advantages to DNOs' control over the network's performance.
%Regulating voltages to stay within statutory operating bands \cite{Yang2014}, shaving peak load to relieve stress from the installed network assets \cite{Bennett2015}, or reducing phase unbalance to increase network efficiency \cite{Wang2015b} are only a few examples of recent research in this field.
%Whilst the questions regarding locating and scaling of ESMU have mostly been addressed, ESMU control can be split into two complementing yet unmarried approaches:
%\begin{enumerate*}
%	\item ``off-line'' control, using load forecasts and ESMU schedules; and
%	\item ``on-line'' control, using Set-Points Control (SPC), Model Predictive Control (MPC) or similar dynamic control methods.
%\end{enumerate*}
%
%Off-line control uses historic data to predict future load patterns, which are used to schedule ESMU operation accordingly.
%Early approaches, e.g. by Oudalov et al. \cite{Oudalov2007}, who used dynamic programming to generate ESMU schedules, had a relatively high forecast error due to the inherent difficulty of predicting future loads, which ultimately limits the ability of given ESMU schedule to i.e. reduce peaks.
%This reason is why recent research either includes uncertainty, like the work by Baker et al. \cite{Baker2017} where uncertainty of wind power was taken into account when scheduling and sizing ESMU, or it frequently re-evaluates ESMU schedules, as done by Wang et.al \cite{Wang2014a}, where ESMU control is adjusted after each decision epoch.
%Despite load forecasts being imperfect, forecasts remain a key component for scheduling ESMU thanks to work like that by Rowe et al. \cite{Rowe2014a}, where a filtering mechanism was proposed for scheduling algorithms to reduce peak load in LV networks in spite of forecast errors.
%Furthermore, most day-ahead forecast only forecast at a temporal resolution down to half-hourly periods.
%As pointed out by Haben et al. \cite{Poghosyan2014, Haben2014}, forecasts at half-hourly resolution yield the best compromise between high accuracy and high temporal resolution, which is why they have become the standard for generating ESMU operating schedules.
%Nonetheless, sub-half-hourly load volatility imposes the biggest stress on the network and cannot be addressed when using this kind of half-hourly forecast, which is why on-line control has been considered as an alternative to off-line control.
%
%One flavour of on-line control is the Set-Point Control (SPC), which is a robust technique that can immediately respond to network changes.
%Since this kind of control runs the risk of reaching shortage or surplus of ESMU stored energy, modifications like hysteresis control \cite{Gybel2012} and ramp-rate control \cite{Such2012} were proposed.
%However, this kind of on-line control is less effective in addressing daily demand peaks, since pure SPC can only react to current network demand and does not respond to general trends or upcoming load events.
%To address these shortcomings SPC has been extended, using short-term load predictions by implementing Model Predictive Control (MPC).
%Some MPC examples include Auto-Regressive (AR) models \cite{Li2009, Nie2011}, fuzzy logic models \cite{Sannomiya2001, Chen2013a}, genetic algorithms \cite{Xia2015a, Liu2015} or Artificial Neural Networks (ANN) \cite{Kalogirou2014, Quan2014, Lee2014, Pezeshki2014, Vaz2016, Reihani2016, Xiao2017}.
%Implementing increasingly complex MPC to support on-line control is therefore a strong research trend, however the computational burden to deliver real-time solutions makes implementation of such systems not yet feasible.

\nomenclature[G]{SPC}{Set-Point Control}
\nomenclature[G]{MPC}{Mode-Predict Control}
\nomenclature[G]{PID}{Proportional Integrating Derivative (control)}


In the preceding chapter an Energy Storage Management Unit (ESMU) is used to improve network operation.
This improvement is achieved by optimally adjusting the device's scheduled three-phase powers.
Any improvement is indicated by a cost reduction, where the underlying cost functions are tied to changes in key network parameters.
The extend to which ESMU is able to improve network operation is then shown by focusing on the minimisation of different cost functions and repetitively optimising and simulating the distribution network.
However, this network improvement is limited by the constraint of having to obey the underlying half-hourly ESMU schedule, despite applying adjustments at a sub-half-hourly level.

In the following chapter, this limiting constraint is lifted, and the corresponding sub-half-hourly ESMU schedule adjustment method is proposed.
This method unifies the benefits from sub-half-hourly demand measurements and half-hourly demand forecasts.
Unlike previous work in the field, the proposed approach reverses the traditional control paradigm to compensate for schedule inaccuracies.
To reiterate, these traditional approaches implemented on-line control mechanisms, e.g. Set-Point Control (SPC), in combination with prediction models in order to adjust and prepare ESMU for future load trends.
In this presented work however, instead of supporting on-line control with real-time load predictions, forecast driven schedules are adjusted using on-line measurements.
This is achieved by first scheduling ESMU operation at half-hourly resolution, i.e. by following a ``peak-shaving'' and ``valley-filling'' behaviour which has been explained in Chapter \ref{ch1}, and then modifying this schedule using MPC.
In this case, MPC is comprised of of a lightweight AR model to assure real-time deployability.
These two control signals are unified using two Proportional Integral Derivative (PID) compensators that are tuned to assure system robustness, regardless of the forecast's erroneousness.
All ESMU schedules are generated under the constraints of a realistic ESMU model, and all demand measurements and corresponding forecasts used in this work are based on real data, provided by the project partner and DNO: \textit{Scottish and Southern Energy Networks} (SSEN).
Results are generated from this realistic (i.e. provided) network load with corresponding load forecasts, and cases are compared against the original and a baseline load case (i.e. traditional off-line control).
It is shown that, even under these imperfect forecast conditions, the proposed schedule adjustment method can successfully reduce sub-half-hourly peaks.
In fact, whilst the probability distribution of the baseline case sat around an average of 1.78kW peak reduction, the proposed method increased the reduction to 5.24kW.
Since this proposed control method is the natural extension of our previous work in \cite{Zangs2016}, it is hereon referred to as ``dynamic control''.

The chapter is organised as follows:
In Section \ref{ch2:sec:system-explanation}, all constituent system components including ESMU model, forecast acquisition and ESMU schedule generation are explained.
%Cost functions to generate optimised half-hourly ESMU schedules are formulated, containing well established parameters like the Peak-to-Average Ratio (PAR) or ``Min-Max'' difference, which have been widely used in DER scheduling and operation \cite{Liu2014, Deng2015, Bayram2015, Zangs2016a}.
Section \ref{ch2:sec:control-of-esmu} presents the dynamic control, including the dual PID setup and MPC.
Section \ref{ch2:sec:case-studies} outlines the different case studies that were used to compare the performance of the dynamic control.
In Section \ref{ch2:sec:results}, all results from these case studies are presented and discussed.
Finally, conclusion and the future work are described in Section \ref{ch2:sec:summary}.












\section{System Explanation}
\label{ch2:sec:system-explanation}

\begin{figure*}[htb]\centering
% Define some block styles
\tikzstyle{box} = [%
	draw,%
	rectangle,%
	%fill=green!20,%
	minimum height=3em,%
	minimum width=5em,%
]
\subfloat[]{%
	\begin{tikzpicture}[node distance=1.5cm, shorten >= 1pt, >=stealth', auto, scale=0.8, transform shape]
		% Define nodes
	    \path (0,0)
	    node [box, fill=blue!20](data) {Data}
	    node [box, fill=blue!20, below of=data](forecast) {Forecast}
	    node [box, fill=blue!20, below of=forecast](schedule) {Schedule}
	    node [box, below of=schedule](battery) {Battery}
	    node [box, below of=battery](network) {Network};
		% Draw lines
		\draw [->] (data) to (forecast);
		\draw [->] (forecast) to (schedule);
		\draw [->] (schedule) to (battery);
		\draw [->] (battery) to (network);
	\end{tikzpicture}%
	\label{ch2:subfig:traditional-forecast-based-system}%
}
\hspace{10mm}
\subfloat[]{
	\begin{tikzpicture}[node distance=1.5cm, shorten >= 1pt, >=stealth', auto, scale=0.8, transform shape]
		% Define nodes
	    \path
	    (0,0) 
	   	node [box, fill=green!20, below of=schedule, xshift=-14.5mm](controller) {Controller}
		node [box, fill=green!20, below of=controller](mpc) {MCP}
	    node [box, right of=controller, xshift=15mm](battery) {Battery}
	    node [box, below of=battery](network) {Network};;
		% Draw lines
		\draw [->] (network) to (mpc);
		\draw [->] (mpc) to (controller);
		\draw [->] (controller) to (battery);
		\draw [->] (battery) to (network);
	\end{tikzpicture}%
	\label{ch2:subfig:traditional-on-line-system}%
}
\hspace{10mm}
\subfloat[]{%
	\begin{tikzpicture}[node distance=1.5cm, shorten >= 1pt, >=stealth', auto, scale=0.8, transform shape]
		% Define nodes
	    \path
	    (0,0)
	    node [box, fill=blue!20](data) {Data}
	    node [box, fill=blue!20, below of=data](forecast) {Forecast}
	    node [box, fill=blue!20, below of=forecast](schedule) {Schedule}
	    node [box, below of=schedule](battery) {Battery}
	    node [box, below of=battery](network) {Network}
	   	node [box, fill=green!20, left of=battery, xshift=-14.5mm](controller) {Controller}
		node [box, fill=green!20, below of=controller](mpc) {MCP};
		% Draw lines
		\draw [->] (data) to (forecast);
		\draw [->] (forecast) to (schedule);
		\draw [->, bend right] (schedule.180) to (controller.90);
		\draw [->] (network) to (mpc);
		\draw [->] (mpc) to (controller);
		\draw [->] (controller) to (battery);
		\draw [->] (battery) to (network);
	\end{tikzpicture}%
	\label{ch2:subfig:proposed-dynamic-control-system}%
}

\caption{Combination of traditional forecast driven BESS control (Subfig. \ref{ch2:subfig:traditional-forecast-based-system}) and traditional on-line system (Subfig. \ref{ch2:subfig:traditional-on-line-system}) results in the proposed dynamic control system (Subfig. \ref{ch2:subfig:proposed-dynamic-control-system}).}
\label{ch2:fig:system-diagram}
\end{figure*}


The presented work is part of the \textit{New Thames Valley Vision} (NTVV) research project and was conducted in collaboration with the British DNO \textit{Scottish and Southern Energy Networks} (SSEN) \cite{NTVV2016}.
From the findings of this research project the diagram in Figure~\ref{ch2:fig:system-diagram} was generated, showing two well established ESMU control approaches and the proposed dynamic control approach.
This figure includes all constituent systems that were used during the ESMU street-level deployment.
The two traditional systems are off-line and on-line ESMU control which are shown in Figure~\ref{ch2:subfig:traditional-forecast-based-system} and \ref{ch2:subfig:traditional-on-line-system}, respectively.
Alongside these two control approaches is the proposed dynamic control system that is shown in Figure~\ref{ch2:subfig:proposed-dynamic-control-system}.
This control approach entails the benefits from both the traditional half-hourly forecast driven and the sub-half-hourly ESMU control system and can therefore be seen as the hybrid of the two traditional systems.
Unlike previous work this hybrid system does not rely on Set-Point Control (SPC) which is adjusted by a MPC to compensate for trends in the load profile.
Instead, it operates by executing a predetermined half-hourly ESMU schedule which is adjusted at sub-half-hourly intervals.
Therefore the necessity of relying on a stable SPC is removed and replaced by a robust schedule execution.
Nonetheless, flexibility is provided by allowing the aforementioned schedule adjustments.
The preceding work by Rowe et al. in \cite{Rowe2014} inspired this hybrid system and used a similar approach that emphasises these benefits of using a hybrid system.
Unlike the work by Rowe et al. in \cite{Rowe2014} however, the proposed hybrid system operates at a higher temporal resolution (it uses a light weight deterministic adjustment method in the form of MPC), and it does not rely on a long forecasting horizon since it recomputes the power adjustments for every single time-step.
As already mentioned, those adjustments are based on MPC-guided instructions and details about this dynamic control are outlined in Section~\ref{ch2:sec:control-of-esmu}.

In this section however the battery model which is used in this work is reminded first.
Also the load data acquisition, forecasting and ESMU schedule generation are outlined where scheduling is performed in accordance to the ESMU model's constraints.

\subsection{ESMU model}
\label{ch2:subsec:esmu-model}

\nomenclature[J]{$C_{bat}$}{Battery capacity in kWh, where $C_{bat} \in \mathbb{Z}^{>0}$}
\nomenclature[J]{$C_{f}$}{Charge factor or ``C-factor'' of the battery, where $C_{f} \in \mathbb{Z}^{>0}$}
\nomenclature[J]{$\mu$}{Self-discharge losses of battery, where $\mu \in (0, 1]$}
\nomenclature[J]{$P_{bat}$}{ESMU power electronic rating, where $P_{bat} \in \mathbb{Z}^{>0}$}
\nomenclature[J]{$\eta$}{Round-trip efficiency of power electronics, where $\eta \in (0, 1]$}
\nomenclature[J]{$t$}{Discrete sample of time, where $t \in \{0, \Delta t, \dots, T\Delta t\}$}
\nomenclature[J]{$T$}{Number of samples during the entire simulation, where $T \in \mathbb{Z}_{>0}$}
\nomenclature[J]{$\Delta t$}{Sub-half-hourly sample period, where $\Delta t \in \mathbb{Z}^{>0}$}
\nomenclature[J]{$SOC(t)$}{Scheduled state of charge at sample $t$}
\nomenclature[J]{$p(t)$}{ESMU power at time $t$, where $\textbf{p} = (p(t))$ and $p(t) \in \mathbb{Z}$}
\nomenclature[J]{$\textbf{p}$}{ESMU power vector, where $\textbf{p} = (p(t))$}
\nomenclature[J]{$p_{bat}(t)$}{Battery power at time $t$, which is derived form $p(t)$, where $\textbf{p}_{bat} = (p_{bat}(t))$ and $p(t) \in \mathbb{Z}$}
\nomenclature[J]{$\textbf{p}_{bat}$}{Battery power vector, where $\textbf{p}_{bat} = (p_{bat}(t))$}

The ESMU model is based on the physical system that was deployed by SSEN during the NTVV project.
Since this model is the same model as the one used in the preceding chapter which has been explained in detail in Section~\ref{ch1:subsec:battery-model}, only the model's final equation (as well as all used parameters) are detailed, hence foregoing the re-deriving of the same battery storage model.
This ESMU model equation is as follows:

\begin{equation}
\begin{split}
	&SOC(t+\Delta t) =\\
	&\begin{cases}
		\eta \left(SOC(t) + \frac{\mu \Delta t p(t)}{C_{bat}(3.6\times10^6)}\right) &\text{if } p(t) \geq 0\\
		\eta \left(SOC(t) + \frac{\Delta t p(t)}{\mu C_{bat}(3.6\times10^6)}\right) &\text{otherwise}
	\end{cases}
\end{split}
	\label{ch2:equ:battery-model-equation}
\end{equation}

Here the next State of Charge, $SOC(t+\Delta t)$, is computed from the current State of Charge, $SOC(t)$, and the current battery power, $p(t)$.
This is done by calculating the current change in SOC as the added energy $\Delta t p(t)$, divided by the total battery capacity $C_\text{bat}$.
Dynamic properties of the model also take into account the energy conversion efficiency, $\eta$, and the self-discharge factor, $\mu$.

For the purpose of the simulation, it is assumed that the battery is initially charged up to 50\%.
Hence, the initial conditions of this model are defined as $SOC(0) = 0.5$, which makes the model valid for a time span of $t \geq 0$, where $t \in \mathbb{Z}_{\geq0}$.

\subsection{Load data and ESMU scheduling}
\label{ch2:subsec:load-data-and-esmu-scheduling}

\nomenclature[J]{$k(t)$}{Sampling time conversion function, linking sub-half hourly samples $t$ at sampling period $\Delta r$ to half-hourly period $30 \Delta t$}

Having established the ESMU model, the procedure to generate a corresponding schedule is explained in this subsection.
This procedure follows the same practice as outlined in the previous chapter (i.e. in Section~\ref{ch1:subsec:esmu-scheduling}) where an ESMU schedule is generated at half-hourly temporal resolution.
Therefore the same synchronisation function, $k(t)$, can be used to link the native sampling period of $\Delta t$ to the schedule's half-hourly period.
Since the sub-half-hourly operation was at a minutely period and the generated schedule is at half-hourly period, this fixed conversion function is defined as:

\begin{equation}
	k(t) := \left\lfloor\frac{t-1}{30\Delta t}\right\rfloor+1
	\label{ch2:equ:sample-conversion-function}
\end{equation}

Having established a means of synchronising the two sampling periods, the shape of the ESMU schedule that would ``smoothen'' the underlying power profile is defined next.
For simplicity linear forwarding was chosen which means that the power assigned at e.g. $t=1$ remains constant over the scheduling period of $30\Delta t$ until $t=31$.
With this assumption the ESMU's SOC can be calculated for each $t$ despite the scheduled power profile only having been defined for every 30$^\text{th}$ $t$.
Furthermore, with this second assumption not only every sub-half-hourly ESMU power can be derived from its half-hourly schedule, but it also enables the calculation of every SOC, i.e. $SOC(t)$ is well defined.

\nomenclature[J]{$p_\text{for}(k(t))$}{Half-hourly load forecast that is used for computing the ESMU schedule, where $p_\text{for}(k(t))\in\mathbb{Z}$}
\nomenclature[J]{$\textbf{p}_\text{for}$}{Half-hourly load forecast vector that is used for computing the ESMU schedule, where $\textbf{p}_\text{for} = (p_\text{for}(k(t)))$}
\nomenclature[J]{$p_\text{sch}(k(t))$}{Half-hourly schedule that is generated from the load forecast, where $p_\text{sch}(k(t))\in\mathbb{Z}$}
\nomenclature[J]{$\textbf{p}_\text{sch}$}{Half-hourly schedule vector that is generated from the load forecast, where $\textbf{p}_\text{sch} = (p_\text{sch}(k(t)))$}

For the generation of the ESMU schedule a load forecast, $\textbf{p}_\text{for}$, was required; here $\textbf{p}_\text{for} = (p_\text{for}(k(t)))$.
Just like the ESMU forecast this forecast is also produced at half-hourly temporal resolution and it was provided by SSEN as part of the NTVV research project.
The task at hand is to find a half-hourly ESMU schedule, $\textbf{p}_\text{sch}$, where $\textbf{p}_\text{sch} = (p_\text{sch}(k(t)))$, that improves the shape of the underlying forecast, e.g. by reducing load peaks.
In order to generate this optimised ESMU schedule a performance metric quantifying improvements had to be defined first.
The remaining task is to now compute a half-hourly schedule, $\textbf{p}_\text{sch}$, that yields the best performance.
This computation is done by minimising several cost-functions.

In Chapter~\ref{ch1} several cost functions were defined.
Here however, three shape dependent cost-functions are used that quantify the profile improvements that are yielded by $\textbf{p}_\text{sch}$.
These costs entailed the Peak-to-Average Ratio (PAR), the difference between the resulting power profile's maximum and minimum (MMD) load, and the magnitude of all power transients (TRA) \cite{Mohsenian-Rad2010, Mostafa2016}.
Although these costs and their benefits have already been presented in Section~\ref{ch1:subsec:esmu-scheduling} of this thesis, they are reminded for convenience.
Before however detailing each of these three cost functions, a notation that simplifies power as, $\textbf{p}$, is introduced:

\begin{equation}
\begin{split}
	p(t) = p_\text{for}(k(t)) + p_\text{sch}(k(t))\\
	\text{ where } (p(t)) = \textbf{p}
\end{split}
\label{ch2:equ:notation-simplification}
\end{equation}

Within this section, the vector $\textbf{p}$ represents the power profile as it would be measured at the substation when both forecast, $p_\text{for}(t)$, and scheduled, $p_\text{sch}(t)$, power were applied.
The first cost function that is used in this chapter addresses the minimisation of PAR and is defined as follows:

\nomenclature[J]{$\zeta_\text{PAR}(\textbf{p})$}{Cost of a power profile $\textbf{p}$, based on Peak-to-Average Ratio (PAR), where $\zeta_\text{PAR}(\textbf{p}) \in \mathbb{R}_{\geq0}$}

\begin{equation}
	\zeta_\text{PAR}(\textbf{p}) := \left(\frac{\max_t|\textbf{p}|}{\overline{\textbf{p}}}\right)^2 - 1
	\label{ch2:equ:cost-par}
\end{equation}

\nomenclature[J]{$T_\text{sch}$}{Length of scheduling horizon, where $T_\text{sch} \in \mathbb{Z}_{>0}$ and $T \geq T_\text{sch}$}

Here, $\overline{\textbf{p}}$ represents the mean power, i.e. $\overline{\textbf{p}} = \frac{\Delta t}{T_\text{sch}}\sum_{t=1}^{T_\text{sch}}p(t)$ and $\overline{\textbf{p}} \in \mathbb{R}$, where $T_\text{sch}$ is the length of the scheduling horizon in regards to the sampling period $\Delta t$.
If the profile $\textbf{p}$ had a lot of spikes then the ratio between its maximum and its mean value is greater than one (or with the $-1$ term greater than zero).
A perfectly flat power profile on the other hand would thus result in cost of zero.
However, due to limited battery capacity achieving such a cost of zero is highly unlikely.
This is why a solution to minimise this cost needs to be found that minimises this cost in accordance to the previously explained ESMU model.
To not only increase the mean power or reduce peak power, the second cost function is defined as the difference between minimum and maximum power of $\textbf{p}$:

\nomenclature[J]{$\zeta_\text{MMD}(\textbf{p})$}{Cost of a power profile $\textbf{p}$, based on the difference between minimum and maximum power, where $\zeta_\text{MMD}(\textbf{p}) \in \mathbb{R}_{\geq0}$}

\begin{equation}
	\zeta_\text{MM}(\textbf{p}) := \max_t(\textbf{p}) - \min_t(\textbf{p}) % \max_t(p(k(t))) - \min_k(p(k(t)))
	\label{ch2:equ:cost-min-max}
\end{equation}

Similar to the PAR this cost also reduces to zero when the resulting power profile is perfectly flat.
Unlike the PAR however, this cost does not incentivise an increase of mean power.
Minimising PAR by itself may result in unnecessary and potentially damaging battery cycling when trying to raise the mean power of the profile, yet this behaviour is avoided when $\zeta_\text{MMD}$ is included alongside $\zeta_\text{PAR}$.
Nonetheless, $\zeta_\text{PAR}$ and $\zeta_\text{MMD}$ only impact the fringes of the resulting half-hourly power profile and could lead to an erratic load profile.
Therefore the third and final cost addresses the interim power volatility by aiming to minimise the largest possible power transient:

\nomenclature[J]{$\zeta_\text{TRA}(\textbf{p})$}{Cost of a power profile $\textbf{p}$, based on largest power transient, where $\zeta_\text{TRA}(\textbf{p}) \in \mathbb{R}_{\geq0}$}

\begin{equation}
	\zeta_\text{TRA}(\textbf{p}) := \max_{t}(p(t+\Delta t)-p(t))^2
	\label{ch2:equ:cost-tra}
\end{equation}

Minimising this final cost has a smoothening effect on the improved half-hourly power profile since a profile with no transients is by definition a flat and smooth profile.
Since all three cost functions are normalised, they are summaries into a single global cost function.
In this cost function only the half-hourly ESMU schedule, $\textbf{p}_\text{sch}$, is used as an input and the forecast, $\textbf{p}_\text{for}$, is kept constant:

\nomenclature[J]{$\zeta(\textbf{p})$}{Global cost for a given power profile $\textbf{p}$}
\begin{equation}
\begin{split}
	\zeta(\textbf{p}_{sch}) := &\zeta_{PAR}(\textbf{p}_{sch}+	\textbf{p}_{for})\\
	&+ \zeta_{MM}(\textbf{p}_{sch}+\textbf{p}_{for})\\
	&+ \zeta_{TRA}(\textbf{p}_{sch}+\textbf{p}_{for})
\end{split}
\label{ch2:equ:cost-global}
\end{equation}

Subject to ESMU constraints, this global cost function is minimised using a standard solver (i.e. Sequential Quadratic Programming - SQP) to yield a ESMU schedule that is optimised for the given forecast:

\begin{equation}
	\min_{p_{schedule}}\zeta(p_{schedule}) \text{ s.t. }
	\begin{cases}
		0 \leq SOC(t) \leq 1\\
		|p_{battery}(t)| \leq C_{battery} \times C_{factor}
	\end{cases}
	\forall t
	\label{ch2:equ:cost-minimisation}
\end{equation}

In order to limit the control's flexibility a State Of Charge tolerance, $SOC_{tol}$, is included in this minimisation problem.
$SOC_{tol}$ defines the maximum allowed deviation from the computed SOC profile without hitting operational limits (i.e. SOC of one or zero) and may take values in the form of $SOC_{tol} \in [0, 0.5)$ where 0 implies no tolerance and 0.5 implies complete flexibility as if no schedule were computed.
For the work at hand a value of 0.1 was chosen to allow a $\pm$10\% energy tolerance band.

\begin{figure}\centering
	\subfloat[]{%
		\includegraphics{_chapter2/fig/day-forecasted}
		\label{ch2:subfig:day-forecasted}
	}\\
	\subfloat[]{%
		\includegraphics{_chapter2/fig/day-actual}
		\label{ch2:subfig:actual-forecasted}
	}
	\caption{An example of applying a half-hourly ESMU schedule to its half-hourly schedule (Subfig. \ref{ch2:subfig:day-forecasted}) and the actual, sub-half-hourly daily load (Subfig. \ref{ch2:subfig:actual-forecasted}).}
	\label{ch2:fig:cost-sample}
\end{figure}

As repetitively mentioned, the ESMU operation that results from this scheduling mechanism is at half-hourly resolution and has therefore limited impact on sub-half-hourly load variation.
To visualise this limitation a singe day's ESMU schedule was generated from its corresponding forecast as defined in Equation~\ref{ch2:equ:cost-minimisation} and plotted in Figure~\ref{ch2:fig:cost-sample}.
In this simple comparison the noticeable discrepancy between the half-hourly ESMU schedule and the actual, sub-half-hourly demand can be observed.
Furthermore, noticeable disparity in peak duration, magnitude and volatility can be noted.
This discrepancy and disparity emphasise the incompatibility issues between half-hourly ESMU schedules and the actual sub-half-hourly load.
As previously discussed benefits of ESMU were intended to mitigate sub-half-hourly load volatility, yet this cannot be achieved when solely applying half-hourly ESMU schedules in an off-line manner.
Therefore, the control strategy to add an on-line component is explained in the next section.


\section{Control of ESMU}
\label{ch2:sec:control-of-esmu}

\begin{figure}[htb]\centering
% Define some block styles
\tikzstyle{box} = [%
	draw,%
	rectangle,%
	%fill=green!20,%
	minimum height=3em,%
	minimum width=5em,%
]
\begin{tikzpicture}[node distance=2cm, shorten >= 1pt, >=stealth', auto, scale=0.8, transform shape]
	% Define nodes
    \path
    (0,0)
    node [box, minimum width=5cm](network) {Network}
    node [box, below of=network, yshift=5mm](battery) {Battery}
    node [draw, circle, below of=battery](plus) {+}
    node [box, fill=green!20, below left of=plus, yshift=-10mm, xshift=-10mm](pid_soc) {PID$_1$}
    node [box, fill=green!20, below right of=plus, yshift=-10mm, xshift=10mm](pid_p) {PID$_2$}
    node [box, fill=blue!20, left of=pid_soc, xshift=-20mm](schedule) {Schedule}
    node [box, fill=green!20, right of=pid_p, xshift=20mm](mpc) {MPC};
%    node [box, below of=schedule](battery) {Battery}
%    node [box, below of=battery](network) {Network}
%   	node [box, fill=green!20, left of=battery, xshift=-14.5mm](controller) {Controller}
%	node [box, fill=green!20, below of=controller](mpc) {MCP};
	% Draw lines
	\draw [->] (battery) to (network);
	\draw [->] (plus) to node[right] {$p(t+\tau)$} (battery);
	\draw [->, bend left] (pid_soc) to node[right, pos=0.2] {$p_{1}(t+\tau)$} (plus);
	\draw [->, bend right] (pid_p) to node[left, pos=0.2] {$p_{2}(t+\tau)$} (plus);
	\draw [->] (battery.180) to [out=180, in=120] node[left, pos=0.2] {$SOC(t)$} (pid_soc.180|-pid_soc.90);
	\draw [->] (network) to [out=340, in=60] node[right] {$\text{ }p_{network}(t)$} (pid_p.0|-pid_p.90);
	\draw [->] (schedule) to node[pos=0.35] {$SOC(k)$} (pid_soc);
	\draw [->] (mpc) to node[pos=0.35] {$\hat{p}(t+\tau)$} (pid_p);
	\draw [->, bend left] (network.0) to (mpc.90);
	
	\draw (-4, -7.25) node [right] {Controller};
	
	\draw [dashed] (-4, -3) -- (4, -3) -- (4, -7) -- (-4, -7) -- (-4, -3);
%	\draw [->, bend right] (schedule.180) to (controller.90);
%	\draw [->] (network) to (mpc);
%	\draw [->] (mpc) to (controller);
%	\draw [->] (controller) to (battery);
%	\draw [->] (battery) to (network);
\end{tikzpicture}%


\caption{Controller}
\label{ch2:fig:system-controller}
\end{figure}


This section explains the dynamic control (i.e. the controller block that is shown in Figure~\ref{ch2:subfig:proposed-dynamic-control-system}), which contains the two PID compensators.
The first PID compensator is fed by the ESMU schedule, and the other is fed by the MPC load estimations.
After the control system is detailed in this section, the auto-regressive models, which were used during the course of this research, are explained, too.

\subsection{Dynamic control}

The content of the dynamic control procedure is shown in Figure~\ref{ch2:fig:system-controller}.
Here, two reference signals are used as inputs to the dynamic control.
The first reference signal is the SOC profile derived from the ESMU scheduled, $SOC(t)$, and the second is an estimated future network power, $\hat{p}_\text{net}(t+\Delta t)$.
These two inputs are fed into compensator PID$_1$ and compensator PID$_2$, respectively.
The output of each compensator is a corrective battery power component that, when summed, yields the next ESMU power, i.e. $p_1(t+\Delta t)$ and $p_2(t+\Delta t)$, which is applied to the ESMU model.
Each PID compensator also receives a feedback signal to compute the internal error states.
More specifically, PID$_1$ receives the most recent SOC value that is obtained from the ESMU model, $SOC^*(t)$, and PID$_2$ receives the network's most recent power demand, $p_\text{net}(t)$ (e.g. through measurements by substation monitoring).

\nomenclature[J]{$p_\text{net}(t)$}{Most recent network demand at sample $t$, where $\textbf{p}_{net} = (p_\text{net}(t))$ and $p_\text{net}(t) \in \mathbb{R}$}
\nomenclature[J]{$SOC^*(t)$}{Battery's state of charge at sample $t$, where $SOC^*(t) \in [0, 1]$}
\nomenclature[J]{$E_\text{SOC}(t)$}{Error in state of charge at sample $t$, where $E_\text{SOC}(t) \in \mathbb{R}$}
\nomenclature[J]{$E_{p}(t)$}{Difference between current and predicting network power at sample $t$, where $E_{p}(t) \in \mathbb{R}$}
\nomenclature[J]{$\hat{p}_\text{net}(t+\Delta t)$}{Predicted next network power at sample $t$}
\nomenclature[J]{$\boldsymbol{\alpha}$}{PID weight vector for SOC compensator PID$_1$, where $\boldsymbol{\alpha} = \{\alpha_P, \alpha_I, \alpha_D\}$ and $\boldsymbol{\alpha} \in \mathbb{R}^3$}
\nomenclature[J]{$\boldsymbol{\beta}$}{PID weight vector for MPC compensator PID$_2$, where $\boldsymbol{\beta} = \{\beta_P, \beta_I, \beta_D\}$ and $\boldsymbol{\beta} \in \mathbb{R}^3$}
\nomenclature[J]{$\textbf{a}$}{Weight vector for compensator input regression of the AR model, where $\textbf{a} = \mathbb{R}^{N}$}
\nomenclature[J]{$\textbf{b}$}{Weight vector for compensator output regression of the AR model, where $\textbf{b} = \mathbb{R}^{N}$}
\nomenclature[J]{$N$}{Number of regressors of the AR model, where $N \in \mathbb{Z}_{>0}$}
\nomenclature[J]{$p_1(t)$}{Corrective ESMU power components from PID$_1$, where $p_1(t) \in \mathbb{R}$}
\nomenclature[J]{$p_2(t)$}{Corrective ESMU power components from PID$_2$, where $p_2(t) \in \mathbb{R}$}
\nomenclature[J]{$SOC_{tol}$}{SOC tolerance, i.e. maximum deviation from the prescheduled SOC profile, where $SOC_{tol} \in [0, 0.5]$}

Inside the PID$_1$ component, a SOC error term, $E_\text{SOC}(t)$, is computed.
This term is the difference between the scheduled SOC profile, $SOC(t)$, and the actual (or simulated) SOC values, $SOC^*(t)$.
The following equation captures this error term.

\begin{equation}
	E_\text{SOC}(t) := SOC^*(t) - SOC(t)
	\label{ch2:equ:soc-error}
\end{equation}

Applying a standard, linearly weighted dynamic gain vector, $\boldsymbol{\alpha}$, to the SOC error, allows the calculation of a corrective ESMU power component dynamically.
Here, $\boldsymbol{\alpha} = \{\alpha_P, \alpha_I, \alpha_D\}$, where being the P, I and D weights, respectively.
This corrective power is denoted as $p_1(t+\Delta t)$, and is defined as follows:

\begin{equation}
\begin{split}
	p_1(t+\Delta t) &:= \alpha_P E_\text{SOC}(t)\\
	&+ \alpha_I \sum_{i=0}^\infty E_\text{SOC}(t-i\Delta t)\\
	&+ \alpha_D \frac{E_\text{SOC}(t)-E_\text{SOC}(t-\Delta t)}{\Delta t}
\end{split}
\label{ch2:equ:corrective-component-soc}
\end{equation}

Here, the integral component removes steady-state error and the instantaneous error differential prevents overshooting.
All in all, this compensator uses present and past values to issue a corrective future ESMU instruction.
Compensator PID$_2$ on the other hand uses values from the present, past and future in order to minimise the power transient and load peaks.

\begin{figure}\centering
% Define some block styles
\tikzstyle{box} = [%
	draw,%
	rectangle,%
	%fill=green!20,%
	minimum height=3em,%
	minimum width=5em,%
]
\tikzstyle{sample} = [draw, circle, fill, scale=0.4]
\tikzstyle{estimate} = [draw, circle, solid, fill=white, scale=0.4]
\begin{tikzpicture}[node distance=2cm, shorten >= 1pt, >=stealth', auto, scale=0.9, transform shape]
	% Draw axis and labels
	\draw [<->] (0, 6) -- (0, 0) -- (12, 0);
	\draw (12, 0) node [anchor=north east] {Time};
	\draw (0, 3) node [anchor=south, rotate=90] {Power};
	% Add axis ticks
	\foreach[count=\i, evaluate=\i as \l using int(\i-1)] \t in {1,4,...,10}
	{
		\draw (\t,0.1) -- (\t,-0.1);
		\ifthenelse{\l=0}
		{\draw (\t,-0.1) node[anchor=north] {$t$}}
		{
			\ifthenelse{\l=1}
			{\draw (\t,-0.1) node[anchor=north] {$t+\Delta t$}}
			{\draw (\t,-0.1) node[anchor=north] {$t+\l\Delta t$}};
		};
	}
	% Draw main power curve
	\draw [thick]
	(0.5, 2.1) to
	(1, 2) node[sample](sam_0) {} to
	(4, 4) node[sample](sam_1) {} to
	(7, 3.5) node[sample](sam_2) {} to
	(10, 2) node[sample](sam_3) {} to
	(11, 1.8);
	
	% Add power sample forward projection
	\draw [dashed] (1, 2) -- (4.6, 2);
	\draw [dashed] (4, 4) -- (7.5, 4);
	\draw [dashed] (7, 3.5) -- (10.5, 3.5);
	\draw [dashed] (10, 2) -- (10.5, 2);
	
	% Plot estimates
	\draw [dotted] (1, 2) to
	(4, 3.7) node[estimate](est_1) {} to
	(7, 3.8) node[estimate](est_2) {} to
	(10,1.7) node[estimate](est_3) {};
	
	% Label the power samples
	\draw [->] (1, 5) node[anchor=south] {$p_\text{net}(t)$} to [out=-135,in=135] (sam_0);
	\draw [->] (4, 5) node[anchor=south] {$p_\text{net}(t+\Delta t)$} to [out=-135,in=135] (sam_1);
	\draw [->] (7, 5) node[anchor=south] {$p_\text{net}(t+2\Delta t)$} to [out=-135,in=135] (sam_2);
	\draw [->] (10, 5) node[anchor=south] {$p_\text{net}(t+3\Delta t)$} to [out=-135,in=135] (sam_3);
	
	% Label the estimates
	\draw [->] (4, 0.75) node[anchor=north] {$\hat{p}_\text{net}(t+\Delta t)$} to [in=-45] (est_1);
	\draw [->] (7, 0.75) node[anchor=north] {$\hat{p}_\text{net}(t+2\Delta t)$} to [in=-45] (est_2);
	\draw [->] (10, 0.75) node[anchor=north] {$\hat{p}_\text{net}(t+3\Delta t)$} to [in=-45] (est_3);
	
	% Label the forward power transients
	\draw [decorate,decoration={brace,amplitude=3pt,mirror,raise=4pt},yshift=0pt,xshift=5mm]
	(4, 2) -- (4, 4) node [anchor=west,black,midway,xshift=2mm] {$E_p(t)$};
	\draw [decorate,decoration={brace,amplitude=3pt,mirror,raise=4pt},yshift=0pt,xshift=3mm]
	(7, 3.5) -- (7, 4) node [anchor=west,black,midway,xshift=2mm] {$E_p(t+\Delta t)$};
	\draw [decorate,decoration={brace,amplitude=3pt,mirror,raise=4pt},yshift=0pt,xshift=3mm]
	(10, 2) -- (10, 3.5) node [anchor=west,black,midway,xshift=2mm] {$E_p(t+2\Delta t)$};
	
\end{tikzpicture}
%	\draw [solid] (12, 6) -- (12, 4.75) -- (10.5, 4.75) -- (10.5, 6) -- (12, 6);
%	\draw (10, 6) node[sample, yshift=-10mm, xshift=25mm](legend_sample) {};
%	\draw node[right of=legend_sample, xshift=-15mm] {$p$};
%	\draw (10, 6) node[estimate, yshift=-20mm, xshift=25mm](legend_estimate) {};
%	\draw node[right of=legend_estimate, xshift=-15mm] {$\hat{p}_\text{net}$};
\caption{Underlying time-series based compensation strategy for compensator PID$_2$.}
\label{ch2:fig:power-transient-minimisation}
\end{figure}


Figure~\ref{ch2:fig:power-transient-minimisation} summarises the time series computations for each power sample at times $t$, $t+\Delta t$, etc.
Ideally, PID$_2$ uses present power readings, $p_\text{net}(t)$, and a power value of the immediate future, i.e. $p_\text{net}(t+\Delta t)$, to compute a power error signal, which is to be reduced to a smallest possible value.
This error signal is defined as:

\begin{equation}
	E_p(t) := p_\text{net}(t+\Delta t) - p_\text{net}(t)
	\label{ch2:equ:power-error-signal}
\end{equation}

However, since the future network power is unknown an ``estimated next power'', $\hat{p}_\text{net}(t+\Delta t)$, is used instead.
This value is the PID$_2$'s input from the MPC and results in an ``estimated power error signal'':

\begin{equation}
	\hat{\epsilon}_{p}(t) = \hat{p}_{network}(t+\tau) - p_{network}(t)
	\label{ch2:equ:power-error-estimate}
\end{equation}

Similarly to PID$_1$, PID$_2$ produces a corrective ESMU power component, $p_2(t)$ that smoothens the resulting power profile.
This corrective ESMU power is also computed using a standard linear weighted dynamic vector $\boldsymbol{\beta}$ (with $\boldsymbol{\beta} = \{\beta_P, \beta_P, \beta_P\}$, being the P, I and D weight, respectively):

\begin{equation}
\begin{split}
	p_2(t+\tau) &:= \beta_P E_{p}(t)\\
	&+ \beta_I \sum_{i=0}^\infty E_{p}(t-i\tau)\\
	&+ \beta_D \frac{E_{p}(t)-E_{p}(t-\tau)}{\tau}
\end{split}
\label{ch2:equ:corrective-component-power}
\end{equation}

Finally, the ``next ESMU power'' can be deduced by adding the two corrective ESMU power components, as shown in the equation below.

\begin{equation}
	p(t+\Delta t) = p_1(t+\Delta t) + p_2(t+\Delta t)
	\label{ch2:equ:next-battery-power}
\end{equation}

Both PID compensators do however depend on correctly chosen weights for $\boldsymbol{\alpha}$ and $\boldsymbol{\beta}$.
Therefore they need to be tuned prior to executing the dynamic control.
For this work a minimisation problem was formulated, based on a cost function, $\zeta^*(\alpha, \beta)$, to deduce the two weight vectors as follows:

\begin{equation}
\begin{split}
	&\min_{\alpha,\beta} \zeta^*(\alpha, \beta) \\
	&\text{ s.t. }
	\begin{cases}
		SOC(t) - SOC_\text{tol} \leq 0\\
		-SOC(t) \leq 0\\
		SOC(t) - 1 \leq 0
	\end{cases}
	\forall t
\end{split}
\label{ch2:equ:cost-weights}
\end{equation}

Here, $\zeta^*(\alpha, \beta)$ is defined as:

\begin{equation}
\begin{split}
	\zeta^*(\alpha,\beta) := \max_t(\textbf{p}_{net} + \textbf{p})\\
	\text{ where } p_\text{net}(t) \in \textbf{p}_{net} \text{ and } p(t) \in \textbf{p}
\end{split}
\label{ch2:equ:dynamic-cost}
\end{equation}

In Equation~\ref{ch2:equ:cost-weights} and \ref{ch2:equ:dynamic-cost}, $\zeta^*(\alpha, \beta)$ represents the sub-half-hourly peak load during a day, when ESMU schedules are adjusted with the corresponding $\boldsymbol{\alpha}$ and $\boldsymbol{\beta}$ weights.
Also, the same SOC tolerance that was used to generate the SOC schedule, i.e. $SOC_{tol}$, is included to prevent the solution from deviating off the prescheduled SOC profile.
To generalise this solution for all load cases, a minimisation problem was formulated to solve multiple daily load profiles in order to find ideal $\boldsymbol{\alpha}$ and $\boldsymbol{\beta}$ weights.
This resulting set of $\boldsymbol{\alpha}$ and $\boldsymbol{\beta}$ weights, therefore guaranteed a convergent and stable solutions.
The details concerning these case studies themselves, are however outlined in Section~\ref{ch2:sec:case-studies}.

\subsection{Model predictive control}

As explained in the literature review in Chapter~\ref{ch-literature}, Model Predictive Control (MPC) is favoured over Set-Point Control (SPC), since it takes into account time-series to produce a behaviour.
With this knowledge, MPC can be used to not only react to recent changes but also to counteract foreseen trends.
Different approaches exist to obtain these foreseen trends and these approaches highly vary in accuracy, computational burden and robustness.
Equally, the characteristics of underlying data which is used to train these models impacts their performances.
For the presented work, an efficient and robust approach is required, since system deployment dictates these functional requirements.
As a result, prediction accuracy is an optional requirement, which becomes important only when the predicting model can issue predictions in real-time and does (for the predicting horizon) remain stationary and bounded.

The simplest form of producing a prediction, is to assume that the currently observed load will also apply in the future.
This kind of prediction does however not take into account model dynamics.
An AR model on the other hand, uses a series of past observations to predict the next.
Yet the further into the future these predictions are made, the less accurate they become.
Therefore, this work only attempts to issue a power prediction for the immediate future, i.e. next sample time at $t+\Delta t$.
Furthermore, to reduce computational burden and guarantee real-time operation, the simplest dynamic model, i.e. an Auto-Regressive (AR) model is chosen instead of e.g. deep artificial neural networks.
Since external forces can and often do impact the behaviour of the model, the AR model is treated as an exogenous model, with a time-series of input powers, $p(t) \in \textbf{p}$, a time-series of predicted ``next powers'', $\hat{p}(t) \in \hat{\textbf{p}}$, and an internal delay function $t-\Delta t$.

\begin{figure}\centering
% Define some block styles
\tikzstyle{box} = [%
	draw,%
	rectangle,%
	%fill=green!20,%
	minimum height=2em,%
	minimum width=2em,%
]
\begin{tikzpicture}[node distance=2cm, shorten >= 1pt, >=stealth', auto, scale=1.0, transform shape]
	\pgfmathsetmacro\N{4}

	% Draw input nodes
    \draw
    (0, 0) node [](input) {$p(t)$}
    (2, 0) node [](input_1) {};
    % Draw output nodes
    \draw
    (9.5, 0) node [](output) {$\hat{p}(t+\Delta t)$}
    (7, 0) node [](output_1) {};
    % Draw main adder node
    \draw (4.5, 0) node [circle, draw](adder_main) {+};
    % Link input, adder and output
    \draw [->] (input) to (adder_main);
    \draw [->] (adder_main) to (output);
    % Draw AR model nodes
    \foreach[count=\i, evaluate=\i as \l using int(\i*3)] \n in {1,2,...,\N}
	{
		\ifthenelse{\n<\N}
		{
	    	\draw (2, 1.5-\l) node [box](z_a\n) {${t-\Delta t}$};
	    	\draw (2, -\l) node [](half_a\n) {};
	    	\draw (2.8, -\l) node [box](a\n) {$a_\n$};
	    	
	    	\draw (7, 1.5-\l) node [box](z_b\n) {${t-\Delta t}$};
	    	\draw (7, -\l) node [](half_b\n) {};
	    	\draw (6.2, -\l) node [box](b\n) {$b_\n$};
	    	
	    	\draw (4.5, -\l) node [circle, draw](adder_\n) {+};
    	}{
	    	\draw (2, 2.95-\l) node [](z_a\n) {};
	    	\draw (7, 2.95-\l) node [](z_b\n) {};
    	}
    	\ifthenelse{\n=1}
    	{
    		\draw [->] (input_1.center) to (z_a\n);
    		\draw [->] (output_1.center) to (z_b\n);
    		\draw [->] (adder_\n) to (adder_main);
    	}{
    		\pgfmathtruncatemacro{\dn}{\n-1};
    		\pgfmathtruncatemacro{\dnn}{\n-2};
    		\ifthenelse{\n<\N}
			{
				\ifthenelse{\dn=1}
				{
					\draw [->] (z_a\dn) to node[left]{$p(t-\Delta t)$} (z_a\n);
				}{
					\draw [->] (z_a\dn) to node[left]{$p(t-\dn\Delta t)$} (z_a\n);
				}
				\ifthenelse{\dnn=0}
				{
					\draw [->] (z_b\dn) to node[right]{$\hat{p}(t)$} (z_b\n);
				}{
					\ifthenelse{\dnn=1}
					{
						\draw [->] (z_b\dn) to node[right]{$\hat{p}(t-\Delta t)$} (z_b\n);
					}{
						\draw [->] (z_b\dn) to node[right]{$\hat{p}(t-\dnn\Delta t)$} (z_b\n);
					}
				}
				\draw [->] (adder_\n) to (adder_\dn);
			}{
				\draw [-] (z_a\dn) to node[left]{$p(t-\dn\Delta t)$} (z_a\n.center);
				\draw [-] (z_b\dn) to node[right]{$\hat{p}(t-\dnn\Delta t)$} (z_b\n.center);
			}
    		\draw [->] (half_a\dn.center) to (a\dn);
    		\draw [->] (a\dn) to (adder_\dn);
    		\draw [->] (half_b\dn.center) to (b\dn);
    		\draw [->] (b\dn) to (adder_\dn);
    	}
    }

\end{tikzpicture}
\caption{Example of exogenous auto-regressive model that is used for model predictive control. Here, ${t-\Delta t}$ indicates the time delay by one sample period.}
\label{ch2:fig:mpc-arx}
\end{figure}


Figure~\ref{ch2:fig:mpc-arx} graphically captures the standard AR model's function tree, which is also represented mathematically in the following equation:

\begin{equation}
	\hat{p}(t+\Delta t) = p(t) + \sum_{i=1}^{N} a_i p(t-i\Delta t)+ \sum_{i=1}^{N} b_i \hat{p}(t-(i-1)\Delta t)
\label{ch2:equ:mpc-arx}
\end{equation}

Values of the two weight vectors $\textbf{a}$ and $\textbf{b}$, where $\textbf{a} = (a_i)$ and $\textbf{b} = (b_i)$, are determined during runtime using the standard adaptive least squares algorithm.
This algorithm dynamically adjusts $\textbf{a}$ and $\textbf{b}$ to minimise the prediction error at each time-step.
Beside finding optimised values for $\textbf{a}$ and $\textbf{b}$, the model's number of regressors, $N$, is also predicted to impact the model's performance ($N$ is also referred to as the ``model length'').
The example in Figure~\ref{ch2:fig:mpc-arx} represents a symmetric model where $N=3$.
This short length however is most likely insufficient in predicting $p(t+\Delta t)$, which is why several models of increasing lengths are assessed and compared.
From this comparison, the impact of $N$ on the models' resulting values of $\hat{p}(t+\Delta t)$, and correspondingly on the performance of the dynamic controller can be discussed.
Details about the cases for different model length are presented in the case studies in Section~\ref{ch2:sec:case-studies}.












\section{Case studies}
\label{ch2:sec:case-studies}

All cases that are used to demonstrate the operation of the proposed hybrid control use 27 days of uninterrupted historical demand data.
In total, five cases are assessed.
Two special cases, respectively case \textbf{O} and case \textbf{B}, assess the performance of the original case, i.e. where no ESMU operation takes place, against a baseline case, i.e.  where traditional off-line ESMU operation that only uses predetermined half-hourly ESMU schedules is referred to as the benchmark case.
The remaining remaining cases, which are explained below, capture different implementations of the dynamic control.
These three case studies are defined: cases \textbf{I}, \textbf{II} and \textbf{III}.
This group of three case studies evaluates the impact of the proposed dynamic control when subjected to realistic (i.e. imperfect) half-hourly load forecasts.
In each of the three cases, a different mechanisms is used to predict the power volatility.
More specifically:
\begin{itemize}
	\item case \textbf{I} implements the simplest prediction mechanism, i.e. it is assumed that the current power measurements repeats,
	\item case \textbf{II} uses the aforementioned MPC, and performance of different AR model lengths is compared, and
	\item case \textbf{III},the third and final case, represents an ideal scenario where perfect foresight is assumed and the exact next load can be estimated.
\end{itemize}
For clarity, all three cases, numbered \textbf{I} to \textbf{III}, are summarised and tabulated in Table~\ref{ch2:tab:cases}.

\begin{table}\centering
	\begin{tabular}{r | c c}
		estimation method & real forecast\\% & ideal forecast\\
		\hline
		perfect foresight & \textbf{I}\\% & \textbf{IV}\\
		MPC (AR/ARX) & \textbf{II}\\% & \textbf{V}\\
		power repetition & \textbf{III}\\% & \textbf{VI}\\
	\end{tabular}
	\caption{Three cases and their dynamic control input assumptions}
	\label{ch2:tab:cases}	
\end{table}

Results from all ESMU cases (\textbf{B}, \textbf{I}, \textbf{II} and \textbf{III}) are first compared against the original, i.e. uncompensated, network load case (\textbf{O}).
Here, by using a sample day, the assessment of load profile improvements are made clear.
Once it is clear how the day's peak is reduced by the algorithm, the daily peak reduction capability from all cases' results are compared.
Rather than assessing the underlying load profile from a time-series perspective, focus is only put on any additional reductions of peak load.
However, the number of days may make it difficult to spot trends and improvements in the data.
Therefore, from the daily peak reduction results, a Probability Density Function (PDF) is derived, which is based on kernel density estimation.
The PDF shows the stochastic improvement of each case in comparison to the original case, i.e. case \textbf{O}.
Finally, to assess the AR model's impact on the peak reduction performance, the simulations are re-run using different AR model lengths ($N$) are and the results are compared using the same PDF comparison method.


\section{Results and discussion}
\label{ch2:sec:results}

All proposed cases are used to control power flow of the ESMU using 27 days of uninterrupted sub-half-hourly load records.
In this section the time-series improvements are presented at first where a day's peak reduction due to the sub-half-hourly schedule adjustment are highlighted.
\hladd{After all, reducing peaks both frees additional resources for future load and reduces ohmic losses in the cables.}
Then the daily peak reduction across the entire dataset is presented and followed by a probability density plot to better compare these findings.
In the end the model's impact on the peak reduction performance is assessed.

\subsection{Time-series analysis}

\begin{figure}\centering
	\subfloat[]{%
		\includegraphics{_chapter2/fig/day-peak-1}
		\label{ch2:subfig:day-peak-total}
	}\\
	\subfloat[]{%
		\includegraphics{_chapter2/fig/day-peak-1-zoom}
		\label{ch2:subfig:day-peak-zoomed}
	}
%	\subfloat[]{%
%		\includegraphics{_chapter2/fig/day-peak-1}
%		\label{ch2:subfig:day-peak-1}
%	}
%	\vspace{0mm}
%	\subfloat[]{%
%		\includegraphics{_chapter2/fig/day-peak-2}
%		\label{ch2:subfig:day-peak-2}
%	}
	\caption{Time series performance over a single day when using realistic load forecasts: (\ref{ch2:subfig:day-peak-total}) total day; (\ref{ch2:subfig:day-peak-zoomed}) zoomed in on critical period}
	\label{ch2:fig:day-peak}
\end{figure}

A single day was plotted in Figure~\ref{ch2:fig:day-peak} which shows the time-series improvements that were yielded by the ESMU operation.
For visual clarity Figure~\ref{ch2:subfig:day-peak-total} and \ref{ch2:subfig:day-peak-zoomed} show, respectively, the entire day and a zoomed in version that focuses on the period of interest where the ESMU impact is most apparent.
It can be observed that the unmodified demand profile (i.e. the original case  \textbf{O}) and the case where scheduled half-hourly ESMU operation is applied (i.e. the  baseline case \textbf{B}) result in noticeably higher load peaks than any of the three adjustment cases.
More specifically, the original peak reduction (which is equal to the scheduled ESMU power) was 1.8kW (or 3.9\% peak reduction).
The average peak reduction when applying adjustments to the ESMU operation was 9.6kW (or 20.6\% peak reduction).
Although it is too early to conclude on any overall performance improvements this time-series modification does show the physical impact of the ESMU schedule adjustments on the network's load profile.
Furthermore Figure~\ref{ch2:subfig:day-peak-total} highlights the volatility of the underlying data which would be neglected for half-hourly ESMU schedules.

Interestingly, both the standard AR and the exogenous AR estimation models that were used in case \textbf{II} performed very similar and show little to no significant difference in peak reduction performance.
Equally noteworthy is the fact that the simplest prediction methods of them all in case \textbf{III} (i.e. the method of assuming a power repetition occurs) yields positive peak power reductions, too.
However, case \textbf{I} slightly outperformed all other cases since perfect knowledge would also imply best results.
Nonetheless, only a small improvement was possible due to the imperfect underlying half-hourly ESMU schedule.
The amount by which the three cases were able to reduce the daily peak load is also indicated by horizontal dashed lines and dots located at the point of peak load for each profile.
These initial findings show that every single version of dynamic control reduces peak load when compared to the baseline case \textbf{B}.
This finding is also tabulated in Table~\ref{ch2:tab:ts-table}, and it suggests that the prediction mechanism by itself did play some role when compensating for demand volatility.

\begin{table}\centering
%	\begin{tabular}{r | c}
%		case & peak load\\% & ideal forecast\\
%		\hline
%		\textbf{O} & 46.6kW\\
%		\textbf{B} & 44.8kW\\
%		\textbf{I} & 36.4kW\\% & \textbf{IV}\\
%		\textbf{II} (AR) & 36.8kW\\% & \textbf{V}\\
%		\textbf{II} (ARX) & 36.6kW\\% & \textbf{V}\\
%		\textbf{III} & 38.4kW\\% & \textbf{VI}\\
%	\end{tabular}
	\begin{tabular}{r | c | c | c | c | c | c}
		\multirow{2}{*}{case} & \multirow{2}{*}{\textbf{O}} & \multirow{2}{*}{\textbf{B}} & \multirow{2}{*}{\textbf{I}} & \textbf{II} & \textbf{II} & \multirow{2}{*}{\textbf{III}}\\% & ideal forecast\\
		& & & & \tiny{(AR)} & \tiny{(ARX)} & \\% & ideal forecast\\
		\hline
		peak & \multirow{2}{*}{46.6} & \multirow{2}{*}{44.8} & \multirow{2}{*}{36.4} & \multirow{2}{*}{36.8} & \multirow{2}{*}{36.6} & \multirow{2}{*}{38.4}\\
		(kW) & & & & & & \\
	\end{tabular}
	\caption{Peak reduction in time-series sample}
	\label{ch2:tab:ts-table}	
\end{table}

However, the general impact of each prediction method on the resulting peak reduction performance can only be assessed if the complete dataset is evaluated.
Hence, the next section compares the daily peak load reduction from the application of each case.

\subsection{Daily peak reduction}

\begin{figure}\centering
	\subfloat[]{%
		\includegraphics{_chapter2/fig/daily-peaks-1}
		\label{ch2:subfig:daily-peaks}
	}\\
	\subfloat[]{%
		\includegraphics{_chapter2/fig/daily-peaks-1-percentage}
		\label{ch2:subfig:daily-peaks-percentage}
	}
	\caption{Daily peak reduction when using realistic forecasts as: (\ref{ch2:subfig:daily-peaks}) peak power values; (\ref{ch2:subfig:daily-peaks-percentage}) percentage of original case \textbf{B}.}
	\label{ch2:fig:daily-peaks}
\end{figure}

In Figure~\ref{ch2:fig:daily-peaks}, every day's power peak was extracted in a similar way to the procedure that was used for Figure~\ref{ch2:fig:day-peak}.
Here the actual power peaks were plotted in Figure~\ref{ch2:subfig:daily-peaks}, and the relative power improvements (i.e. ratio to the baseline power peaks from case \textbf{B}) were plotted in Figure~\ref{ch2:subfig:daily-peaks-percentage}.
From both plots it can be seen that controlling ESMU using the proposed dynamic control \hladd{(\textbf{I}, \textbf{II}, \textbf{III}) }lowers peak load.
This is true even when the underlying ESMU schedule originally worsened and increased peak load\hladd{ (see \textbf{B} vs \textbf{O})}.
Such a behaviour can be observed clearly during \hlrem{e.g.} days 6 and 25, where the half-hourly ESMU schedule\hladd{ based on \textbf{B}} increased the actual load peak\hladd{ from case \textbf{O}} by 2.8kW and 2.5kW, respectively.
The ESMU schedule adjustment mechanisms \hladd{(\textbf{I}, \textbf{II}, \textbf{III}) }however compensated for this error, but in those two cases the compensation was not enough to reduce peak power below the original value.
Day 26 on the other hand experienced a similar increase in peak power during the baseline case \hlrem{(i.e. case }\textbf{B}\hlrem{)} by 1.2kW, but the \hladd{proposed }power adjustment mechanism \hladd{according to \textbf{II}}corrected this forecast error and reduced the final peak power below the original value.

Nonetheless, the sensitivity to the underlying power prediction approaches becomes apparent when having this larger set of peak reduction results to compare the dynamic control's performance against its baseline cases.
As seen in Figure~\ref{ch2:subfig:daily-peaks-percentage} the scenario with perfect foresight (i.e. case \textbf{I}) frequently outperformed all other cases since it appears to achieve largest peak power reduction from the baseline case.
During some days however (i.e. day 3, 4 and 24) the compensators could not correctly compensate despite the perfect foresight.
This behaviour was unexpected, but it turns out that the discrepancy between the underlying half-hourly BESS schedule and the actual load curve (i.e. due to erroneous half-hourly load forecasts) caused the dynamic control to reach its SOC tolerance limit.
Reaching its limit during those three days consequently worsened the daily peak.
The simplest of all cases on the other hand (i.e. case \textbf{III}) yielded a constant but small reduction when compared to the baseline case.
Case \textbf{II} seems to perform similar, but slightly better than case \textbf{III}.
One could therefore assume that by maintaining a constant error in the power prediction does positively skew the results when already subjected to low-resolution forecasting errors.
Whether this assumption holds can however not be said with the presented analysis and instead, in order to obtain an more general picture of the overall peak reduction performance, the Probability Density Function (PDF) had to be estimated and analysed for all cases.
This is done in the following section.

\subsection{Probability of peak reduction}
\label{ch2:subsec:probability-of-peak-reduction}

\begin{figure}\centering
	\includegraphics{_chapter2/fig/pdf-1-avg}
%	\subfloat[]{%
%		\includegraphics{_chapter2/fig/pdf-1}
%		\label{ch2:subfig:peak-pdf-1}
%	}
%	\vspace{0mm}
%	\subfloat[]{%
%		\includegraphics{_chapter2/fig/pdf-2}
%		\label{ch2:subfig:peak-pdf-2}
%	}
	\caption{Probability of load peak when using realistic forecasts.}
	\label{ch2:fig:peak-pdf}
\end{figure}

With the use of the standard kernel density estimation, the PDF is plotted in Figure~\ref{ch2:fig:peak-pdf}.
The data used to generate these plots is the same data as shown in Figure~\ref{ch2:fig:daily-peaks}.
Now however, the probability of a peak power occurring is linked to the magnitude of this peak.
It can be seen that case \textbf{O} has the highest probability around a peak load of 45kW, whilst case \textbf{B} has its highest probability around a peak load of 42kW.
This indicates that there is even a high probability that the half-hourly ESMU schedule has a positive impact on the load peaks.
When adjusting this schedule by using the proposed dynamic control, this peak was however lowered further.
Case \textbf{I} performed best by having a most probable peak power of 36.1kW.
Case \textbf{II} achieve the second best values at 36.7kW and 36.8kW (for AR and ARX case, respectively) whilst the simplest prediction mechanism has its peak power probability maximised at 38.4kW.

\begin{figure}[htb]\centering
	\includegraphics[width=\linewidth]{_chapter2/fig/difference-pdf-1-avg}
%	\subfloat[]{%
%		\includegraphics[width=\linewidth]{_chapter2/fig/difference-pdf-1}
%		\label{ch2:subfig:peak-diff-pdf-1}
%	}
%	\vspace{0mm}
%	\subfloat[]{%
%		\includegraphics[width=\linewidth]{_chapter2/fig/difference-pdf-2}
%		\label{ch2:subfig:peak-diff-pdf-2}
%	}
	\caption{Probability of load peak reduction when using realistic forecasts.}
	\label{ch2:fig:peak-diff-pdf}
\end{figure}

Figure~\ref{ch2:fig:peak-diff-pdf} takes this analysis even further where only the difference in peak load to the original case (i.e. case \textbf{O}) is plotted.
Now the ESMU impact can easily be seen since a high probability of positive peak load reduction indicates a beneficial impact of the ESMU operation.
A negative peak load reduction (i.e. increased peak load) would therefore indicate a worse performance.
As expected, case \textbf{B} has a slight positive impact on the system whilst a cumulative probability of more than 25\% (i.e. area under curve of case \textbf{O}) to the left of 0kW suggests that the peak might be worsened one in four times.
The dynamic control with its simplest prediction method however (i.e. case \textbf{III}) lowered this probability to already \hlrem{11.8\%}\hladd{7.4\%}.
The perfect foresight model (i.e. case \textbf{I}) performed \hlrem{best at} only \hlrem{7.4\%}\hladd{at 11.8\%} and the MPC based cases (i.e. case \textbf{II}) achieved an average of 8.0\% probability of worsening the peak power.
\hladd{The fact, that the perfect foresight model \textbf{III} could not reduce the probability of peaks as well as the simple model \textbf{I} is likely due to the used power profiles and forecast errors.
However, the mean probabilities (that are discussed below) differ as expected.}

\hladd{In the following, the mean peak reduction of the base case \textbf{B} (i.e. where the cumulative probability reaches 50\%) is treated as the benchmark for peak reduction.
This probability is reached at 1.7kW or in other words:}\hlrem{
Beside the reduced probability of missing or worsening peak load, the probability of having a larger positive impact is also increased when using the dynamic control.
Whilst} the probability of reducing load peaks by 1.7kW or more was at 50\% for case \textbf{B}\hladd{.
The}\hlrem{,} \hladd{simplest }case \textbf{III} \hladd{however }increased this probability to 77.7\%, case \textbf{II} to 84.5 5\% (AR) / 83.1\% (ARX), and the \hladd{perfect foresight }case \textbf{I} to 79.8\%.
The reason why this simplest case achieved a slightly lower value than the AR/ARX cases was due to aforementioned discrepancy between actual and forecasted load profiles.
Due to the discrepancy in erroneousness the chosen SOC tolerance was exhausted and lead to some worsening cases that negatively skewed results of the perfect foresight case (i.e. case \textbf{I}).
Nonetheless, when comparing the three dynamic control cases with each other as done in Figure~\ref{ch2:fig:peak-diff-pdf}, then it can be seen that case \textbf{II} using an AR model for MPC performed best at reducing peak loads for the used dataset.

\subsection{Impact of varying the model's length}

The subsequent results are intended to show whether the length of the AR/ARX model impacted the peak reduction performance.
To do so, the same procedure was use as shown in Section~\ref{ch2:subsec:probability-of-peak-reduction}, but the length of the AR and ARX models was varied from five minutes to two hours.
Therefore the MPC of the dynamic control took into account a longer power history to potentially improve the prediction of the next power.

\begin{figure}\centering
	\includegraphics{_chapter2/fig/difference-pdf-1}
%	\subfloat[]{%
%		\includegraphics{_chapter2/fig/pdf-1}
%		\label{ch2:subfig:peak-pdf-1}
%	}
%	\vspace{0mm}
%	\subfloat[]{%
%		\includegraphics{_chapter2/fig/pdf-2}
%		\label{ch2:subfig:peak-pdf-2}
%	}
	\caption{Probability of peak load reduction for different prediction mechanisms and different AR/ARX model lentgths.}
	\label{ch2:fig:peak-pdf-multi-length}
\end{figure}

Similar to Figure~\ref{ch2:fig:peak-diff-pdf}, Figure~\ref{ch2:fig:peak-pdf-multi-length} shows the probability for the difference in peak power between the original case (\textbf{O}) and all other cases.
In this plot however all PDFs for the different model lengths have been included (whilst the previous study only showed the inter-model means).
It can be seen that both the AR and ARX case (i.e. case \textbf{II}) performed noticeably better than the baseline case \textbf{B}.
Despite the varying model length all PDFs appear to peak around a reduction performance of 5kW.
Therefore one may assume that the length of the chosen models does not significantly impact the results.

\begin{figure}\centering
	\includegraphics{_chapter2/fig/ar-length-peak-comparison-1}
%	\subfloat[]{%
%		\includegraphics{_chapter2/fig/pdf-1}
%		\label{ch2:subfig:peak-pdf-1}
%	}
%	\vspace{0mm}
%	\subfloat[]{%
%		\includegraphics{_chapter2/fig/pdf-2}
%		\label{ch2:subfig:peak-pdf-2}
%	}
	\caption{Visualisation of the peak power distribution for different AR/ARX model lengths.}
	\label{ch2:fig:boxplot-multi-length}
\end{figure}

This assumption is also supported by the boxplots in Figure~\ref{ch2:fig:boxplot-multi-length} where the peak power distributions are visualised for all different model lengths and the six different case studies.
It can be seen that the different AR/ARX model lengths (i.e. case \textbf{II}) outperforms both the original and baseline cases (i.e. case \textbf{O} and case \textbf{B}, respectively).
All in all, a certain variation in peak reduction performance can be observed, but no apparent trend.
Therefore the assumption that the model length impacts the performance of the dynamic control is true, but for the used data the assumption that a longer model generally yields better results is not.


\section{Discussion}
\label{ch2:sec:discussion}


\section{Conclusion}
\label{ch2:sec:conclusion}

\chapter{Benefits of de-synchronising control instructions for peak-shaving algorithm}
\label{ch3}

\section{Overview}
\label{ch3:sec:overview}

In previous chapters the the question regarding how one can optimally control a single battery energy storage has been addressed.
It was shown that half-hourly forecasts can be used to predict demand due to customer behaviours.
With this knowledge, Battery Energy Storage Systems (BESS) can be scheduled shave peak load in order to avoid overloading the already stressed system.
However, sub-half-hourly issues cannot be addressed by traditional BESS schedules, which is why two successive sub-half-hourly power adjustment methods were developed.
The first method improved network operation by focusing on the underlying three-phase network topology, whilst strictly following the underlying half-hourly SOC plan.
The second method on the other hand alleviated this constraint by adjusting total power flow instead.
Benefits from using BESS schedules complement dynamic feedback and yield improved power profiles with reduced peak load.

The logical next step is to take such schedules and apply them to multiple, distributed batteries.
To prevent the negative impact from battery charging, particularly when dealing with the home-charging of Electric Vehicles (EVs), their charging scheduling needs to be coordinated.
As already discussed in Section \ref{ch-review}, multiple EV scheduling methods exist.
Approaches propose demand prioritisation, multi-tariff environments or other game theory approaches to maximise global benefit whilst reducing the individual's disadvantages.
Here, coordination of charge scheduling signals becomes a vital requirement to react to other EV's charging plans.
This statement is commonly acknowledged, yet in a system of distributed scheduling the assumption of perfect knowledge about the environment no longer holds.

In fact, during distributed scheduling, control instructions that may be broadcasted by each EV, to inform all other EVs in the system of updated charging plan, need not or cannot be sent at certain times unless message synchronisation is guaranteed.
Therefore, this chapter studies the impact of desynchronising message propagation by adding transmission jitter to the updating broadcasts.
Here, EVs are used since their so called ``smart-charging'' behaviour is intended to avoid home-charging related load spikes.
A robust smart-charging algorithm to determine multiple EVs' charging schedule is presented and executed in both a synchronised and a desynchronised messaging scenario.
This smart-charging algorithm is designed to prevent aggregated demand, due to EV charging, from reaching certain peak power.
A Multi-Agent System (MAS) is implemented to perform the distributed scheduling using the Foundation for Intelligent Physical Agents (FIPA) compliant agents.

Results show how synchronised scheduling leads to expected outcomes that have also been established in literature.
However, adding jitter to message broadcasting significantly changes the same algorithm's behaviour.
Differences regarding rate of convergence and criteria for stability are most noticeable.
In this chapter, a brief summary of obtaining simplistic EV demand and how this is used in a typical (i.e. robust and converging) scheduling algorithm is detailed in Section \ref{ch3:sec:ev-coordination}.
Next, the distributed control and the reasons for choosing a multi-agent system is presented in Section \ref{ch3:sec:distributed-systems}, alongside the two cases for synchronised and desynchronised information propagation.
Section \ref{ch3:sec:results} presents and discusses the results from these two cases, upon which conclusions is drawn and presented in Section \ref{ch3:sec:summary}.













\chapter{Cooperative Operation of Distributed Batteries without Communications Infrastructure Needs}
\label{ch4}

\singlespacing
\epigraph{\textit{M. Zangs, P. Adams, et.al., ``Distributed Energy Storage Control for Dynamic Load Impact Mitigation,'' Energies, vol. 9, no. 8, p. 647, Aug. 2016}}{--- Available: http://dx.doi.org/10.3390/en9080647}
\doublespacing

\section{Overview}
\label{ch4:sec:overview}

This chapter addresses the question how multiple batteries could be coordinated collectively...

\section{Summary}
\label{ch4:sec:summary}

In this chapter, Chapter~\ref{ch4} an algorithm is proposed for distributed battery energy storage in order to mitigate the negative impact of highly variable and uncontrolled loads - such as the charging of EVs.
Unlike previous algorithms, the improved algorithm (i.e. AIMD+) only uses local bus voltage measurements to issue control instructions.
It implements a reference voltage profile which is derived from power flow analysis of the distribution network for its individualised set-point control.
Taking the distance to the feeding substation into account allows an optimisation of the algorithm's parameters for each BESS.
Simulations were performed on the IEEE EU LV Test feeder and a set of real UK LV network models that were provided by SSEN.
Comparisons were made of the standard AIMD algorithm with a fixed voltage threshold against the proposed AIMD+ algorithm with its individualised control.
A set of European demand profiles and a realistic EV travel model were used to feed load data into the simulations.
For all conducted simulations, network performance was improved by using the AIMD and AIMD+ algorithm in the distributed BESS.
However, AIMD+ frequently outperformed the traditional AIMD control.
More specifically, although the improved algorithm only resulted in a comparable reduction of voltage variation, it did outperform AIMD when decreasing line utilisation, thus reducing the frequency of line overloads.
Additionally, the same algorithm equalised the cycling and utilisation of battery energy storage to make better use of the deployed battery assets.
Despite being data driven, all findings indicate a similar improvement in performance when extending the traditional AIMD based control to AIMD+.
As a result, \ref{objective-4} (which was outlined in Section~\ref{ch-introduction:sec:problem-statement} of this thesis) has been met since all the aforementioned benefits were achieved without the use of any ICT infrastructure whilst satisfying all assumptions that were outlined in Section~\ref{ch4:subsec:assumptions}.

\chapter{Conclusion}
\label{ch-conclusion}

Throughout this thesis, aspects concerning the scheduling and control of energy storage devices on the LV network for system support have been studied.
More specifically, each chapter has presented its own findings and conclusions, and thus this thesis presents several contributions to knowledge.
Therefore, this chapter, Chapter~\ref{ch-conclusion}, is going to retrospectively summarise to those contributions to knowledge and link them back to the overarching problem statement presented in Section~\ref{ch-introduction:sec:problem-statement}.
At first, Chapter~\ref{ch-conclusion} summarises the main findings in Section~\ref{ch-conclusion:main-findings}.
Then, the contributions to knowledge are presented in Section~\ref{ch-conclusion:knowledge-contribution}, before the limitations of the conducted research and the potential future work are, respectively, being discussed in Section~\ref{ch-conclusion:research-limits} and Section~\ref{ch-conclusion:future-work}.

\section{Overview of Main Findings}
\label{ch-conclusions:sec:main-findings}

The problem statement of this thesis, which has been presented in Chapter~\ref{ch-introduction}, is summarised as follows:

\begin{itemize}
	\item
	The aim was to investigate how BESS in the LV network should be controlled in order to achieve best possible network support, including the reduction of peak load, voltage deviations and phase unbalance.
	\item 
	To assess the impact of BESS on the topology of the LV network, simulations were run to compare on-line and off-line control performance.
	\item
	Given that BESS operated flexibly but had a limited energy resource, and often had a predetermined half-hourly schedule, the presented research studied whether sub-half-hourly corrections could improve the performance of LV networks by incorporating load forecasts.
	\item
	Additionally the aim was extended to assess the effects of, and to develop algorithms for, desynchronised and communication less BESS control.
	This was done since distributed BESS (usually controlled using ICT) is expected to proliferate within the LV networks of the UK.
\end{itemize}

The reviewed literature in Chapter~\ref{ch-literature} as well as the findings from Chapter~\ref{ch1} emphasised the need for improving methods of control for energy storage in the LV network.
In Chapter~\ref{ch1} a set of key network parameters were introduced to highlight the breadth of possible network improvement functions.
Using the LV connected BESS, its impact on each of these key network parameters was assessed by optimising each parameter through its corresponding cost function.
The same BESS would have been operated traditionally with a half-hourly schedule that dictates the active powers of the device.
Using this operation as a benchmark, sub-half-hourly phasor adjustments were proposed to tune the BESS operation to achieve optimal impact for any given key network parameter, but without violating the higher resolution power constraints.
As shown in several resulting time-series plots in Section~\ref{ch1:sec:results-and-discussion} that were summarised in Table~\ref{ch1:tab:cost-table}, optimising BESS operation for certain key network parameters had two resulting impacts:

\begin{enumerate}
	\item The minimisation of a cost function, derived from a specific key network parameter, results in BESS operation  improving the associated network operation. For example, when minimising the cost that was linked to distribution losses, then a mean reduction in losses of 5.0kWh was achieved instead of a 1.2kWh reduction which would have been the result for traditional BESS scheduling.
	\item The minimisation of a cost function, derived from a specific key network parameter, also results in BESS operation that impacts other parameters, indirectly associated to the same network operation. For example, when minimising the cost that was linked to BESS voltage deviation, then the worst voltage deviation, the worst line loadings and the network's neutral currents were also reduced, but voltage deviation, line loadings and power factor at substation level were worsened.
\end{enumerate}

In Chapter~\ref{ch1} it was shown that this second impact is not necessarily positive.
Instead, a statistically significant positive impact (i.e. where $p<0.05$) was proven for only certain pairs of network parameters (for example and as stated earlier, maximum bus voltage deviation, phase unbalance and neutral power when minimising voltage deviation at ESMU level).
Nonetheless, showing that sub-half-hourly phase power adjustments can result in improved network operation formed the basis for the next chapter, Chapter~\ref{ch2}, where the half-hourly active power constraints were eliminated.

Chapter~\ref{ch2} presented a novel approach in combining both on-line and off-line energy storage control to dynamically reduce both daily and minutely load peaks.
An average peak load reduction of 5kW was achieved for the best algorithm configuration without reaching a surplus or shortage of stored energy since a half-hourly BESS schedule (similar to Chapter~\ref{ch1}) was followed.
Unlike the preceding chapter however, the BESS control in Chapter~\ref{ch2} had operational flexibility within a certain tolerance band of 10\% around its predetermined half-hourly schedule.
Combined with a predictor to estimate the sub-half-hourly power volatility, results were achieved that noticeably reduced load peaks in comparison to the traditional forecast driven control.
In fact, as shown in Figure~\ref{ch2:fig:peak-diff-pdf}, the mean peak load reduction increased from 1.7kW to around 5kW for different prediction mechanisms.
These findings from Chapter~\ref{ch1} and Chapter~\ref{ch2} thus form the contribution to knowledge regarding \ref{objective-1} and \ref{objective-2}, respectively.

The findings in Chapter~\ref{ch1} and Chapter~\ref{ch2} did not take into consideration any issues regarding the inter-device communication.
For example, they simply used all network information when computing BESS control instructions without considering possible latency issues.
Chapter~\ref{ch3} therefore developed a new smart-charging algorithm and used a novel MAS implementation that operated in an intentionally desynchronised manner.
This desynchronisation was to assess the algorithm performance when the previously assumed communication infrastructure becomes less reliable.
Since uncoordinated EV charging is expected to put the most significant load on the LV network, any algorithm failure (like failure to coordinate this charging) would become noticeable.
And indeed, the results in Chapter~\ref{ch3} showed that the converging behaviour of the algorithm became less sensitive to its control parameters in the desynchronised environment, when compared to the traditional synchronised algorithm execution.
For example, when extreme control parameter values were chosen, an oscillating behaviour was observed for the synchronised case which lead to the repetitive allocation of a 200kW charging spike.
However, this oscillating behaviour disappeared in the desynchronised case which meant that the algorithm converged on a global level.
In this desynchronised case the algorithm's performance and convergence became less sensitive to the choice of control parameter values.
This fact became particularly apparent when the overall performance of avoiding charging peaks between the synchronised case (i.e. Figure~\ref{ch3:fig:all-sync}) and the fully desynchronised case (i.e. Figure~\ref{ch3:fig:all-async-irregular}) was compared.
Chapter~\ref{ch3} therefore achieved \ref{objective-3} by developing a robust smart-charging algorithm that is thoroughly assessed in regards to possible communication desynchronisation.

From the lessons learnt in Chapter~\ref{ch3} and to circumvent the need for a communication infrastructure altogether, Chapter~\ref{ch4} proposed a communication-less control method for distributed BESS to reduce peak load, voltage deviation and unequal asset utilisation.
This communication-less control was achieved by using individualised control parameters in a modified AIMD algorithm.
Dynamic loads (i.e. uncoordinated EVs) were co-located to BESS in order to maximise the stress on the LV network that the developed control algorithm had to mitigate.
The results showed that for different EV uptake levels, BESS could always yield improvements for both AIMD and the proposed AIMD+ control methods.
However, as seen in Figure~\ref{ch4:fig:storage-aimd}, only the latter method did compensate uniformly across the LV network since it took into account the network specific voltage characteristics like the voltage drop along the feeder.
Therefore, these findings formed the contribution to knowledge regarding \ref{objective-4}.

The research over these four chapters has shown that energy storage algorithms can be improved by merging on-line and off-line control at high and low temporal resolution.
Additionally, the research has shown that desynchronised control instructions can yield significantly different operation of otherwise synchronised control algorithms.
However, this issue can be avoided when mitigating the need for communication technology altogether.
In each chapter, this thesis comprehensively tested the presented control algorithms on real demand data, allowing it to encapsulate varying demand behaviour and characteristics at both high and low temporal resolutions.
All findings were generated from the available datasets and were therefore subject to its properties of comprehensively capturing typical demand behaviours.
Despite potential limitations in using said datasets, the overarching finding from Chapter~\ref{ch1} to Chapter~\ref{ch4} is, that there is currently no single control algorithm that consistently outperforms all proposed aspects of the covered research.
However, results showed that focused control can be tuned to achieve a significantly higher positive impact on a narrow set of key parameters, which is why the chapters that implement such methods did also present the means of implementing their control in regards to the available data (thus achieving the subsequent network improvements which were derived from data driven simulations).
All objectives that were set out in the problem statement of this thesis have been met by making the aforementioned key contributions.
One can therefore conclude that the research presented in this thesis is beneficial to both industry and the academic research community.
These contributions to knowledge, possible research limitations and future work discussing these benefits, are outlined in the subsequent sections.


\section{Knowledge Contribution}
\label{ch-conclusion:knowledge-contribution}

In Chapter~\ref{ch-literature}, the literature is reviewed that surrounds the current and present control methods of DNO owned storage devices on the LV network.
This literature review supports the thesis problem statement in Section~\ref{ch-introduction:sec:problem-statement} since it concluded with the identified gaps in literature in which further investigation and research was deemed necessary and beneficial for both the industry and academic research community.
All chapters that are presented in this thesis make contributions within these identified gaps, and these contributions are summarised here:

\begin{itemize}
	\item
	In accordance with \ref{objective-1}, a closed-loop phasor adjustment method is presented to control DNO owned BESS to maximise its beneficial impact on key network parameters of the LV network.
	Findings in Chapter~\ref{ch1} show how issues including e.g. voltage deviation, neutral currents, phase unbalance and losses can be individually reduced when adjusting BESS operation at a sub-half-hourly resolution, even when the device is constrained by an active power schedule at half-hourly time scale.
	However, this constraint still imposes limitations to the otherwise flexible BESS operation, but it also shows the benefits that can be achieved despite this constraint.
	\item
	In accordance with \ref{objective-2}, a dynamic schedule correcting BESS control method is presented to control DNO owned BESS to maximise its capabilities at reducing both daily demand peaks, i.e. at half-hourly resolution, whilst also mitigating volatile load peaks, i.e. at sub-half-hourly resolution.
	Findings in Chapter~\ref{ch2} show how the control method outperforms traditional BESS control and how the probability of reducing peak load can be noticeably increased.
	However, to achieve this improvement, the implicit assumption of an ICT infrastructure could limited algorithm deployability when distributed across multiple devices.
	\item
	In accordance with \ref{objective-3}, a smart-charging algorithm for distributed control of an EV fleet was developed and deployed on a standardised MAS, which was desynchronised to assess the algorithm's performance.
	Findings in Chapter~\ref{ch3} show that the execution of the algorithm becomes less dependent on the underlying control parameters executed in a desynchronised environment, yet the overall performance of the algorithm remains intact.
	However, mitigating the need for ICT altogether would not only circumvent the issue of potential desynchronisation, but it would also lower deployment requirements and system cost.
	\item
	In accordance with \ref{objective-4}, a communication-less control method for distributed BESS was developed to assess its ability at reducing the negative impact from the charging of co-located EVs.
	Findings in Chapter~\ref{ch4} show that the developed AIMD+ algorithm does not only reduce peak loads or voltage deviation, but it also equalises the asset utilisation across the entire feeder.
	However, without any communication infrastructure, the performance of the proposed algorithm may be unable to address issues like e.g. phase unbalance, that have already been assessed in Chapter~\ref{ch1}.
\end{itemize}









\section{Research Limits}
\label{ch-conclusion:research-limits}



\section{Future Work}
\label{ch-conclusions:sec:future-work}

\hladd{The models, algorithms and control methods presented in this thesis provide a first step of assessing the impact of BESS on LV distribution networks.
Regarding the limitations outlined in Section~}\ref{ch-conclusions:sec:research-limits}\hladd{ it is worth considering possible steps to improve the models from an academic point of view and eventually make them ``industry ready''.
These two considerations are discussed in the subsequent subchapters before concluding on the entire work presented in this thesis.}

\subsection{Modelling}

\hladd{As mentioned before, the used data driven models have their limitations and it is worth considering the possible next steps to bring them closer to reality.
The developed BESS model for example did perform well enough to accurately schedule the ESMU operation during the field trials of the \textit{NTVV} project.
However, there are still certain enhancements that may be considered in future research to extend this BESS model.
The non-linearity regarding the battery's charging behaviour for instance may be included since more battery characteristics have been discovered over the recent past.
Characteristics regarding constant-current constant-voltage charging paradigms, temperature and battery condition have become better understood.
Also, from the preliminary lessons learnt during the field trials, safety mechanisms that operate independent of the BESS instructions (i.e. instructions sent from a control centre) might be included into the model.
After all, it was found that those safety mechanisms limited BESS operation without any warning.
Predicting when these mechanisms may activate would allow a more schedule generation.
The data collected during this field trial is therefore a great starting point to continue the research and better the BESS model.
In future work, such data could also aid the simulation battery control algorithms, since could also take into account the battery conditions itself to maximise its lifetime.
However, since the research aim of this thesis was on the development of BESS control methods and not on the development of better BESS models, this endeavour lies outside the research scope.}

\hladd{Equally, the EV model where it is assumed that vehicles charge at home and begin their charging process immediately after being connected may be considered already outdated.
This so called ``dumb-charging'' may still be seen as the baseline when assessing the impact of EVs on power networks, but more sophisticated and coordinated charging mechanisms (like the one hinted at in Chapter~}\ref{ch3}\hladd{) would certainly mitigate the EV impact.
Therefore, the inclusion of smart charging is seen as a future work, since the implementation, validation and extension of available smart-charging schemes currently lies beyond the scope of this thesis.}

\hladd{Regarding the use of network models to assess the impact on the LV network, and maybe even use real-time simulations to support control, is also a hot topic to consider as future work.
After all, power flow solvers like OpenDSS, PowerSim, GridLab-D or similar tools are frequently used throughout literature to create accurate network assessments.
Having learnt the standardised network model structure that is specified by the IEEE will allow any future work to utilise the updated collection of network models in an IEEE compliant manner.
Such compliance would not only allow a better comparative assessment of network power flow solutions, but also of network failures, harmonic studies and islanded operation - previously this would not have easily been possible.
Furthermore, with improved and updated network information, simulation tools would also give more accurate results (for example regarding the location and scaling of LV assets).}

\hladd{In fact, private correspondence with employers at the Electric Power Research Institute (EPRI) and the National Energy Research Laboratory (NERL) in the United States have become possible as a result of this doctoral research and a cooperative development to extend OpenDSS for multiple platforms is currently ongoing.
This cooperation entails the acceleration of simulations of network models by parallelising the power flow solver and porting the execution to different programming languages.
Therefore the possibility of extending the number of simulated trials would better the certainty of performance before committing to field trials.
However, for the scope of the conducted research, the number and variety of network models is sufficient.
After all the industrial partner (i.e. SSEN) assured the model's accuracy and ability to qualify as ``typical UK distribution networks''.}

\subsection{Realisation considerations for DNO}

\hladd{Beside the already mentioned safety aspect and improved modelling to establish a foundation for field trials, DNOs need to consider aspects of ownership, data privacy and data security, too.
Who will own the BESS and operate it is a frequently discussed issue.
After all, if the DNO owned the BESS, they would most likely want it to operate at maximum power to fully utilise the asset and get the largest return for their investment.
If BESS was owned by private households on the other hand, a reduction of the electricity bill would be more welcome than providing network support.
Establishing means to compensate or pay private owners for providing network support or using variable energy prices is considered as an incentive so that private owners partake in network support.
National (or international) legislation to enable such transactions is still missing, but would certainly provide a framework for cooperation between DNOs and households.
Also, restrictions to assure that the scale of network support assets does not exceed the requirements would need to be established, too.
In order to address this issue, DNOs may want to enquire how much households are willing to invest and get in return when providing network support using BESS.
It would also be interesting to know whether network-independence or a lower carbon footprint is a better incentive to enter the network support market.}

\hladd{Although the topic of data privacy and security is not part of the presented research as such, it is still worth discussing possible considerations in light of era of big data with Industry 4.0 and the Internet of Things.
The field of computer science being at the forefront of (especially by using example artificial intelligence - AI) evaluating big data has lead to the result that AI is also being used for controlling power network (at least in simulations).
On may even say that this trend of computer based decision making and controlling is becoming increasingly popular in the area of power delivery.
With the ever increasing abilities of computers, AI based control approaches are likely to outperform traditional deterministic and probabilistic methods.
It is therefore worth stating that future work is very likely to include an AI aspects for controlling a network support asset.}

\hladd{However, using large datasets of their customers requires DNOs to assure privacy of said customers.
Although the rollout of smart-meters in the UK was envisioned to supply DNOs with more feedback data regarding customer behaviour, using this data correctly is a challenging topic, too.
This is not only the case for privacy but also safety, especially when considering the issue of desynchronising message propagation as presented in this thesis.
A new inter-disciplinary research category may potentially focus on power delivery, control systems and telecommunication issues particularly targeted at AI in power networks.
However, these areas lie outside the scope of the work that was presented in this thesis.}



%next line adds the Bibliography to the contents page
\addcontentsline{toc}{chapter}{Bibliography}
%uncomment next line to change bibliography name to references
%\renewcommand{\bibname}{References}
\bibliography{../Papers/Bibliography/library}        %use a bibtex bibliography file refs.bib
\bibliographystyle{ieeetr}  %use the plain bibliography style


%now enable appendix numbering format and include any appendices
\appendix
\chapter{Additional Results}
\label{appx-a:additional-results}

\section{Improving operation performance of battery schedules at sub-half-hourly resolution}
\label{appx-a:ch1}

\begin{figure}\centering
\subfloat[Voltage levels as measured at the substation]{\includegraphics{_chapter1/fig/appendix/ts-substation-voltages___}}\\	
\subfloat[Cost associated with the voltage levels as measured at the substation]{\includegraphics{_chapter1/fig/appendix/ts-substation-voltages__}}
\caption{Additional substation voltage level comparison between base, normal and the case where the ESMU's schedule was adjusted.}
\end{figure}

\begin{figure}\centering
\subfloat[ESMU voltage levels]{\includegraphics{_chapter1/fig/appendix/ts-esmu-voltages___}}\\	
\subfloat[Cost associated with the ESMU voltage levels]{\includegraphics{_chapter1/fig/appendix/ts-esmu-voltages__}}
\caption{Additional ESMU voltage level comparison between base, normal and the case where the ESMU's schedule was adjusted.}
\end{figure}

\begin{figure}\centering
\subfloat[Highest and lowest voltage levels in entire network]{\includegraphics{_chapter1/fig/appendix/ts-all-voltages___}}\\	
\subfloat[Cost associated with highest and lowest voltage levels in entire network]{\includegraphics{_chapter1/fig/appendix/ts-all-voltages__}}
\caption{Additional voltage level comparison between base, normal and the case where the ESMU's schedule was adjusted.}
\end{figure}

\begin{figure}\centering
\subfloat[Highest and lowest phase power]{\includegraphics{_chapter1/fig/appendix/ts-phase-unbalance___}}\\	
\subfloat[Phase unbalance cost]{\includegraphics{_chapter1/fig/appendix/ts-phase-unbalance__}}
\caption{Additional phase unbalance cost comparison between base, normal and the case where the ESMU's schedule was adjusted.}
\end{figure}

\begin{figure}\centering
\subfloat[Network load]{\includegraphics{_chapter1/fig/appendix/ts-power-factor___}}\\	
\subfloat[Power factor]{\includegraphics{_chapter1/fig/appendix/ts-power-factor__}}
\caption{Additional power factor cost comparison between base, normal and the case where the ESMU's schedule was adjusted.}
\end{figure}

\begin{figure}\centering
\subfloat[Utilisation of the substation fuse]{\includegraphics{_chapter1/fig/appendix/ts-fuse-utilisation___}}\\	
\subfloat[Cost associated with the utilisation of the substation fuse]{\includegraphics{_chapter1/fig/appendix/ts-fuse-utilisation__}}
\caption{Additional comparison of the substation fuse utilisation between base, normal and the case where the ESMU's schedule was adjusted.}
\end{figure}

\begin{figure}\centering
\subfloat[The highest line utilisation of any line in the entire network]{\includegraphics{_chapter1/fig/appendix/ts-all-line-utilisation___}}\\	
\subfloat[The highest cost associated to the highest line utilisation of any line in the entire network]{\includegraphics{_chapter1/fig/appendix/ts-all-line-utilisation__}}
\caption{Additional line utilisation comparison between base, normal and the case where the ESMU's schedule was adjusted.}
\end{figure}

\begin{figure}\centering
\subfloat[Distribution losses]{\includegraphics{_chapter1/fig/appendix/ts-losses___}}\\	
\subfloat[Cost associated to distribution losses]{\includegraphics{_chapter1/fig/appendix/ts-losses__}}
\caption{Additional comparison of distribution loss cost between base, normal and the case where the ESMU's schedule was adjusted.}
\end{figure}
\newpage

\begin{figure}\centering
	\includegraphics{_chapter1/fig/appendix/minimising-substation-voltage-deviation}
	\caption{Cost difference spread, based on the ESMU schedule adjustment to minimise substation voltage deviation}
\end{figure}

\begin{figure}\centering
	\includegraphics{_chapter1/fig/appendix/minimising-battery-voltage-deviation}
	\caption{Cost difference spread, based on the ESMU schedule adjustment to minimise ESMU's PCC voltage deviation}
\end{figure}

\begin{figure}\centering
	\includegraphics{_chapter1/fig/appendix/minimising-maximum-voltage-deviation}
	\caption{Cost difference spread, based on the ESMU schedule adjustment to minimise the maximum voltage deviation on any bus of the network}
\end{figure}

\begin{figure}\centering
	\includegraphics{_chapter1/fig/appendix/minimising-phase-unbalance}
	\caption{Cost difference spread, based on the ESMU schedule adjustment to minimise the network's phase unbalance}
\end{figure}

\begin{figure}\centering
	\includegraphics{_chapter1/fig/appendix/minimising-neutral-power}
	\caption{Cost difference spread, based on the ESMU schedule adjustment to minimise the network's power flow in the neutral conductor}
\end{figure}

\begin{figure}\centering
	\includegraphics{_chapter1/fig/appendix/minimising-power-factor}
	\caption{Cost difference spread, based on the ESMU schedule adjustment to minimise the network's offset to unity power factor}
\end{figure}

\begin{figure}\centering
	\includegraphics{_chapter1/fig/appendix/minimising-substation-fuse-loading}
	\caption{Cost difference spread, based on the ESMU schedule adjustment to minimise the substation's fuse utilisation}
\end{figure}

\begin{figure}\centering
	\includegraphics{_chapter1/fig/appendix/minimising-maximum-line-loading}
	\caption{Cost difference spread, based on the ESMU schedule adjustment to minimise the maximum line utilisation of any line in the network}
\end{figure}

\begin{figure}\centering
	\includegraphics{_chapter1/fig/appendix/minimising-losses}
	\caption{Cost difference spread, based on the ESMU schedule adjustment to minimise distribution losses}
\end{figure}
\newpage

\subsection{Probability Density Analysis}

The details described in this section address the prerequisites for the performed null hypothesis test in Section \ref{ch1:subsec:probability-density-analysis}.
These steps were beyond the content of the corresponding chapter, yet have been included for completeness sake.
Here, each step to condition the raw data for the $t$-test is explained in detail.

The original data is a highly volatile, non-stationary and has a non-gaussian probability distribution.

\newpage
\chapter{Multi-Agent Systems}
\label{appx-b:multi-agent-systems}

This appendix, Appendix~\ref{appx-b:multi-agent-systems}, presents additional details on the MAS implementation.
More specifically, the method used to implement FIPA are presented, and the main communication protocols that war used within this method are detailed.

\section{FIPA Implementation}

\nomenclature[G]{JADE}{JAVA Agent Communication Language}
\nomenclature[G]{ACL}{Agent Communication Language}

The Foundation for Intelligent Physical Agents (FIPA) has established a standard set of protocols that allow agents to interact with each other.
These protocols form the so called Agent Communication Language (ACL)
Telecom Italia has successively begun to develop a JAVA Agent Development Framework (JADE) that puts the entire ACL at the programmer's disposal.
Published under LGPL (i.e. the Lesser General Public License Version 2), JADE is a free software package that can easily be used to construct large MASs.

In order to perform optimisation functions however, a way to interact with OpenDSS was required.
On Microsoft Windows, the ActiveX COM server provided a simple access point to MATLAB and OpenDSS specific functions, and the JAVA COM Bridge (JACOB) made this server accessible to the JAVA run-time environment.

JADE and JACOB were, respectively, obtained from the following two sources:

\begin{itemize}
	\item JADE: \textit{http://jade.tilab.com}
	\item JACOB: \textit{https://sourceforge.net/projects/jacob-project/}
\end{itemize}

By including the \textit{jade.jar} and \textit{jacob.jar}, and the corresponding Dynamic Linked Libraries (DLLs) \textit{jacob-1.8-M2-x86.dll} and  \textit{jacob-1.8-M2-x64.dll}, FIPA was fully implemented and linked to MATLAB and OpenDSS.

\section{Communication Protocols}

The main three protocols that were used within Chapter~\ref{ch3} are:

\begin{enumerate}
	\item FIPA Query Protocol (FIPA-standard-SC00027H)
	\item FIPA Brokering Protocol (FIPA-standard-SC00033H)
	\item FIPA ContractNet Protocol (FIPA-standard-SC00029H)
\end{enumerate}

The flowcharts for these three protocols were taken from the corresponding standards and, for completeness, are explained in the following three subsections.

\subsection{FIPA Query Protocol}
\label{appx-b:subsec:fipa-query-protocol}

\begin{figure}\centering
	\includegraphics[width=0.5\linewidth]{_appendices/_a2/fig/fipa-query}
	\caption{FIPA Query Protocol flow chart}
	\label{appx-b:fig:fipa-query}
\end{figure}

Figure~\ref{appx-b:fig:fipa-query} shows the complete flow chart of the FIPA Query protocol.
This protocol is initiated by an ``\textit{initiator}'' that send a ``\textit{query}'' message (either ``if'' or ``reference'' message) to a ``\textit{Participant}''.
In Chapter~\ref{ch3}, the initiators were the brokering agents of the loads and the participant were the brokering agents of the energy supplier.
The \textit{participant} replies either with an ``\textit{agree}'' to inform the \textit{initiator} that the query is received, or a ``\textit{refuse}'' message is sent to terminate the communication.
After an \textit{agree} message, the \textit{participant} sends the required information in an ``\textit{inform}'' message (as a reply to the ``if'' or ``reference'' query), or a ``\textit{failure}'' is sent when no data is available.
In Chapter~\ref{ch3}, the data that is sent in the \textit{inform} messages includes the daily load profile onto which the EV agents should superimpose their charging demand.

\subsection{FIPA Brokering protocol}
\label{appx-b:subsec:fipa-brokering-protocol}

\begin{figure}\centering
	\includegraphics[width=0.66\linewidth]{_appendices/_a2/fig/fipa-brokering}
	\caption{FIPA Brokering Protocol flow chart}
	\label{appx-b:fig:fipa-brokering}
\end{figure}

The brokering protocol, as shown in Figure~\ref{appx-b:fig:fipa-brokering}, is used to delegate an agents task to a different agent in order to free up its own computational resources.
In Chapter~\ref{ch3} for instance, the load agents never communicate with the energy supplier directly, since the applying and undoing of power profiles is delegated to their brokering agents.
The protocol is initiated by assigning a broker to a load agent by sending a ``\textit{proxy}'' message.
This message contains the required information for the broker, like the power profile a buying broker should apply.
If the broker can fulfil this request, then an ``\textit{agree}'' message is sent, otherwise a ``\textit{refuse}'' message is sent.
The broker uses the FIPA Query protocol, as explained in Section~\ref{appx-b:subsec:fipa-query-protocol}, to obtain a list of broker agents that are linked to energy suppliers, which can be used to apply the load's demand profile.
However, if no such broker is found, then a ``\textit{failure}'' (i.e. ``no match'') message is sent.
Alternatively, the broker begins its delegating task and it forwards the requested demand profile to the corresponding energy supplier (i.e. it uses the FIPA ContractNet protocol as outlined in the next section, Section~\ref{appx-b:subsec:fipa-contract-net-protocol}).
If an error occurs during this delegating process, then a ``\textit{failure}'' message is sent (i.e. ``proxy failure'' or ``inform falure'').
Upon successful delegation, the broker replies to the ``\textit{Initiator}'' with a ``\textit{reply}'' message that contains information about the applied demand profile.
Theoretically, this information can also contain pricing information, yet this feature was disregarded since it lies outside the scope of this thesis.

\subsection{FIPA ContractNet Protocol}
\label{appx-b:subsec:fipa-contract-net-protocol}

\begin{figure}\centering
	\includegraphics[width=0.5\linewidth]{_appendices/_a2/fig/fipa-contract-net}
	\caption{FIPA ContractNet Protocol flow chart}
	\label{appx-b:fig:fipa-contract-net}
\end{figure}

Figure~\ref{appx-b:fig:fipa-contract-net} shows the FIPA ContractNet Protocol that allows an agent to negotiate a binding contract.
After executing this dual handshake protocol, all contract participants are informed about the final contract decision and no information is lost during the message exchange.
The protocol is initiated by an ``\textit{Initiator}'', who sends a ``\textit{Call For Proposal}'' (\textit{cfp}) to $m$ ``\textit{Participant}s''.
This \textit{cfp} contains a deadline within which all agents that do want to participate should reply.
They can reject their participation by sending a ``\textit{refuse}'' message, or acknowledge their participation by sending a ``\textit{propose}'' message that also contains proposition information (for example pricing information).
Once all participants have replied or the deadline has expired, the \textit{initiator} continues executing.
It collects and assesses all proposals, choses the accepted and rejected ones and, respectively, issues ``\textit{accept}'' and ``\textit{reject}'' notifications.
The participating agents reply with an ``\textit{inform}'' message if they acknowledge the ``accept`` or ``reject'' message, and in case of an error, they reply with a ``failure'' message.







\chapter{Stochastic EV Demand Model}
\chapter{Network Simulation Interface}

\section{OpenDSS}

\section{Java}

\section{MATLAB}

\section{Python}
\end{document}
