\section{System Modelling}
\label{ch4:sec:system-modelling}

In this section, the underlying assumptions to validate the research are addressed. Next, a model to describe EV charging behaviour is explained. This is followed by a model of the BESS. Finally, the network models used to simulate the power distribution networks are explained.

\subsection{Assumptions}
\label{ch4:subsec:assumptions}

For this work, several underlying assumption were made to obtain the models:
\begin{enumerate}[leftmargin=*,labelsep=5mm]
\item The uptake of EVs is assumed to increase and, hence, to have a significant impact on the normal operation of the low voltage distribution network. This assumption is based on a well-established prediction that the majority of EV charging will take place at home \cite{Munkhammar2015a}.
\item The transition from internal combustion engine-powered vehicles to EVs is assumed to not impact the users' driving behaviour. Similar to \cite{Dallinger2012}, this assumption allows the utilisation of recent vehicle mobility data \cite{MiD2008} to generate leaving, driving and arriving probabilities, from which the EV charging demand can be determined.
\item The transition to low carbon technologies will increase the variability of electricity demand, and therefore, grid-supporting devices, such as BESS, are anticipated to play a more important role \cite{FES2015}. Hence, alongside a high uptake of EVs, an increased adoption of distributed BESS devices is assumed.
\item It is assumed that BESS solutions, or more specifically battery energy storage solutions, start the simulations at 50\% SOC and are not 100\% efficient at storing and releasing electrical energy, as in \cite{Rowe2014a}. Additionally, its utilisation will degrade the energy storage capability and performance over time, as shown in \cite{Laresgoiti2015}. Therefore, the requirements for equal and fair storage usage is of high importance.
\item It is assumed that the load profiles provided by the IEEE Power and Energy Society (PES) are sufficient as base load profiles for all simulations.
\end{enumerate}

\subsection{EV charging behaviour}
\label{ch4:subsec:ev-charging-behaviour}

From publicly-available car mobility data \cite{Dallinger2012, MiD2008} an empirical model was developed to capture the underlying driving behaviour. The raw data, $n_{r}(t)$, represents the probabilities of starting a trip during a 15-min period of a weekday. Three continuous normal distribution functions, each defined as:
\begin{equation}
\hat{n}_x(t) = \beta_x\frac{1}{\sigma_x\sqrt{2\pi}} \exp\left[-\frac{\left(^t/_{24}-\mu_x\right)^2}{2\sigma_x^2}\right] \;\text{where}\; t = [0, 24]
\end{equation}
were used to represent vehicles leaving in the morning, $\hat{n}_{m}(t)$, lunch time, $\hat{n}_{l}(t)$, and in the evening, $\hat{n}_{e}(t)$. The aggregate probability of these three functions was optimised using a Generalised Reduced Gradient (GRG) algorithm to fit the original data. In order to represent a symmetric commuting behaviour, i.e., vehicles departing in the morning and returning during the evening, an equality amongst the three probabilities was defined as follows:
\begin{equation}
0 = \int_{0}^{24}\left[\hat{n}_m(t) + \hat{n}_l(t) - \hat{n}_e(t)\right]dt
\end{equation}

The resulting parameters from the GRG fitting of the three distribution functions are tabulated in Table~\ref{table-starting-a-trip-probability}. Additionally, the resulting departure probabilities, as well as the reference data $n_r(t)$ are shown in Figure~\ref{fig-starting-a-trip-probability}.

\begin{table}\centering
\begin{tabular}{cccc}
\hline
\textbf{Equation} \boldmath{$\hat{n}_x(t)$} & \boldmath{$\mu_x$} \textbf{(Mean)} & \boldmath{$\sigma_x$} \textbf{(SD)} & \boldmath{$\beta_x$} \textbf{(Weight)} \\
\hline
$\hat{n}_m(t)$ & 0.3049 & 0.0488 & 0.00206 \\
$\hat{n}_l(t)$ & 0.4666 & 0.0829 & 0.00314 \\
$\hat{n}_e(t)$ & 0.7042 & 0.0970 & 0.00521\\
\hline
\end{tabular}
\caption{Parameters for normal distributions.}
\label{table-starting-a-trip-probability}
\end{table}\vspace{-6pt}


\begin{figure}\centering
 \includegraphics[width=0.8\textwidth]{foo}
 \caption{The probability of starting a trip at a particular time during a weekday, extrapolated into three normal distributions (RMS error: $9.482\%$).}
 \label{fig-starting-a-trip-probability}
\end{figure}

Statistical data capturing the probability distribution of a trip being of a certain distance were also extracted from the dataset. This was done for both the weekdays $w_{wd}(d)$ and weekends $w_{we}(d)$. The Weibull function was chosen to be fitted against the extracted probability distributions and is defined as:

\begin{equation}
\hat{w}_x(d) :=
	\begin{cases}
		\frac{k_x}{\gamma_x}\left(\frac{d}{\gamma_x}\right)^{k_x-1}\exp\left[-\left(\frac{d}{\gamma_x}\right)^{k_x}\right] &\text{if } d \geq 0 \\
		0 &\text{if } d < 0
	\end{cases}
\end{equation}

Performing the curve fitting using the GRG optimisation algorithm, a weekday trip distance distribution, $\hat{w}_{wd}(d)$, and a weekend trip distribution, $\hat{w}_{we}(d)$, could be estimated. The computed function parameters for these two estimated distribution functions are tabulated in Table~\ref{table-trip-distance-probailility}. Their resulting probability distributions are plotted for comparison against the real data, $w_{wd}(d)$ and $w_{we}(d)$, in Figure~\ref{fig-trip-distance-probability}.\\

\begin{table}\centering 
\begin{tabular}{ccc}
\hline
\textbf{Equation} \boldmath{$\hat{w}_x(d)$} & \boldmath{$\gamma_x$} \textbf{(Scale)} & \boldmath{$k_x$} \textbf{(Shape)} \\
\hline
$\hat{w}_{wd}(t)$ & 15.462 & 0.6182 \\
$\hat{w}_{we}(t)$ & 38.406 & 0.4653\\
\hline
\end{tabular}
\caption{Parameters for Weibull distributions.}
\label{table-trip-distance-probailility}
\end{table}\vspace{-6pt}


\begin{figure}\centering
 \includegraphics[width=0.8\textwidth]{foo}
 \caption{The probability of a trip being of a particular distance during a weekday, extrapolated into a Weibull distribution (RMS error: $3.791\%$).}
 \label{fig-trip-distance-probability}
\end{figure}
 
In addition to these probabilities, an average driving speed of 56 kmh (35 mph) and an average driving energy efficiency of 0.1305 kWh/kmh (0.21 kWh/mph) are taken from \cite{UKGovernmentDigitalService2013}. Using the predicted driving distance and average driving speed with the driving energy efficiency, it is possible to estimate an EV's energy demand upon arrival. Starting to charge from this arrival time until the energy demand has been met allows the generation of an estimated charging profile of a single EV. To do this, a maximum charging power of the U.K.'s average household circuit rating (i.e., 7.4 kW) and an immediate disconnection of the EV upon charge completion were assumed \cite{EVHomeCharging}.

Generating several of those charging profiles and aggregating them produces an estimated charging demand for an entire fleet of EVs. To provide an example, charge demand profiles for 50 EVs were generated, aggregated and plotted in Figure~\ref{fig-aggregated-ev-power}. This plot shows the expected magnitude and variability in energy demand that is required to charge several EVs at consumers' homes based on the vehicles' daily usage.

This model's EV charging behaviour has been implemented to reflect EV demand if applied today without widespread smart charging infrastructure. It does therefore reflect the worst case scenario. Future smart-charging schemes would mitigate the currently present collective EV charging spike, yet the implementation and validation of available smart-charging schemes lies beyond the scope of this paper. This model's data were used to feed additional demand into the power network models, which are outlined in the next section.

\begin{figure}\centering
 \includegraphics[width=0.8\textwidth]{foo}
 \caption{Excerpt from the aggregated 50 EVs; charging powers that were each generated from the empirical models.}
 \label{fig-aggregated-ev-power}
\end{figure}

\subsection{Battery Modelling}

For this work, a well-established model that has been used in previous publications by this research group was used \cite{Rowe2013, Rowe2014, Rowe2014a}. This model consists of a battery with a self-discharge loss that is dependent on the current battery's State Of Charge (SOC) and an energy conversion loss to represent the energy lost when charging or discharging this battery. 

%\begin{table}\centering
%\begin{tabular}{cl}%{r p{10cm} }
%{\bf Parameter} & {\bf Description} \\
%\hline
%ine
%$P_{bat}(t)$ & Battery power at time $t$\\
%$SOC(t)$ & Battery state of charge at time $t$\\
%$\delta_{SOC}(t)$ & Change in SOC during time period $\tau$\\
%$\mu$ & Self-discharge loss factor\\
%$\eta$ & Energy conversion efficiency\\
%$SOC_{min}$ & Minimum rated SOC for limited battery operation\\
%$SOC_{max}$ & Maximum rated SOC for limited battery operation\\
%$C$ & Battery capacity\\
%$P_{max}$ & Power rating of battery\\
%\end{tabular}
%\caption{Table of the notation used in this section.}% This table has not been referred to within the text of the manuscript.
%\label{table-notation-ev-model}
%\end{table}

When an ideal battery charges or discharges, the change in SOC is related by the battery power, $P_{bat}$. When sampling battery operation at a regular period, $\tau$, then the energy transferred into the battery can be described as $P_{bat}(t)\tau$. The change in SOC for this ideal battery, $\delta_{SOC}$, is therefore defined as:\vspace{6pt}
\begin{equation}
\delta_{SOC}(t) := \frac{P_{bat}(t)\tau}{C} = \text{SOC}(t) - \text{SOC}(t-\tau)
\end{equation}

The self-discharge loss is added to this ideal battery model to represent the continual loss of energy in the battery typical of chemical energy storage. This self-discharge loss, $\delta_{SOC,self\text{-}discharge}$, is proportional to the current SOC and is determined using the self-discharge loss factor, $\mu$:
\begin{equation}
	\delta_{SOC,self\text{-}discharge}(t) := \mu SOC(t)
\end{equation}

Additionally, to represent the losses in the power electronics and energy conversion process, an energy conversion loss, $\delta_{SOC,conversion}$, is defined. This loss is proportional to the rate at which the battery's SOC changes, by using the energy conversion efficiency, $\hat{\eta}$ as follows:
\begin{equation}
	\delta_{SOC,conversion}(t) := \hat{\eta}\delta_{SOC}(t)
\end{equation}

Here, the conversion losses in the power electronics are reflected as an asymmetric efficiency, which depends on the direction of the flow of energy. This is done by charging the battery at a lower power when consuming energy and discharging it more quickly when releasing energy. Mathematically, this can be represented as:
\begin{equation}
\hat{\eta} =
	\begin{cases}
		\eta &\text{if } \delta_{SOC}(t) \geq 0 \\
		\frac{1}{\eta} &\text{if } \delta_{SOC}(t) < 0
	\end{cases}
\end{equation}

When substituting the self-discharge loss and conversion losses, respectively $\delta_{SOC,self\text{-}discharge}$ and $\delta_{SOC,conversion}$, into the SOC evolution equation, the full battery model can be summarised as follows:
\begin{equation}
\begin{split}
	\text{SOC}(t) :&= \delta_{SOC}(t-\tau) - \delta_{SOC,self\text{-}discharge}(t-\tau) - \delta_{SOC,conversion}(t)\\
	&= (1-\mu)\delta_{SOC}(t-\tau) - \hat{\eta}\delta_{SOC}(t)	
\end{split}
\end{equation}

In addition, both the SOC and the $P_{bat}$ are constrained due to the device's maximum and minimum energy storage capabilities, respectively $SOC_{max}$ and $SOC_{min}$, and maximum charge and discharge rate, $P_{max}$. These limitations are captured in Equations (\ref{equ-SoC-range}) and (\ref{equ-charge-discharge-range}), respectively.
\begin{equation}
\text{SOC}_{min} \leq \text{SOC}(t) \leq \text{SOC}_{max}
\label{equ-SoC-range}
\end{equation}
\begin{equation}
\left|P_{bat}(t)\right| \leq \text{P}_{max}
\label{equ-charge-discharge-range}
\end{equation}

\subsection{Network Models}

To simulate the low-voltage energy distribution networks, the Open Distribution System Simulator (OpenDSS) developed by the Electronic Power Research Institute (EPRI) was used. It requires element-based network models, including line, load and transformer information, and generates realistic power flow results.

\begin{figure}\centering
 \subfloat[]{%
 \includegraphics[width=0.5\linewidth]{foo}%
 \label{fig-eu-lv-test-feeder}%
 }
 \subfloat[]{%
 \includegraphics[width=0.5\linewidth]{foo}%
 \label{fig-uk-feeder}%
 }
 \caption{Sample Open Distribution System Simulator (OpenDSS) power flow plots of the used power networks. Consumers are indicated as red crosses and 11/0.416-kV substations are marked with a green square. (\textbf{a}) IEEE Power and Energy Society (PES) EU Low Voltage Test Feeder plot; (\textbf{b}) Scottish and Southern Energy Power Distribution (SSE-PD) Common Information Model (CIM) (UK) feeder plot.}
 %There is no explanation for * and the green square in the capition.
 \label{fig-feeders}%please reference in the main text before presenting
\end{figure}

Simulations were conducted using the IEEE's European Low Voltage Test Feeder \cite{EULVFeeder2015} and six detailed U.K. feeder models, that are based on real power distribution networks and provided by Scottish and Southern Energy Power Distribution (SSE-PD). The SSE-PD circuit models were provided as Common Information Models (CIM) during the collaboration on the New Thames Valley Vision Project Project (NTVV) \cite{NTVV2016}. An example of the IEEE EU LV Test feeder and a U.K. feeder provided by SSE-PD are shown in Figure~\ref{fig-feeders}a and Figure~\ref{fig-feeders}b, respectively. A summary of these model's parameters is given in the Table~\ref{table-model-parameters}.

\begin{table}\centering
\begin{tabular}{lccccccc}%{r p{10cm} }
\multirow{2}{*}{\textbf{Parameter}} & \textbf{IEEE EU} & \multicolumn{6}{c}{\multirow{2}{*}{\textbf{SSE-PD LV Feeders}}}\\
 & \textbf{LV Test Feeder} & \\
\hline
Network number & 1 $^1$ & 2 $^1$ & 3 & 4 & 5 & 6 & 7\\
\hline
Number of customers & 55 & 56 & 53 & 91 & 59 & 88 & 37\\
\hline
Median load per customer (VA) & 227 & 227 & 231 & 241 & 224 & 237 & 237\\
\hline
Maximum load per customer (kVA)& 16.8 & 16.8 & 16.8 & 19.5 & 16.8 & 19.5 & 16.8\\
\hline
Customer connection & Single-phase & \multicolumn{6}{c}{Single-phase}\\
\hline
Median substation load (kVA)& 24.4 & 24.9 & 23.9 & 41.9 & 25.6 & 38.9 & 16.3\\
\hline
Maximum load per customer (kVA)& 72.6 & 72.7 & 72.2 & 92.9 & 73.5 & 89.6 & 60.5\\ 
\hline
\multirow{2}{*}{Feeder line model} & Three-phase & \multicolumn{6}{c}{Three-phase}\\
 & implicit-neutral & \multicolumn{6}{c}{explicit-neutral}\\
\end{tabular}
\caption{Network model parameters.}
\label{table-model-parameters}
\begin{tabular}{ccc}
\multicolumn{1}{c}{\footnotesize $^1$ These networks are shown in Figure~\ref{fig-feeders}.}
\end{tabular}
\end{table}

Throughout this paper, all excerpt and time series results were extracted from experiments with the IEEE EU LV Test feeder (i.e., Network No. 1). All concluding results are based on an aggregation of all networks to include network diversity in the analysis.

The model-derived EV data and IEEE EU LV Test feeder consumer demand profiles were used in all simulations. The resultant demand profiles represent the total daily electricity demand of households with EVs. These profiles were sampled at $\tau = 1\text{ min}$. The OpenDSS simulation environment was controlled using MATLAB, achieved through OpenDSS's Common Object Model (COM) interface and accessible using Microsoft's ActiveX server bridge.