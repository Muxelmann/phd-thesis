\section{Summary}
\label{ch4:sec:summary}

In this chapter, Chapter~\ref{ch4} an algorithm is proposed for distributed battery energy storage in order to mitigate the negative impact of highly variable and uncontrolled loads - such as the charging of EVs.
Unlike previous algorithms, the improved algorithm (i.e. AIMD+) only uses local bus voltage measurements to issue control instructions.
It implements a reference voltage profile which is derived from power flow analysis of the distribution network for its individualised set-point control.
Taking the distance to the feeding substation into account allows an optimisation of the algorithm's parameters for each BESS.
Simulations were performed on the IEEE EU LV Test feeder and a set of real UK LV network models that were provided by SSEN.
Comparisons were made of the standard AIMD algorithm with a fixed voltage threshold against the proposed AIMD+ algorithm with its individualised control.
A set of European demand profiles and a realistic EV travel model were used to feed load data into the simulations.
For all conducted simulations, network performance was improved by using the AIMD and AIMD+ algorithm in the distributed BESS.
However, AIMD+ frequently outperformed the traditional AIMD control.
Although the improved algorithm only resulted in a comparable reduction of voltage variation (for example voltage deviation was reduced by an average of 4\% when using AIMD and an average of 5\% when using AIMD+ in a scenario of 50\% BESS uptake), it did outperform AIMD when decreasing line utilisation, thus reducing the frequency of line overloads.
But the AIMD+ algorithm made better use of the deployed battery assets since it always reduced the corresponding metric (i.e. $\zeta^{**}_\textbf{D}$) by an additional 7.1\% in comparison to the AIMD algorithm's results (i.e. for $\zeta^{**}_\textbf{C}$).
Additionally, the same AIMD+ algorithm equalised the cycling and utilisation of battery energy storage to since the BESS located at the top of the feeder were also included into the distributed BESS operation.
Quantitatively, this was shown by a reduction in the PAR of battery cycling, where the AIMD+ algorithm outperformed the AIMD algorithm by an average of 1.3 for all storage uptake scenarios (i.e. when comparing $\zeta^{***}_\textbf{D}$ with $\zeta^{***}_\textbf{C}$).
Despite being data driven, all findings indicate a similar improvement in performance when extending the traditional AIMD based control to AIMD+.
As a result, \ref{objective-4} (which was outlined in Section~\ref{ch-introduction:sec:problem-statement} of this thesis) has been met since all the aforementioned benefits were achieved without the use of any ICT infrastructure whilst satisfying all assumptions that were outlined in Section~\ref{ch4:subsec:assumptions}.
