\section{Related Work}
\label{ch4:sec:related-work}

Existing literature addresses the usage of energy storage units in low-voltage distribution networks to assure voltage security \cite{Sugihara2013, Toledo2013, Marra2013, Mokhtari2013, Atia2016}. An approach used by, e.g., Mokhtari et al. in \cite{Mokhtari2013} relies on bus voltage and network load measurements to prevent system overloads. Yet, these kinds of storage control systems do require communication infrastructures to relay the network information and control instructions. This requirement has also been addressed in the comprehensive review on storage allocation and application methods by Hatziargyriou et al. \cite{Hatziargyriou2015}. In the presented work, a control algorithm is proposed that removes the need for such an inter-BESS communication, since it only uses local voltage measurements to infer the network operation. Yet, to prevent conflicting device behaviour, the underlying coordination mechanism is of particular importance. Assuring convergence, the AIMD algorithm is perfectly suited for such coordinated control.

Originally, AIMD algorithms were applied to congestion management in communications networks using the TCP protocol \cite{Chiu1989}, to maximise utilisation while ensuring a fair allocation of data throughput amongst a number of competing users \cite{Wirth2014}. AIMD-type algorithms have previously been applied to power sharing scenarios in low voltage distribution networks, where the limited resource is the availability of power from the substation's transformer.

For instance, such an algorithm was first proposed for EV charging by St{\"{u}}dli et al. \cite{Studli2012}, requiring a one-way communications infrastructure to broadcast a ``capacity event'' \cite{Studli2014, Studli2014a}. Later, their work was further developed to include vehicle-to-grid applications with reactive power support \cite {Studli2015}. The battery control algorithm proposed in this paper builds upon the algorithm used by Mareels et al. \cite{Mareels2014}, where EV charging was organised by including bidirectional power flow and the use of a reference voltage profile derived from network models. Similar to the work by Xia et al. \cite{Xia2014}, who utilised local voltage measurements to adjust the charging rate, only voltage measurements at the batteries' connection sites were used in this work to control the batteries' operations.

Previous research is therefore extended by the work presented here, as previous work has only utilised common set-point thresholds for controlling each of the DERs. The approach proposed in this paper ensures that unavoidable voltage drops along the feeder do not skew the control decisions, and voltage oscillations caused by demand variation are taken into control considerations. In contrast to previous work, where substation monitoring was used to inform control units of the transformer's present operational capacity, the proposed AIMD+ algorithm does not require this information and, hence, does not require such an extensive communications infrastructure.