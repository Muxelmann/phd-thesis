\section{Introduction}
\label{ch2:sec:introduction}

Outages are becoming more frequent in the UK as nearly half a million customers experienced service disruptions in December 2013 alone \cite{Ofgem2014}.
Latest figures from the UK energy regulator \textit{OFGEM} indicate that \pounds44.6 million worth of penalties were accumulated due to customer interruptions and customer minutes lost in a single year \cite{Ofgem2012}.
Underlying causes that lead to these penalties were both due to natural events, e.g. extreme weather, as well as system overloads due to increasing demand for electricity.
The latter issue is only amplified since major focus of UK energy policies has been put on transitioning towards a low carbon economy \cite{HMGovernment2009, RoyalAcademyofEngineering2010}.
Particularly the decarbonisation of heat and transport sectors are two areas of significant strategic focus and Low Carbon Technology (LCT) such as photovoltaic installations, electric vehicles and heat pumps are expected to contribute significantly to this transition.

However, as adaptation of these LCTs increases and they start to penetrate power distribution networks, stress on these networks will continue to increase even further, which may result in additional service disruptions.
Furthermore, the uptake of LCTs is not expected to progress evenly throughout the entire power network, and instead clusters of early adopters are predicted to form, leading to certain Low-Voltage (LV) networks to exceed their operational constraints even at relatively low national rate of LCT adaption \cite{Poghosyan2014}.
Traditional network planning approaches to circumvent constraint violations, follow the commonly used practice of aggregating a large number of customers and designing the power delivery network to cater for their largest probable demand, i.e. the After Diversity Maximum Demand (ADMD) method \cite{Richardson2010a}.
This ADMD method has remained the same for many years and uses historical load analysis and standard growth assumptions that are both no longer valid in this unprecedented LCT uptake scenario \cite{Yunusov2016}.
To make things worse, LV networks in the UK are generally unmonitored once installed.
Distribution Network Operators (DNOs) have become aware of this issue and are developing updated planning strategies involving ``smart'' and ``flexible'' electricity grids.
However, in situ equipment that will become subject to the same adaptation of LCT needs to be managed actively via innovation in the use of existing and new technologies; otherwise both frequency of service disruptions and customer minutes lost will increase alongside the proliferation of LCTs \cite{Ault2008a}.

Two solutions exist, allowing DNOs to support LV network's operation: 
\begin{enumerate*}
	\item reinforcement of in situ network assets;
	\item deployment of network support equipment.
\end{enumerate*}
Whilst network reinforcement would certainly address immediate issues of current network capacity constraints, it is also the more expensive and disruptive option.
More specifically, customer will need to deal with outages during periods of asset upgrades (e.g. transformer upgrade and line re-conductoring after secondary transformers' tap settings have been adjusted).
Therefore, alternatives to defer or avoid network reinforcements have been sought and assessed \cite{Harrison2007, Zangs2016a, VanderKlauw2016d, Greenwood2017}.
Most promising alternatives are to install flexible and controllable Distributed Energy Resources (DERs), or more specifically: Battery Energy Storage Solutions (BESS) \cite{Wade2010}.
BESS has not only seen significant advancements in technology, but also received increasing attention in both academic studies and industry trials \cite{Palizban2016}.


Installing BESS on a strategic location in the LV network brings several advantages to DNOs' control over the network's performance.
Regulating voltages to stay within statutory operating bands \cite{Yang2014}, shaving peak load to relieve stress from the installed network assets \cite{Bennett2015}, or reducing phase unbalance to increase network efficiency \cite{Wang2015b} are only a few examples of recent research in this field.
Whilst the questions regarding locating and scaling of BESS have mostly been addressed, BESS control can be split into two complementing yet unmarried approaches:
\begin{enumerate*}
	\item ``off-line'' control, using load forecasts and BESS schedules; and
	\item ``on-line'' control, using Set-Points Control (SPC), Model Predictive Control (MPC) or similar dynamic control methods.
\end{enumerate*}

Off-line control uses historic data to predict future load patterns, which are used to schedule BESS operation accordingly.
Early approaches, e.g. by Oudalov et al. \cite{Oudalov2007}, who used dynamic programming to generate BESS schedules, had a relatively high forecast error due to the inherent difficulty of predicting future loads, which ultimately limits the ability of given BESS schedule to i.e. reduce peaks.
This reason is why recent research either includes uncertainty, like the work by Baker et al. \cite{Baker2017} where uncertainty of wind power was taken into account when scheduling and sizing BESS, or it frequently re-evaluates BESS schedules, as done by Wang et.al \cite{Wang2014a}, where BESS control is adjusted after each decision epoch.
Despite load forecasts being imperfect, forecasts remain a key component for scheduling BESS thanks to work like that by Rowe et al. \cite{Rowe2014a}, where a filtering mechanism was proposed for scheduling algorithms to reduce peak load in LV networks in spite of forecast errors.
Also, most day-ahead forecast only forecast at a temporal resolution down to half-hourly periods.
As pointed out by Haben et al. \cite{Poghosyan2014, Haben2014}, forecasts at half-hourly resolution yield the best compromise between high accuracy and high temporal resolution, which is why they have become the standard for generating BESS operating schedules.
Nonetheless, high-resolution load volatility, imposing most stress on the network, cannot be addressed when using this kind of low-resolution forecast, which is why on-line control has been considered as an alternative to off-line control.

One flavour of on-line control is the Set-Point Control (SPC), which is a robust technique that can immediately respond to network changes.
Since this kind of control runs the risk of reaching shortage or surplus of BESS stored energy, modifications like hysteresis control \cite{Gybel2012} and ramp-rate control \cite{Such2012} were proposed.
However, this kind of on-line control is less effective in addressing daily demand peaks, since pure SPC can only react to current network demand and does not respond to general trends or upcoming load events.
To address these shortcomings SPC has been extended, using short-term load predictions by implementing Model Predictive Control (MPC).
Some MPC examples include Auto-Regressive (AR) models \cite{Li2009, Nie2011}, fuzzy logic models \cite{Sannomiya2001, Chen2013a}, genetic algorithms \cite{Xia2015a, Liu2015} or Artificial Neural Networks (ANN) \cite{Kalogirou2014, Quan2014, Lee2014, Pezeshki2014, Vaz2016, Reihani2016, Xiao2017}.
Implementing increasingly complex MPC to support on-line control is therefore a strong research trend, however the computational burden to deliver real-time solutions makes implementation of such systems not yet feasible.
Therefore, the control approach is reversed in this work, and instead of supporting on-line control with real-time load forecasts, pre-determined BESS schedules are adjusted using real-time measurements.

In this paper a BESS schedule adjustment method is proposed that unifies the benefits from high-resolution demand measurements and low-resolution demand forecasts.
The proposed method combines BESS scheduling, following a ``peak-shaving'' and ``valley-filling'' behaviour, and a MPC, comprising of a lightweight AR model, using two  Proportional Integral Derivative (PID) compensators.
To assure future implementability, these components were chosen to form a lightweight, stable and robust system.
All BESS schedules were generated under the constraints of a realistic BESS model, and all demand measurements and corresponding forecasts used in this work were provided by the project partner and DNO: \textit{Scottish and Southern Energy Networks} (SSEN).
Results are generated from realistic (i.e. provided) network loads with corresponding load forecasts, and cases are compared against the original and a baseline load case (i.e. traditional off-line control).
It is shown that, even under these imperfect forecast conditions, the proposed schedule adjustment method can successfully reduce high-resolution peaks.
In fact, whilst the probability distribution of the baseline case sat around an average of 1.78kW peak reduction, the proposed method increased the reduction to 5.24kW.
Since this proposed control method is the natural extension of our previous work in \cite{Zangs2016}, it is hereon referred to as ``dynamic control''.

The paper is organised as follows:
In Section \ref{ch2:sec:system-explanation} of this paper, all constituent system components including BESS model, forecast acquisition and BESS schedule generation are explained.
%Cost functions to generate optimised half-hourly BESS schedules are formulated, containing well established parameters like the Peak-to-Average Ratio (PAR) or ``Min-Max'' difference, which have been widely used in DER scheduling and operation \cite{Liu2014, Deng2015, Bayram2015, Zangs2016a}.
Section \ref{ch2:sec:control-of-bess} presents the dynamic control, including the dual PID setup and MPC.
Section \ref{ch2:sec:case-studies} outlines the different case studies that were used to compare the performance of the dynamic control.
In Section \ref{ch2:sec:results}, all results from these case studies are presented and discussed.
Finally, conclusion and the future work are described in Section \ref{ch2:sec:conclusion}.










