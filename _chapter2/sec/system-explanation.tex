\section{System Explanation}
\label{ch2:sec:system-explanation}

\begin{figure*}[htb]\centering
% Define some block styles
\tikzstyle{box} = [%
	draw,%
	rectangle,%
	%fill=green!20,%
	minimum height=3em,%
	minimum width=5em,%
]
\subfloat[]{%
	\begin{tikzpicture}[node distance=1.5cm, shorten >= 1pt, >=stealth', auto, scale=0.8, transform shape]
		% Define nodes
	    \path (0,0)
	    node [box, fill=blue!20](data) {Data}
	    node [box, fill=blue!20, below of=data](forecast) {Forecast}
	    node [box, fill=blue!20, below of=forecast](schedule) {Schedule}
	    node [box, below of=schedule](battery) {Battery}
	    node [box, below of=battery](network) {Network};
		% Draw lines
		\draw [->] (data) to (forecast);
		\draw [->] (forecast) to (schedule);
		\draw [->] (schedule) to (battery);
		\draw [->] (battery) to (network);
	\end{tikzpicture}%
	\label{ch2:subfig:traditional-forecast-based-system}%
}
\hspace{10mm}
\subfloat[]{
	\begin{tikzpicture}[node distance=1.5cm, shorten >= 1pt, >=stealth', auto, scale=0.8, transform shape]
		% Define nodes
	    \path
	    (0,0) 
	   	node [box, fill=green!20, below of=schedule, xshift=-14.5mm](controller) {Controller}
		node [box, fill=green!20, below of=controller](mpc) {MCP}
	    node [box, right of=controller, xshift=15mm](battery) {Battery}
	    node [box, below of=battery](network) {Network};;
		% Draw lines
		\draw [->] (network) to (mpc);
		\draw [->] (mpc) to (controller);
		\draw [->] (controller) to (battery);
		\draw [->] (battery) to (network);
	\end{tikzpicture}%
	\label{ch2:subfig:traditional-on-line-system}%
}
\hspace{10mm}
\subfloat[]{%
	\begin{tikzpicture}[node distance=1.5cm, shorten >= 1pt, >=stealth', auto, scale=0.8, transform shape]
		% Define nodes
	    \path
	    (0,0)
	    node [box, fill=blue!20](data) {Data}
	    node [box, fill=blue!20, below of=data](forecast) {Forecast}
	    node [box, fill=blue!20, below of=forecast](schedule) {Schedule}
	    node [box, below of=schedule](battery) {Battery}
	    node [box, below of=battery](network) {Network}
	   	node [box, fill=green!20, left of=battery, xshift=-14.5mm](controller) {Controller}
		node [box, fill=green!20, below of=controller](mpc) {MCP};
		% Draw lines
		\draw [->] (data) to (forecast);
		\draw [->] (forecast) to (schedule);
		\draw [->, bend right] (schedule.180) to (controller.90);
		\draw [->] (network) to (mpc);
		\draw [->] (mpc) to (controller);
		\draw [->] (controller) to (battery);
		\draw [->] (battery) to (network);
	\end{tikzpicture}%
	\label{ch2:subfig:proposed-dynamic-control-system}%
}

\caption{Combination of traditional forecast driven BESS control (Subfig. \ref{ch2:subfig:traditional-forecast-based-system}) and traditional on-line system (Subfig. \ref{ch2:subfig:traditional-on-line-system}) results in the proposed dynamic control system (Subfig. \ref{ch2:subfig:proposed-dynamic-control-system}).}
\label{ch2:fig:system-diagram}
\end{figure*}


The presented work is part of the \textit{New Thames Valley Vision} (NTVV) research project and was conducted in collaboration with the British DNO \textit{Scottish and Southern Energy Networks} (SSEN) \cite{NTVV2016}.
From the findings of this research project the diagram in Figure~\ref{ch2:fig:system-diagram} was generated, showing two well established ESMU control approaches and the proposed dynamic control approach.
This figure includes all constituent systems that were used during the ESMU street-level deployment.
The two traditional systems are off-line and on-line ESMU control which are shown in Figure~\ref{ch2:subfig:traditional-forecast-based-system} and \ref{ch2:subfig:traditional-on-line-system}, respectively.
Alongside these two control approaches is the proposed dynamic control system that is shown in Figure~\ref{ch2:subfig:proposed-dynamic-control-system}.
This control approach entails the benefits from both the traditional half-hourly forecast driven and the sub-half-hourly ESMU control system and can therefore be seen as the hybrid of the two traditional systems.
Unlike previous work this hybrid system does not rely on Set-Point Control (SPC) which is adjusted by a MPC to compensate for trends in the load profile.
Instead, it operates by executing a predetermined half-hourly ESMU schedule which is adjusted at sub-half-hourly intervals.
Therefore the necessity of relying on a stable SPC is removed and replaced by a robust schedule execution.
Nonetheless, flexibility is provided by allowing the aforementioned schedule adjustments.
The preceding work by Rowe et al. in \cite{Rowe2014} inspired this hybrid system and used a similar approach that emphasises these benefits of using a hybrid system.
Unlike the work by Rowe et al. in \cite{Rowe2014} however, the proposed hybrid system operates at a higher temporal resolution (it uses a light weight deterministic adjustment method in the form of MPC), and it does not rely on a long forecasting horizon since it recomputes the power adjustments for every single time-step.
As already mentioned, those adjustments are based on MPC-guided instructions and details about this dynamic control are outlined in Section~\ref{ch2:sec:control-of-esmu}.

In this section however the battery model which is used in this work is reminded first.
Also the load data acquisition, forecasting and ESMU schedule generation are outlined where scheduling is performed in accordance to the ESMU model's constraints.

\subsection{ESMU model}
\label{ch2:subsec:esmu-model}

\nomenclature[J]{$C_{bat}$}{Battery capacity in kWh, where $C_{bat} \in \mathbb{Z}^{>0}$}
\nomenclature[J]{$C_{f}$}{Charge factor or ``C-factor'' of the battery, where $C_{f} \in \mathbb{Z}^{>0}$}
\nomenclature[J]{$\mu$}{Self-discharge losses of battery, where $\mu \in (0, 1]$}
\nomenclature[J]{$P_{bat}$}{ESMU power electronic rating, where $P_{bat} \in \mathbb{Z}^{>0}$}
\nomenclature[J]{$\eta$}{Round-trip efficiency of power electronics, where $\eta \in (0, 1]$}
\nomenclature[J]{$t$}{Discrete sample of time, where $t \in \{0, \Delta t, \dots, T\Delta t\}$}
\nomenclature[J]{$T$}{Number of samples during the entire simulation, where $T \in \mathbb{Z}_{>0}$}
\nomenclature[J]{$\Delta t$}{Sub-half-hourly sample period, where $\Delta t \in \mathbb{Z}^{>0}$}
\nomenclature[J]{$SOC(t)$}{Scheduled state of charge at sample $t$}
\nomenclature[J]{$p(t)$}{ESMU power at time $t$, where $\textbf{p} = (p(t))$ and $p(t) \in \mathbb{Z}$}
\nomenclature[J]{$\textbf{p}$}{ESMU power vector, where $\textbf{p} = (p(t))$}
\nomenclature[J]{$p_{bat}(t)$}{Battery power at time $t$, which is derived form $p(t)$, where $\textbf{p}_{bat} = (p_{bat}(t))$ and $p(t) \in \mathbb{Z}$}
\nomenclature[J]{$\textbf{p}_{bat}$}{Battery power vector, where $\textbf{p}_{bat} = (p_{bat}(t))$}

The ESMU model is based on the physical system that was deployed by SSEN during the NTVV project.
Since this model is the same model as the one used in the preceding chapter which has been explained in detail in Section~\ref{ch1:subsec:battery-model}, only the model's final equation (as well as all used parameters) are detailed, hence foregoing the re-deriving of the same battery storage model.
This ESMU model equation is as follows:

\begin{equation}
\begin{split}
	&SOC(t+\Delta t) =\\
	&\begin{cases}
		\eta \left(SOC(t) + \frac{\mu \Delta t p(t)}{C_{bat}(3.6\times10^6)}\right) &\text{if } p(t) \geq 0\\
		\eta \left(SOC(t) + \frac{\Delta t p(t)}{\mu C_{bat}(3.6\times10^6)}\right) &\text{otherwise}
	\end{cases}
\end{split}
	\label{ch2:equ:battery-model-equation}
\end{equation}

Here the next State of Charge, $SOC(t+\Delta t)$, is computed from the current State of Charge, $SOC(t)$, and the current battery power, $p(t)$.
This is done by calculating the current change in SOC as the added energy $\Delta t p(t)$, divided by the total battery capacity $C_\text{bat}$.
Dynamic properties of the model also take into account the energy conversion efficiency, $\eta$, and the self-discharge factor, $\mu$.

For the purpose of the simulation, it is assumed that the battery is initially charged up to 50\%.
Hence, the initial conditions of this model are defined as $SOC(0) = 0.5$, which makes the model valid for a time span of $t \geq 0$, where $t \in \mathbb{Z}_{\geq0}$.

\subsection{Load data and ESMU scheduling}
\label{ch2:subsec:load-data-and-esmu-scheduling}

\nomenclature[J]{$k(t)$}{Sampling time conversion function, linking sub-half hourly samples $t$ at sampling period $\Delta r$ to half-hourly period $30 \Delta t$}

Having established the ESMU model, the procedure to generate a corresponding schedule is explained in this subsection.
This procedure follows the same practice as outlined in the previous chapter (i.e. in Section~\ref{ch1:subsec:esmu-scheduling}) where an ESMU schedule is generated at half-hourly temporal resolution.
Therefore the same synchronisation function, $k(t)$, can be used to link the native sampling period of $\Delta t$ to the schedule's half-hourly period.
Since the sub-half-hourly operation was at a minutely period and the generated schedule is at half-hourly period, this fixed conversion function is defined as:

\begin{equation}
	k(t) := \left\lfloor\frac{t-1}{30\Delta t}\right\rfloor+1
	\label{ch2:equ:sample-conversion-function}
\end{equation}

Having established a means of synchronising the two sampling periods, the shape of the ESMU schedule that would ``smoothen'' the underlying power profile is defined next.
For simplicity linear forwarding was chosen which means that the power assigned at e.g. $t=1$ remains constant over the scheduling period of $30\Delta t$ until $t=31$.
With this assumption the ESMU's SOC can be calculated for each $t$ despite the scheduled power profile only having been defined for every 30$^\text{th}$ $t$.
Furthermore, with this second assumption not only every sub-half-hourly ESMU power can be derived from its half-hourly schedule, but it also enables the calculation of every SOC, i.e. $SOC(t)$ is well defined.

\nomenclature[J]{$p_\text{for}(k(t))$}{Half-hourly load forecast that is used for computing the ESMU schedule, where $p_\text{for}(k(t))\in\mathbb{Z}$}
\nomenclature[J]{$\textbf{p}_\text{for}$}{Half-hourly load forecast vector that is used for computing the ESMU schedule, where $\textbf{p}_\text{for} = (p_\text{for}(k(t)))$}
\nomenclature[J]{$p_\text{sch}(k(t))$}{Half-hourly schedule that is generated from the load forecast, where $p_\text{sch}(k(t))\in\mathbb{Z}$}
\nomenclature[J]{$\textbf{p}_\text{sch}$}{Half-hourly schedule vector that is generated from the load forecast, where $\textbf{p}_\text{sch} = (p_\text{sch}(k(t)))$}

For the generation of the ESMU schedule a load forecast, $\textbf{p}_\text{for}$, was required; here $\textbf{p}_\text{for} = (p_\text{for}(k(t)))$.
Just like the ESMU forecast this forecast is also produced at half-hourly temporal resolution and it was provided by SSEN as part of the NTVV research project.
The task at hand is to find a half-hourly ESMU schedule, $\textbf{p}_\text{sch}$, where $\textbf{p}_\text{sch} = (p_\text{sch}(k(t)))$, that improves the shape of the underlying forecast, e.g. by reducing load peaks.
In order to generate this optimised ESMU schedule a performance metric quantifying improvements had to be defined first.
The remaining task is to now compute a half-hourly schedule, $\textbf{p}_\text{sch}$, that yields the best performance.
This computation is done by minimising several cost-functions.

In Chapter~\ref{ch1} several cost functions were defined.
Here however, three shape dependent cost-functions are used that quantify the profile improvements that are yielded by $\textbf{p}_\text{sch}$.
These costs entailed the Peak-to-Average Ratio (PAR), the difference between the resulting power profile's maximum and minimum (MMD) load, and the magnitude of all power transients (TRA) \cite{Mohsenian-Rad2010, Mostafa2016}.
Although these costs and their benefits have already been presented in Section~\ref{ch1:subsec:esmu-scheduling} of this thesis, they are reminded for convenience.
Before however detailing each of these three cost functions, a notation that simplifies power as, $\textbf{p}$, is introduced:

\begin{equation}
\begin{split}
	p(t) = p_\text{for}(k(t)) + p_\text{sch}(k(t))\\
	\text{ where } (p(t)) = \textbf{p}
\end{split}
\label{ch2:equ:notation-simplification}
\end{equation}

Within this section, the vector $\textbf{p}$ represents the power profile as it would be measured at the substation when both forecast, $p_\text{for}(t)$, and scheduled, $p_\text{sch}(t)$, power were applied.
The first cost function that is used in this chapter addresses the minimisation of PAR and is defined as follows:

\nomenclature[J]{$\zeta_\text{PAR}(\textbf{p})$}{Cost of a power profile $\textbf{p}$, based on Peak-to-Average Ratio (PAR), where $\zeta_\text{PAR}(\textbf{p}) \in \mathbb{R}_{\geq0}$}

\begin{equation}
	\zeta_\text{PAR}(\textbf{p}) := \left(\frac{\max_t|\textbf{p}|}{\overline{\textbf{p}}}\right)^2 - 1
	\label{ch2:equ:cost-par}
\end{equation}

\nomenclature[J]{$T_\text{sch}$}{Length of scheduling horizon, where $T_\text{sch} \in \mathbb{Z}_{>0}$ and $T \geq T_\text{sch}$}

Here, $\overline{\textbf{p}}$ represents the mean power, i.e. $\overline{\textbf{p}} = \frac{\Delta t}{T_\text{sch}}\sum_{t=1}^{T_\text{sch}}p(t)$ and $\overline{\textbf{p}} \in \mathbb{R}$, where $T_\text{sch}$ is the length of the scheduling horizon in regards to the sampling period $\Delta t$.
If the profile $\textbf{p}$ had a lot of spikes then the ratio between its maximum and its mean value is greater than one (or with the $-1$ term greater than zero).
A perfectly flat power profile on the other hand would thus result in cost of zero.
However, due to limited battery capacity achieving such a cost of zero is highly unlikely.
This is why a solution to minimise this cost needs to be found that minimises this cost in accordance to the previously explained ESMU model.
To not only increase the mean power or reduce peak power, the second cost function is defined as the difference between minimum and maximum power of $\textbf{p}$:

\nomenclature[J]{$\zeta_\text{MMD}(\textbf{p})$}{Cost of a power profile $\textbf{p}$, based on the difference between minimum and maximum power, where $\zeta_\text{MMD}(\textbf{p}) \in \mathbb{R}_{\geq0}$}

\begin{equation}
	\zeta_\text{MM}(\textbf{p}) := \max_t(\textbf{p}) - \min_t(\textbf{p}) % \max_t(p(k(t))) - \min_k(p(k(t)))
	\label{ch2:equ:cost-min-max}
\end{equation}

Similar to the PAR this cost also reduces to zero when the resulting power profile is perfectly flat.
Unlike the PAR however, this cost does not incentivise an increase of mean power.
Minimising PAR by itself may result in unnecessary and potentially damaging battery cycling when trying to raise the mean power of the profile, yet this behaviour is avoided when $\zeta_\text{MMD}$ is included alongside $\zeta_\text{PAR}$.
Nonetheless, $\zeta_\text{PAR}$ and $\zeta_\text{MMD}$ only impact the fringes of the resulting half-hourly power profile and could lead to an erratic load profile.
Therefore the third and final cost addresses the interim power volatility by aiming to minimise the largest possible power transient:

\nomenclature[J]{$\zeta_\text{TRA}(\textbf{p})$}{Cost of a power profile $\textbf{p}$, based on largest power transient, where $\zeta_\text{TRA}(\textbf{p}) \in \mathbb{R}_{\geq0}$}

\begin{equation}
	\zeta_\text{TRA}(\textbf{p}) := \max_{t}(p(t+\Delta t)-p(t))^2
	\label{ch2:equ:cost-tra}
\end{equation}

Minimising this final cost has a smoothening effect on the improved half-hourly power profile since a profile with no transients is by definition a flat and smooth profile.
Since all three cost functions are normalised, they are summaries into a single global cost function.
In this cost function only the half-hourly ESMU schedule, $\textbf{p}_\text{sch}$, is used as an input and the forecast, $\textbf{p}_\text{for}$, is kept constant:

\nomenclature[J]{$\zeta(\textbf{p})$}{Global cost for a given power profile $\textbf{p}$}
\begin{equation}
\begin{split}
	\zeta(\textbf{p}_{sch}) := &\zeta_{PAR}(\textbf{p}_{sch}+	\textbf{p}_{for})\\
	&+ \zeta_{MM}(\textbf{p}_{sch}+\textbf{p}_{for})\\
	&+ \zeta_{TRA}(\textbf{p}_{sch}+\textbf{p}_{for})
\end{split}
\label{ch2:equ:cost-global}
\end{equation}

Subject to ESMU constraints, this global cost function is minimised using a standard solver (i.e. Sequential Quadratic Programming - SQP) to yield a ESMU schedule that is optimised for the given forecast:

\begin{equation}
	\min_{p_{schedule}}\zeta(p_{schedule}) \text{ s.t. }
	\begin{cases}
		0 \leq SOC(t) \leq 1\\
		|p_{battery}(t)| \leq C_{battery} \times C_{factor}
	\end{cases}
	\forall t
	\label{ch2:equ:cost-minimisation}
\end{equation}

In order to limit the control's flexibility a State Of Charge tolerance, $SOC_{tol}$, is included in this minimisation problem.
$SOC_{tol}$ defines the maximum allowed deviation from the computed SOC profile without hitting operational limits (i.e. SOC of one or zero) and may take values in the form of $SOC_{tol} \in [0, 0.5)$ where 0 implies no tolerance and 0.5 implies complete flexibility as if no schedule were computed.
For the work at hand a value of 0.1 was chosen to allow a $\pm$10\% energy tolerance band.

\begin{figure}\centering
	\subfloat[]{%
		\includegraphics{_chapter2/fig/day-forecasted}
		\label{ch2:subfig:day-forecasted}
	}\\
	\subfloat[]{%
		\includegraphics{_chapter2/fig/day-actual}
		\label{ch2:subfig:actual-forecasted}
	}
	\caption{An example of applying a half-hourly ESMU schedule to its half-hourly schedule (Subfig. \ref{ch2:subfig:day-forecasted}) and the actual, sub-half-hourly daily load (Subfig. \ref{ch2:subfig:actual-forecasted}).}
	\label{ch2:fig:cost-sample}
\end{figure}

As repetitively mentioned, the ESMU operation that results from this scheduling mechanism is at half-hourly resolution and has therefore limited impact on sub-half-hourly load variation.
To visualise this limitation a singe day's ESMU schedule was generated from its corresponding forecast as defined in Equation~\ref{ch2:equ:cost-minimisation} and plotted in Figure~\ref{ch2:fig:cost-sample}.
In this simple comparison the noticeable discrepancy between the half-hourly ESMU schedule and the actual, sub-half-hourly demand can be observed.
Furthermore, noticeable disparity in peak duration, magnitude and volatility can be noted.
This discrepancy and disparity emphasise the incompatibility issues between half-hourly ESMU schedules and the actual sub-half-hourly load.
As previously discussed benefits of ESMU were intended to mitigate sub-half-hourly load volatility, yet this cannot be achieved when solely applying half-hourly ESMU schedules in an off-line manner.
Therefore, the control strategy to add an on-line component is explained in the next section.
