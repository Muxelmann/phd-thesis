\section{Summary}
\label{ch2:sec:summary}

In this chapter a dynamic control method is proposed to address \ref{objective-2} in the objective list which is presented in Section~\ref{ch-introduction:sec:problem-statement}.
The proposed method adjusts half-hourly Energy Storage Management Unit (ESMU) schedules on a sub-half-hourly basis in order to minimise otherwise neglected sub-half-hourly power spikes without risking a shortage or surplus of ESMU stored energy.
Reason behind this is that traditional load forecasts usually lack in accuracy and temporal resolution which makes it hard to schedule ESMU in an optimal manner.
Recent research implements derivations of Set-Point Control (SPC) which is typically guided by a short-term Model Predictive Control (MPC) mechanism to address both load volatility and to prepare ESMU for upcoming load spikes.
Those approaches do however not utilise the information, quality and operational certainty that would be provided by load forecasts and ESMU schedules.
The proposed dynamic control addresses this shortcoming by approaching the problem from the opposite direction: i.e. it adjusts a predetermined ESMU schedule based on two linked PID compensators.

The first compensator was designed to minimise the deviation from the prescheduled ESMU's State Of Charge (SOC) profile and the second compensator was designed to minimise the load volatility.
For the second compensator to operate however, a short-term predictive model was used to estimate the load power in the immediate future (i.e. the next time step).
Different light weight and well established mechanisms were used to implement this predictive model in order to assure real-time operation and robustness of the system.

Simulating these different models to guide schedule adjustments yielded positive results for each test case that used dynamic control.
In fact, dynamic control outperformed the baseline case in nearly every case (where the baseline case is the scenario of applying a traditional half-hourly ESMU schedule in an off-line manner).
Whilst this baseline operation did also increase peak load under severe forecast errors, the best performing dynamic control case was able to reduce the probability of increasing peak loads by a factor of 3.12.
Also, the length of the underlying prediction models was varied form 5 minutes to 2 hours in order to assess the impact of this variation on the performance of the dynamic control.
It was determined that there is no linear correlation between the models' lengths and the performance of the dynamic control.
Instead, the dynamic control operated with slight performance variations, yet always outperformed the original and baseline cases.
Those performance gains were achieved despite the fact that all cases used the same underlying ESMU schedule that were generated from realistic (hence imperfect and erroneous) load forecasts.
In conclusion, \ref{objective-2} which is defined in Section~\ref{ch-introduction:sec:problem-statement} has been successfully met with the provided data.

The work presented here demonstrates how imperfect ESMU schedules may still be used to yield more beneficial impacts on the overall load profile.
With future deployment of ESMU throughout Low-Voltage power distribution networks, advanced ESMU control is necessary to assure their impact is in accordance to volatile demand.
Control, like the one proposed here can take into account the complete range of demand volatility and when implemented correctly, can defer or avoid network reinforcement altogether.
This is particularly true since ongoing electrification of UK heat and transport sectors change consumers' electricity consumption and increase stress on power distribution networks.
