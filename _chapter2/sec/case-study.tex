\section{Case studies}
\label{ch2:sec:case-studies}

The aim of the presented work is to show that BESS operation, that is guided by imperfect forecasts, can be improved to yield further reductions of peak load.
Therefore, six different case studies are defined and split into two distinct groups.
Here, the first group addresses the the impact of dynamic control when subjected to realistic (i.e. imperfect) half-hourly load forecasts, whilst the second group assumes perfect half-hourly load forecasts.
Since half-hourly temporal resolution does not allow to compensate for sub-half-hourly demand volatility, each group is sub-divided into three cases, where different mechanisms to predict the volatility are presented.
The first case in each group implements the simplest prediction mechanism, i.e. it is assumed that the current power measurements repeats.
For the second case, the more sophisticated MPC is used and different lengths of AR models are assessed.
The third and final case is the ideal case where perfect foresight is given.
These six cases are numbered \textbf{I} to \textbf{VI} and, for clarity, they are tabulated in Table \ref{ch2:tab:cases}, too.

\begin{table}\centering
	\begin{tabular}{r | c c}
		estimation method & real forecast\\% & ideal forecast\\
		\hline
		perfect foresight & \textbf{I}\\% & \textbf{IV}\\
		MPC (AR/ARX) & \textbf{II}\\% & \textbf{V}\\
		power repetition & \textbf{III}\\% & \textbf{VI}\\
	\end{tabular}
	\caption{Three cases and their dynamic control input assumptions}
	\label{ch2:tab:cases}	
\end{table}

Results from each case are compared against the baseline, \textbf{B}, where the half-hourly BESS schedule is simulated without and sub-half-hourly adjustments.
In this work, two comparison steps are performed.
The first step compares the impact of dynamic BESS control for a single day, i.e. time-series comparison to get this first impression is presented in the results section.
Secondly, the entire dataset is compared by only focusing on the daily power peak, and the magnitude of their reduction is compared.