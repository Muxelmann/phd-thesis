\section{Case studies}
\label{ch2:sec:case-studies}

Two cases are defined that the performance of the proposed dynamic control is compared against: case \textbf{O} and case \textbf{B}.
More specifically, case \textbf{O}, original case, is the scenario where no BESS operation takes place.
Traditional off-line BESS operation that only uses predetermined half-hourly BESS schedules is referred to as the benchmark case, or case \textbf{B}.
All remaining cases, which are explained below, are capture different versions of the dynamic control.

In addition to the two comparison cases, three more different case studies are defined: cases \textbf{I}, \textbf{II} and \textbf{III}.
This group of three case studies evaluates the impact of the proposed dynamic control when subjected to realistic (i.e. imperfect) half-hourly load forecasts.
In each of the three cases, a different mechanisms is used to predict the power volatility.
More specifically, case \textbf{I} implements the simplest prediction mechanism, i.e. it is assumed that the current power measurements repeats, case \textbf{II} uses the aforementioned MPC, and performance of different AR model lengths is compared, and the third and final case, case \textbf{III}, represents an ideal scenario where perfect foresight is assumed and the exact next load can be estimated.
For clarity, all three cases, numbered \textbf{I} to \textbf{III}, are summarised and tabulated in Table \ref{ch2:tab:cases}.

\begin{table}[htb]\centering
	\begin{tabular}{r | c c}
		& real forecast & ideal forecast\\
		\hline
		repeated power estimation & \textbf{I} & \textbf{IV}\\
		MPC power estimation & \textbf{II} & \textbf{V}\\
		perfect power foresight & \textbf{III} & \textbf{VI}\\
	\end{tabular}
	\caption{Six cases and their dynamic control input assumptions}
	\label{ch2:tab:cases}	
\end{table}

Results from all BESS cases (\textbf{B}, \textbf{I}, \textbf{II} and \textbf{III}) are first compared against the original, i.e. uncompensated, network load case (\textbf{O}).
Here, by using a sample day, the assessment of load profile improvements are made clear.
Once it is clear how each day's peak is reduced by the algorithm, the daily peak reduction capability from all cases' results are compared.
Rather than assessing the underlying load profile from a time-series perspective, only focus is put on any further peak load reductions.
However, the number of days may make it difficult to spot trends and improvements in the data.
Therefore, from the daily peak reduction results, a Probability Density Function (PDF) is derived, which is based on kernel density estimation.
The PDF shows the stochastic improvement of each case type in comparison to the original case, i.e. case \textbf{O}.
