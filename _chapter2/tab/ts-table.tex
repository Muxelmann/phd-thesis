\begin{table}\centering
%	\begin{tabular}{r | c}
%		case & peak load\\% & ideal forecast\\
%		\hline
%		\textbf{O} & 46.6kW\\
%		\textbf{B} & 44.8kW\\
%		\textbf{I} & 36.4kW\\% & \textbf{IV}\\
%		\textbf{II} (AR) & 36.8kW\\% & \textbf{V}\\
%		\textbf{II} (ARX) & 36.6kW\\% & \textbf{V}\\
%		\textbf{III} & 38.4kW\\% & \textbf{VI}\\
%	\end{tabular}
	\begin{tabular}{r | c | c | c | c | c | c}
		\multirow{2}{*}{case} & \multirow{2}{*}{\textbf{O}} & \multirow{2}{*}{\textbf{B}} & \multirow{2}{*}{\textbf{I}} & \textbf{II} & \textbf{II} & \multirow{2}{*}{\textbf{III}}\\% & ideal forecast\\
		& & & & \tiny{(AR)} & \tiny{(ARX)} & \\% & ideal forecast\\
		\hline
		peak & \multirow{2}{*}{46.6} & \multirow{2}{*}{44.8} & \multirow{2}{*}{36.4} & \multirow{2}{*}{36.8} & \multirow{2}{*}{36.6} & \multirow{2}{*}{38.4}\\
		(kW) & & & & & & \\
	\end{tabular}
	\caption{Peak reduction in time-series sample}
	\label{ch2:tab:ts-table}	
\end{table}