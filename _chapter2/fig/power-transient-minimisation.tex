\begin{figure}[htb]\centering
% Define some block styles
\tikzstyle{box} = [%
	draw,%
	rectangle,%
	%fill=green!20,%
	minimum height=3em,%
	minimum width=5em,%
]
\tikzstyle{sample} = [draw, circle, fill, scale=0.4]
\tikzstyle{estimate} = [draw, circle, solid, fill=white, scale=0.4]
\begin{tikzpicture}[node distance=2cm, shorten >= 1pt, >=stealth', auto, scale=0.9, transform shape]
	% Draw axis and labels
	\draw [<->] (0, 6) -- (0, 0) -- (12, 0);
	\draw (12, 0) node [anchor=north east] {Time};
	\draw (0, 3) node [anchor=south, rotate=90] {Power};
	% Add axis ticks
	\foreach[count=\i, evaluate=\i as \l using int(\i-1)] \t in {1,4,...,10}
	{
		\draw (\t,0.1) -- (\t,-0.1);
		\ifthenelse{\l=0}
		{\draw (\t,-0.1) node[anchor=north] {$t$}}
		{
			\ifthenelse{\l=1}
			{\draw (\t,-0.1) node[anchor=north] {$t+\tau$}}
			{\draw (\t,-0.1) node[anchor=north] {$t+\l\tau$}};
		};
	}
	% Draw main power curve
	\draw [thick]
	(0.5, 2.1) to
	(1, 2) node[sample](sam_0) {} to
	(4, 4) node[sample](sam_1) {} to
	(7, 3.5) node[sample](sam_2) {} to
	(10, 2) node[sample](sam_3) {} to
	(11, 1.8);
	
	% Add power sample forward projection
	\draw [dashed] (1, 2) -- (4.6, 2);
	\draw [dashed] (4, 4) -- (7.5, 4);
	\draw [dashed] (7, 3.5) -- (10.5, 3.5);
	\draw [dashed] (10, 2) -- (10.5, 2);
	
	% Plot estimates
	\draw [dotted] (1, 2) to
	(4, 3.7) node[estimate](est_1) {} to
	(7, 3.8) node[estimate](est_2) {} to
	(10,1.7) node[estimate](est_3) {};
	
	% Label the power samples
	\draw [->] (1, 5) node[anchor=south] {$p(t)$} to [out=-135,in=135] (sam_0);
	\draw [->] (4, 5) node[anchor=south] {$p(t+\tau)$} to [out=-135,in=135] (sam_1);
	\draw [->] (7, 5) node[anchor=south] {$p(t+2\tau)$} to [out=-135,in=135] (sam_2);
	\draw [->] (10, 5) node[anchor=south] {$p(t+3\tau)$} to [out=-135,in=135] (sam_3);
	
	% Label the estimates
	\draw [->] (4, 0.75) node[anchor=north] {$\hat{p}(t+\tau)$} to [in=-45] (est_1);
	\draw [->] (7, 0.75) node[anchor=north] {$\hat{p}(t+2\tau)$} to [in=-45] (est_2);
	\draw [->] (10, 0.75) node[anchor=north] {$\hat{p}(t+3\tau)$} to [in=-45] (est_3);
	
	% Label the forward power transients
	\draw [decorate,decoration={brace,amplitude=3pt,mirror,raise=4pt},yshift=0pt,xshift=5mm]
	(4, 2) -- (4, 4) node [anchor=west,black,midway,xshift=2mm] {$\epsilon_p(t)$};
	\draw [decorate,decoration={brace,amplitude=3pt,mirror,raise=4pt},yshift=0pt,xshift=3mm]
	(7, 3.5) -- (7, 4) node [anchor=west,black,midway,xshift=2mm] {$\epsilon_p(t+\tau)$};
	\draw [decorate,decoration={brace,amplitude=3pt,mirror,raise=4pt},yshift=0pt,xshift=3mm]
	(10, 2) -- (10, 3.5) node [anchor=west,black,midway,xshift=2mm] {$\epsilon_p(t+2\tau)$};
	
\end{tikzpicture}
%	\draw [solid] (12, 6) -- (12, 4.75) -- (10.5, 4.75) -- (10.5, 6) -- (12, 6);
%	\draw (10, 6) node[sample, yshift=-10mm, xshift=25mm](legend_sample) {};
%	\draw node[right of=legend_sample, xshift=-15mm] {$p$};
%	\draw (10, 6) node[estimate, yshift=-20mm, xshift=25mm](legend_estimate) {};
%	\draw node[right of=legend_estimate, xshift=-15mm] {$\hat{p}$};
\caption{Underlying time-series based compensation strategy for compensator PID$_2$. Here, $p_{network}$ and $\hat{p}_{network}$ are abbreviated to $p$ and $\hat{p}$, respectively.}
\label{ch2:fig:power-transient-minimisation}
\end{figure}
