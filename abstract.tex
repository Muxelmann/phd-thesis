%\setlength{\topmargin}{-.5in}

\epigraph{\textit{It is not a dream. It is a simple feat of scientific electrical engineering. Electric power can drive the world's machinery without the need of coal, oil or gas. Although perhaps humanity is not yet sufficiently advanced to be willingly lead by the inventor's keen searching sense. Perhaps it is better in this present world of ours where a revolutionary idea may be hampered in its adolescence. All this that was great in the past was ridiculed, condemned, combatted, suppressed only to emerge all the more triumphantly from the struggle. [...] Our duty is to lay the foundation for those who are to come and to point the way, yes humanity will advance with giant strides. We are whirling through endless space with an inconceivable speed, all around everything is spinning, everything is moving, everywhere there is energy.}}{--- Nicola Tesla}


\chapter*{Abstract}

\addcontentsline{toc}{chapter}{Abstract}


%enter text for the abstract below

%% version 3
British Distribution Network Operators (DNOs) are facing challenges due to the\linebreak
energy sectors transitioning into a low carbon economy. This thesis aims to present\linebreak
novel methods to aid DNOs in operating their Low-Voltage (LV) networks despite this\linebreak
ongoing transition and its entailed challenges. The presented methods are realised\linebreak
with the use of Battery Energy Storage Solutions (BESS) and they develop BESS\linebreak
energy management algorithms whilst focusing on communication regimes and\linebreak
sub-half-hourly volatility in demand. Consequently, improving LV network operation\linebreak
mainly considers the reduction of peak power flow, but also includes reducing\linebreak
energy losses, voltage deviation, the magnitude of neutral currents and phase un-\linebreak
balance. Without these methods, DNOs would have to rely on traditional network re-\linebreak
enforcements so that LV networks are kept within statutory voltage bands, for\linebreak
example. Extending current literature with methods to control a single energy\linebreak
resource and a distributed BESS - whilst considering requirements for communication\linebreak
systems that may effect BESS control - is the main contribution of this thesis.

The BESS control algorithm developed in this thesis is designed to incorporate half-hourly forecasts and sub-half-hourly load volatility.
Resulting key network parameters and their interplay are identified and daily load peaks, caused by load volatility, could be reduced by an average of 3.8kW (from 45kW).
Methods are developed and address challenges for controlling a single BESS.
Neglected challenges are addressed in the subsequent BESS control methods where a desynchronised Multi-Agent Network (MAS) and communication-less BESS control fill this gap.
Results show how internal algorithm behaviour changes when desynchronising the communication environment, but without impacting the global performance of the distributed BESS.
Also, real-time performance of the communication less control algorithm is studied on different basis to show how effects from uncoordinated Low-Carbon Technologies (LTCs) like Electric Vehicle (EV) charging, can be successfully mitigated.
All objectives aligning with the aforementioned achievements have been met and the comparable storage control techniques in literature are either met or exceeded in performance when subjected to the available datasets.
 