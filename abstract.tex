%\setlength{\topmargin}{-.5in}

\epigraph{\textit{It is not a dream. It is a simple feat of scientific electrical engineering. Electric power can drive the world's machinery without the need of coal, oil or gas. Although perhaps humanity is not yet sufficiently advanced to be willingly lead by the inventor's keen searching sense. Perhaps it is better in this present world of ours where a revolutionary idea may be hampered in its adolescence. All this that was great in the past was ridiculed, condemned, combatted, suppressed only to emerge all the more triumphantly from the struggle. [...] Our duty is to lay the foundation for those who are to come and to point the way, yes humanity will advance with giant strides. We are whirling through endless space with an inconceivable speed, all around everything is spinning, everything is moving, everywhere there is energy.}}{--- Nicola Tesla}


\chapter*{Abstract}

\addcontentsline{toc}{chapter}{Abstract}


%enter text for the abstract below

British national low-carbon targets have resulted in a transition within the UK\linebreak
energy and transport sectors. This transition entails the development and uptake of\linebreak
new Low-Carbon Technologies (LCTs). Some technologies (e.g. photovoltaic) can\linebreak
offer demand alleviating services since they can supply electricity to the grid,\linebreak
whilst others (e.g. Electric Vehicles - EVs) will increase demand for grid supplied\linebreak
electricity. Distribution System Operators (DNOs), having to keep the distribution\linebreak
system within statutory constraints, are thus facing significant challenges like\linebreak
more volatility in demand, higher voltage deviation and larger phase unbalance.\linebreak
Energy storage is seen as a feasible alternative to conventional reinforcement\linebreak
activities, since they can be managed actively. Developing novel control for\linebreak
Battery Energy Storage Solutions (BESS) is thus the main focus of this thesis.\linebreak
Extending current literature by developing control algorithms for a single \linebreak
energy system and distributed BESS whilst taking into account the communication\linebreak
requirements for effective control is the main contribution of this thesis.

%Beginning the transition of UK energy systems to reach national low-carbon\linebreak
%economy targets is expected to put significant strain onto the existing power\linebreak
%networks. Reinforcing the Low-Voltage (LV) distribution network, to assure its\linebreak
%operation remains within statutory constraints, will become essential and necessary\linebreak
%when e.g. Low-Carbon Technologies (LCTs) like heat pumps, photovoltaic panels or\linebreak
%Electric Vehicles (EVs) begin penetrating the network. The expected impact\linebreak
%such as higher voltage deviation, phase unbalance, neutral current and larger\linebreak
%mean power flow, will make the Distribution Network Operator's (DNO's) role of\linebreak
%ensuring their LV networks remain within the aforementioned statutory limits \linebreak
%a significant challenge. The deployment of Battery Energy Storage Solution \linebreak
%(BESS) is seen as a feasible alternative to conventional network reinforcement.\linebreak
%Energy storage control, for both single, DNO owned and multiple, privately owned\linebreak
%energy storage devices, to improve LV network operation without the need for any\linebreak
%reinforcement, is the main contribution of this thesis.

More specifically, this thesis explores control methods for the unique characteristics of volatile demand profiles at high temporal resolutions.
The novel energy storage control algorithms are designed to incorporate both half-hourly forecasts and sub-half-hourly load volatility in order to take into account both the long term demand trends and its high volatility.
Methods developed throughout this thesis address the challenges when controlling a single (i.e. DNO owned) energy store and those challenges that are encountered in distributed battery systems.
Results show how energy storage can effectively improve network operation, even when the underlying demand forecast was erroneous.
Real-time control algorithms are studied on a time-series and statistical basis to assess the performance of the developed control methods.
All key objectives have been met for each of the presented contributions, and the comparable storage control techniques in literature are either met or exceeded in performance when subjected to the available datasets.

 