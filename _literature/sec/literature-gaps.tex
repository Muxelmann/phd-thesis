\section{Summary of gaps in literature}
\label{ch-literature:sec:literature-gaps}

In this chapter, Chapter \ref{ch-literature}, the current and future roles for energy storage have been laid out.
When focusing on BESS applications that support DNO owned networks, i.e. to enable the integration of LCTs and DERs within the LV distribution network without the need for network reinforcements, two key functions have emerged:
\begin{enumerate*}
	\item limiting voltage deviation to within statutory regulations, and
	\item avoid thermal constraints by solving the power flow problem.
\end{enumerate*}
Since DNOs had little experience with using BESS in their LV networks, several research projects and field trials were undertaken over the past decade.
So far, this research has already focused on sizing, locating and operating BESS.
From the presented literature, BESS control methods can be split into two  categories that still remain unlinked: off-line control (e.g. scheduled or forecast driven control) and on-line control (e.g. SPC or MPC).
Whereas off-line control takes into account daily load trends (i.e. at half-hourly resolution), it cannot compensate for load volatility due to DERs and LTCs (i.e. at sub-half-hourly resolution).
On-line control methods on the other hand are designed to react quickly when system changes occur (i.e. at sub-half-hourly resolution), but they cannot efficiently include daily or weekly load patterns (i.e. at half-hourly resolution) due to the increase in model complexity.
On the basis of the gaps in literature, as highlighted in the literature review, Chapter \ref{ch-literature}, and the problem statement of this thesis, which is stated in Section \ref{ch-introduction:sec:problem-statement}, research \ref{objective-1} and \ref{objective-2} were derived.
Furthermore, as the number of DERs increases throughout the grid, methods to manage them need to become more sophisticated, too.
%Yet when system complexity exceeds the available computational resource, then centralised control cannot deliver real-time instructions.
%As a result, control is being distributed using different implementations of MAS, e.g. Internet of Things (IoT) or Virtual Power Plant (VPP), where several intelligent agents collaborate to improve the network's operation.
However, all developed algorithms to control DERs either explicitly or implicitly assume synchronisation amongst all controlled entities, which need not be the case in reality.
Assessing how information desynchronisation impacts the performance of a distributed algorithm is still an open research question that is adressed by \ref{objective-3}.
\ref{objective-4} aims to extend a distributed control algorithm by developing an method that no longer depends on communication systems.
To summarise, the problems that arises from the identified gaps in literature are:

\begin{itemize}
	\item how to assign pre-scheduled BESS half-hourly power profiles to a three-phase network at sub-half-hourly resolution in order to yield the largest positive impact for the underlying power distribution system (\ref{objective-1}),
	\item how to adjust a half-hourly BESS schedule, which is based on a realistic but erroneous load forecast, based on sub-half-hourly load variations to minimise daily peak demands at both temporal resolutions (\ref{objective-2}),
	\item how large the impact will be on the performance of a scheduling and control algorithm when information exchange or message passing amongst the distributed control entities becomes desynchronised (\ref{objective-3}), and
	\item how multiple BESS can be coordinated in a communication-less environment to circumvent the need for ICT whilst contributing to voltage stability and thermal constraints without allocating their energy resources unevenly (\ref{objective-4}).
\end{itemize}

Despite some of the literature including aspects of the proposed research, none of them answer the research questions that are identified above.
The novelty of the research in this thesis consists of combining on-line and off-line control, as well as to assess and extend the control of distributed BESS.
All contributions and corresponding publications, as outlined in Section \ref{ch-introduction:sec:contributions} and \ref{ch-introduction:sec:publications}, respectively, reflect upon the novelty of the presented research against the objectives, their aim and gaps in literature upon which they are founded.
