\section{Topology and challenges of the UK low-voltage power distribution network}
\label{ch-literature:sec:topology-of-lv-network}

Today's electricity network in the UK has evolved over the past couple of centuries and is based on an interconnected high-voltage gird.
This grid is also known as the transmission network and it connects centralised power stations to the distribution networks.
Those distribution networks supply electricity to all loads across the UK, which include industrial customers as well as urban and rural domestic customers.

The entire structure of the network is a three-phase Alternating Current (AC) system since this allowed easy voltage level conversion with the use of transformers, i.e. without the need of power electronics.
In the UK, the highest voltage level for generation and transmission is 400kV.
Such a high voltage requires a relatively small current to transmit the generated bulk power, which in turn reduces conduction losses and maximises the efficiency of the high-voltage network.
Regional supply points step-down this high voltage to 132kV\footnote{In some cases regional supply points provide 127kV instead of 132kV.} to deliver power to Distribution Network Operators (DNOs).
From the primary level of the distribution network and onwards, this so called medium-voltage is stepped down to 33kV, then 11kV and finally 400V, in order to cater for heavy industry, medium clients and domestic customers, respectively.

\nomenclature[G]{DER}{Distributed Energy Resource}
\nomenclature[G]{ADMD}{After Diversity Maximum Demand}

All primary substations in the UK are equipped with voltage regulation equipment, i.e. on-line tap-changers, to increase or decrease their voltage on the secondary transformer side.
This function is to keep voltages of the distribution grid within statutory operating bands, i.e. 230V +10\% -6\%, as specified by the Electricity Supply Quality and Continuity Regulation (ESQRC) \cite{HealthandSafetyExecutive2002} and Engineering Recommendation G59 \cite{EnergyNetworksAssociation2013}.
In the UK, all customers at the 230V distribution level are connected to one of the three phases, and by randomly choosing their allocation, network balance was hoped to be achieved.
Also, for a long time customers were of a consumptive nature which meant that power flows from higher voltage levels towards the lower voltage levels.
Therefore, traditional network planning approaches to circumvent constraint violations, follow the commonly used practice of aggregating a large number of customers and designing the power delivery network to cater for their largest probable demand, i.e. the After Diversity Maximum Demand (ADMD) method \cite{Richardson2010a}.
Yet this ADMD method has remained the same for many years and uses historical load analysis and standard growth assumptions that are both no longer valid in this unprecedented LCT uptake scenario \cite{Yunusov2016}.


Firstly, because this assumption injection of power from Distributed Energy Resources (DERs), e.g. rooftop solar PV, can reverse the power flow, and secondly, larger electrified, e.g. home-charging Electric Vehicles (EVs), are predicted to significantly increase demand at peak times.
Combined with an increased phase unbalance, traditional network management methods can no longer cope with such challenges, which means that in situ equipment needs to be managed actively via innovation in the use of existing and new technologies; otherwise not only the frequency of constraint violations  will increase, but also the frequency of service disruptions and customer minutes is expected to rise alongside the proliferation of LCTs \cite{Ault2008a}.
One such technology, which is the main focus of the presented research, is the installation and management of battery storage \cite{Chen2009}.















