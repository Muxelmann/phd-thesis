\section{Energy storage projects in LV application}
\label{ch-literature:sec:energy-storage}

According to the Department of Energy's global energy storage database, there are more than 1200 energy storage projects, and as of 2016 27 of those are installed in the UK; they accumulate to a UK energy storage capacity of 33GWh \cite{Garton2016}.
61\% of all global energy storage projects use electro-chemical energy storage technology, and 49\% of those energy storage projects use Battery Energy Storage Systems (BESS) that are rated at less than 250kW.
Their corresponding small form-factor makes such BESS suitable for deployment in distribution networks, and as shown in the energy storage database 131 of these projects are indeed used for LV network support \cite{DOE-GESD}.

Different roles for energy storage have already been surveyed in Section \ref{ch-introduction:subsec:role-of-energy-storage-a-survey}. 
In this section however, Section \ref{ch-literature:sec:energy-storage}, applications within the LV networks are reviewed, since BESS impact on distribution networks is the main focus of the research that is presented in this thesis.
Projects and findings concerning the integration of LCTs is addressed in this literature review with particular emphasis on voltage control mechanisms and power flow management.

\subsubsection{Voltage control}

\nomenclature[G]{RER}{Renewable Energy Resource}

Due to the volatile nature of Renewable Energy Resources (RERs), management of BESS has become an active field of research \cite{Grillo2012, Rowe2014a, Li2016, Hosseina2016a}, in order to guarantee the supply of electric energy.
In cases where these RERs are not located in a centralised location like traditional power plants, but spread throughout the distribution network, issues arising from Distributed Energy Resources (DERs) need to be addressed, too.
Such issues include voltage deviation, phase unbalance and power factor degradation.

Challenges arising from DERs have been summarised in \cite{Ferreira2013a, Bravo2015}, where the most dominant challenge of limiting voltage deviation is being discusses.
In their studies, both local and global voltage control approaches are compared against an unmanaged BESS operation scenario, and through simulations of power distribution networks it is shown how voltage levels can be brought back within the statutory voltage band.
In the UK, this statutory voltage band is defined in the Electricity Supply Quality and Continuity Regulation (ESQRC) and ranges around 230V by +10\% and -6\%, i.e. 253V and 216V, respectively.

Electrical storage demonstration projects that took place within the UK have been summarised by Lyons et.al. in \cite{Lyons2015a}.


the opposite side of the spectrum, voltage drop due to new distributed loads, e.g. Electric Vehicles (EVs), 


