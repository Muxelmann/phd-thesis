\section{Overview}
\label{ch-literature:sec:overview}

With the ongoing electrification and decarbonisation of the heat and transport sector, demand across the electricity network is expected to double by 2050 \cite{Wilks2010}.
One contributor towards this increasing demand is the expected uptake of LCTs as they start to penetrate power distribution networks.
Also, this uptake of LCTs is expected to not progress evenly throughout the network; instead clusters of early adopters are predicted to form \cite{Poghosyan2014}.
Such a scenario will result in LV networks to exceed their operational constraints even whilst the rate of LCT adaptation is at a relatively low national rate \cite{Poghosyan2014}.
Conventional reinforcement to extend the network's capacity is effective but expensive.
Together with recent availability of load information, due to the distribution and installation of smart-meters, the opportunity arises for DNOs to develop energy storage control strategies in order to achieve the best possible performance and add most benefits to their distribution networks.

In fact, according to the Department of Energy's global energy storage database, there are more than 1200 energy storage projects worldwide.
In the UK, as of 2016, 27 of those are installed and accumulate to an energy storage capacity of 33GWh \cite{Garton2016}.
Out of all global energy storage projects, 61\% use ``electro-chemical energy storage technology'', i.e. rechargeable batteries, and 49\% of those Battery Energy Storage Systems (BESS) are rated at less than 250kW.
Their sizes and ratings make such BESS suitable for deployment in distribution networks, and the figures in the energy storage database indicate that worldwide 131 of these projects are indeed used for support of the secondary distribution network \cite{DOE-GESD}.

A general survey of different roles for energy storage has already been presented in Section \ref{ch-literature:sec:role-of-energy-storage-a-survey}.
However, to align with the focus of this thesis on improving UK power distribution networks, Chapter \ref{ch-literature} will provide an extensive review on BESS research and applications that support LV network operation; i.e. projects concerning voltage control mechanisms and power flow management.
The structure of this chapter is as follows.
First, in Section \ref{ch-introduction:subsec:topology-of-lv-network}, an overview of the UK power distribution network is given.
Then, the literature regarding energy storage projects and their applications in LV networks is reviewed in Section \ref{ch-literature:sec:energy-storage}.
Afterwards, in Section \ref{ch-literature:sec:control-of-energy-storage}, different control approaches for energy storage are reviewed and compared, and they are linked to the research objectives that were outlined in Section \ref{ch-introduction:sec:problem-statement}.
In the end, in Section \ref{ch-literature:sec:literature-gaps}, the gaps are summarised to emphasise the research contribution and support the problem statement of this thesis.
