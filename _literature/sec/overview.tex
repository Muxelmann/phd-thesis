\section{Overview}
\label{ch-literature:sec:overview}

With the ongoing electrification and decarbonisation of the heat and transport sectors in the UK, demand across the electricity network is expected to double by 2050 \cite{Wilks2010}.
One contributor towards this increasing demand is the expected uptake of LCTs as they start to penetrate power distribution networks.
As discussed in Chapter \ref{ch-introduction} of this thesis, conventional reinforcement to upgrade the network's infrastructure in order to counteract capacity shortages is effective but expensive.
Instead, this PhD research focuses on the improvement of grid operation by controlling BESS in the LV distribution network, and together with recent availability of load information, due to the distribution and installation of smart-meters, the opportunity arises for DNOs to develop energy storage control strategies in order to achieve the best possible performance and add most benefits to their distribution networks.

In fact, energy storage as an alternative to grid reinforcement has seen an increasing interest in industry since, according to the Department of Energy's global energy storage database, there are more than 1200 energy storage projects worldwide.
In the UK, as of 2016, 27 projects are installed and accumulate to an energy storage capacity of 33GWh \cite{Garton2016}.
Out of all global energy storage projects, 61\% use ``electro-chemical energy storage technology'', i.e. rechargeable batteries, and 49\% of those BESS are rated at less than 250kW.
Their sizes and ratings make such BESS suitable for deployment in distribution networks, and the figures in the energy storage database indicate that worldwide 131 of these projects are indeed used for support of the secondary distribution network \cite{DOE-GESD}.

The range of applications for energy storage in the electricity grid has grown significantly over the past decades.
Therefore, the first section of this chapter, Section~\ref{ch-literature:sec:role-of-energy-storage-a-survey}, presents an extensive survey of roles for electrical energy storage solutions and narrows the focus on those roles that are applicable for the conducted PhD research.
Section~\ref{ch-literature:sec:energy-storage} then provides an extensive review of already conducted BESS research projects that support LV network operation; i.e. concerning voltage control and power flow management.
Next, Section~\ref{ch-literature:sec:control-of-energy-storage} presents and reviews different control methods for grid connected energy resources that have either been used in the already discussed BESS research projects, or that have been studied theoretically.
Particular focus is put on comparing off-line and on-line control, centralised and distributed control, and communication-less control.
In the end of this chapter, in Section~\ref{ch-literature:sec:literature-gaps}, the gaps and research opportunities are summarised to link to the research contributions and support the problem statement of this thesis.
