\section{Overview}
\label{ch-review:sec:overview}

Electrical energy storage systems have been discussed for a long time, and their important role in future energy systems has been identified as early as the 70s by Kalhammer \cite{Kalhammer1979}. Phasing out oil and gas powered plants to replace them with sustainable energy sources, and enabling the uptake of future transportation (e.g. EVs) was already envisioned. Back then, from the grid operating perspective, power levelling was seen as the major contribution that energy storage systems could provide. This prediction turned out to be true since, according to the EPRI, 127GW of grid scale energy storage was installed using pumped hydro. This kind of of energy storage represented 99\% of the world's bulk energy storage capacity in 2012 \cite{TheEconomist2012a}. The requirement for increased energy storage capacity lead to a significant uptake in installed energy capacity, especially of pumped hydro-electric energy storage \cite{Rehman2015}. As shown in the latest reviews and datasets \cite{Barbour2016, Barbour2015}, Kalhammer's prediction was proven correct, since global pumped hydro has tripled since his publication.

Technology advancements and increasing popularity of renewable energy sources, combined with government incentives to support their uptake, lead to a significant rise in Distributed Energy Resources (DERs). Yet to allow DERs to be installed without significant negative impact on the local MV or LV networks, required more than large scale power levelling systems could provide. Alternative energy storage solutions were evaluated, and one of the first comprehensive reviews of these different energy storage technologies was published by McLaron and Cairns \cite{McLarnon1989} in 1989. Back then, the electricity grid was still predominantly supplied from fossil fuelled power plants, which resulted in their main arguments for energy storage being the reduction of network losses. Nowadays, the functions provided by energy storage services has become as diverse as the underlying technologies themselves. This becomes clear from several comprehensive reviews of different technologies that can be used for storing electrical energy \cite{Ibrahim2008, Chen2009, Hadjipaschalis2009, Luo2015}. These authors highlighted the need for ancillary services to support the integration of volatile renewable energy sources and proposed technologies covering smaller pumped hydro-electric, thermal, flywheel, compressed air, fuel cell, and battery energy storage. 

Whilst 