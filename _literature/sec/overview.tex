\section{Overview}
\label{ch-review:sec:overview}

Electrical energy storage systems have been discussed for a long time, and their important role in future energy systems has already been identified in the 70s by Kalhammer \cite{Kalhammer1979}.
Phasing out oil and gas powered plants to replace them with sustainable energy sources, and enabling the uptake of future transportation (e.g. EVs) was already envisioned.
Back then, power levelling functions were seen as the major contributions that could be provided by energy storage systems to the electricity grid.
Reasons behind this contribution were a reduction in network losses that are associated with ramping up and powering down the production of energy, so that it would match variations in demand.
With increasing demand for electricity, the power fluctuations increased accordingly, which in turn increased the demand for additional energy storage capacity, leading to a significant uptake in installed energy capacity.
According to the EPRI, 127GW of grid scale energy storage have been installed \cite{Rehman2015}, where 99\% of the world's bulk energy storage capacity consisted of pumped hydro-electric energy storage \cite{TheEconomist2012a}.
In combination with the latest reviews and datasets \cite{Barbour2016, Barbour2015}, Kalhammer's prediction on the demand for energy storage was therefore proven correct; capacity of global pumped hydro has tripled since his publication.
Nonetheless, despite its large scale, the benefits of such bulk energy storage systems have become more limited in today's power system.
The main reason behind this limitation is its lack of flexibility.

Whilst traditional power systems were designed for a smooth unidirectional energy transmission, recent DER installation caused highly fluctuating demand and sometimes bidirectional power flow.
Volatile renewable DERs like PV, have become popular due to government subsidisation.
In fact, due to the German government's feed in tariff, PV subsidisation triggered an annual capacity increase of nearly 33GW in 2012 \cite{Hockenos2013}.
Together with electrification of household appliances and the uptake of LTCs, household electricity demand is becoming more volatile \cite{Jewell1987}.
The mitigation of ramping and transmission losses is not the only 

Traditional power systems were designed to cater for unidirectional power flow.
Here, connecting a large scale energy storage system to the beginning of the MV network would allow planned injection of energy into the distribution network to counteract power spikes.
Losses caused by ramping up and down supply are mitigated.

Whilst traditional power systems were designed to cater for unidirectional energy delivery, DERs that are located at the end of the distribution network, inject energy into the system.
Mitigating $\text{CO}_2$ emissions, reducing energy costs and relieving stress from the power distribution networks are only some incentives behind the DER uptake \cite{Peng2009}.
High subsidisation of PV installations by the German government lead to a solar boom in 2012, with an annual capacity installation of nearly 33GW \cite{Hockenos2013}.
Beside the environmental and financial benefits, such renewable DERs do have a critical issue: increasing uncertainty of supply.
Whilst traditional supply could be regulated with ease, solar and wind power output are prone to cloud cover and varying wind speeds, respectively \cite{Jewell1987}.
The energy they feed into the power network is therefore highly volatile and, depending on the scale of the installed DER, may be large in magnitude, too.
Unlike bulk energy storage, this volatile supply causes more issues than benefits.

Technology advancements and increasing popularity of renewable energy sources, combined with government incentives to support their uptake, also lead to a significant rise in Distributed Energy Resources (DERs).
Yet to allow DERs to be installed without significant negative impact on the local MV or LV networks, required functions that large scale power levelling systems could not provide. More specifically, fast response to counteract highly volatile loads or unpredictable and distributed DERs; e.g. home PV installations \cite{Jewell1987}. Alternative energy storage solutions were evaluated, and one of the first comprehensive reviews of these different energy storage technologies was published by McLaron and Cairns \cite{McLarnon1989} in 1989. Back then, the electricity grid was still predominantly supplied from fossil fuelled power plants, which resulted in their main arguments for energy storage being the reduction of network losses. Nowadays, the functions provided by energy storage services has become as diverse as the underlying technologies themselves. This becomes clear from several comprehensive reviews of different technologies that can be used for storing electrical energy \cite{Ibrahim2008, Chen2009, Hadjipaschalis2009, Luo2015}. These authors highlighted the need for ancillary services to support the integration of volatile renewable energy sources and proposed technologies covering smaller pumped hydro-electric, thermal, flywheel, compressed air, fuel cell, and battery energy storage. 

Whilst 