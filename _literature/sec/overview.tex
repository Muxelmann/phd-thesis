\section{Overview}
\label{ch-literature:sec:overview}

With the ongoing electrification and decarbonisation of the heat and transport sector, demand across the electricity network is expected to double by 2050 \cite{Wilks2010}.
One main contribution towards this increasing demand is the expected uptake in LCTs as they start to penetrate power distribution networks.
The uptake of LCTs is however not expected to progress evenly throughout the entire power network, and instead clusters of early adopters are predicted to form.
Such a scenario result in certain LV networks to exceed their operational constraints even at relatively low national rate of LCT adaption \cite{Poghosyan2014}.
Conventional reinforcement to extend the network's capacity is effective but expensive.
Together with recent availability of load information, due to the distribution and installation of smart-meters, the opportunity arises for DNOs to develop energy storage control methods in order to achieve the greatest performance and add most benefits to their distribution networks.

This chapter, Chapter \ref{ch-literature}, is going to review the current literature on current energy storage projects and different control approaches for energy storage.
In the end, gaps in this literature are identified to highlight possible research contribution and support the problem statement of this work.