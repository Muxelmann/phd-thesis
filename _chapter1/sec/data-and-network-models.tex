\section{Network, Data and Battery Models}
\label{ch1:sec:data-and-network-models}

All key network parameters and their associated cost functions have now been established.
In this section the network model and power flow simulation, from which all aforementioned key network parameters were extracted is explained.
Also, all data that is used throughout the research, as it is presented in this chapter, is included presented, too.

\subsection{Standardised Network Model}
\label{ch1:subsec:standardised-network-model}

The IEEE Power and Energy Society (IEEE-PES) provided several multi-node test feeder cases.
These test cases used to be limited to distribution networks in the United States.
In 2015 however, they published a standardised model of a LV distribution network for the UK power network.
This model is called the ``European Low Voltage Test Feeder'' and an OpenDSS compatible version can be obtained from their website, too \cite{DistributionTestFeeders2017}.
Within the context of this work, the feeder is referred to as the ``LV Test Case'' and a network plot of this feeder has been included for reference.

\begin{figure}\centering
	\includegraphics[width=0.8\textwidth]{_chapter1/fig/network-plot-LVTestCase}
	\caption{A power flow plot of the IEEE-PES European Test Case Feeder, i.e. a LV distribution network in the UK.}
	\label{ch1:fig:network-plot-LVTestCase}
\end{figure}

Here, a substation (triangle in north west) provides power to the feeder, and the amount of power is visualised by using a thicker line plot.
There are 55 single-phase households connected to the substation and for this particularly case, are each customer drew 2kW.

\subsection{Load Profiles: Minutely and Half-Hourly}

Alongside the LV Test Case, 100 minutely demand profiles were supplied; each being 24h long.
Therefore, by assigning one load profile to each customer, a series of 1440 snapshot simulations could be run in OpenDSS in order to simulate the variation and volatility in demand over an entire day.
Since the provided power values are only of active power, a standardised power factor of 95\% was assumed for all loads to calculate their reactive component.
This non-unity power factor simulates a certain inductive load, which would otherwise be neglected.

Computing the ESMU schedule at minutely resolution, or any sub-half-hourly resolution, is very computationally demanding and highly ineffective since cutting edge power forecasts have only now reached half-hourly resolution.
Therefore, the sub-half-hourly profile must be down-sampled or extrapolated to a coarser resolution.
This extrapolation is achieved by first calculating the total daily demand profile at sub-half-hourly resolution, i.e. $s_{network\;\;load}(t)$:

\begin{equation}
	s_\text{net}(t) := \sum_{i=1}^{I} s_{\text{load},i}(t)
	\text{ where } I \in \textbb{Z}_{\geq 0}
\end{equation}

Then, the day is split into a number of half-hourly time-slots, $T^*$, and the average power was computed over each of those time-slots.
It it worth mentioning, that the number of time-slots for the sub-half-hourly load profile, $T$, must be divisible by the number of time-slots for the half-hourly (or coarser) profile $T^*$; i.e. $T \overset{!}{=} \alpha T^* \text{ where } \alpha \in \mathbb{N}$.
The resulting half-hourly network load, $s_{network\;\;load}(t^*)$, is thus extrapolated as follows:

\begin{equation}
	s^{*}_\text{net}(t) = \frac{1}{K}\sum_{\tau=k(t)K}^{(k(t)+1)K-1}s_\text{net}(\tau) \forall t \text{ where } K \in \textbb{Z}_{>0}
	\label{ch1:equ:down-sampling-to-half-hourly}
\end{equation}

To illustrate how the original sub-half-hourly network load is extrapolated into the resulting half-hourly demand, both are plotted in a figure below.

\begin{figure}\centering
	\includegraphics{_chapter1/fig/sub-half-horuly-demand-comparison}
	\caption{Highly variable and volatile demand profile vs half-hourly demand (i.e. a forecast under perfect foresight conditions)}
	\label{ch1:fig:sub-half-horuly-demand-comparison}
\end{figure}

Here, in Figure \ref{ch1:fig:sub-half-horuly-demand-comparison}, it can be observed how the high variability and volatility in power is removed in the half-hourly profile.
When generating the ESMU schedules (which is addressed in Section \ref{ch1:subsec:esmu-schedule-generation}) these variations are neglected and the unwanted peak power demands are hence no longer sufficiently compensated.

\subsection{Battery Model}

The deployed ESMU systems themselves consist of two parts: the Power Management Unit (PMU) and the Energy Storage Unit (ESU).
The PMU controls the phase powers converters and links the battery packs to the grid.
It is rated for a maximum per phase power of 12kVA and, beside battery charging and discharging, can perform filtering functions, e.g. compensating for  harmonic distortion.
The ESU is a modular container of 12.5kWh of Li-Ion energy storage, that can be aggregated to increase the energy storage capacity.
All battery monitoring, conditioning and regulation is performed internally and need not be considered for the scope of this work.
However, control instructions that are sent to the ESMU system itself must not request the device to operate outside its own specifications.

In order to simulate this ESMU system and its energy flow behaviour, a model had to be derived from the given device specifications.
This model includes both an inversely proportional charge-discharge efficiency, and a standby loss.
The charge-discharge efficiency is related to the efficiency of the PMU's power converters, which are quoted to have a round trip efficiency of 98\%.
The standby loss on the other hand is associated to the nominal control system's power demand as well as the battery's self-discharge rate.
It has become common practice to assess an energy storage's charge level as the State of Charge (SOC).
This SOC is described as:

\begin{equation}
	SOC(t) := \frac{E_\text{bat}(t)}{C_\text{bat}}
	\label{ch1:equ:state-of-charge-definition}
\end{equation}

Here, the SOC is defined for any given time $t$, as the actual energy stored in the ESU, $E_{battery}(t)$, divided by the capacity of the system, $C_{battery}$.
With the conversion efficiency as the factor $\eta$, the battery charge and discharge power, $s_{battery}(t)$ (in kW)\footnote[1]{It is worth mentioning, that $s$ is chosen for apparent power (i.e. active and reactive power in kVA), yet $s_{batery}$ is purely active (i.e. in kW).}, can be calculated for any given ESMU power, $\textbf{s}_{ESMU}(t)$ (where $s_{ESMU,p}(t) \in \textbf{s}_{ESMU}(t)$).

\begin{equation}
	s_{battery}(t) = 
	\begin{cases}
		\eta\text{Re}\left\{\sum_{p=1}^{P}s_{ESMU,p}(t)\right\} &\text{ if } \sum_{p=1}^{P}s_{ESMU,p}(t) \geq 0\\
		\frac{1}{\eta}\text{Re}\left\{\sum_{p=1}^{P}s_{ESMU,p}(t)\right\} &\text{ otherwise}
	\end{cases} \forall t \text{ and } P \in \mathbb{N}
	\label{ch1:equ:battery-power-definition}
\end{equation}

Assuming this charge/discharge power is applied and constant during a sample period of $t_s$, an equation for the optimum change in stored energy can be formulated.

\begin{equation}
	\Delta E_\text{bat}(t) = s_\text{bat}(t)\Delta t \forall t
	\label{ch1:equ:change-in-energy}
\end{equation}

As already mentioned, the energy level is however affected by standby losses, which are captured by a self-discharge factor, $\mu$.
Therefore, when adding the change in energy level to the current energy level, in order to calculate the next energy level, this factor needs to be taken into account.

\begin{equation}
	E_\text{bat}(t+\Delta t) = \mu\left(\Delta E_\text{bat}(t) + E_\text{bat}(t)\right)
\label{ch1:equ:next-energy}
\end{equation}

In an ideal case, $\mu = 1$, where no energy would be lost in the storage system.
Finally, by adding the current energy level to the change in energy level, and by substituting Equation \ref{ch1:equ:state-of-charge-definition}, the next SOC can be found:

\begin{equation}
	SOC(t+1) = \mu\left(\frac{s_\text{bat}(t)\Delta t}{C_\text{bat}} + SOC(t)\right) \forall t
	\label{ch1:equ:next-state-of-charge}
\end{equation}

A flowchart to visually capture the model, as it is explained from Equation \ref{ch1:equ:state-of-charge-definition} to \ref{ch1:equ:next-state-of-charge}, is presented below.

\begin{figure}\centering
% Define some block styles
\tikzstyle{input} = [%
	draw,%
	ellipse,%
	fill=green!20,%
	minimum height=2em,%
]
\tikzstyle{result} = [%
	draw,%
	ellipse,%
	fill=blue!20,%
	minimum height=2em,%
]
\tikzstyle{output} = [%
	draw,%
	ellipse,%
	fill=yellow!20,%
	minimum height=2em,%
]
\tikzstyle{decision} = [%
	diamond,%
	draw,%
	text width=4.5em,%
	text badly centered,%
	inner sep=0pt%
]
\begin{tikzpicture}[node distance=3cm, shorten >= 1pt, >=stealth', auto]

	% Define nodes
    \node (power_esmu) [input] {$\textbf{s}_\text{ESMU}(t)$};
    \node (is_charging) [decision, right of=power_esmu] {charging\footnotemark[2]};
    \node (charging) [state, above right of=is_charging] {$\times\eta$};
    \node (discharging) [state, below right of=is_charging] {$\times\frac{1}{\eta}$};
    \node (power_battery) [result, below right of=charging] {$s_\text{bat}(t)$};
    \node (sample) [state, right of=power_battery] {$\times \Delta t$};
    \node (change_energy_battery) [result, below of=sample, yshift=5mm] {$\Delta E_\text{bat}(t)$};
    \node (add) [state, below of=change_energy_battery, yshift=5mm] {$+$};
    \node (current_energy_battery) [result, left of=add] {$E_\text{bat}(t)$};
    \node (multiply) [state, left of=current_energy_battery] {$\times C_\text{bat}$};
    \node (current_soc) [input, left of=multiply] {$SOC(t)$};
    \node (losses) [state, below of=add, yshift=5mm] {$\times \mu \Delta t$};
	\node (next_energy) [result, left of=losses, xshift=-5mm] {$E_\text{bat}(t+1)$};
	\node (divide) [state, left of=next_energy, xshift=-5mm] {$\div C_\text{bat}$};
	\node (next_soc) [output, left of=divide, xshift=-5mm] {$SOC(t+1)$};

	% Draw lines
	\draw [->] (power_esmu) to (is_charging);
	\draw [->, bend left] (is_charging) to node {yes} (charging);
	\draw [->, bend right] (is_charging) to node [swap] {no} (discharging);
	\draw [->, bend left] (charging) to (power_battery);
	\draw [->, bend right] (discharging) to (power_battery);
	\draw [->] (power_battery) to (sample);
	\draw [->] (sample) to (change_energy_battery);
	\draw [->] (change_energy_battery) to (add);
	\draw [->] (current_soc) to (multiply);
	\draw [->] (multiply) to (current_energy_battery);
	\draw [->] (current_energy_battery) to (add);
	\draw [->] (add) to (losses);
	\draw [->] (losses) to (next_energy);
	\draw [->] (next_energy) to (divide);
	\draw [->] (divide) to (next_soc);
	
\end{tikzpicture}
\caption{Flowchart to calculate the next SOC (i.e. $SOC(t+\Delta t)$) based on current ESMU power (i.e. $\textbf{s}_\text{ESMU}(t)$) and current SOC (i.e. $SOC(t)$)}
\end{figure}

\footnotetext[2]{In the flowchart ``charging'' implies that $\sum_{\phi=1}^{\Phi}s_{\text{ESMU},\phi}(t) \geq 0$ as explained in Equation \ref{ch1:equ:battery-power-definition}.}

Here, all green and blue fields indicate, respectively, model inputs and results.
The white states represent operations applied onto those inputs and results and in the end yield the output, i.e. the yellow field.
