\section{Summary}
\label{ch1:sec:summary}

In this chapter, a method to adjust an ESMU schedule to support network operation has been developed and tested.
The underlying schedule was a half-hourly dis/charge profile that the battery had to follow.
Through the adjustment of the three-phase ESMU power, both active and reactive power could be adjusted in order to minimise a set of defined costs.
The results indicate, that when explicitly focusing on the improvement of certain key network parameters, then in every case the derived cost reduced.
Furthermore, for most cost minimisation attempts, different costs were impacted, too (i.e. loss minimisation impacted nearly all other costs).
Using cumulative cost differences it was shown that a net cost reduction could be achieved, when implementing the proposed schedule adjustment method.
\textcolor{red}{\hl{This fact is also the reason why the resulting probability density analysis, using the two-paired $t$-test and Kolmogorov-Smirnov test, disproved the null hypothesis as it was formulated for this chapter.}}
Therefore, there is strong evidence that adjustments of ESMU schedules on a sub-half-hourly bases have positive impacts on the distribution network's operation.

The main limitation of the proposed method is however the battery's half-hourly schedule.
It dictates the total active power that has to be injected into or absorbed from the distribution network.
Also, this schedule inadvertently dictates the remaining overhead in reactive power that may be compensated for on each phase.
Therefore, the next chapter addresses how the active power schedule may be corrected in real-time.