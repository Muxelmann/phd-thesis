\section{Summary}
\label{ch1:sec:summary}

In this chapter, a method to adjust three-phase ESMU powers on a sub-half-hourly basis to support network operation, whilst following a pre-determined half-hourly schedule, is proposed and tested.
The ESMU schedule is tailored to result in a ``peak-shaving'' and ``valley-filling'' behaviour and uses a realistic ESMU model to meet any operational constraints.
A set of key network parameters to indicate the performance of the network, were used in a corresponding set of cost functions.
By adjusting the ESMU's active and reactive powers, each cost could be minimised and therefore network operation is improved.

Results indicate that when explicitly focusing on the improvement of certain key network parameters, then the derived cost reduces for every single case.
Nonetheless, any cost minimisation had an effect on different costs (e.g. loss minimisation positively impacted nearly all other costs).
Using cumulative cross-cost differences, it is shown that a net cost reduction is achieved, simply by implementing the proposed ESMU power adjustment method top of the normal execution of a half-hourly schedule.
This fact is also supported by the statistical sensitivity analysis, using the two-paired $t$-test.
Using this test after having correctly preconditioned the data to meet the test's prerequisites, the initial null hypothesis is successfully be disproven; i.e. the hypothesis that sub-half hourly power adjustments have no impact.
Hence there is strong evidence that those power adjustments do have a positive impact on the distribution network's operation, and the first objective of this thesis, which is outlined in Section~\ref{ch-introduction:sec:problem-statement}, has been met.

The main limitation of the proposed method is however the battery's half-hourly schedule.
It dictates the active power that has to be injected into or absorbed from the distribution network.
Also, this schedule inadvertently dictates the remaining overhead in reactive power that may be compensated for on each phase.
Therefore, the next chapter in this thesis presents a method of dynamically adjusting this scheduled power profile in real-time without violating any physical constraints.
