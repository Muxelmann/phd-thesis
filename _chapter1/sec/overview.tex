\section{Overview}
\label{ch1:sec:overview}

Due to the previously identified trend in energy demand, future network load is expected to increase in both magnitude and volatility.
As a result, DNOs have two choices to address the issues that are expected to result from increased network stress.
They can either invest in network reinforcement or install network support equipment.
Since the former option has more costs than benefits, which has been outlined in Chapter \ref{ch-introduction} (e.g. decommissioning cost, installation cost, service disruption, etc.), the installation of network support equipment was favoured.
As mentioned in Section \ref{ch-introduction:sec:ntvv}, \textit{SSEN} decided to deploy an Energy Storage Management Unit (ESMU) in some of their Low-Voltage (LV) power distribution networks.
Within the scope of their research, ESMU had to be controlled to benefit the network, without exceeding or violating its operational constraints.
In order to achieve this kind of operation, ESMU was scheduled.
During this scheduled operation of ESMU, the system either consumes or injects power, according to a predetermined plan that changes at regular intervals.
For historic reasons and system compliance, this interval was chosen to be of 30 minutes or at half-hourly period.

Since the ESMU schedule was generated based upon a demand forecast, any resulting impact on the LV network operation is then based upon two factors:

\begin{enumerate}
	\item quality of the underlying forecast that was used to generate the ESMU schedule, and
	\item network parameters that are used to quantify the improvements that were experienced by the network when the half-hourly schedule is applied.
\end{enumerate}

Our previous research focused on improving half-hourly network operation to e.g. reduce peak load \cite{Rowe2014a, Yunusov2011}.
However, in that research, high-resolution demand variability has not been taken into account.
Therefore, the previously used performance parameters and the corresponding measure of success did not effectively address the ESMU's capability at mitigating negative impacts from minutely (or sub-half-hourly) demand.

In this chapter, a closed-loop optimisation method is proposed that adjusts the scheduled phase powers at a sub-half-hourly resolution, in order to improve network operation, whilst maintaining the same average active power flow during the corresponding half-hourly period.
Unlike previous work in the field, this approach guarantees the correct execution of the predetermined ESMU schedule, despite allowing ESMU to respond to high-resolution variations in network load.

In order to investigate how network operation may be improved, a collection of commonly used parameters are summarised in a set of cost functions.
Initially, these cost functions are minimised on an individual basis to inspect their separate impact.
Then, all cost functions are combined using the method of a weighted sum, forming a global cost function, which is used for the final analysis of positive impact.
For each optimisation approach, power flow simulations are run on a standardised UK power distribution feeder model in the simulation environment OpenDSS.
To statistically determine that the proposed cost minimisation approach resulted in better network performance, a one-tailed null hypothesis is formulated, and rejected with $p<0.05$.
The null hypothesis for this piece of work (i.e. the assumption to be rejected), is:

\textit{Adjusting an ESMU schedule based on sub-half-hourly measurements of key network parameters cannot significantly improve network operation.}

The term ``\textit{network measurements}'' and the correlated measure of ``\textit{improvement}'' are explained in the next section.
All acquired data and the power network models used for this piece of work are shown afterwards.
Subsequently, the closed-loop optimisation method is presented.
At the end of this chapter, all results are presented, discussed and summarised.
