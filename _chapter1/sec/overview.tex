\section{Overview}
\label{ch1:sec:overview}

Since the increasing domestic demand for electric energy is expected to put significant strain on existing power distribution networks, DNOs have a binary choice to address this issue.
Either they invest in substantial network reinforcement, resulting in unavoidable cost and service disruption, or alternative network support mechanisms need to be installed.
As mentioned in the Introduction in Section \ref{ch-introduction:sec:ntvv}, DNOs prefer the latter option since it yields long term flexibility at significantly lower cost; more specifically, SSE-EN decided to deploy ESMU in LV power distribution networks.
To operate the ESMU so that its promised battery's life expectancy is reached and operational constraints are not violated, scheduled operation was chosen.
To reiterate, during the scheduled operation of a ESMU it consumes or injects power according to a preset plan, which changes at regular intervals.
For historic reasons and compliance, this interval was chosen to be of 30 minutes or half-hourly period.

Resulting operation of the LV network is based upon two factors:

\begin{enumerate}
	\item the underlying forecast that was used to generate the ESMU schedule, and
	\item the network parameters used to quantify the improvements that were experienced by the network if the half-hourly schedule had been applied.
\end{enumerate}

Our previous research focused on improving half-hourly network operation \cite{Rowe2014a, Yunusov2011}.
Yet limiting performance parameters and the measure of success to a high level at half-hourly resolution does not effectively address and mitigate the negative impact of sub-half-hourly (i.e. minute by minute) variations in demand.

Therefore, in this chapter, a closed-loop optimisation method is proposed that adjusts a ESMU schedule in a sub-half-hourly manner in order to improve network operation whilst maintaining the same average power flow during the half-hourly period.
Unlike previous work in the field, this approach guarantees the correct execution of the predetermined ESMU schedule, whilst allowing higher ESMU responsiveness to high resolution variations in power demand.

In order to investigate how network operation may be improved, several, commonly used parameters are summarised in a set of cost functions.
Initially, these cost functions are minimised individually to inspect their resulting individual impact.
Then, they are combined in an optimal manner using a weighted sum cost function, and the final improvement is analysed.
For each optimisation approach, power flow simulations are run on a standardised UK power distribution feeder.
Finally, to statistically determine that the proposed cost minimisation approach resulted in better network performance, a one-tailed null hypothesis is formulated, and rejected with $p<0.05$.
The null hypothesis for this piece of work (i.e. the assumption to be rejected), is:

\textit{Adjusting a ESMU schedule based on instantaneous network measurements yields no significant improvement in network operation.}

The term ``\textit{network measurements}'' and the correlated measure of ``\textit{improvement}'' is explained in the next section.
All data acquisition and the power network models used for this piece of work are shown next.
Then, the closed-loop optimisation method is presented.
All results are discussed and, in the last section, the null hypothesis rejection is is summarised.
