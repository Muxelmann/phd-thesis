\section{Introduction}
\label{ch1:sec:introduction}

The adoption of electric vehicles (EV) is seen as a potential solution to the decarbonisation of future transport networks, offsetting emissions from conventional internal combustion engine vehicles. The current rate of EV uptake is anticipated to increase with improved driving range, reduced cost of purchase and greater emphasis on leading an environmentally-friendly lifestyle \cite{Shah2015}. It is predicted that by 2030, there will be three million plug-in hybrid electric vehicles (PHEV) and EVs sold in Great Britain and Northern Ireland \cite{DBER2008}, and it is expected that by 2020, every tenth car in the United Kingdom will be electrically powered \cite{Ecolan2013}. It is anticipated that the majority of PHEV/EV will be charged at~home, putting additional stress on the existing local low voltage distribution network, which must then cater for the increased demand in energy \cite{Clement-Nyns2010, PieltainFernandez2011}. Uncontrolled charging of multiple PHEV/EV can raise the daily peak power demand, which leads to: increased transmission line losses, higher voltage drops, equipment overload, damage and failure \cite{Hadley2009, Putrus2009, Pillai2010, Zhou2014}. Accommodating the increased demand and mitigation of such failures is a major area of research interest, with the focus mainly placed on the coordinating and support of home charging.

Demand Side Management (DSM) strategies for Distributed Energy Resources (DER), aim to alleviate the impacts of PHEV/EV home-charging and are a favoured solution. Mohsenian-Rad et al. in \cite{Mohsenian-Rad2010} developed a distributed DSM algorithm that implicitly controls the operation of loads, based~on game theory and the network operator's ability to dynamically adjust energy prices. Focusing on financial incentive-driven DSM strategies, in \cite{Deilami2011}, a Time-Of-Use (TOU) tariff and real-time load management strategy was proposed, where disruptive charging is avoided by allocating higher prices to times of peak demand. Financial incentives have also become a drive towards optimising the operation of Battery Energy Storage Solutions (BESS) and Distributed Generation (DG) when including PHEV/EV into the problem formulation \cite{Masoum2015}.

Research focused on grid support has been driven by the need to deliver long-term savings and to avoid the immediate costs and disruption of network reinforcements and upgrades. This~area of research proposes the implementation of alternative solutions to support the adoption of low carbon technologies, such as EVs, heat pumps and the electrification of consumer products. To reduce the resulting increased peak demand, Mohsenian-Rad et al. developed an approach of direct interaction between grid and consumer to achieve valley-filling, by means of dynamic game theory \cite{Mohsenian-Rad2010}. In \cite{Karfopoulos2013}, a Multi-Agent System (MAS) was used to manage flexible loads for the minimisation of cost in a dynamic game. The use of aggregators has been proposed to allow the participation of a number of small providers to participate in network support, such as grid frequency response \cite{Wu2012a, Samadi2012, Xu2015b}. Yet~without the availability of power demand forecasts, real-time control needs to be implemented.

Real-time DSM can either be implemented in a centralised or distributed control approach. In~the~former, a central controller relays control signals to its aggregated DERs, whereas the latter allows each DER to control itself. A common form of controlling DERs in this mode of operation is set-point control \cite{Leadbetter2012685}. Using set-point control on multiple identically-configured DERs would yield optimal operation conditions if each DER's control parameters (e.g., bus voltage) were~shared. In a system without sharing network information, DER control algorithms have to be improved to prevent, for example, devices located furthest from the substation from being used more frequently than others.

This paper therefore presents an individualised BESS control algorithm that lets distributed batteries respond to fluctuations in real-time local bus voltage readings. The proposed algorithm is based on the robust Additive Increase Multiplicative Decrease (AIMD) type algorithm, yet implements a set-point adjustment based on the location of the controlled BESS. It will be shown how these home-connected batteries can mitigate the impact of additional loads (i.e., EV uptake), whilst assuring that all BESS are cycled equally.

The key contribution of this work can be summarised as a novel distributed battery storage algorithm for mitigating the negative impact of dynamic load uptake on the low-voltage network. This~algorithm uses an individualised set-point control to regulate bi-directional battery power flow~and, for convergence, extends the traditional AIMD algorithm. As a result, the developed battery control method reduces voltage deviation, over-currents and the inequality of battery usage. Reducing this usage inequality leads to a homogeneous usage of all of the distributed batteries and,~hence, prevents unequal degradation rates and unfair device utilisation.

The remainder of this paper is organised as follows: Section \ref{ch1:sec:related-work} gives some background to related work on AIMD algorithms on which this research is based. Section \ref{ch1:sec:system-modelling} outlines the EV, network and storage models used in the research. Additionally, it explains the assumptions that accommodate and justify these models. Section \ref{ch1:sec:storage-control} elaborates on the proposed AIMD control algorithm (AIMD+). Next, Section \ref{ch1:sec:scenarios-and-comparison-metrics} details the implementation and scenarios used for a set of test cases. For later comparison, this section also outlines a set of comparison metrics. Section \ref{ch1:sec:results-and-discussion} presents and discusses the results, followed by the conclusion in Section \ref{ch1:sec:conclusion}.