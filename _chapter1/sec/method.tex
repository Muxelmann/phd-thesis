\section{Optimisation method}
\label{ch1:sec:closed-loop-optimisation-method}

Previously, in Section~\ref{ch1:sec:key-network-parameters}, the key network parameters and associated cost functions have been established, and the data, models and schedule generation is explained in Section~\ref{ch1:sec:data-and-network-models}.
A summary of all key network parameters that are used in this work is listed below, but the full notation is defined in the nomenclature of this thesis:

\begin{itemize}
	\item substation phase voltages, $\textbf{v}_{ss}(t) = (v_{\text{ss},\phi}(t))$ where $\textbf{v}_{ss}(t) \in \mathbb{R}^\Phi$,
	\item ESMU phase voltages, $\textbf{v}_\text{ESMU}(t) = (v_{\text{ESMU},\phi}(t))$ where $\textbf{v}_\text{ESMU}(t) \in \mathbb{C}^\Phi$,
	\item all load voltages, $\textbf{C}_\text{load}(t) = (v_{\text{load},i,\phi}(t))$ where $\textbf{V}_\text{load}(t) \in \mathbb{C}^{I\times\Phi}$,
	\item substation apparent phase power, $\textbf{s}_{ss}(t) = (s_{\text{ss},\phi}(t))$ where $\textbf{s}_{ss}(t) \in \mathbb{C}^\Phi$,
	\item substation phase currents, $\textbf{i}_{ss}(t) = (i_{\text{ss},\phi}(t))$ where $\textbf{i}_{ss}(t) \in \mathbb{R}^\Phi$,
	\item all line currents, $\textbf{I}_\text{line}(t) = (i_{\text{line},l,\phi}(t))$ where $\textbf{I}_\text{line}(t) \in \mathbb{R}^{L\times\Phi}$, and
	\item all network losses, $s_\text{losses}(t)$ where $s_\text{losses}(t) \in \mathbb{C}$.
\end{itemize}

In this section, the method of adjusting the predetermined ESMU schedule on a sub-half-hourly basis is presented.
This method is designed to improve network performance which in turn is indicated by the aforementioned key network parameters.
After detailing the method itself the generation and assessment of all results are explained. 

\subsection{Closed-loop schedule adjustment}

A global cost function is generated to summarise and combine all eight costs that were derived from the key network parameters, and that were defined in Equations~\ref{ch1:equ:voltage-deviation}, \ref{ch1:equ:load-voltage-deviation}, \ref{ch1:equ:unbalance-cost}, \ref{ch1:equ:power-factor}, \ref{ch1:equ:neutral-load}, \ref{ch1:equ:losses}, \ref{ch1:equ:fuse-utilisation}, and \ref{ch1:equ:line-utilisation}.
This cost function is defined as follows:

\begin{multline}
	\zeta(\textbf{v}_\text{ss}(t), \textbf{v}_\text{ESMU}(t), \textbf{v}_{\text{load}}(t), \textbf{s}_{ss}(t), \textbf{i}_{ss}(t), \textbf{i}_{\text{line}}(t), s_\text{losses}(t), \boldsymbol{\alpha}) :=\\
	\alpha_1 \sum_{\phi=1}^\Phi\zeta_\text{voltage}(v_{\text{ss},\phi}(t))%\\
	+ \alpha_2 \sum_{\phi=1}^\Phi\zeta_\text{voltage}(v_{\text{ESMU},\phi}(t))%\\
	+ \alpha_3 \zeta_\text{\text{load} voltage}(\textbf{v}_{\text{load}}(t))\\
	+ \alpha_4 \zeta_\text{unbalance}(\textbf{s}_{ss}(t))%\\
	+ \alpha_5 \zeta_\text{PF}(\textbf{s}_{ss}(t))%\\
	+ \alpha_6 \zeta_{\text{neutral load}}(\textbf{s}_{ss}(t))\\
	+ \alpha_7 \zeta_\text{fuse utilisation}(\textbf{i}_{ss}(t))%\\
	+ \alpha_8 \zeta_\text{\text{line} utilisation}(\textbf{i}_{\text{line}}(t))%\\
	+ \alpha_9 \zeta_\text{losses}(s_\text{losses}(t)) \\
	 \text{where } \phi \in \{1, \dots, \Phi\} \text{ and } \Phi \in \mathbb{Z}_{>0} \text{ and } \boldsymbol{\alpha} = \{\alpha_1, \dots, \alpha_9\}
\label{ch1:equ:weighted-sum-cost-function}
\end{multline}

Here, $\boldsymbol{\alpha}$ is a binary choice vector with which the weight of the global cost function can easily be adjusted.
In other words, this vector allows the global cost to target any specific key network improvement which is based on a specific cost.
To simplify the notation, and since all key network parameters are outputs of the power flow simulations and not directly adjustable, the global cost function is shortened to $\zeta(\boldsymbol{\alpha})$.

\begin{figure}\centering
% Define some block styles
\tikzstyle{title} = [%
	rectangle,%
%	minimum height=2em,%
	text centered%
]
\tikzstyle{function} = [%
	rectangle,%
	draw,%
	fill=white,%
%	minimum height=2em,%
	rounded corners=5pt,%
]

\begin{tikzpicture}[node distance=3cm, shorten >= 1pt, >=stealth', auto, anchor=west]
	
	\def\title_offset{-1.33};
	\def\image_offset{-2.66};
	
	\node at (0, 0) (closed_loop_optimisation) [title] {Closed-Loop Optimisation};

	\node at (0, \title_offset+\image_offset) (opendss_logo) [rectangle] {\includegraphics[height=7.5mm]{_chapter1/fig/OpenDSS}};
	\node (simulation_text) [rectangle, right of=opendss_logo, xshift=-1.5cm] {Simulation};
	\node (simulation) [function, fit={(opendss_logo) (simulation_text)}] {};
	\node at (0, \title_offset+\image_offset) (opendss_logo) [rectangle] {\includegraphics[height=7.5mm]{_chapter1/fig/OpenDSS}};
	\node (simulation_text) [rectangle, right of=opendss_logo, xshift=-1.5cm] {Simulation};
	
	\node at (0, \title_offset) (matlab_cost) [rectangle] {\includegraphics[height=7.5mm]{_chapter1/fig/MATLAB}};
	\node (cost_function_text) [rectangle, right of=matlab_cost, xshift=-1.15cm] {Cost Function};
	\node (cost_function) [function, fit={(matlab_cost) (cost_function_text)}] {};
	\node at (0, \title_offset) (matlab_cost) [rectangle] {\includegraphics[height=7.5mm]{_chapter1/fig/MATLAB}};
	\node (cost_function_text) [rectangle, right of=matlab_cost, xshift=-1.15cm] {Cost Function};
	
	\node at (7, \title_offset) (matlab_optimiser) [rectangle] {\includegraphics[height=7.5mm]{_chapter1/fig/MATLAB}};
	\node (optimiser_text) [rectangle, right of=matlab_optimiser, xshift=-1.5cm] {Optimiser};
	\node (optimiser) [function, anchor=center, fit={(matlab_optimiser) (optimiser_text)}] {};
	\node at (7, \title_offset) (matlab_optimiser) [rectangle] {\includegraphics[height=7.5mm]{_chapter1/fig/MATLAB}};
	\node (optimiser_text) [rectangle, right of=matlab_optimiser, xshift=-1.5cm] {Optimiser};
	
	\node at (8, \title_offset+\image_offset) (addition) [state, fill=white] {$+$};
	
	\node at (12.5, \title_offset-2) (battery) [rectangle] {\includegraphics[height=15mm]{_chapter1/fig/battery}};
	\node (model_text) [rectangle, below of=battery, yshift=1.9cm] {Model};
	\node (model) [function, fit={(battery) (model_text)}] {};
	\node at (12.5, \title_offset-2) (battery) [rectangle] {\includegraphics[height=15mm]{_chapter1/fig/battery}};
	\node (model_text) [rectangle, below of=battery, yshift=1.9cm] {Model};
	
	\node (output) [function, below of=addition, fill=yellow!20, yshift=1cm, inner sep=1em] {Output};
	
	\begin{scope}[on background layer]
		\node [draw=black, fill=green!20, fit={%
			(closed_loop_optimisation)%
			(simulation)%
			(cost_function)%
			(optimiser)}] {};
	\end{scope}

	\draw [->, bend left] (simulation) to node {key parameters} (cost_function);
	\draw [->] (cost_function) to node [above] {$\zeta(\boldsymbol{\alpha})$} (optimiser);
	\draw [->, bend left] (optimiser) to node [left] {$\delta \textbf{s}_{ESMU}(t)$} (addition);
	\draw [->] (addition) to node [above] {$\textbf{s}_{ESMU}(t) + \delta \textbf{s}_{ESMU}(t)$} (simulation);
	\draw [->] (model.west|-addition) to node [above, pos=0.3] {$\textbf{s}_{ESMU}(t)$} (addition);
	\draw [->, dashed, bend right, color=red] (model.north-|model.center) to node [midway, above, yshift=1mm] {constraints} (optimiser.east|-optimiser.center);
	\draw [->] (addition) -- (output);
\end{tikzpicture}
\caption{ESMU schedule adjustment flow diagram}
\label{ch1:fig:closed-loop-optimisation}
\end{figure}

\nomenclature[I]{$\delta\textbf{s}_\text{ESMU}(t)$}{Three-phase apparent ESMU power adjustment vector at time $t$, where $(\delta s_{\text{ESMU},\phi}(t)) = \delta \textbf{s}_\text{ESMU}(t)$ and $\delta \textbf{s}_\text{ESMU}(t) \in \mathbb{C}^\Phi$}
\nomenclature[I]{$\delta s_{\text{ESMU}, \phi}(t)$}{Three-phase apparent ESMU power adjustment vector at time $t$, where $(\delta s_{\text{ESMU},\phi}(t)) = \delta \textbf{s}_\text{ESMU}(t)$ and $\delta s_{\text{ESMU},\phi}(t) \in \mathbb{C}$}

The underlying method that performs the proposed closed-loop optimisation is shown in Figure~\ref{ch1:fig:closed-loop-optimisation}.
For each time-slot, $t$, the pre-scheduled ESMU power vector, $\textbf{s}_\text{ESMU}(t)$, is extracted and adjusted by an offset vector, $\delta \textbf{s}_\text{ESMU}(t)$.
An optimal offset vector is found through iterative optimisation to minimise the global cost function, $\zeta(\boldsymbol{\alpha})$.
This minimisation is achieved by repetitively running power flow simulations of the IEEE distribution feeder and adjusting $\delta\textbf{s}_\text{ESMU}(t)$.
Once the adjusted ESMU schedule (i.e. $\textbf{s}_\text{ESMU}(t) + \delta \textbf{s}_\text{ESMU}(t)$) has converged and a solution has been found, then the closed-loop optimisation process terminates and the simulation begins optimising the next time slot (i.e. $t+\Delta t$).
One additional constraint is defined since $\delta \textbf{s}_\text{ESMU}(t)$ must not impact the underlying half-hourly ESMU schedule.
This constraint assures that the sum of all phase powers in the adjustment vector equates to zero; hence keeping the internal battery's charging-discharge power the same.
Including the previously mentioned battery system constraints which ensure that the ESMU operates within its technical limitations, the minimisation problem for the closed-loop optimisation mechanism is formulated as follows:

\begin{equation}
\begin{split}
	\min_{\delta \textbf{s}_{ESMU}(t)} \zeta(\boldsymbol{\alpha})
	&\text{ s.t.}
	\begin{cases}
		\sum_{p=1}^{P} \text{Re} \left(\delta s_{ESMU,p}(t)\right) = 0 \forall t\\
		s_{battery}(t) \leq C_{factor}\times C_{battery} \forall t\\
		\left|s_{ESMU,p}(t)\right| \leq S_{rating} \forall p \forall t\\
		0 \leq SOC(t) \leq 1 \forall t
	\end{cases}\\
	&\text{ where } \delta s_{ESMU,p}(t) \in \textbf{s}_{ESMU}(t) \text{ and } P \in \mathbb{N}
\end{split}
\label{ch1:equ:closed-loop-minimisation}
\end{equation}


\subsection{Execution and result assessment procedure}
\label{ch1:subsec:method-execution}

Being able to focus the global cost function in Equation~\ref{ch1:equ:weighted-sum-cost-function} on improving a particular key network parameter by adjusting the binary choice vector $\boldsymbol{\alpha}$, and after having established how the closed-loop optimising method aims to achieve these improvements, the performance assessment procedure is introduced.
The complete evaluation procedure and assessment is summarised in Figure~\ref{ch1:fig:method-evaluation-flowchart}.

\begin{figure}\centering

% Define some block styles
\tikzstyle{data} = [rectangle, draw, inner sep=1em, text centered, rounded corners=5pt, fill=white]
\tikzstyle{method} = [rectangle, draw, inner sep=1em, text centered, align=center]
\tikzstyle{output} = [rectangle, draw, minimum height=1cm, minimum width=2.5cm, text centered, rounded corners=5pt, fill=yellow!20]

\scalebox{0.7}{%
\begin{tikzpicture}[node distance=3cm, shorten >= 1pt, >=stealth', auto, anchor=west]
	
	\node (centre) [] {};
	
	\node (power_profiles) [data, left of=centre, xshift=1cm] {Power Profiles};
	\node (battery_model) [data, right of=centre, xshift=-1cm] {Battery Model};
	
	\node (inputA) [left of=battery_model, yshift=1cm, xshift=-9cm] {};
	\node (inputB) [left of=battery_model, yshift=-1cm, xshift=-9cm] {};
	\draw [decorate,decoration={brace,mirror,amplitude=10pt,raise=4pt},yshift=0pt] (inputA) -- (inputB) node [left,black,midway,align=center,xshift=-10mm,rotate=90,anchor=center] {Input Data};
	
	\node (optimisation) [method, below right of=battery_model, yshift=-1.5cm, fill=green!20] {Closed-Loop Optimisation Method};
	\node (normal_run) [method, left of=optimisation, xshift=-3.5cm, fill=blue!20] {Normal Simulation};
	\node (base_run) [method, left of=normal_run, xshift=-1.5cm, fill=red!20] {Base Simulation};
	
	\node (simA) [left of=battery_model, yshift=-2.2cm, xshift=-9cm] {};
	\node (simB) [left of=battery_model, yshift=-5.2cm, xshift=-9cm] {};
	\draw [decorate,decoration={brace,mirror,amplitude=10pt,raise=4pt},yshift=0pt] (simA) -- (simB) node [left,black,midway,align=center,xshift=-10mm,rotate=90,anchor=center] {Simulations};
	
	\draw [->, out=270, in=120] (power_profiles.south) to (optimisation.north);
	\draw [->, out=270, in=90] (power_profiles.south) to (normal_run.north);
	\draw [->, out=270, in=60] (power_profiles.south) to (base_run.north);
	\draw [->, out=270, in=120] (battery_model.south) to (optimisation.north);
	\draw [->, out=270, in=60] (battery_model.south) to (normal_run.north);

	\def\offsetx{-5};
	\def\offsety{-4};
	\node (alpha_base) [output, below of=centre, yshift=\offsety cm, xshift=\offsetx cm-3cm, fill=red!20] {$base$};
	\draw [->, out=270, in=60] (base_run) to (alpha_base);
	\foreach \i [evaluate=\i as \x using 3*\i/2+\offsetx] in {0,2,...,8}
	{
		\foreach \j [evaluate=\j as \y using -1.5*\j] in {0, 1}
		{
			\pgfmathtruncatemacro{\k}{\i+1-\j}
			\if\k0
				\node (alpha_normal) [output, below of=centre, yshift=\offsety cm+\y cm, xshift=\y cm+\x cm, fill=blue!20] {$normal$};
				\begin{scope}[on background layer]\draw [->, out=270, in=60] (normal_run) to (alpha_normal);\end{scope}
			\else
				\node (alpha_\k) [output, below of=centre, yshift=\offsety cm+\y cm, xshift=\y cm+\x cm] {$\zeta(\alpha_{\k} = 1)$};
				\begin{scope}[on background layer]
					\ifnum7>\k
						\draw [->, out=270, in=60] (optimisation) to (alpha_\k);
					\else
						\ifnum7=\k
							\draw [->, out=270, in=90] (optimisation) to (alpha_\k);
						\else
							\draw [->, out=270, in=120] (optimisation) to (alpha_\k);
						\fi
					\fi
				\end{scope}
			\fi
		}
	}
	
	
	\node (dataA) [left of=battery_model, yshift=-6cm, xshift=-9cm] {};
	\node (dataB) [left of=battery_model, yshift=-9.4cm, xshift=-9cm] {};
	\draw [decorate,decoration={brace,mirror,amplitude=10pt,raise=4pt},yshift=0pt] (dataA) -- (dataB) node [left,black,midway,align=center,xshift=-10mm,rotate=90,anchor=center] {Results or\\Datasets};
	
	\foreach \i [evaluate=\i as \k using int(round(\i-1))] in {0,1,2,...,10}
	{
		\pgfmathsetmacro\x{1.5*\i}
		\pgfmathsetmacro\y{-\i*0.5};
		
		\ifnum1>\k
			\ifnum0=\i
				\node (assess_base) [method, below of=alpha_base, yshift=\y cm, xshift=\x cm, anchor=north, fill=red!20] {$base$ assessment};
				\draw [->] (alpha_base) to (assess_base);
			\else
				\node (assess_normal) [method, below of=alpha_base, yshift=\y cm, xshift=\x cm, anchor=north, fill=blue!20] {$normal$ assessment};
				\draw [->] (alpha_normal) to (assess_normal);
			\fi
		\else
			\node (assess_\k) [method, below of=alpha_base, yshift=\y cm, xshift=\x cm, anchor=north, fill=yellow!20] {$\zeta(\alpha_{\k}=1)$ assessment};
			\draw [->] (alpha_\k) to (assess_\k);
		\fi
	}


	\node (braceA) [below of=assess_base, yshift=1.5cm, xshift=-1.5cm] {};
	\node (braceB) [left of=assess_9, yshift=-1.5cm, xshift=1.5cm] {};
	\draw [decorate,decoration={brace,mirror,amplitude=10pt,raise=4pt},yshift=0pt] (braceA) -- (braceB) node [below,black,midway,yshift=-8mm,anchor={north east},align=right] {Data Assessments\\\footnotesize Each assessment is comparing \\\footnotesize all nine cost-functions};
	
\end{tikzpicture}%
}
\caption{Method execution and results assessment flowchart}
\label{ch1:fig:method-evaluation-flowchart}
\end{figure}

Since the global cost function can be focused in nine distinct ways using $\boldsymbol{\alpha}$, eleven datasets of simulation results can be assessed and compared.
These additional two results are obtained from a \textit{base} simulation and a \textit{normal} simulation (and nine cost driven i.e. \textit{optimisation} simulations).
For the \textit{base} simulation the outcome is generated by applying just the daily power profiles without any ESMU intervention.
Therefore this case represents the baseline of network performance which should be improved by any ESMU intervention.
The \textit{normal} simulation is the simplest of all ESMU interventions since the ESMU executes its normal (or traditional) half-hourly schedule without any additional modifications.
Comparing results from the \textit{base} and \textit{normal} simulations will show the direct impact of the traditional ESMU operation on network performance.
The remaining nine datasets are results of the nine different cost driven simulations where the ESMU schedule is adjusted on a sub-half-hourly level for each simulation.
This adjustment is designed to minimise one underlying cost-function, whilst conforming to the ESMU's overall half-hourly charging and discharging profile.
In order to treat each cost-function separately $\boldsymbol{\alpha}$ is set to focus on each cost independently, for example by setting $\alpha_1 = 1 \text{ and } \alpha_2 = \alpha_3 = \dots = \alpha_9 = 0$.
For simplicity the flowchart in Figure~\ref{ch1:fig:method-evaluation-flowchart} abbreviates the specific cost choice by only indicating which entry in the $\boldsymbol{\alpha}$ vector is set to $1$, for example $\zeta(\alpha_1=1)$.

Once all eleven simulations have completed their corresponding datasets are assessed in an identical manner so that their impact on network performance can be compared.
This comparison is broken into three parts for all dataset:

\begin{enumerate}
	\item \textbf{Time Series Analysis} - 
	The underlying profiles are plotted and compared against their respective counterpart cases in order to link the immediate network impacts to their physical meaning.
	Here the used performance metric is the cost that is calculated from the simulation specific cost function.
	For the same profiles their corresponding cost profiles are calculated and plotted in the results section.
	This is done to highlight how the profiles are interpreted by the cost-functions in terms of improvement (i.e. lower cost) or worsening (i.e. increased cost).
	\item \textbf{Difference Analysis} - 
	The difference in cost profiles compared to the respective \textit{base} or \textit{normal} case is calculated and boxplots of these differences are presented in the results section to show the statistical spread of improvements or worsening.
	For these plots a generally positive boxplot indicates a general improvement of the underlying network parameters whilst a generally negative boxplot does indicate worse performance in regards to the underlying network parameters.
	\item \textbf{Probability Density Analysis} - 
	A set of Probability Density Functions (PDF) is derived for each cost profile using the well established and standardised kernel density estimation.
	These PDFs indicate the probability that a certain cost value occurs.
	An improvement is noted when the PDF is shifted towards the lower cost values, whereas a shift towards higher cost values worsened the network performance.
\end{enumerate}

The above cost function is however limited to target each individual cost separately since $\boldsymbol{\alpha}$ is a binary choice vector.
As discussed in the future work in Chapter~\ref{ch-conclusions:sec:future-work} a multi-objective problem may be considered as a next step where costs are weighted against each other.
After all, the difference of importance of each cost that DNOs may care about could then be reflected accordingly.




