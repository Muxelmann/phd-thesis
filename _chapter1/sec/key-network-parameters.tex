\section{Key Network Parameters and Derived Cost Functions}
\label{ch1:sec:key-network-parameters}

Two distinct approaches have emerged to quantitatively improve the performance of a system: either ``cost'' is reduced or ``utility'' is maximised.
Both approaches rely on a mathematical explanation of underlying features that relate to performance of the system.
The choice for this piece of work was to associate a cost to each key network parameter, for the reason that cost functions can be minimised to a finite value, i.e. zero.
Utility maximisation on the other hand is a theoretically unbound problem that can only reach a maximum, if its maximum can be estimated in advance.
In other words, solutions to a cost function where the resulting cost is zero, are by definition part of the set of optimal solutions.
Determining the set of optimal solutions for the maximisation of a utility function is however more difficult.

\nomenclature[I]{$t$}{Time-steps of the simulation, where $t \in \{1,\Delta t, 2\Delta t,\dots,T\}$ (Chapter \ref{ch1})}
\nomenclature[I]{$\Delta t$}{Sample time, where $\Delta t \in \mathbb{Z}_{\geq0}$ (Chapter \ref{ch1})}
\nomenclature[I]{$T$}{Length of simulation, where $T \in \mathbb{Z}_{\geq0}$ (Chapter \ref{ch1})}

With this in mind, the key network parameters are defined and their corresponding cost functions are introduced.
In this piece of work, power flow simulations are run at discrete times, $t$, which are separated by a sampling period $\Delta t$. 
The model used for these simulations is the IEEE LV Test Case, which consists of 906 three phase buses, resulting in a total of 2718 nodes.
For each node, complex currents and voltages can be obtained, making the number of parameters to chose from nearly inexhaustible.
In reality however, a power distribution network can only be observed at a limited number of measuring points.
For the NTVV project, these points were at the substation and the ESMU's Point of Common Coupling (PCC).
Therefore, all derived network parameters that could be obtained in reality are seen as ``realistic parameters'', despite the fact that all key network parameters are extracted from power flow simulations.
The remaining key network parameters, i.e. those that could not easily be obtained in reality, are therefore referred to as ``theoretical parameters''.

Due to the high number of theoretical parameters, only a subset of those theoretical parameters is used.
The choice of parameters is based on their importance, role and impact on the actual network operation.
A list of all realistic and theoretical key network parameters is presented below, and in this list all theoretical key network parameters are marked with a dagger ($\dagger$).

\begin{itemize}
	\item Voltages at substation transformer's secondary winding
	\item Voltages at ESMU's PCC
	\item Voltages at customer lateral$^{\dagger}$
	\item Total power flow
	\item Substation line utilisation
	\item Maximum line utilisation$^{\dagger}$
	\item Distribution losses$^{\dagger}$
\end{itemize}

In the following subsections, all key network parameters' associated cost functions are explained and formulated.

\subsection{Voltages at substation}
\label{ch1:subsec:voltages-at-substation}

\nomenclature[I]{$v_{\text{ss},\phi}$}{Phase voltage at substation for phase $\phi$ at time $t$, where $(v_{\text{ss},\phi}(t)) = \textbf{v}_{ss}(t)$ (Chapter \ref{ch1})}
\nomenclature[I]{$\textbf{v}_{ss}(t)$}{Phase voltage vector at time $t$, where $\textbf{v}_\text{ss}(t) \in \mathbb{R}^{\Phi}$ (Chapter \ref{ch1})}
\nomenclature[I]{$\phi$}{Phase number, where $\phi \in \{1, \dots, \Phi\}$ (Chapter \ref{ch1})}
\nomenclature[I]{$\Phi$}{Number of phases, where $\Phi \in \mathbb{Z}_{>0}$ here $\Phi = 3$ (Chapter \ref{ch1})}
\nomenclature[I]{$\zeta_\text{voltage}(\textbf{v}(t))$}{Voltage deviation cost for voltage vector $\textbf{v}$ at time $t$, where $\zeta_\text{voltage}(\textbf{v}(t)) \in \mathbb{R}_{\geq0}$ (Chapter \ref{ch1})}
\nomenclature[I]{$V_\text{ss}$}{Nominal substation voltage, where $V_\text{ss} \in \mathbb{R}$ (Chapter \ref{ch1})}
\nomenclature[I]{$V_h$}{High-voltage threshold of statutory voltage band, where $V_h \in \mathbb{R}$ (Chapter \ref{ch1})}
\nomenclature[I]{$V_l$}{Low-voltage threshold of statutory voltage band, where $V_l \in \mathbb{R}$ (Chapter \ref{ch1})}

In the UK, LV networks operate at a nominal voltage of 230V Phase-to-Neutral (P2N) or 400V Phase-to-Phase (P2P).
Substations supply electricity to a three-phase cable, i.e. the feeder, and  link to MV distribution networks, which operate at 11kV P2P.
In an ideal case the voltage measured at the substation transformer's secondary winding remains constant as load changes.
But in reality, internal losses (e.g. conductive losses and magnetic leakage) lead to a dropping voltage level, when load increases.
Therefore, any deviation from the substation's nominal voltage can be seen as an indication of suboptimal network operation.

The ``voltage deviation cost function'' $\zeta_\text{voltage}(\textbf{v}(t))$ captures this suboptimal operation.
This cost function is defined for a multi-phase complex voltage vector as $\textbf{v}(t)$ where $(v_\phi(t)) = \textbf{v}(t)$, where $\phi$ is the phase number and where $t$ the time at which the measurement was taken.
Both phase and time are discrete, i.e. $\phi \in \{1,\dots,\Phi\}$ where $\Phi \in \mathbb{Z}_{>0}$ and $t \in \mathbb{Z}_{\geq0}$.
When using the three-phase substation voltage vector, $\textbf{v}_{ss}(t)$ (where $(v_{\text{ss},\phi}(t)) = \textbf{v}_{ss}(t)$), with this cost function, any drop in transformer voltage results in a positive cost.

\begin{equation}
\begin{split}
	\zeta_\text{voltage}(\textbf{v}(t)) :=& \sum_{\phi=1}^{\Phi}{\begin{cases}
		\zeta_h(v_{\phi}(t)) & \text{if } V_{ss} \leq v_\phi\\
		\zeta_l(v_{\phi}(t)) & \text{otherwise}\\
	\end{cases}} \forall t\\
	&\text{ where } \Phi \in \mathbb{N} \text{ s.t. } V_l < V_{ss} < V_h
\end{split}
\label{ch1:equ:voltage-deviation}
\end{equation}

In this cost function, $\zeta_h(v)$ and $\zeta_l(v)$ are two functions that convert a single voltage value, i.e. $v_p$, into a normalised positive cost.
If the voltage $v_p$ is greater than or equal to the nominal substation voltage, $V_{ss}$, then the result from $\zeta_h(v)$ is used as a cost; otherwise the result from $\zeta_l(v)$ is used.
In order to define these two functions, the corresponding high and low voltage thresholds, $V_h$ and $V_l$ respectively, need to be introduced.
These two thresholds are based on the nominal LV voltage range of +10\% -6\% around $V_n$, i.e. 230V P2N.

\begin{equation}
	\zeta_h(v) := \alpha \left|\frac{v-V_{ss}}{V_h-V_{ss}}\right|^{\beta}
	\label{ch1:equ:high-voltage-threshold-cost-complete}
\end{equation}

\begin{equation}
	\zeta_l(v) := \alpha \left|\frac{V_{ss}-v}{V_{ss}-V_l}\right|^{\beta}
	\label{ch1:equ:low-voltage-threshold-cost-complete}
\end{equation}

In this context, $\alpha$ is the function's weight that linearly scales the cost function, and $\beta$ regulates the function's offset gradient.
More specifically, $\alpha$ determines the value of the cost functions for the values of $V_l$ and $V_h$, and may take any value in $(0, \infty)$.
For example, when $\alpha = 1$, then $\zeta_{h}(v_l) = 1$.
$\beta$ on the other hand may take any value in the range of $[2, \infty)$, to assure a continuously differentiable cost function.
For this work, $\alpha$ and $\beta$ were treated as constants and set to $1$ and $2$, respectively.
Substituting these values into Equations \ref{ch1:equ:high-voltage-threshold-cost-complete} and \ref{ch1:equ:low-voltage-threshold-cost-complete}, simplifies them to:

\begin{equation}
	\zeta_h(v) := \left|\frac{v-V_{ss}}{V_h-V_{ss}}\right|^{2}
	\label{ch1:equ:high-voltage-threshold-cost-simple}
\end{equation}

\begin{equation}
	\zeta_l(v) := \left|\frac{V_{ss}-v}{V_{ss}-V_l}\right|^{2}
	\label{ch1:equ:low-voltage-threshold-cost-simple}
\end{equation}


In this voltage cost function, $\Phi$ represents the number of phases (i.e. $\Phi = 3$), and $\zeta_h(v)$ and $\zeta_l(v)$ are two functions that convert a single voltage value, i.e. $v_\phi$, into a normalised positive cost based upon the direction of voltage deviation.
High and low voltage thresholds, respectively $V_h$ and $V_l$, are introduced in order to define these two functions.
When choosing these two thresholds, then they must also satisfy the following inequality:

\begin{equation}
	V_l < V_\text{ss} < V_h
\end{equation}


For the work presented here, these two thresholds are based on the UK's nominal LV voltage range of +10\% -6\% around $V_n$, i.e. 230V P2N.
As a result, the following upper and lower threshold functions are defined, in order to form a continuously differentiable cost function with a single zero tangent.

\begin{equation}
	\zeta_h(v) := \left|\frac{v-V_\text{ss}}{V_h-V_\text{ss}}\right|^{2}
	\label{ch1:equ:high-voltage-threshold-cost-simple}
\end{equation}

\begin{equation}
	\zeta_l(v) := \left|\frac{V_\text{ss}-v}{V_\text{ss}-V_l}\right|^{2}
	\label{ch1:equ:low-voltage-threshold-cost-simple}
\end{equation}

Substations may boost the voltage above the nominal LV voltage level, since voltage levels drop continuously along a purely consumptive feeder.
The impact on the cost function $\zeta_\text{voltage}(\textbf{v})$ when $V_\text{ss}$ is boosted is shown in Figure \ref{ch1:fig:voltage-deviation} (for simplicity the a single-phase voltage vector is shown, i.e. $\Phi = 1$).

\begin{figure}\centering
	\includegraphics{_chapter1/fig/voltage-deviation}
	\caption{Cost function $\zeta_\text{voltage}(v_\phi)$ values for different substation voltages}	
	\label{ch1:fig:voltage-deviation}
\end{figure}

In this figure, it can be seen that $\zeta_\text{voltage}(\textbf{v})$ at the thresholds $V_l$ and $V_h$ equates to one, and to zero at the set substation voltage, even when this voltage is boosted.
This intentional feature is demonstrated by raising $V_\text{ss}$ from $V_n$ by +4\% and +8\%.

\subsection{Voltages at ESMU's PCC}
\label{ch1:subsec:voltages-at-esmu}

At the ESMU's Point of Common Coupling (PCC), the device has access to all three phases of the feeder.
Where ESMU should be connected in order to achieve a best possible network impact, is explained in Section \ref{ch1:sec:data-and-network-models}.
Neglecting the exact ESMU location for now, one can assume that the line voltage along the feeder will change and, with increasing load, most likely drop.
Reasons behind this voltage drop are the resistive and inductive losses in the distribution lines.
These losses are amplified with proximity to the substation, since load currents form ``down stream'' customers are aggregated.
For purely consumptive loads (i.e. no distributed generation), and particularly under heavy load conditions, this voltage is likely drop below the statutory nominal operation band.
As mentioned in literature, this threshold is an operational constraint of LV networks and must not be violated to assure correct appliance operation and prevent financial penalisation.

In order to mitigate this voltage drop, power is injected into the feeder at the ESMU's PCC.
Doing so boosts the voltage at that location, since the portion of load current that would normally be supplied by the substation is now delivered by the ESMU.
The effect of this power injection is sketched in the Figure \ref{ch1:fig:sketch-voltage-esmu} below.

\begin{figure}\centering
	\subfloat[Voltage drop along feeder without ESMU intervention]{%
		\includegraphics[width=0.45\textwidth]{_chapter1/fig/sketch-voltage-esmu-normal}%
		\label{ch1:subfig:sketch-voltage-esmu-normal}%
		}
	\hspace{5mm}
	\subfloat[Voltage drop along feeder with ESMU intervention]{%
		\includegraphics[width=0.45\textwidth]{_chapter1/fig/sketch-voltage-esmu-boost}%
		\label{ch1:subfig:sketch-voltage-esmu-boost}%
		}
	\caption{Benefits of ESMU power injection on the voltage drop along the feeder}
	\label{ch1:fig:sketch-voltage-esmu}
\end{figure}

\nomenclature[I]{$v_{\text{ESMU},\phi}(t)$}{Phase voltage at ESMU for phase $\phi$ at time $t$, where $(v_{\text{ESMU},\phi}(t)) = \textbf{v}_{ESMU}(t)$ and $v_{\text{ESMU},\phi}(t) \in \mathbb{R}$ (Chapter \ref{ch1})}

In this figure, an expected voltage drop along the entire feeder is sketched. 
It can be seen how the line voltage of the feeder's tailing section drops below $V_l$.
In contrast, Figure \ref{ch1:subfig:sketch-voltage-esmu-boost} shows how the ESMU's intervention can alleviate some load and raise the trailing voltage levels above $V_l$.

This realistic key network parameter is also used in a cost function, since the Phase to Neutral (P2N) voltage at the ESMU's PCC can easily be measured.
To do so, the three-phase ESMU voltage at its PCC, $\textbf{v}_{ESMU}(t)$ (where $(v_{\text{ESMU},\phi}(t)) = \textbf{v}_{ESMU}(t)$), is used in the same cost that was used with the substation transformer's voltage, i.e. Equation \ref{ch1:equ:voltage-deviation}.
Therefore, the resulting cost can be formulated as $\zeta_\text{voltage}(\textbf{v}_{ESMU}(t))$.

\subsection{Voltages at customer laterals}
\label{ch1:subsec:voltages-at-customers}

As mentioned in Chapter \ref{ch-introduction:sec:overview}, the allowable voltage range at customers is defined by the Electricity Safety, Quality and Continuity Regulations (ESQCR).
However, monitoring those voltages in real-time to assure they obey these regulations is costly, which is why they are often left unmonitored.
Ultimately, when it comes to voltage level correction, only substation voltage levels and customer voltage levels need to be controlled.
This is to assure proper transformer operation (e.g. to maximise its lifespan) and to prevent penalisation due to customer voltage violation (i.e. fining according to the ESQCR).
As explained in Section \ref{ch1:subsec:voltages-at-esmu}, ESMU can impact voltage levels for all customers but, as mentioned above, these voltage levels cannot be measured in reality.
In simulations however, all load voltages can easily be extracted, which is why they are treated as theoretical key network parameters.

To illustrate this load voltage drop, a snapshot OpenDSS simulation was run on the used network model with all load consuming  8kW of power\footnote[1]{Whilst historic and recent loads may reach values of this magnitude quite infrequently, future customer demand with the aggregated effect home-charging of EVs is expeceted to yield extreme scenarios like this.}.
In Figure \ref{ch1:fig:voltage-drop-for-loads-along-feeder}, the load bus voltages are plotted against the loads' distances to their feeding substation was drawn.

\begin{figure}\centering
	\includegraphics{_chapter1/fig/voltage-drop-for-loads-along-feeder}
	\caption{Voltage at the loads in the IEEE LV Test Case network for a total load of 440kVA against distance between the corresponding load and substation: for the quadratic fit $R^2=58.76\%$}
	\label{ch1:fig:voltage-drop-for-loads-along-feeder}
\end{figure}

\nomenclature[I]{$v_{\text{load},i,\phi}(t)$}{Phase voltage of load $i$ for phase $\phi$ at time $t$, where $(v_{\text{load},i,\phi}(t)) = \textbf{v}_{load}(t)$ and $v_{\text{load},i,\phi}(t) \in \mathbb{R}$ (Chapter \ref{ch1})}
\nomenclature[I]{$\zeta_\text{load voltage}(\textbf{v}(t))$}{Voltage deviation cost for load voltage vector $\textbf{v}$ at time $t$ and $\zeta_\text{load voltage}(\textbf{v}(t)) \in \mathbb{R}_{\geq0}$ (Chapter \ref{ch1})}

In this figure, two observations can be made.
For one, it can be seen that phases are significantly unbalanced.
Secondly, customers further than 200m from the substation experience low-voltage events.
As already proposed in the previous section, ESMU can reduce the number of such low-voltage events, but including a cost for each load would add significant computational burden onto the solving algorithm.

To address this problem more efficiently, the previously defined voltage cost function (i.e. Equation \ref{ch1:equ:voltage-deviation}) is expanded to only return a single value from all customer voltages.
More specifically, instead of including the cost for every single customer's voltage deviation, only the worst deviation is used.
By reducing this number to the worst case, any implemented solver needs not consider the otherwise large number of parameters, and only needs to focus on the minimisation of a single value.
This focus is of particular importance, especially if the impact of the ESMU on some customers' voltages is comparatively low, since e.g. an aggregated deviation cost could potentially obfuscate the solver and prevent it from targeting the worst case.
Therefore, for this work, the customer (or load) voltage is defined as $\textbf{v}_{load}(t)$ (where $v_{\text{load},i,\phi} \in \textbf{v}_{load}$) and used in the new cost function, $\zeta_\text{load voltage}(\textbf{v}_{load}(t))$, which is defined as:

\begin{equation}
	\zeta_\text{load voltage}(v_{i,p}) := \max_{i}{\zeta_\text{voltage}(v_{i,p})} \forall i \text{ where } i \in [1, \dots I]
	\label{ch1:equ:load-voltage-deviation}
\end{equation}

\subsection{Total power flow}
\label{ch1:subsec:total-power-flow}

Beside having to keep voltages within their specified operational boundaries, DNOs want to assure that the distribution network operates both in an efficient and ideal manner.
Determining how ideal a three-phase network operates can be done by e.g. assessing the balance of the network's phases.
The frequently neglected disturbance due to unbalanced phases may not have an immediate impact, but the negative long term effects (e.g. asymmetric load on transformers, rotating machines and increased neutral current) weaken the network's assets and must not be neglected.

Customer phasing procedures in the UK increase the problem of phase unbalance even more, since phases are chosen arbitrarily and customers are connected via single-phase laterals.
These lateral or the link to the feeder is established by connecting the customer's supply cable to a phase and the neutral conductor.
Randomly assigning customers to phases should distribute load evenly across all three phases, which in theory should balance the three-phase network load, yet in reality this is not the case.
Even if the number of customers per phase was the same, the probability that all customers' load profiles are identical is very low.
Therefore, the likeliness that LV distribution feeders in the UK are unbalanced is very high.

\nomenclature[I]{$\textbf{s}_{ss}(t)$}{Apparent multi-phase power at substation level at time $t$, where $\textbf{s}_{ss}(t) \in \mathbb{C}^\Phi$ (Chapter \ref{ch1})}
\nomenclature[I]{$s_{\text{ss},\phi}(t)$}{Apparent single-phase power at substation level for phase $\phi$ at time $t$, where $(s_{\text{ss},\phi}(t)) = \textbf{s}_{ss}(t)$ (Chapter \ref{ch1})}
\nomenclature[I]{$\text{UF}(\textbf{x})$}{Function calculating the Unbalance Factor (UF) for any multidimensional vector $\textbf{x}$, where $(x_n) = \textbf{x}$, $n \in \mathbb{Z}_{>0}$ and $\text{UF}(\textbf{x}) \in \mathbb{R}_{\geq0}$ (Chapter \ref{ch1})}

Since substation monitoring is capable of providing reliable three-phase power measurements, they can be used as realistic key network parameters from which the network's unbalance can be determined.
Deriving phase unbalance from this power vector is done by following the American National Standards Institute's (ANSI) definition of Unbalance Factor (UF) \cite{ANSI-MB-1-2011}:

\begin{equation}
	\text{UF}(\textbf{x}) := \frac{\max_n |\bar{\textbf{x}} - x_n|}{\bar{\textbf{x}}}
	\label{ch1:equ:unbalance-equation}
\end{equation}

Here, $\textbf{x}$ is a three phase vector, where $x_n \in \textbf{x} \text{ for } n=[1, 2, 3]$.
$x_n$ may be a voltage, current or power measurement per phase, but for context of this work $x_n$ was chosen to be the power flow into the network.
For clarity, the notation $\bar{\textbf{s}}$ is used to define the mean of all three phase values.
The mean's definition is give below:

\begin{equation}
	\bar{\textbf{x}} := \frac{1}{3}\sum_n^3{x_i}
\end{equation}

\nomenclature[I]{$\zeta_\text{unbalance}(\textbf{s}(t))$}{Power unbalance for multi-phase apparent power vector $\textbf{s}$ at time $t$, where $\zeta_\text{unbalance}(\textbf{s}(t)) \in \mathbb{R}_{\geq0}$ (Chapter \ref{ch1})}

From the substation monitoring, a three-phase substation power vector, $\textbf{s}_{ss}(t)$ (where $s_{\text{ss},\phi} \in \textbf{s}_{ss}$), can be used to calculate the network's phase unbalance.
Since this forms another realistic key network parameter, the UF in Equation \ref{ch1:equ:unbalance-equation} was formulated into a cost function.
The resulting cost function $\zeta_\text{unbalance}(\textbf{s}_{ss})$ is thus defined as:

\begin{equation}
	\begin{split}
		\zeta_\text{unbalance}(\textbf{s}):=&\text{UF}(\textbf{s}) - 1\\
		=&\frac{\max_p\left|\bar{\textbf{s}} - s_p\right|}{\bar{\textbf{s}}}\\
		=&\frac{\max_p\left|\left(\frac{1}{3}\sum_p^3{s_p}\right) - s_p\right|}{\frac{1}{3}\sum_p^3{s_p}}
	\end{split}
	\label{ch1:equ:unbalance-cost}
\end{equation}

Since the lowest value the UF can take is one, its corresponding cost function had to be adjusted so that it can reach a lowest value of zero instead.
This adjustment can be observed by the subtraction of one at the end of the definition of the cost function, $\zeta_\text{unbalance}(\textbf{s})$.
An sample illustration, showing how this cost behaves as phase unbalance increases, is included in the Figure \ref{ch1:fig:power-unbalance}, which shown below.

\begin{figure}\centering
	\includegraphics[width=5cm]{foo}
	\caption{Sample network imbalance for different phase loadings as defined in ANSI/NEMA MG 1-2011}
	\label{ch1:fig:power-unbalance}
\end{figure}

Here, it can be seen how $\zeta_\text{unbalance}(\textbf{s})$ varies with an increasing separation of the three phases' power values.

\nomenclature[I]{$\zeta_\text{PF}(\textbf{s}(t))$}{Power Factor (PF) cost function for multi-phase apparent power vector $\textbf{s}$ at time $t$, where $\zeta_\text{PF}(\textbf{s}(t)) \in \mathbb{R}_{\geq0}$ (Chapter \ref{ch1})}

Additionally, to assess the effective utilisation of the power distribution network, the Power Factor (PF) divergence is also included as a cost.
PF is the ratio between the supplied active (p) and apparent power (s).
It therefore gives an indication of how much ``good''\footnote[1]{Reactive power is used to maintain magnetic fields in rotating machines, yet this can be supplied by local reactive power compensators and thus need not occupy otherwise free power transmission resources.} power is being consumed by the system.
Experts would agree that keeping the PF of a system close to unity indicates that the system only requires active power to operate and therefore uses the lowest amount transmission resources.
In order to asses the proximity to unity PF a cost function, $\zeta_\text{PF}(\textbf{s}_{ss}(t))$, is defined, using a multi-phase power vector as input:

\begin{equation}
	\zeta_\text{PF}(\textbf{s}(t)):= 1 - \sum_{\phi=1}^{\Phi}\frac{\text{Re}(s_\phi(t))}{|s_\phi(t)|} \forall t \text{ where } (s_\phi(t)) = \textbf{s}(t) \text{ and } \Phi \in \mathbb{Z}_{>0}
\label{ch1:equ:power-factor}
\end{equation}

Here, deviating from a unity PF per phase increases the associated cost, whilst achieving a perfect PF for each phase results in a total cost of zero.

\nomenclature[I]{$\zeta_\text{neutral load}(\textbf{s}(t))$}{Neutral load cost function for multi-phase apparent power vector $\textbf{s}$ at time $t$, where $\zeta_\text{neutral load}(\textbf{s}(t)) \in \mathbb{R}_{\geq0}$ (Chapter \ref{ch1})}

Lastly, the already mentioned neutral current is also a result of both three-phase unbalance and non-unity PF.
Since in perfectly balanced systems all three phases are $120^\circ$ out of phase, the sum of the instantaneous powers should equate to zero.
This summation would also result in no neutral current flowing in the system.
However, in an unbalanced system the power transmitted through the neutral conductor deviates significantly from zero.
This negative impact of neutral currents is further amplified, since power distribution cables often use neutral conductors with significantly smaller cross-section than those used for line conductors.
Therefore, any additional power flow in this neutral conductor will further deviate neutral voltages from ground.
Furthermore, since it also exhausts the neutral conductor's power carrying capability, the system becomes more prone to failures.
To address this last point, and before dealing with line utilisation, a cost function that deals with the neutral load $\zeta_\text{neutral load}(\textbf{s}_{ss}(t))$ is defined as follows:

\begin{equation}
	\zeta_\text{neutral load}(\textbf{s}(t)) := \left|\sum_{\phi=1}^{\Phi} s_\phi(t)e^{\frac{j2\phi\pi}{\Phi}}\right| \text{ where } \textbf{s}(t) = (s_\phi(t)) \text{ and } \Phi \in \mathbb{Z}_{>0}
\label{ch1:equ:neutral-load}
\end{equation}

In a three phase scenario, each phase power, $s_{\text{ss},\phi}(t)$, is therefore rotated by an integer multiple of $120^\circ$ before adding them to obtain the neutral load vector.
Then, to obtain $\zeta_\text{neutral load}(\textbf{s}_{ss}(t))$, the magnitude of this apparent power vector is computed, which is also the size of the neutral conductor's load.

\subsection{Substation line utilisation}
\label{ch1:subsec:substation-line-utilisation}

Although phase unbalance deteriorates the efficiency and life expectancy of three-phase network assets, high power demand puts strain on the physical cables themselves.
This is predominantly due to resistive and inductive losses heating the cables, and bringing them closer to their operational limits.
Therefore, cables have an assigned thermal rating which must not be exceeded to prevent permanent cable damage or network failures.
At substation level, to prevent over-currents, fuses or reclosers are installed that will disconnect the network under fault or high demand conditions.
To quantify whether the substation fuse is approaching its tripping point, its nominal rating is used as reference.

\nomenclature[I]{$I_\text{fuse}$}{Nominal fuse rating at substation, where $I_\text{fuse} \in \mathbb{R}$ (Chapter \ref{ch1})}
\nomenclature[I]{$i_{\text{ss},\phi}(t)$}{Single-phase substation current for phase $\phi$ at time $t$, where $i_{\text{ss},\phi}(t) \in \mathbb{R}$ (Chapter \ref{ch1})}
\nomenclature[I]{$\textbf{i}_{ss}(t)$}{Multi-phase substation current at time $t$, where $\textbf{i}_{ss}(t) \in \mathbb{R}^\Phi$ (Chapter \ref{ch1})}
\nomenclature[I]{$\zeta_\text{fuse utilisation}(\textbf{i}_{ss}(t))$}{Fuse utilisation cost, derived from multi-phase substation current vector $\textbf{i}_{ss}$ at time $t$, where $\zeta_\text{fuse utilisation}(\textbf{i}_{ss}(t)) \in \mathbb{R}_{\geq0}$ (Chapter \ref{ch1})}

For the context of this work, this nominal fuse rating, $I_\text{fuse}$, is a fixed value for the fuse at the substation and must not be exceeded.
Using the three-phase current vector (obtained via substation monitoring), $\textbf{i}_{ss}(t)$ (where $(i_{\text{ss},\phi}(t)) = \textbf{i}_{ss}(t)$) a cost function, $\zeta_\text{fuse utilisation}(\textbf{i}_{ss}(t))$, can be defined as follows:

\begin{equation}
	\zeta_\text{utilisation}(\textbf{i}(t)) :=%
	\left|\frac{\sum_{p=1}^{P}{i_p(t)}}{i_{fuse}}\right|^2 \forall t%
	\text{ where } p = [1, \dots, P]%
	\text{ and } P \in \mathbb{N}_{>0}
	\label{ch1:equ:fuse-utilisation}
\end{equation}

A plot has been included in the Figure \ref{ch1:fig:fuse-utilisation}, which illustrates how this quadratic cost behaves as substation current increases.

\begin{figure}\centering
	\includegraphics{_chapter1/fig/fuse-utilisation}
	\caption{Cost of line or fuse utilisation against network current}
	\label{ch1:fig:fuse-utilisation}
\end{figure}

For this simple case, the substation line rating was set as $i_{fuse}=400\text{A}$, and the total substation current is the sum of all three phase currents.

\subsection{Maximum line utilisation}
\label{ch1:subsec:maximum-line-utilisation}

Just like voltage levels at each customer's bus are a theoretical key network parameters, currents in all lines are theoretical key network parameters, too.
Whilst optimising the aforementioned current at substation level would prevent unintentional feeder disconnection and equipment damage, the lines' ratings impose distributed limits, too.
For instance, the main three-phase wire has to be of sufficient scale in order to deliver several 100s of Amperes to a collection of customers
Feeder branches on the other hand, with a lower load count, may only experience 10s of Amperes and can be of smaller scale.
Generally, as distance to the substation increases, fewer down stream customers are connected, which allows the feeding cables to be down scaled.
This network topology is very common for radial distribution networks, since it saves significant equipment cost without compromising network integrity.
Yet with the advent of DG and electrified LCTs, the feeder's branches are expected to experience increasingly larger currents.

\nomenclature[I]{$l$}{Line number, where $l \in [1, \dots, L]$ (Chapter \ref{ch1})}
\nomenclature[I]{$L$}{Number of lines, where $L \in \mathbb{Z}^{>0}$ (Chapter \ref{ch1})}
\nomenclature[I]{$i_{\text{line},l,\phi}(t)$}{Single-phase line current for phase $\phi$ of line $l$ at time $t$, where $i_{\text{line},l,\phi}(t) \in \mathbb{R}$ (Chapter \ref{ch1})}
\nomenclature[I]{$\textbf{i}_\text{line}(t)$}{Multi-phase line currents at time $t$, where $(i_{\text{line},l,\phi}(t)) = \textbf{i}_\text{line}(t)$, and $\textbf{i}_\text{line}(t) \in \mathbb{R}^{L\times\Phi}$ (Chapter \ref{ch1})}
\nomenclature[I]{$I_{\text{nom},l}$}{Nominal line current for line $l$, $I_{\text{nom},l} \in \mathbb{R}$ (Chapter \ref{ch1})}

This is why the previous cost function, as it was defined in Equation \ref{ch1:equ:fuse-utilisation}, has to be expanded to take all line currents and ratings into account.
Since this work was based on network simulations, each line current, $i_{\text{line},l,\phi}(t)$ (where $l$ represents the line number and $p$ the phase in that line), could easily be extracted.
Collecting them in $\textbf{i}_{line}(t)$ (where $(i_{\text{line},l,\phi}(t)) = \textbf{i}_\text{line}(t)$) allows the formulation of an extended line utilisation cost function, $\zeta_\text{line utilisation}(\textbf{i}_\text{line}(t))$.
Addressing the multitude of line currents just like the multitude of customer voltages was dealt with, $\zeta_\text{line utilisation}(\textbf{i}_\text{line}(t))$ can be defined as follows:

\begin{equation}
\begin{split}
	\zeta_\text{line utilisation}(\textbf{i}(t)) :=& %
	\max_{l}{\left|\frac{\sum_{p=1}^{P}{i_{l,p}(t)}}{i_{nom,l}}\right|^2} \forall t \text{ where } l \in [1, \dots, L] \\
	&\text{ and } p \in [1, \dots, P] \text{ and } L \in \mathbb{N} \text{ and } P \in \mathbb{N}
\end{split}
\label{ch1:equ:line-utilisation}
\end{equation}

In this quadratic cost function, $I_{\text{nom},l}$ is the nominal rating of line $l$ in the network.
Also, and similar to Equation \ref{ch1:equ:load-voltage-deviation}, by considering only the maximum line utilisation, computational burden is reduced whilst parameter dependent sensitivity is increased.

\subsection{Distribution losses}
\label{ch1:subsec:losses}

When it comes to profit margins, energy losses in a distribution network are unwanted, since nobody pays for undelivered energy.
Although the losses a single distribution network are small in comparison to the losses in the entire electricity grid, the aggregate effect of reducing those losses could have a noticeable impact.
To put this into perspective, the losses in the IEEE LV Test Case network were 58kW, when simulated under the same high demand scenario which was used for voltage drop visualisation in Section \ref{ch1:subsec:voltages-at-customers}.
This equates to 12\% of the total network demand ($\text{i.e. }\frac{s_\text{losses}(t)}{\sum_{\phi=1}^\Phi{s_{\text{ss},\phi}(t)}} \approx \frac{58kW}{484kW}$).
Since this is a high network load, losses would be noticeably lower for notmal network operation.
This is made apparent in Figure \ref{ch1:fig:losses-against-network-demand}, where the uniform network load is varied and the corresponding losses are plotted against this variation.

\begin{figure}\centering
	\includegraphics{_chapter1/fig/losses-against-network-demand}
	\caption{Losses against increasing power demand}
	\label{ch1:fig:losses-against-network-demand}
\end{figure}

\nomenclature[I]{$\zeta_\text{losses}(s(t))$}{Losses based cost function, where $\zeta_\text{losses}(s(t)) \in \mathbb{R}_{\geq0}$ (Chapter \ref{ch1})}
\nomenclature[I]{$s_\text{losses}(t)$}{Total apparent power losses in the network $s_\text{losses}(t) \in \mathbb{C}$ (Chapter \ref{ch1})}

In Figure \ref{ch1:fig:losses-against-network-demand}, the region where losses exceed 5\% of the total network power is highlighted in red.
These preliminary results were found from power flow simulations.
In reality however, losses cannot be determined this easily.
Therefore, the network losses, $s_\text{losses}(t)$, are seen as theoretical key network parameters and used in the final cost function $\zeta_\text{losses}(s(t))$, which is simply defined as:

\begin{equation}
	\zeta_\text{losses}(s(t)) := |s(t)| 
	\label{ch1:equ:losses}
\end{equation}

