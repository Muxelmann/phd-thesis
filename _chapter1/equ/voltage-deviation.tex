\begin{equation}
\begin{split}
	\zeta_\text{voltage}(\textbf{v}(t)) :=& \sum_{\phi=1}^{\Phi}{\begin{cases}
		\zeta_h(v_{\phi}(t)) & \text{if } V_\text{ss} \leq v_\phi\\
		\zeta_l(v_{\phi}(t)) & \text{otherwise}\\
	\end{cases}} \forall t\\
	&\text{ where } \Phi \in \mathbb{Z}^{>0}
\end{split}
\label{ch1:equ:voltage-deviation}
\end{equation}

In this voltage cost function, $\Phi$ represents the number of phases (i.e. $\Phi = 3$), and $\zeta_h(v)$ and $\zeta_l(v)$ are two functions that convert a single voltage value, i.e. $v_p$, into a normalised positive cost based upon the direction of voltage deviation.
E.g. if the voltage $v_p$ is greater than or equal to the nominal substation voltage, $V_\text{ss}$, then the result from $\zeta_h(v)$ is used as a cost; otherwise the result from $\zeta_l(v)$ is used.
In order to define these two functions, the corresponding high and low voltage thresholds, respectively $V_h$ and $V_l$, are introduced.
With those high and low voltage bands, $V_\text{ss}$ has to be chosen in order to satisfy the following inequality:

\begin{equation}
	V_l < V_\text{ss} < V_h
\end{equation}

For the presented work, these two voltage thresholds are based on the UK's nominal LV voltage range of +10\% -6\% around $V_n$, i.e. 230V P2N.

\begin{equation}
	\zeta_h(v) := \alpha \left|\frac{v-V_\text{ss}}{V_h-V_\text{ss}}\right|^{\beta}
	\label{ch1:equ:high-voltage-threshold-cost-complete}
\end{equation}

\begin{equation}
	\zeta_l(v) := \alpha \left|\frac{V_\text{ss}-v}{V_\text{ss}-V_l}\right|^{\beta}
	\label{ch1:equ:low-voltage-threshold-cost-complete}
\end{equation}

In this context of defining $\zeta_{voltage}$, the variable $\alpha$ is used as the functions' linear weight that scales the corresponding cost, and the variable $\beta$ linearly increases the functions' gradients as voltage continues to deviate.
More specifically, $\alpha$ determines the value of the functions at voltages $V_l$ and $V_h$, where $\alpha \in \mathbb{R}^{>0}$; for example, when $\alpha = 1$, then $\zeta_{h}(v_l) = 1$.
$\beta$ on the other hand may take any value in the range of $\mathbb{R}^{>2}$, to assure a continuously differentiable cost function.
Both $\alpha$ and $\beta$ were treated as constants and, for the scope of this work, set to $1$ and $2$, respectively.
Substituting these values into Equations \ref{ch1:equ:high-voltage-threshold-cost-complete} and \ref{ch1:equ:low-voltage-threshold-cost-complete}, simplifies the high and low cost functions to:

\begin{equation}
	\zeta_h(v) := \left|\frac{v-V_\text{ss}}{V_h-V_\text{ss}}\right|^{2}
	\label{ch1:equ:high-voltage-threshold-cost-simple}
\end{equation}

\begin{equation}
	\zeta_l(v) := \left|\frac{V_\text{ss}-v}{V_\text{ss}-V_l}\right|^{2}
	\label{ch1:equ:low-voltage-threshold-cost-simple}
\end{equation}
