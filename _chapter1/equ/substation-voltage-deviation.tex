\begin{equation}
	\zeta_\text{substation voltage}(v_{p}) := \sum_{p=1}^{3}{\begin{cases}
		\zeta_h(v_p) & \text{if } V_{ss} \leq v_p\\
		\zeta_l(v_p) & \text{otherwise}\\
	\end{cases}}
	\label{ch1:equ:substation-voltage-deviation}
\end{equation}

Here, $\zeta_h(v)$ and $\zeta_l(v)$ are functions that convert the single phase voltage $v_p$ into a normalised and increasing, positive cost.
If the voltage $v_p$ is greater than or equal to the nominal substation voltage $V_{ss}$, then the result from $\zeta_h(v)$ is used, otherwise the result from $\zeta_l(v)$ is used.
To define these two functions, the nominal voltage $V_n$, of 230V P2N, and the corresponding high and low voltage thresholds, $V_h$ and $V_l$ respectively, need to be introduced.

\begin{equation}
	\zeta_h(v) := \left|\left(\frac{v-V_{ss}}{V_h-V_n}\right)^2-1\right|^{-1}
	\label{ch:equ:high-voltage-threshold-cost}
\end{equation}

\begin{equation}
	\zeta_l(v) := \left|\left(\frac{V_{ss}-v}{V_n-V_l}\right)^2-1\right|^{-1}
	\label{ch:equ:low-voltage-threshold-cost}
\end{equation}