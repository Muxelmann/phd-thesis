\section{Overview}
\label{ch3:sec:overview}

In previous chapters the the question regarding how one can optimally control a single battery energy storage has been addressed.
It was shown that half-hourly forecasts can be used to predict demand due to customer behaviours.
With this knowledge, Battery Energy Storage Systems (BESS) can be scheduled to shave peak load in order to avoid overloading the already stressed system.
But sub-half-hourly issues could not be addressed by traditional BESS schedules, which is why two successive sub-half-hourly power adjustment methods were developed.
The first method improved network operation by focusing on the underlying three-phase network topology, whilst strictly following the underlying half-hourly SOC plan.
The second method on the other hand alleviated this constraint by adjusting total power flow instead.
Benefits from using BESS schedules complement dynamic feedback and yield improved power profiles with reduced peak load.

The next step is to take such schedules and apply them to multiple, distributed batteries.
To prevent the negative impact from battery charging, particularly when dealing with the home-charging of Electric Vehicles (EVs), their charging scheduling needs to be coordinated.
As already discussed in the literature review in Chapter \ref{ch-literature}, multiple control methods to coordinate Distributed Energy Resources (DER) that including EV charge scheduling methods, exist \cite{Atia2016, Bidram2012, Bidram2014, Dolan2012, Gill2014, Guerrero2008, Guerrero2013, Sugihara2013, Toledo2013, Wang2016, Vovos2007, Guerrero2013a, Mansouri-Samani1993, Marra2013, Mokhtari2013}. 
Those approaches propose demand prioritisation, multi-tariff environments and even other game theory based methods to maximise utility or to reduce cost.
In the context of EV charging, coordination their charge scheduling signals becomes a vital requirement to react to other EV's charging plans.
This statement is commonly acknowledged since most smart charging research focuses on algorithm improvements, but in a distributed system this scheduling assumption of perfect knowledge exchange no longer holds.

In fact, during distributed EV scheduling, control instructions that may be broadcasted by one EV to inform all other EVs in the system of e.g. an updated charging plan, need not or cannot be received and responded to at the exact same times unless some synchronisation amongst all EVs is guaranteed.
Therefore, this chapter assesses the impact of desynchronising message propagation by adding transmission jitter to the updating broadcasts, since the effect on the performance of a traditionally synchronised EV scheduling algorithm is unknown, and by doing so this chapter addresses objective 3 of this thesis, this was outlined in Section \ref{ch-introduction:sec:problem-statement}.
The charging of EVs is explicitly assessed instead of managing a collection of BESSs, since storage is able to release energy and thus provide grid support.
Traditional EVs on the other hand do not have such capabilities and need to be coordinated in order to avoid home-charging related load spikes.
To achieve this coordination, a robust smart-charging algorithm to determine multiple EVs' charging schedule is presented and executed in both a synchronised and a desynchronised messaging scenario.
A Multi-Agent System (MAS) is implemented to perform the distributed scheduling using the Foundation for Intelligent Physical Agents (FIPA) compliant agents.

Results show how synchronised scheduling leads to expected outcomes that have also been established in literature.
However, adding jitter to message broadcasting significantly changes the same algorithm's behaviour.
Differences regarding rate of convergence and criteria for stability are most noticeable.
The structure of this chapter is as follows:
First, the EV demand and scheduling mechanism to coordinate the synchronised and desynchronised smart-charging is explained in Section \ref{ch3:sec:ev-coordination}.
Next, in Section \ref{ch3:sec:distributed-systems}, the distributed control system for the chosen MAS is presented, alongside the two cases for synchronised and desynchronised information propagation.
Section \ref{ch3:sec:results} presents and discusses the results from these two cases, upon which a conclusion is drawn in Section \ref{ch3:sec:summary}.
