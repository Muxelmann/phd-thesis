\section{Summary}
\label{ch3:sec:summary}

When designing a smart-charging algorithm to distribute the EV load over the entire day and thus avoid new demand spikes, coordination between EVs is usually achieved by the means of ICT.
In this chapter, Chapter~\ref{ch3}, such an algorithm was developed to assure that the coordinated charging of an EV fleet dos not add a new demand spike onto the base power profile.
This algorithm was then deployed on a MAS and controlled using two parameters, i.e. $\alpha$ and $\beta$, that allowed each agent to, respectively, undo and reassign an amount of its charging profile.
By repeating this behaviour of undoing and reassigning fractions of the charging profile, agents were able to respond to each other and avoid simultaneous charging.
Two performance metrics, i.e. $\zeta^\text{PAR}$ and $\zeta^\text{TRA}$, indicated, respectively, the spikiness and volatility of the final power profile.
Reducing these metrics is therefore the key function of the smart-charging algorithm, despite the algorithm not being metric dependent.

Originally however, the presented smart-charging algorithm was designed for synchronised MAS execution, which means that all agents obtain a network update and update their charging profile at exactly the same time.
By desynchronising the agent communication, the output's parameter dependence significantly changed when compared to the synchronised execution of the algorithm.
In fact regular and irregular desynchronisation yielded much lower values for $\zeta^\text{PAR}$ and $\zeta^\text{TRA}$, as seen in Section~\ref{ch3:subsec:algorithm-performance-desynchronised-regular} and Section~\ref{ch3:subsec:algorithm-performance-desynchronised-irregular}.
Convergence towards the final values on the other hand, did remained similar to the synchronised algorithm execution despite the difference in MAS execution.
Therefore, the algorithm's valley-filling behaviour was still upheld, yet the interplay between agents that implement this algorithm significantly changed the outcome of the aggregated result.
This work thus completes \ref{objective-3} of this thesis, which was outlined in Section~\ref{ch-introduction:sec:problem-statement}, since it shows the capabilities of a smart-charging algorithm and highlights the importance of considering agent de/synchronisation when developing a multi-controller DSM network.
Such findings are especially relevant due to the inherent difficulty and cost associated with the synchronisation of a distributed control system.
More specifically, synchronisation becomes particularly difficult when the network size and number of controllers increases.
With lightweight algorithms like the one proposed in this chapter, Chapter~\ref{ch3}, synchronisation can be neglected without sacrificing algorithm performance.
Nonetheless, this finding is true for any smart algorithm, as long as the algorithm is studied in both a synchronised and desynchronised test environment; which is however done very seldom.
This inherent difficulty of designing and implementing any smart algorithm with ICT, would thus raise the question if it is possible to design a cooperative algorithm that does not rely on ICT.
The subsequent chapter, Chapter~\ref{ch4}, intends to answer this question.






