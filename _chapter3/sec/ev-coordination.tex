\section{Coordination of EV charging}
\label{ch3:sec:ev-coordination}

In this section an algorithm for EV charging is presented which is implemented in both a synchronised and desynchronised case.
The Irish load dataset \cite{IrishData2002} is used in combination with EV energy demand to test the proposed smart-charging algorithm.
Performance of the algorithm at preventing new power spikes from occurring is then assessed with the use of standard performance metrics that have already been used throughout this thesis (i.e. the PAR and TRA metric).
Finally, to study the convergence of the algorithm a convergence criteria as well as rate of convergence are presented, too.

\subsection{EV Demand}

\nomenclature[K]{$T^\text{sch}$}{Scheduling horizon for EV charging, where $T^\text{sch} \in \mathbb{Z}^{>0}$}
\nomenclature[K]{$u$}{EV unit number, where $u \in [1, \dots, U]$}
\nomenclature[K]{$n$}{Iteration number of EV scheduling algorithm, where $N \in [1, \dots, N]$}
\nomenclature[K]{$U$}{Number of EVs that need to be scheduled, where $U \in \mathbb{Z}^{>0}$}
\nomenclature[K]{$N$}{Number of algorithm iterations to schedule multiple EVs, where $N \in \mathbb{Z}^{>0}$}
\nomenclature[K]{$p^\text{EV}_{u,n}(t)$}{Scheduled EV charging power, for EV $u$ at algorithm iteration $n$ for time $t$, where $p^\text{EV}_{u,n}(t) \in \mathbb{R}^{\geq 0}$}
\nomenclature[K]{$\textbf{p}^\text{EV}_{u, n}$}{Scheduled EV charging power vector, for EV $u$ at algorithm iteration $n$, where $\textbf{p}^\text{EV}_{u,n} = (p^\text{EV}_{u,n}(t))$}

\nomenclature[K]{$P^\text{min}_{u}$}{Minimum EV charging power }
\nomenclature[K]{$P^\text{min}_{u}$}{Maximum EV charging power }

In order to simulate the fleet of charging EVs, $U$ where $U \in \mathbb{Z}^{>0}$, each EV, $u$ where $u \in \{1, \dots, U\}$, is modelled as a load that needs to consume a certain amount of energy, $E_u$ where $E_u \in \mathbb{R}^{\geq0}$, over the course of a scheduling horizon, $T^\text{sch}$ where $T^\text{sch} \in \mathbb{Z}^{>0}$.
Unlike typical load profiles (for example household load profiles) EVs are modelled to not have a predetermined load profile.
Instead they are flexible so they schedule their own demand, $p^\text{EV}_{u,n}(t)$ where $\textbf{p}^\text{EV}_{u,n} = (p^\text{EV}_{u,n}(t))$, at any moment time, $t$ where $t \in \mathbb{Z}^{\geq0}$.
In other words, EVs can autonomously assign their own charging plan over the predetermined number of future time-slots, $T^\text{sch}$.
Due to limitations in on-board power electronics each EV's maximum charge rate, $P^\text{max}_{u}$ where $P^\text{max}_{u} \in \mathbb{R}^{>0}$, is restricted and may not be exceeded.
In order to meet the EV's charging demand over the scheduling horizon, $T^\text{sch}$, a soft minimum charging power, $P^\text{min}_{u}$ where $P^\text{min}_{u} \in \mathbb{R}^{>0}$, is also introduced:

\begin{equation}
	P_{min,u} := \frac{E_u}{H}
	\label{ch3:equ:power-charge-minimum}
\end{equation}

Although the upper limit represents the technical restriction of the on board charging equipment, this lower limit is a necessity to assure the demanded energy is charged over the entire charging period and thus this limit is also used to initiate the scheduling procedure itself.
This charging procedure is explained in Section~\ref{ch3:subsec:scheduling-algorithm}.
EVs utilise their agent system to purchase energy quantities for each time-slot, $t$, and also sell or ``undo'' some of the already acquired energy quantities if it contributes towards the lowering of a potential load spike.

\subsection{Baseline Load}

\nomenclature[K]{$\Delta t$}{Sample period for EV scheduling, where $\Delta t \in \mathbb{R}^{>0}$}

Historic customer load profiles were used in this work to represent real power consumption in simulations \cite{IrishData2002}.
For every of the containing 543 loads, this dataset consisted of 7392 demand readings which were sampled at half-hourly period (i.e. $\Delta t = 0.5$ hours).
A single scheduling horizon was defined as one day (i.e. $T^\text{sch}=48$ samples).

In this MAS related context, each household dispatches its agents once to acquire the household's half-hourly energy demand for the entire day; thus making the assumption that demand foresight is available.
It is worth mentioning that perfect demand foresight, especially for domestic loads, is difficult to realise.
Imperfect foresight would result in sub-optimal power profile allocation, for example.
Therefore, and target the focus of this chapter on determining the difference between synchronised and desynchronised algorithm execution, the assumption of perfect foresight is made.
After each household has acquired or reserved its daily demand by issuing an energy request, the entire network demand is known to the energy supplier and can be relayed to all EV agents when they query for the overall demand profile.
This ability is exploited when scheduling and negotiating the unknown EV charging profiles.
More specifically, all EV's agents communicate with the supplier's agent to optimally embed their charging profiles (i.e. $p^\text{EV}_{u,n}(t)$) within this aggregated baseline load.

\subsection{Scheduling Algorithm}
\label{ch3:subsec:scheduling-algorithm}

For the EV charging coordination strategy, an algorithm was designed that generates charging profiles for each EV so that the network power, $p^\text{net}(t)$ where $p^\text{net}(t) \in \mathbb{R}$, is optimised.
Here an optimal network power implies that when adding all aggregated charging profiles to the network's baseline load, $p^\text{base}(t)$  where $p^\text{base}(t) \in \mathbb{R}^{>0}$, no additional power spikes occur in the resulting power profile (i.e. $p^\text{net}(t) = p^\text{base}(t) + \sum_{u=1}^U p^\text{EV}_{u,n}(t)$ where $p^\text{EV}_{u,n}(t)$ is the EV charging power).
The charging profiles are generated by repetitively querying energy supplies for the network's baseline load, adjusting individual EV charging profiles and resubmitting the adjusted charging profile.
As already stated, the common assumption when designing such a scheduling algorithm is that all scheduling entities are synchronised (i.e. they wait for each other) before querying for the network's baseline load.
For visualisation the message exchange between two loads and a supplier is shown in Figure~\ref{ch3:fig:agent-synchronisation}.
This visualisation also includes the agent synchronisation event.

\begin{figure}\centering
\tikzstyle{box} = [%
	draw,%
	rectangle,%
	%fill=green!20,%
	minimum height=2em,%
	minimum width=2em,%
]
\tikzstyle{bracket} = [%
	decorate,%
	decoration={brace,amplitude=5pt,raise=1mm},%
	yshift=0mm,%
	color=black!50,%
]
\tikzstyle{bracket_text} = [%
	black,%
	midway,%
	xshift=2mm,%
	align=left,%
	color=black!50,%
]
\begin{tikzpicture}[node distance=2cm, shorten >= 1pt, >=stealth', auto, scale=1, transform shape]
	\pgfmathsetmacro\N{4}

	% Draw supplyer
    \draw (0, 0) node [box, fill=black!10](supplier) {Supplier};
    \draw (5, 0) node [box, fill=black!10](load1) {Load$_1$};
    \draw (10, 0) node [box, fill=black!10](load2) {Load$_2$};
    
    \draw (0, -11.5) node [](s_end) {};
    \draw (0, -10) node [](s_sync) {};
    
    \draw (5, -1) node [](l1_q1) {};
    \draw (5, -2) node [](l1_r1) {};
    \draw (5, -5) node [](l1_u1) {};
    \draw (5, -6) node [](l1_a1) {};
    \draw (5, -11) node[](l1_q2) {};
    
    \draw (10, -1.1) node [](l2_q1) {};
    \draw (10, -3.0) node [](l2_r1) {};
    \draw (10, -7.5) node [](l2_u1) {};
    \draw (10, -8.5) node [](l2_a1) {};
    \draw (10, -11.1) node[](l2_q2) {};
    
    
    \draw [dotted] (supplier.south) to (s_end.center);
    \draw [dotted] (load1.south) to (s_end.center-|l1_q1.center);
    \draw [dotted] (load2.south) to (s_end.center-|l2_q1.center);
    
    \draw [->] (l1_q1.center) to node[pos=0.5, above]{query$(l_1)$} (supplier.center|-l1_q1);
    \draw [->] (supplier.center|-l1_q1) to
    			(supplier.center|-l1_r1) to node[pos=0.5, above]{reply$(l_1)$} (l1_r1.center);
    \draw [bracket] (l1_q1) -- (l1_r1) node[bracket_text]{waiting};
    \draw [->] (l1_r1.center) to 
    			(l1_u1.center) to node[pos=0.5, above]{update$(l_1)$} (supplier.center|-l1_u1);
    \draw [bracket] (l1_r1) -- (l1_u1) node[bracket_text]{scheduling};
    \draw [->] (supplier.center|-l1_u1) to
    			(supplier.center|-l1_a1) to node[pos=0.5, above]{ack$(l_1)$} (l1_a1.center);
    \draw [bracket] (l1_u1) -- (l1_a1) node[bracket_text]{waiting};
    \draw [bracket] (l1_a1) -- (l1_a1|-s_sync.90) node[bracket_text]{syncing};
    
    \draw [->] (l2_q1.center) to node[pos=0.25, above]{query$(l_2)$} (supplier.center|-l2_q1);
    \draw [->] (supplier.center|-l2_q1) to
    			(supplier.center|-l2_r1) to node[pos=0.75, above]{reply$(l_2)$} (l2_r1.center);
    \draw [bracket] (l2_q1) -- (l2_r1) node[bracket_text]{waiting};
    \draw [->] (l2_r1.center) to
    			(l2_u1.center) to node[pos=0.25, above]{update$(l_2)$} (supplier.center|-l2_u1);
    \draw [bracket] (l2_r1) -- (l2_u1) node[bracket_text]{scheduling};
    \draw [->] (supplier.center|-l2_u1) to
    			(supplier.center|-l2_a1) to node[pos=0.75, above]{ack$(l_2)$} (l2_a1.center);
    \draw [bracket] (l2_u1) -- (l2_a1) node[bracket_text]{waiting};
    \draw [bracket] (l2_a1) -- (l2_a1|-s_sync.90) node[bracket_text]{syncing};
    
    \draw [thick] (supplier.180|-s_sync) to node[pos=1, right]{SYNC} (load2.0|-s_sync);
    
    \draw [->] (l1_q2.center) to node[pos=0.5, above]{query$(l_1)$} (supplier.center|-l1_q2);
    \draw [->] (l2_q2.center) to node[pos=0.25, above]{query$(l_2)$} (supplier.center|-l2_q2);
    
    \draw [-] (supplier.center|-l1_q2) to (s_end.center);
\end{tikzpicture}	
\caption{Agent synchronisation before re-scheduling their EVs charging profile.}
\label{ch3:fig:agent-synchronisation}
\end{figure}

In Figure~\ref{ch3:fig:agent-synchronisation} the horizontal arrows indicate messages being sent from loads (i.e. EV agents) to a supplier and vertical lines indicate processing or idle time.
Here a single scheduling iteration is shown, which can be broken into the sub-processes of: querying, scheduling, updating and synchronising.
From top to bottom, the sequential execution of these sub-processes is as follows
First, both load$_1$ and load$_2$ query the supplier for the currently known network load (i.e. query($l_1$) and query($l_2$)).
This network load is used to schedule their power profiles to ``fill valleys'' (i.e. only charge EVs during periods of low demand).
Upon receipt of a reply from the energy supplier (i.e. reply($l_1$) and reply($l_2$)) both loads immediately start scheduling their profiles.
In the example above load$_1$ found a solution before load$_2$ and can therefore inform the supplier about its intended load profile sooner.
It does so by sending an update (i.e. update$(l_1)$) to the supplier.
Subsequently querying the supplier for an updated network load would be premature since the other load (i.e. load$_2$) has not yet generated and updated its load profile.
Therefore a synchronisation mechanism had to be used which forces load$_1$ to wait until all loads have sent updates to the supplier.
In this example load$_1$ waits until load$_2$ has sent an update and the corresponding profile was acknowledged by the supplier (i.e. ack$(l_2)$).
Only after this had happened, a synchronisation event would be triggered (i.e. \textit{SYNC} event).
After this synchronisation event the next algorithm iteration is initiated and the procedure repeats.
Since all subsequent iterations are similar to the one shown in Figure~\ref{ch3:fig:agent-synchronisation} only the two querying messages of the second iteration are shown.

Although timing and message exchange has been outlined, the mechanism to allocate and reallocate charging powers in order to achieve the aforementioned valley filling behaviour has not yet been defined.
This behaviour is shown in Figure~\ref{ch3:fig:valley-filling-procedure} where several iterations numbered $n$ (where $n \in [1, \dots, N]$ where $N \in \mathbb{Z}^{>0}$) are shown and for each subsequent iteration some amount of prescheduled power is reallocated to different time-slots.
It is this reallocation to different time-slots that reduces and eventually completely mitigates EV charging spikes.

\begin{figure}\centering
\subfloat[]{
\begin{tikzpicture}[node distance=2cm, shorten >= 1pt, >=stealth', auto, scale=1, transform shape]
	\draw [<-] (0, 3) to node[pos=0.5,rotate=90,anchor=south]{power} (0, 0);
	\draw [->] (0, 0) to node[pos=0.5,anchor=north]{time} (3, 0);
	\draw (2.25, 0.3) node [fill=green!10,minimum width=0.5cm,minimum height=2.2cm,anchor=south](rect_add) {};
	\draw [very thick]
	(0.0, 2.0) -- (0.5, 2.0) --
	(0.5, 1.3) -- (1.0, 1.3) --
	(1.0, 1.1) -- (1.5, 1.1) --
	(1.5, 0.6) -- (2.0, 0.6) --
	(2.0, 0.3) -- (2.5, 0.3) --
	(2.5, 1.4) -- (3.0, 1.4);
	\draw [thick, green]
	(0.0, 2.0) -- (0.5, 2.0) --
	(0.5, 1.3) -- (1.0, 1.3) --
	(1.0, 1.1) -- (1.5, 1.1) --
	(1.5, 0.6) -- (2.0, 0.6) --
	(2.0, 2.5) -- (2.5, 2.5) --
	(2.5, 1.4) -- (3.0, 1.4);
\end{tikzpicture}
\label{ch3:subfig:valley-filling-1}
}
\hspace{10mm}
\subfloat[]{
\begin{tikzpicture}[node distance=2cm, shorten >= 1pt, >=stealth', auto, scale=1, transform shape]
	\draw [<-] (0, 3) to node[pos=0.5,rotate=90,anchor=south]{power} (0, 0);
	\draw [->] (0, 0) to node[pos=0.5,anchor=north]{time} (3, 0);
	\draw (2.25, 2.0) node [fill=red!10,minimum width=0.5cm,minimum height=0.5cm,anchor=south](rect_sub) {};
	\draw (1.75, 0.6) node [fill=green!10,minimum width=0.5cm,minimum height=0.5cm,anchor=south](rect_add) {};
	\draw [very thick]
	(0.0, 2.0) -- (0.5, 2.0) --
	(0.5, 1.3) -- (1.0, 1.3) --
	(1.0, 1.1) -- (1.5, 1.1) --
	(1.5, 0.6) -- (2.0, 0.6) --
	(2.0, 0.3) -- (2.5, 0.3) --
	(2.5, 1.4) -- (3.0, 1.4);
	\draw [thick, green]
	(0.0, 2.0) -- (0.5, 2.0) --
	(0.5, 1.3) -- (1.0, 1.3) --
	(1.0, 1.1) -- (1.5, 1.1) --
	(1.5, 1.1) -- (2.0, 1.1) --
	(2.0, 2.0) -- (2.5, 2.0) --
	(2.5, 1.4) -- (3.0, 1.4);
	\draw [->, bend right] (rect_sub.180) to (rect_add.90);
\end{tikzpicture}
\label{ch3:subfig:valley-filling-2}
}
\hspace{10mm}
\subfloat[]{
\begin{tikzpicture}[node distance=2cm, shorten >= 1pt, >=stealth', auto, scale=1, transform shape]
	\draw [<-] (0, 3) to node[pos=0.5,rotate=90,anchor=south]{power} (0, 0);
	\draw [->] (0, 0) to node[pos=0.5,anchor=north]{time} (3, 0);
	\draw (2.25, 1.6) node [fill=red!10,minimum width=0.5cm,minimum height=0.4cm,anchor=south](rect_sub_1) {};
	\draw (1.75, 1.0) node [fill=green!10,minimum width=0.5cm,minimum height=0.4cm,anchor=south](rect_add_1) {};
	\draw (1.25, 1.1) node [fill=green!10,minimum width=0.5cm,minimum height=0.3cm,anchor=south](rect_add_2) {};
	\draw [fill=red!10,red!10] (1.5, 1.0) rectangle (2.0,1.1);
	\draw [very thick]
	(0.0, 2.0) -- (0.5, 2.0) --
	(0.5, 1.3) -- (1.0, 1.3) --
	(1.0, 1.1) -- (1.5, 1.1) --
	(1.5, 0.6) -- (2.0, 0.6) --
	(2.0, 0.3) -- (2.5, 0.3) --
	(2.5, 1.4) -- (3.0, 1.4);
	\draw [thick, green]
	(0.0, 2.0) -- (0.5, 2.0) --
	(0.5, 1.3) -- (1.0, 1.3) --
	(1.0, 1.4) -- (1.5, 1.4) --
	(1.5, 1.4) -- (2.0, 1.4) --
	(2.0, 1.6) -- (2.5, 1.6) --
	(2.5, 1.4) -- (3.0, 1.4);
	\draw [->, bend right] (rect_sub_1.180) to (rect_add_2.45);
\end{tikzpicture}
\label{ch3:subfig:valley-filling-3}
}
\vspace{1mm}
\subfloat[]{
\begin{tikzpicture}[node distance=2cm, shorten >= 1pt, >=stealth', auto, scale=1, transform shape]
	\draw [<-] (0, 3) to node[pos=0.5,rotate=90,anchor=south]{power} (0, 0);
	\draw [->] (0, 0) to node[pos=0.5,anchor=north]{time} (3, 0);
	\draw [very thick]
	(0.0, 2.0) -- (0.5, 2.0) --
	(0.5, 1.3) -- (1.0, 1.3) --
	(1.0, 1.1) -- (1.5, 1.1) --
	(1.5, 0.6) -- (2.0, 0.6) --
	(2.0, 0.3) -- (2.5, 0.3) --
	(2.5, 1.4) -- (3.0, 1.4);
	\draw [thick, green]
	(0.0, 2.0) -- (0.5, 2.0) --
	(0.5, 1.42) -- (1.0, 1.42) --
	(1.0, 1.42) -- (1.5, 1.42) --
	(1.5, 1.42) -- (2.0, 1.42) --
	(2.0, 1.42) -- (2.5, 1.42) --
	(2.5, 1.42) -- (3.0, 1.42);
	\draw (1.5, 3.3) node[anchor=south] {$\vdots$};
\end{tikzpicture}
\label{ch3:subfig:valley-filling-4}
}
\caption{Charging power (green line) allocation on top of base network load (black line) for valley-filling behaviour. Here $n=1$ for Fig. \ref{ch3:subfig:valley-filling-1}, $n=2$ for Fig. \ref{ch3:subfig:valley-filling-2}, $n=3$ for Fig. \ref{ch3:subfig:valley-filling-3}, and $n=N$ for Fig. \ref{ch3:subfig:valley-filling-4}.}
\label{ch3:fig:valley-filling-procedure}	
\end{figure}

\nomenclature[K]{$p^\text{base}_{n}(t)$}{Baseline load at time $t$, where $p^\text{base}_{n}(t) \in \mathbb{Z}^{\geq0}$}
\nomenclature[K]{$\textbf{p}^\text{base}_{n}$}{Baseline load vector, where $p^\text{base}_{n}(t) \in \textbf{p}^\text{base}_{n}$}
\nomenclature[K]{$\hat{p}^\text{base}_{n}(t)$}{Temporary demand at time $t$, i.e. the aggregate of all EV charge vector and the baseline load vector, where $\hat{p}^\text{base}_{n}(t) \in \mathbb{Z}^{\geq0}$}
\nomenclature[K]{$\hat{\textbf{p}}_{\text{base},n}$}{Temporary demand vector, i.e. the aggregate of all EV charge vector and the baseline load vector, where $\hat{p}^\text{base}_{n}(t) \in \hat{\textbf{p}}_{\text{base},n}$}

For every iteration, $n$, in Figure~\ref{ch3:fig:valley-filling-procedure}, charging profiles are added onto a baseline network load, $\textbf{p}^\text{base}_{n}$, where $\textbf{p}^\text{base}_{n} = (p^\text{base}_{n}(t))$.
This base load is shown as the bold black line and does not change throughout the EV scheduling procedure.
For every iteration numbered $n$, the charging profile of EV number $u$ is defined as $p^\text{EV}_{u,n}$, where $\textbf{p}^\text{EV}_n = (p^\text{EV}_{u,n}(t))$ (i.e. $\textbf{p}^\text{EV}_n$ consists of all EV charging profiles at iteration $n$).
During the first iteration however (for example Figure~\ref{ch3:subfig:valley-filling-1} where $n=1$) this charging profile is determined by assigning the maximum EV charging power to the time-slots of lowest load, until the total EV energy demand is met (i.e. consecutive time-slots $t$ are chosen where $t = \text{argmin}(\textbf{p}^\text{base})$).
Resulting is an aggregated charging power that is likely to contain at least one new charging spike since all EVs scheduled their profiles based upon the same knowledge of $\textbf{p}^\text{base}_{n}$.
This spike is seen on an updated (i.e. temporary) demand profile, $\hat{\textbf{p}}^\text{base}_{n}$, where $\hat{\textbf{p}}^\text{base}_{n} = (\hat{p}^\text{base}_{n}(t))$, is defined as:

\begin{equation}
	\hat{p}^\text{base}_{n}(t) := p^\text{base}_{n}(t) + \sum_{u=1}^{U} p_{u,n}(t) \forall t, n
	\label{ch3:equ:updated-demand-profile}
\end{equation}

\nomenclature[K]{$\hat{p}^\text{EV}_{u, n}(t)$}{Temporary charging power during iteration number $n$, for EV $u$, at time $t$, where $\hat{p}^\text{EV}_{u, n}(t) \in \textbf{Z}^{\geq0}$}
\nomenclature[K]{$\hat{\textbf{p}}^\text{EV}_{u, n}$}{Temporary charging vector during iteration number $n$, for EV $u$, where $\hat{p}^\text{EV}_{u, n}(t) \in \hat{\textbf{p}}^\text{EV}_{u, n}$}

For the next iteration $n+1$ (for example Figure~\ref{ch3:subfig:valley-filling-2} where $n=2$) a proportion of the previously scheduled power vector $\textbf{p}^\text{EV}_{n-1}$ is ``undone''.
Subsequently, the spike in the resulting $\hat{\textbf{p}}^\text{base}_{n}$ is reduced, yet the energy that has been undone needs to be reallocated to meet the EVs' demands.
The amount by which $\textbf{p}^\text{EV}_{n-1}$ is reduced is determined by the ``\textit{undoing}'' parameter $\alpha$, where $\alpha \in [0, 1)$.
A new reduced (i.e. temporary) charging vector $\hat{p}^\text{EV}_{u, n}(t)$ is therefore defined as:

\begin{equation}
	\hat{p}^\text{EV}_{u,n}(t) := p^\text{EV}_{u,n-1}(t)(1-\alpha)
	\label{ch3:equ:temporary-charging-power}
\end{equation}

Using this temporary charging power the temporary energy demand, $\hat{E}_{u,n}$, that needs to be reallocated can also be defined by including the sampling period $\Delta t$:

\begin{equation}
	\hat{E}_{u,n} := E_u - \sum_{\tau=1}^{T_\text{sch}} \hat{p}_{\text{EV},u,n}(\tau)\Delta t \forall u \text{ and } n > 1
	\label{ch3:equ:temporary-charging-demand}
\end{equation}

To include the first iteration of the algorithm, Equation~\ref{ch3:equ:temporary-charging-demand} needs to be expanded to redefine $\hat{E}_{u,n}$ for all successive algorithm iterations $n$:

\begin{equation}
	\hat{E}_{u,n} :=
	\begin{cases}
		E_u &\text{if } n=1\\
		E_u - \sum_{\tau=1}^{T_\text{sch}} \hat{p}_{\text{EV},u,n}(\tau)\Delta t &\text{otherwise}
	\end{cases}
	\forall u \text{ and } \forall n
	\label{ch3:equ:temporary-charging-demand-expanded}
\end{equation}

Following the similar procedure as for the first iteration, $\hat{E}_{u,n}$ is then allocated to different time-slots where the rule of performing the power allocation (i.e. allocating $p^\text{EV}_{u,n}(t)$ based on $p^\text{EV}_{u,n-1}(t)$) is defined as follows:

\begin{equation}
\begin{split}
	p_{u,k}(n) :=
	\begin{cases}
		\frac{\hat{E}_u(n)}{\kappa}\beta &\text{if } \frac{\hat{E}_u(n)}{\kappa}\beta \leq P_{max,u}\\
		P_{max,i} &\text{otherwise}
 	\end{cases}
 	\forall u  \\\text{ where } k = \text{argmin}_k(\textbf{p}_{base}(n)) \text{ and } n > 1
\end{split}
\label{ch3:equ:valley-filling-equation}
\end{equation}

\nomenclature[K]{$\alpha$}{Undoing parameter to remove a portion of the temporary energy demand, $\hat{E}_{u,n}$, where $\alpha \in (0, 1]$}
\nomenclature[K]{$\beta$}{Allocation parameter to assign a portion of the temporary energy demand, $\hat{E}_{u,n}$, where $\beta \in (0, 1]$}

Here $\beta$ is the maximum ``allocation'' parameter, where $\beta \in (0, 1]$, and this parameter limits the power that may be allocated to any successive time-slot (i.e. $t-\Delta t$).
To not exceed the EV's maximum charging power any value in the charging vector, $\textbf{p}^\text{EV}_{u,n}$, is capped to $P^\text{max}_{u}$.
If $\beta$ is chosen as one then the undone energy is allocated as quickly as possible, but this can lead to new power spikes and this slow the converging behaviour.
For smaller values of $\beta$ on the other hand, the undone charge is reallocated in smaller portions and the old power spike is distributed gradually.
If this value is too small however then the total EV demand, $E_u$, cannot be allocated across the entire scheduling horizon, $T^\text{sch}$.
The constraint in Equation~\ref{ch3:equ:valley-filling-equation} thus guarantees that EV demand is allocated over this finite scheduling horizon.
By doing so, the temporary energy demand equates to the original energy demand after the new charging powers are assigned to their corresponding time-slot (i.e. after $\textbf{p}^\text{EV}_{u,n}$ is updated).

In any following algorithm iteration (i.e. $n>2$ as shown in Figure~\ref{ch3:subfig:valley-filling-3}) each EV's charging profile is adjusted and spread further over the base load, $\textbf{p}^\text{base}$.
In the end (i.e. when $n=N$) the ideal EV charging profiles add to the base load in such a way that the resulting network load has an optimally filled valley.
This valley filling behaviour was thus achieved by the ``undoing'' and ``allocation'' of EV charging power from one algorithm iteration to the next.
For the purpose of this simulation, the algorithm terminates when the final iteration is reached (i.e. $n=N$) regardless of the final network load's shape.
Here the rate of convergence of the algorithm differs based upon the choice of $\alpha$ and $\beta$ values.
This rate can be estimated and compared for each choice of $\alpha$ and $\beta$ since the simulations terminate after the same number of iterations.
However, convergence is in fact guaranteed when selecting values of $\alpha < 1$ and $\beta < 1$ since, in those cases, the algorithm satisfies the D'Alembert Criterion; the criterion requires a continuous but not necessarily regular reduction in successive outputs.

To summarise this section, the complete EV scheduling algorithm was developed by: 
\begin{enumerate*}
	\item defining the message exchange and synchronisation mechanism, which is shown in Figure~\ref{ch3:fig:agent-synchronisation};
	\item formulating the initial and successive ``undoing'' of charging power, as shown in Equation~\ref{ch3:equ:temporary-charging-power}; and
	\item defining the iterative update and ``allocation'' of the temporary energy demand, as defined in Equation~\ref{ch3:equ:temporary-charging-demand-expanded}.
\end{enumerate*}
For clarity, this smart charging algorithm's pseudocode that performs the complete valley filling procedure has been included below; in Algorithm \ref{ch3:alg:valley-filling}.

\begin{algorithm}
 \SetKwFunction{Query}{query}
 \SetKwFunction{Update}{update}
 \SetKwFunction{Break}{break}
 \SetKwFunction{Sync}{synchronising}
 \KwData{$\textbf{p}_{\text{base},n}$, $E_u$, $P_{\text{max},u}$, $P_{\text{min},u}$, $\Delta t$, $T_\text{sch}$}
 \KwResult{$\textbf{p}_{\text{EV},u,n}$}
 \For{$n\leftarrow 1$ \KwTo $N$}{
  \tcp{Query for base load}\label{cmt}
  $\textbf{p}_{\text{base},n} \leftarrow$ \Query{}\;
  \tcp{Forward and undo previous schedule}\label{cmt}
  \eIf{$n>1$}{
    $\textbf{p}_{\text{EV},u,n} \leftarrow \textbf{p}_{\text{EV},u,n-1}\alpha$\;
   }{
    $\textbf{p}_{\text{EV},u,n} \leftarrow [0, 0, \dots, 0]$\;
  }
  \tcp{Determine unallocated energy}\label{cmt}
  $\hat{E}_{u,n} = E_u - \sum_{\tau=1}^{T_\text{sch}} p_{\text{EV},u,n}(\tau)\Delta t$\;
  \tcp{Fill valley}\label{cmt}
  \For{$\tau\leftarrow \text{argmin}(\textbf{p}_{\text{base},n})$ \KwTo $\text{argmax}(\textbf{p}_{\text{base},n})$}{
   \eIf{$p_{\text{EV},u,n}(\tau) + \frac{\hat{E}_{u,n}}{\Delta t}\beta \leq P_{\text{max},u}$}{
    $p_{\text{EV},u,n}(\tau) \leftarrow p_{\text{EV},u,n}(\tau) + \frac{\hat{E}_{u,n}}{\Delta t}\beta$\;
   }{
    $p_{\text{EV},u,n}(\tau) \leftarrow P_{\text{max},u}$\;
   }
   $\hat{E}_{u,n} = E_u - \sum_{\tau=1}^{T_\text{sch}} p_{\text{EV},u,n}(\tau) \Delta t$\;
   \tcp{Once EV profile is found, send update}\label{cmt}
   \If{$\hat{E}_{u,n} = 0$}{
    \Update{$\textbf{p}_{\text{EV},u,n}$}\;
   	\Break{}\;
   }
  }
  \Sync{}\;
 }
 \caption{Robust valley filling algorithm for a single EV in}
 \label{ch3:alg:valley-filling}
\end{algorithm}













