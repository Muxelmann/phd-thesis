\section{Coordination of EV charging}
\label{ch3:sec:ev-coordination}

Here, an algorithm for EV charging is presented, which is later implemented in both a synchronised and desynchronised case.
Real load data is used in combination with EV demand to evaluate the performance of the algorithm at preventing new power spikes from occurring.
Convergence of the algorithm is studied and convergence criteria as well as rate of convergence are also presented.

This section is structured as follows.
First, the means and assumptions for calculating EV demand is defined.
Then the real load data is introduced and explained.
The EV scheduling algorithm is introduced next, before the performance parameters are presented.

\subsection{EV Demand}

EVs were modelled as loads that, over the course of a scheduling horizon, $H$, each need to consume a certain amount of energy, $E_u$, to simulate charging their batteries.
Each EV, i.e. $u$, is part of a fleet of charging and coordinated EVs, i.e. $u \in [1, \dots, U]$.
Unlike typical loads (e.g. households), EVs do not have a predetermined load profile in this simulation and are therefore flexible to schedule their demand, $p_{u,k}$, where $p_{u,k} \in \textbf{p}_u$.
In other words, they need to autonomously assign each charging plan over a certain number of future time-slots.
Due to limitations in on-board power electronics, each EV's maximum charge rate, $P_{max,u}$, is restricted and may not be exceeded.
Equally, in order to meet the EV's charging demand over the scheduling horizon, $H$, a soft minimum charging power, $P_{min,u}$ is also introduced:

\begin{equation}
	P_{min,u} := \frac{E_u}{H}
	\label{ch3:equ:power-charge-minimum}
\end{equation}

Although the upper limit is hard, i.e. caused by technical restrictions, this lower limit is a necessity to initiate the scheduling procedure, which will become apparent when the scheduling algorithm is explained.
Using MAS, EVs utilise their broker agents to purchase energy quantities for each time-slot, $k$, and also sell or ``undo'' some of the already acquired energy quantities if it contributed towards a new load spike.

\subsection{Base Load}

To represent real power consumption in simulations, historic customer load profiles were used in this work \cite{IrishData2002}.
This dataset consisted of 7392 demand readings for 543 loads, which were sampled at half-hourly period, i.e. $\kappa = 0.5$ hours.
A single scheduling horizon was defined as one day, therefore $H=48$ samples.

In this context, each household, i.e. physical entity, dispatches its broker agents to order the household's daily energy demand; therefore it a scenario with some foresight is assumed.
After having issued this energy request, the entire network demand is known to the supplier and can be relayed to all broker agents when they query for it.
This ability is exploited when scheduling and negotiating the unknown EV charging profiles.
More specifically, all households' broker agents communicate with the suppler's broker agents to optimally embed their charging profiles, $p_u(k)$, within this aggregated base load, $d(k)$.

\subsection{Scheduling Algorithm}

For the EV charging coordination strategy, an algorithm was designed that generates optimised charging profiles for each EV.
Here, optimality implies that when adding all aggregated charging profiles to the network's base demand, $d(k)$, no additional spikes occur in the resulting demand profile, $p(k)$.
These profiles were generated by repetitively querying for the network's base load, adjusting individual EV charging profiles, and resubmitting the adjusted charging profile.
As already stated, the common assumption when designing such a scheduling algorithm is that all scheduling entities are synchronised, i.e. wait for each other, before querying for the network's base load.
For visualisation, the message exchange bewteen two loads and a supplier including a synchronisation time is shown in Figure \ref{ch3:fig:agent-synchronisation}.

\begin{figure}\centering
\tikzstyle{box} = [%
	draw,%
	rectangle,%
	%fill=green!20,%
	minimum height=2em,%
	minimum width=2em,%
]
\tikzstyle{bracket} = [%
	decorate,%
	decoration={brace,amplitude=5pt,raise=1mm},%
	yshift=0mm,%
	color=black!50,%
]
\tikzstyle{bracket_text} = [%
	black,%
	midway,%
	xshift=2mm,%
	align=left,%
	color=black!50,%
]
\begin{tikzpicture}[node distance=2cm, shorten >= 1pt, >=stealth', auto, scale=1, transform shape]
	\pgfmathsetmacro\N{4}

	% Draw supplyer
    \draw (0, 0) node [box, fill=black!10](supplier) {Supplier};
    \draw (5, 0) node [box, fill=black!10](load1) {Load$_1$};
    \draw (10, 0) node [box, fill=black!10](load2) {Load$_2$};
    
    \draw (0, -11.5) node [](s_end) {};
    \draw (0, -10) node [](s_sync) {};
    
    \draw (5, -1) node [](l1_q1) {};
    \draw (5, -2) node [](l1_r1) {};
    \draw (5, -5) node [](l1_u1) {};
    \draw (5, -6) node [](l1_a1) {};
    \draw (5, -11) node[](l1_q2) {};
    
    \draw (10, -1.1) node [](l2_q1) {};
    \draw (10, -3.0) node [](l2_r1) {};
    \draw (10, -7.5) node [](l2_u1) {};
    \draw (10, -8.5) node [](l2_a1) {};
    \draw (10, -11.1) node[](l2_q2) {};
    
    
    \draw [dotted] (supplier.south) to (s_end.center);
    \draw [dotted] (load1.south) to (s_end.center-|l1_q1.center);
    \draw [dotted] (load2.south) to (s_end.center-|l2_q1.center);
    
    \draw [->] (l1_q1.center) to node[pos=0.5, above]{query$(l_1)$} (supplier.center|-l1_q1);
    \draw [->] (supplier.center|-l1_q1) to
    			(supplier.center|-l1_r1) to node[pos=0.5, above]{reply$(l_1)$} (l1_r1.center);
    \draw [bracket] (l1_q1) -- (l1_r1) node[bracket_text]{waiting};
    \draw [->] (l1_r1.center) to 
    			(l1_u1.center) to node[pos=0.5, above]{update$(l_1)$} (supplier.center|-l1_u1);
    \draw [bracket] (l1_r1) -- (l1_u1) node[bracket_text]{scheduling};
    \draw [->] (supplier.center|-l1_u1) to
    			(supplier.center|-l1_a1) to node[pos=0.5, above]{ack$(l_1)$} (l1_a1.center);
    \draw [bracket] (l1_u1) -- (l1_a1) node[bracket_text]{waiting};
    \draw [bracket] (l1_a1) -- (l1_a1|-s_sync.90) node[bracket_text]{syncing};
    
    \draw [->] (l2_q1.center) to node[pos=0.25, above]{query$(l_2)$} (supplier.center|-l2_q1);
    \draw [->] (supplier.center|-l2_q1) to
    			(supplier.center|-l2_r1) to node[pos=0.75, above]{reply$(l_2)$} (l2_r1.center);
    \draw [bracket] (l2_q1) -- (l2_r1) node[bracket_text]{waiting};
    \draw [->] (l2_r1.center) to
    			(l2_u1.center) to node[pos=0.25, above]{update$(l_2)$} (supplier.center|-l2_u1);
    \draw [bracket] (l2_r1) -- (l2_u1) node[bracket_text]{scheduling};
    \draw [->] (supplier.center|-l2_u1) to
    			(supplier.center|-l2_a1) to node[pos=0.75, above]{ack$(l_2)$} (l2_a1.center);
    \draw [bracket] (l2_u1) -- (l2_a1) node[bracket_text]{waiting};
    \draw [bracket] (l2_a1) -- (l2_a1|-s_sync.90) node[bracket_text]{syncing};
    
    \draw [thick] (supplier.180|-s_sync) to node[pos=1, right]{SYNC} (load2.0|-s_sync);
    
    \draw [->] (l1_q2.center) to node[pos=0.5, above]{query$(l_1)$} (supplier.center|-l1_q2);
    \draw [->] (l2_q2.center) to node[pos=0.25, above]{query$(l_2)$} (supplier.center|-l2_q2);
    
    \draw [-] (supplier.center|-l1_q2) to (s_end.center);
\end{tikzpicture}	
\caption{Agent synchronisation before re-scheduling their EVs charging profile.}
\label{ch3:fig:agent-synchronisation}
\end{figure}

In this figure, the horizontal arrows indicate messages being sent from loads (i.e. EV agents) to a supplier and vertical lines indicate processing or idle time.
Shown within Figure \ref{ch3:fig:agent-synchronisation} is a single scheduling iteration, which can be broken into several sub-processes of querying, scheduling, updating and synchronising.
From top to bottom, the temporal execution of these sub-processes is as follows:
First, both load$_1$ and load$_2$ query the supplier for the currently known network load (i.e. query($l_1$) and query($l_2$)).
This network load is used to schedule their power profiles to ``fill valleys'', i.e. only charge their EVs during periods of low demand.
Upon receipt of a  reply from the energy supplier (i.e. reply($l_1$) and reply($l_2$)), both loads immediately start scheduling these profiles.
As shown in the example, load$_1$ found a solution before load$_2$ and can therefore inform the supplier about its intended load profile sooner, by sending an update (i.e. update$(l_1)$) to the supplier.
Subsequently querying the supplier for an updated network load would be premature, since other loads (i.e. load$_2$) has not yet generated and updated its load profile.
Therefore, a synchronisation mechanism has to be used, forcing load$_1$ to wait until all loads have sent updates to the supplier.
In this example, load$_1$ waits until load$_2$ has send an updated and the corresponding profile was acknowledged by the supplier (i.e. ack$(l_2)$).
Only thereafter, a synchronisation event may be triggered (i.e. \textit{SYNC} event).
After this synchronisation event, the next algorithm iteration is initiated and the procedure repeats.
Here, the second iteration's two querying messages are shown in the example.

%It is worth mentioning that this synchronisation is removed by purposefully desynchronising agents, and the impact of this desynchronisation on the algorithm is studied.
%However, the common algorithm design approach, where the presupposition that all loads will have successfully transmitted their charging profiles to the substation before the algorithm continues, is used in the design of the following algorithm.

\hl{Rewriting the entire algorithm explanation... The paper is very unclear!}

With this synchronised message and information exchange, the underlying procedure that assigns EV charging power at times of low energy demand still needs to be defined.
Figure \ref{ch3:fig:valley-filling} shows a graphical representation of the iterative allocation and reallocation of charging power as the valley filling algorithm progresses.

\begin{figure}\centering
\subfloat[]{%
\begin{tikzpicture}[node distance=2cm, shorten >= 1pt, >=stealth', auto, scale=1, transform shape]
	\draw [<-] (0, 3) to node[pos=0.5,rotate=90,anchor=south]{power} (0, 0);
	\draw [->] (0, 0) to node[pos=0.5,anchor=north]{time} (3, 0);
	\draw (2.25, 0.3) node [fill=green!10,minimum width=0.5cm,minimum height=2.2cm,anchor=south](rect_add) {};
	\draw [very thick]
	(0.0, 2.0) -- (0.5, 2.0) --
	(0.5, 1.3) -- (1.0, 1.3) --
	(1.0, 1.1) -- (1.5, 1.1) --
	(1.5, 0.6) -- (2.0, 0.6) --
	(2.0, 0.3) -- (2.5, 0.3) --
	(2.5, 1.4) -- (3.0, 1.4);
	\draw [thick, green]
	(0.0, 2.0) -- (0.5, 2.0) --
	(0.5, 1.3) -- (1.0, 1.3) --
	(1.0, 1.1) -- (1.5, 1.1) --
	(1.5, 0.6) -- (2.0, 0.6) --
	(2.0, 2.5) -- (2.5, 2.5) --
	(2.5, 1.4) -- (3.0, 1.4);
\end{tikzpicture}%
\label{ch3:subfig:valley-filling-1}%
}
\hspace{10mm}
\subfloat[]{%
\begin{tikzpicture}[node distance=2cm, shorten >= 1pt, >=stealth', auto, scale=1, transform shape]
	\draw [<-] (0, 3) to node[pos=0.5,rotate=90,anchor=south]{power} (0, 0);
	\draw [->] (0, 0) to node[pos=0.5,anchor=north]{time} (3, 0);
	\draw (2.25, 2.0) node [fill=red!10,minimum width=0.5cm,minimum height=0.5cm,anchor=south](rect_sub) {};
	\draw (1.75, 0.6) node [fill=green!10,minimum width=0.5cm,minimum height=0.5cm,anchor=south](rect_add) {};
	\draw [very thick]
	(0.0, 2.0) -- (0.5, 2.0) --
	(0.5, 1.3) -- (1.0, 1.3) --
	(1.0, 1.1) -- (1.5, 1.1) --
	(1.5, 0.6) -- (2.0, 0.6) --
	(2.0, 0.3) -- (2.5, 0.3) --
	(2.5, 1.4) -- (3.0, 1.4);
	\draw [thick, green]
	(0.0, 2.0) -- (0.5, 2.0) --
	(0.5, 1.3) -- (1.0, 1.3) --
	(1.0, 1.1) -- (1.5, 1.1) --
	(1.5, 1.1) -- (2.0, 1.1) --
	(2.0, 2.0) -- (2.5, 2.0) --
	(2.5, 1.4) -- (3.0, 1.4);
	\draw [->, bend right] (rect_sub.180) to (rect_add.90);
\end{tikzpicture}%
\label{ch3:subfig:valley-filling-2}%
}
\hspace{10mm}
\subfloat[]{%
\begin{tikzpicture}[node distance=2cm, shorten >= 1pt, >=stealth', auto, scale=1, transform shape]
	\draw [<-] (0, 3) to node[pos=0.5,rotate=90,anchor=south]{power} (0, 0);
	\draw [->] (0, 0) to node[pos=0.5,anchor=north]{time} (3, 0);
	\draw (2.25, 1.6) node [fill=red!10,minimum width=0.5cm,minimum height=0.4cm,anchor=south](rect_sub_1) {};
	\draw (1.75, 1.0) node [fill=green!10,minimum width=0.5cm,minimum height=0.4cm,anchor=south](rect_add_1) {};
	\draw (1.25, 1.1) node [fill=green!10,minimum width=0.5cm,minimum height=0.3cm,anchor=south](rect_add_2) {};
	\draw [fill=red!10,red!10] (1.5, 1.0) rectangle (2.0,1.1);
	\draw [very thick]
	(0.0, 2.0) -- (0.5, 2.0) --
	(0.5, 1.3) -- (1.0, 1.3) --
	(1.0, 1.1) -- (1.5, 1.1) --
	(1.5, 0.6) -- (2.0, 0.6) --
	(2.0, 0.3) -- (2.5, 0.3) --
	(2.5, 1.4) -- (3.0, 1.4);
	\draw [thick, green]
	(0.0, 2.0) -- (0.5, 2.0) --
	(0.5, 1.3) -- (1.0, 1.3) --
	(1.0, 1.4) -- (1.5, 1.4) --
	(1.5, 1.4) -- (2.0, 1.4) --
	(2.0, 1.6) -- (2.5, 1.6) --
	(2.5, 1.4) -- (3.0, 1.4);
	\draw [->, bend right] (rect_sub_1.180) to (rect_add_2.45);
\end{tikzpicture}%
\label{ch3:subfig:valley-filling-3}%
}
\vspace{1mm}
\subfloat[]{%
\begin{tikzpicture}[node distance=2cm, shorten >= 1pt, >=stealth', auto, scale=1, transform shape]
	\draw [<-] (0, 3) to node[pos=0.5,rotate=90,anchor=south]{power} (0, 0);
	\draw [->] (0, 0) to node[pos=0.5,anchor=north]{time} (3, 0);
	\draw [very thick]
	(0.0, 2.0) -- (0.5, 2.0) --
	(0.5, 1.3) -- (1.0, 1.3) --
	(1.0, 1.1) -- (1.5, 1.1) --
	(1.5, 0.6) -- (2.0, 0.6) --
	(2.0, 0.3) -- (2.5, 0.3) --
	(2.5, 1.4) -- (3.0, 1.4);
	\draw [thick, green]
	(0.0, 2.0) -- (0.5, 2.0) --
	(0.5, 1.42) -- (1.0, 1.42) --
	(1.0, 1.42) -- (1.5, 1.42) --
	(1.5, 1.42) -- (2.0, 1.42) --
	(2.0, 1.42) -- (2.5, 1.42) --
	(2.5, 1.42) -- (3.0, 1.42);
	\draw (1.5, 3.3) node[anchor=south] {$\vdots$};
\end{tikzpicture}%
\label{ch3:subfig:valley-filling-4}%
}
\caption{Charging power allocation for valley-filling behaviour}
\label{ch3:fig:valley-filling}	
\end{figure}

During the first iteration, i.e. $n=1$, a base network load, $\textbf{p}_{base}(n)$ (black line) indicates the underlying and unmodifiable load that is expected to occur.
Following the same annotation that was used in previous chapters, $k$ represents a half-hourly time-slot (i.e. sample period $\kappa = 30\text{minutes}$), so that each power value $p_{base,k}(n) \in \textbf{p}_{base}(n)$.
Here, $n$ represents the iteration number, where $n \in [1, \dots, N]$.
An EV charging power $p_{u,k}(n)$ (green line) is added to the time where demand is lowest.
The addition of this charging power is assigned for each EV, $u$, at time $k$, during the iteration step $n$, and for convenience this adding process is highlighted in green, too.
Therefore, each EV's charging profile consists of several charge power values, where $p_{u,k}(n) \in \textbf{p}_u(n)$.


During subsequent iterations, i.e. $n>1$, a proportion of the applied charging power, which potentially contributed to a new demand peak, is subtracted and added to new demand troughs.
The magnitude of the reduction is determined by an ``\textit{undoing}'' parameter, $\alpha$, where $\alpha \in [0, 1)$.
As seen in the above figure, the undone energy is highlighted in red and the corresponding addition of unassigned charge demand is, again, highlighted in green (an arrow emphasises the direction of reallocation).
\hl{Here, it is important to note that for all steps where power is added to the the EV's charging profile, none of the values must exceed the EV's maximum charging power, }$P_{max,u}$.
In fact, in order to spread out the added powers over multiple periods of low demand, a second so called ``\textit{spreading}'' parameter is used, $\beta$, where $\beta \in [0, 1)$.
Therefore, the subsequent equation that determines charge power allocation for each iteration of the scheduling algorithm is defined as follows:

\begin{equation}
\begin{split}
	p_{u,k}(n) :=
	\begin{cases}
		\frac{\hat{E}_u(n)}{\kappa}\beta &\text{if } \frac{\hat{E}_u(n)}{\kappa}\beta \leq P_{max,u}\\
		P_{max,i} &\text{otherwise}
 	\end{cases}
 	\forall u  \\\text{ where } k = \text{argmin}_k(\textbf{p}_{base}(n)) \text{ and } n > 1
\end{split}
\label{ch3:equ:valley-filling-equation}
\end{equation}

An intermediate demanded energy, $\hat{E}_u(n)$, is defined, which captures the total energy that needs to be allocated in the charge profile.
In the first iteration, this energy is equal to the total, actual EV energy demand, $E_u$, and for all subsequent iterations, this energy is equal to the undone energy, i.e.:

\begin{equation}
	\hat{E}_u(n) :=
	\begin{cases}
		E_u &\text{if } n = 1\\
		\alpha E_u &\text{otherwise}
	\end{cases}
	\forall u
	\label{ch3:equ:intermediate-demand-energy}
\end{equation}

Since the energy that is charged by the resulting charging profile has to equate to the original charging demand, Equation \ref{ch3:equ:intermediate-demand-energy} can be expanded into a recursive formula:

\begin{equation}
	\hat{E}_u(n) :=
	\begin{cases}
		E_u &\text{if } n = 1\\
		\alpha \sum_{k=1}^K p_{u,k}(n-1)\kappa &\text{otherwise}
	\end{cases}
	\forall u
	\label{ch3:equ:intermediate-demand-energy-expanded}
\end{equation}

Instead of being based on the general energy demand, this form shows how the algorithm's power reduction mechanism acts on the previous iteration's power profile.
Therefore, the undone power portion can be defined as:

\begin{equation}
	\hat{p}_{u,k}(n) = \alpha p_{u,k}(n-1) \text{ where } n > 1
	\label{ch3:equ:undone-power}
\end{equation}

To summarise: the message exchange is explained in Figure \ref{ch3:fig:agent-synchronisation}; the power allocation captured to achieve valley filling is explained in Equation \ref{ch3:equ:valley-filling-equation}; and the recursive intermediate demand is defined in Equation \ref{ch3:equ:intermediate-demand-energy-expanded}.
When combined, the complete smart valley filling algorithm can be defined as shown in Algorithm \ref{ch3:alg:valley-filling}.

\begin{algorithm}
 \SetKwFunction{Query}{query}
 \SetKwFunction{Update}{update}
 \SetKwFunction{Break}{break}
 \SetKwFunction{Sync}{synchronising}
 \KwData{$\textbf{p}_{\text{base},n}$, $E_u$, $P_{\text{max},u}$, $P_{\text{min},u}$, $\Delta t$, $T_\text{sch}$}
 \KwResult{$\textbf{p}_{\text{EV},u,n}$}
 \For{$n\leftarrow 1$ \KwTo $N$}{
  \tcp{Query for base load}\label{cmt}
  $\textbf{p}_{\text{base},n} \leftarrow$ \Query{}\;
  \tcp{Forward and undo previous schedule}\label{cmt}
  \eIf{$n>1$}{
    $\textbf{p}_{\text{EV},u,n} \leftarrow \textbf{p}_{\text{EV},u,n-1}\alpha$\;
   }{
    $\textbf{p}_{\text{EV},u,n} \leftarrow [0, 0, \dots, 0]$\;
  }
  \tcp{Determine unallocated energy}\label{cmt}
  $\hat{E}_{u,n} = E_u - \sum_{\tau=1}^{T_\text{sch}} p_{\text{EV},u,n}(\tau)\Delta t$\;
  \tcp{Fill valley}\label{cmt}
  \For{$\tau\leftarrow \text{argmin}(\textbf{p}_{\text{base},n})$ \KwTo $\text{argmax}(\textbf{p}_{\text{base},n})$}{
   \eIf{$p_{\text{EV},u,n}(\tau) + \frac{\hat{E}_{u,n}}{\Delta t}\beta \leq P_{\text{max},u}$}{
    $p_{\text{EV},u,n}(\tau) \leftarrow p_{\text{EV},u,n}(\tau) + \frac{\hat{E}_{u,n}}{\Delta t}\beta$\;
   }{
    $p_{\text{EV},u,n}(\tau) \leftarrow P_{\text{max},u}$\;
   }
   $\hat{E}_{u,n} = E_u - \sum_{\tau=1}^{T_\text{sch}} p_{\text{EV},u,n}(\tau) \Delta t$\;
   \tcp{Once EV profile is found, send update}\label{cmt}
   \If{$\hat{E}_{u,n} = 0$}{
    \Update{$\textbf{p}_{\text{EV},u,n}$}\;
   	\Break{}\;
   }
  }
  \Sync{}\;
 }
 \caption{Robust valley filling algorithm for a single EV in}
 \label{ch3:alg:valley-filling}
\end{algorithm}













