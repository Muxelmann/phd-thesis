\section{Coordination of EV charging}
\label{ch3:sec:ev-coordination}

Here, an algorithm for EV charging is presented, which is later implemented in both a synchronised and desynchronised case.
Real load data is used in combination with EV demand to evaluate the performance of the algorithm at preventing new power spikes from occurring.
Convergence of the algorithm is studied and convergence criteria as well as rate of convergence are also presented.

This section is structured as follows.
First, the means and assumptions for calculating EV demand is defined.
Then the real load data is introduced and explained.
The EV scheduling algorithm is introduced next, before the performance parameters are presented.

\subsection{EV Demand}

EVs were modelled as loads that, over the course of a scheduling horizon, $H$, each need to consume a certain amount of energy, $E_u$, to simulate charging their batteries.
Each EV, i.e. $u$, is part of a fleet of charging and coordinated EVs, i.e. $u \in [1, \dots, U]$.
Unlike typical loads (e.g. households), EVs do not have a predetermined load profile in this simulation and are therefore flexible to schedule their demand, $p_u(k)$, where $p_u(t) \in \textbf{p}(k)$.
In other words, they need to autonomously assign each charging plan over a certain number of future time-slots.
Due to limitations in on-board power electronics, each EV's maximum charge rate, $P_{u,max}$, is restricted and may not be exceeded.
Equally, in order to meet the EV's charging demand over the scheduling horizon, $H$, a soft minimum charging power, $P_{u,min}$ is also introduced:

\begin{equation}
	P_{\text{min},u} := \frac{E_u}{H}
	\label{ch3:equ:power-charge-minimum}
\end{equation}

Although the upper limit is hard, i.e. caused by technical restrictions, this lower limit is a necessity to initiate the scheduling procedure, which will become apparent when the scheduling algorithm is explained.
Using MAS, EVs utilise their broker agents to purchase energy quantities for each time-slot, $k$, and also sell or ``undo'' some of the already acquired energy quantities if it contributed towards a new load spike.

\subsection{Base Load}

To replicate real power consumption in the simulations, historic customer load profiles were used in this work \cite{IrishData2002}.
This dataset consisted of 7392 demand readings for 543 loads, which were sampled at half-hourly period, i.e. $\kappa = 0.5$ hours.
A single scheduling horizon was defined as one day, therefore $H=48$ samples.

In this context, each household, i.e. physical entity, dispatches its broker agents to order the household's daily energy demand; therefore it a scenario with some foresight is assumed.
After having issued this energy request, the entire network demand is known to the supplier and can be relayed to all broker agents when they query for it.
This ability is exploited when scheduling and negotiating the unknown EV charging profiles.
More specifically, all households' broker agents communicate with the suppler's broker agents to optimally embed their charging profiles, $p_u(k)$, within this aggregated base load, $d(k)$.

\subsection{Scheduling Algorithm}

For this work's EV charging coordination strategy, an algorithm is designed to generate optimised charging profiles for each EV.
Optimality implies that aggregation of all charging profiles and adding the aggregate onto the network's base demand, $d(k)$, does not result in additional spikes in the resulting demand profile, $p(k)$.
This optimality is achieved by repetitively querying the network's base load and adjusting the individual charging profiles according to the obtained information.
The updated charging profile is then reapplied and the new base load is queried.
This loop continues until either convergence towards a global solution has been achieved or a certain number of iterations has been reached.

\hl{Rewrite the entire algorithm explanation... The paper is very unclear!}

In order to control the algorithm, two parameters are used, i.e. $\alpha$ and $\beta$.
Here, $\alpha \in (0, 1]$ and $\beta \in (0, 1]$, too.
These two parameters control the convergence rate of the algorithm, as well as its resultant steady-state ripple.
Since the algorithm is designed to update each EV's charging profile based on a preceding state (i.e. $n-1$), the algorithm's initialisation is explained before expanding on the updating procedure.

\subsubsection{Algorithm Initialisation ($n=1$)}

Having obtained the network's base demand profile (following the procedure mentioned above), the initial charging profile for each EV can be defined:

\begin{equation}
	p_u(k) = 
	\label{ch3:equ:initial-charging-profile}
\end{equation}












