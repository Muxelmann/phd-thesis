\section{Coordination of EV charging}
\label{ch3:sec:ev-coordination}

In this section, an algorithm for EV charging is presented, which is implemented in both a synchronised and desynchronised case.
The Irish load dataset \cite{IrishData2002} is used in combination with EV energy demand to test the proposed smart-charging algorithm.
Performance of the algorithm at preventing new power spikes from occurring is then assessed with the use of standard performance metrics.
Finally, to study the convergence of the algorithm, a convergence criteria as well as rate of convergence are presented, too.

\subsection{EV Demand}

\nomenclature[K]{$T_\text{sch}$}{Scheduling horizon for EV charging, where $T_\text{sch} \in \mathbb{Z}^{>0}$}
\nomenclature[K]{$u$}{EV unit number, where $u \in [1, \dots, U]$}
\nomenclature[K]{$n$}{Iteration number of EV scheduling algorithm, where $N \in [1, \dots, N]$}
\nomenclature[K]{$U$}{Number of EVs that need to be scheduled, where $U \in \mathbb{Z}^{>0}$}
\nomenclature[K]{$N$}{Number of algorithm iterations to schedule multiple EVs, where $N \in \mathbb{Z}^{>0}$}
\nomenclature[K]{$p_{\text{EV},u,n}(t)$}{Scheduled EV charging power, for EV $u$ at algorithm iteration $n$ for time $t$, where $p_{\text{EV},u,n}(t) \in \mathbb{R}^{\geq 0}$}
\nomenclature[K]{$\textbf{p}_{\text{EV}, u, n}$}{Scheduled EV charging power vector, for EV $u$ at algorithm iteration $n$, where $\textbf{p}_{\text{EV},u,n} = (p_{\text{EV},u,n}(t))$}

\nomenclature[K]{$P_{\text{min},u}$}{Minimum EV charging power }
\nomenclature[K]{$P_{\text{min},u}$}{Maximum EV charging power }

In order to simulate the charging of an EV's battery, each EV, $u$, was modelled as a load that over the course of a scheduling horizon, $T_\text{sch}$ where $T_\text{sch} \in \mathbb{Z}^{>0}$, needs to consume a certain amount of energy, $E_u$.
A predetermined number of EVs, $U$, makes up a coordinated fleet of charging EVs, i.e. $u \in \{1, \dots, U\}$.
Unlike typical load profiles (e.g. household load profiles), EVs are modelled to not have a predetermined load profile and they are therefore flexible to schedule their demand, $p_{\text{EV},u,n}(t)$, at any moment time $t$, where $\textbf{p}_{\text{EV},u,n} = (p_{\text{EV},u,n}(t))$.
In other words, EVs can autonomously assign their own charging plan over the predetermined number of future time-slots, $T_\text{sch}$.
Due to limitations in on-board power electronics, each EV's maximum charge rate, $P_{\text{max},u}$, is restricted and may not be exceeded.
Equally, in order to meet the EV's charging demand over the scheduling horizon, $T_\text{sch}$, a soft minimum charging power, $P_{\text{min},u}$ is also introduced:

\begin{equation}
	P_{min,u} := \frac{E_u}{H}
	\label{ch3:equ:power-charge-minimum}
\end{equation}

Although the upper limit represents a technical restrictions, this lower limit is a necessity to assure the demanded energy is charged over the entire charging period, and this limit is used to initiate the scheduling procedure itself.
This charging procedure is explained in Section~\ref{ch3:subsec:scheduling-algorithm}.
Using MAS, EVs utilise their agent systems to purchase energy quantities for each time-slot, $t$, and also sell or ``undo'' some of the already acquired energy quantities if it helps lowering a potential load spike.

\subsection{Base Load}

\nomenclature[K]{$\Delta t$}{Sample period for EV scheduling, where $\Delta t \in \mathbb{R}^{>0}$}

To represent real power consumption in simulations, historic customer load profiles were used in this work \cite{IrishData2002}.
This dataset consisted of 7392 demand readings for 543 loads, which were sampled at half-hourly period, i.e. $\Delta t = 0.5$ hours.
A single scheduling horizon was defined as one day, therefore $T_\text{sch}=48$ samples.

In this context, each household dispatches its agents once to acquire the household's half-hourly energy demand for the entire day; making the assumption that some demand foresight is available.
After having acquired or reserved the daily demand by issued an energy request, the entire network demand is known to the energy supplier and can be relayed to all EV agents when they query for it.
This ability is exploited when scheduling and negotiating the unknown EV charging profiles.
More specifically, all EV's agents communicate with the suppler's agent to optimally embed their charging profiles, $p_{\text{EV},u,n}(t)$, within this aggregated base load.

\subsection{Scheduling Algorithm}
\label{ch3:subsec:scheduling-algorithm}

For the EV charging coordination strategy, an algorithm was designed that generates optimised charging profiles for each EV.
Here, optimality implies that when adding all aggregated charging profiles to the network's base demand, $p_\text{base}(t)$, no additional power spikes occur in the resulting power profile, i.e. $p_\text{base}(t) + \sum_{u=1}^U p_{EV,u,N}(t)$.
The charging profiles are generated by repetitively querying for the network's base load, adjusting individual EV charging profiles, and resubmitting the adjusted charging profile.
As already stated, the common assumption when designing such a scheduling algorithm is that all scheduling entities are synchronised, i.e. wait for each other, before querying for the network's base load.
For visualisation, the message exchange between two loads and a supplier including a synchronisation time is shown in Figure~\ref{ch3:fig:agent-synchronisation}.

\begin{figure}\centering
\tikzstyle{box} = [%
	draw,%
	rectangle,%
	%fill=green!20,%
	minimum height=2em,%
	minimum width=2em,%
]
\tikzstyle{bracket} = [%
	decorate,%
	decoration={brace,amplitude=5pt,raise=1mm},%
	yshift=0mm,%
	color=black!50,%
]
\tikzstyle{bracket_text} = [%
	black,%
	midway,%
	xshift=2mm,%
	align=left,%
	color=black!50,%
]
\begin{tikzpicture}[node distance=2cm, shorten >= 1pt, >=stealth', auto, scale=1, transform shape]
	\pgfmathsetmacro\N{4}

	% Draw supplyer
    \draw (0, 0) node [box, fill=black!10](supplier) {Supplier};
    \draw (5, 0) node [box, fill=black!10](load1) {Load$_1$};
    \draw (10, 0) node [box, fill=black!10](load2) {Load$_2$};
    
    \draw (0, -11.5) node [](s_end) {};
    \draw (0, -10) node [](s_sync) {};
    
    \draw (5, -1) node [](l1_q1) {};
    \draw (5, -2) node [](l1_r1) {};
    \draw (5, -5) node [](l1_u1) {};
    \draw (5, -6) node [](l1_a1) {};
    \draw (5, -11) node[](l1_q2) {};
    
    \draw (10, -1.1) node [](l2_q1) {};
    \draw (10, -3.0) node [](l2_r1) {};
    \draw (10, -7.5) node [](l2_u1) {};
    \draw (10, -8.5) node [](l2_a1) {};
    \draw (10, -11.1) node[](l2_q2) {};
    
    
    \draw [dotted] (supplier.south) to (s_end.center);
    \draw [dotted] (load1.south) to (s_end.center-|l1_q1.center);
    \draw [dotted] (load2.south) to (s_end.center-|l2_q1.center);
    
    \draw [->] (l1_q1.center) to node[pos=0.5, above]{query$(l_1)$} (supplier.center|-l1_q1);
    \draw [->] (supplier.center|-l1_q1) to
    			(supplier.center|-l1_r1) to node[pos=0.5, above]{reply$(l_1)$} (l1_r1.center);
    \draw [bracket] (l1_q1) -- (l1_r1) node[bracket_text]{waiting};
    \draw [->] (l1_r1.center) to 
    			(l1_u1.center) to node[pos=0.5, above]{update$(l_1)$} (supplier.center|-l1_u1);
    \draw [bracket] (l1_r1) -- (l1_u1) node[bracket_text]{scheduling};
    \draw [->] (supplier.center|-l1_u1) to
    			(supplier.center|-l1_a1) to node[pos=0.5, above]{ack$(l_1)$} (l1_a1.center);
    \draw [bracket] (l1_u1) -- (l1_a1) node[bracket_text]{waiting};
    \draw [bracket] (l1_a1) -- (l1_a1|-s_sync.90) node[bracket_text]{syncing};
    
    \draw [->] (l2_q1.center) to node[pos=0.25, above]{query$(l_2)$} (supplier.center|-l2_q1);
    \draw [->] (supplier.center|-l2_q1) to
    			(supplier.center|-l2_r1) to node[pos=0.75, above]{reply$(l_2)$} (l2_r1.center);
    \draw [bracket] (l2_q1) -- (l2_r1) node[bracket_text]{waiting};
    \draw [->] (l2_r1.center) to
    			(l2_u1.center) to node[pos=0.25, above]{update$(l_2)$} (supplier.center|-l2_u1);
    \draw [bracket] (l2_r1) -- (l2_u1) node[bracket_text]{scheduling};
    \draw [->] (supplier.center|-l2_u1) to
    			(supplier.center|-l2_a1) to node[pos=0.75, above]{ack$(l_2)$} (l2_a1.center);
    \draw [bracket] (l2_u1) -- (l2_a1) node[bracket_text]{waiting};
    \draw [bracket] (l2_a1) -- (l2_a1|-s_sync.90) node[bracket_text]{syncing};
    
    \draw [thick] (supplier.180|-s_sync) to node[pos=1, right]{SYNC} (load2.0|-s_sync);
    
    \draw [->] (l1_q2.center) to node[pos=0.5, above]{query$(l_1)$} (supplier.center|-l1_q2);
    \draw [->] (l2_q2.center) to node[pos=0.25, above]{query$(l_2)$} (supplier.center|-l2_q2);
    
    \draw [-] (supplier.center|-l1_q2) to (s_end.center);
\end{tikzpicture}	
\caption{Agent synchronisation before re-scheduling their EVs charging profile.}
\label{ch3:fig:agent-synchronisation}
\end{figure}

In this figure, the horizontal arrows indicate messages being sent from loads (i.e. EV agents) to a supplier and vertical lines indicate processing or idle time.
Shown within Figure~\ref{ch3:fig:agent-synchronisation} is a single scheduling iteration, which can be broken into the sub-processes of: querying, scheduling, updating and synchronising.
From top to bottom, the sequential execution of these sub-processes is as follows:

First, both load$_1$ and load$_2$ query the supplier for the currently known network load (i.e. query($l_1$) and query($l_2$)).
This network load is used to schedule their power profiles to ``fill valleys'', i.e. only charge EVs during periods of low demand.
Upon receipt of a reply from the energy supplier (i.e. reply($l_1$) and reply($l_2$)), both loads immediately start scheduling their profiles.
In the example above, load$_1$ found a solution before load$_2$ and can therefore inform the supplier about its intended load profile sooner, by sending an update (i.e. update$(l_1)$) to the supplier.
Subsequently querying the supplier for an updated network load would be premature, since the other load (i.e. load$_2$) has not yet generated and updated its load profile.
Therefore, a synchronisation mechanism had to be used, forcing load$_1$ to wait until all loads have sent updates to the supplier.
Here, load$_1$ waits until load$_2$ has sent an update and the corresponding profile was acknowledged by the supplier (i.e. ack$(l_2)$).
Only after this had happened, a synchronisation event would be triggered (i.e. \textit{SYNC} event).
After this synchronisation event, the next algorithm iteration is initiated and the procedure repeats.
Since all subsequent iterations are similar to the one shown in Figure~\ref{ch3:fig:agent-synchronisation}, only the two querying messages of the  second iteration are shown.

Although timing and message exchange has been defined, the mechanism to allocate and reallocate charging powers in order to achieve a valley filling behaviour has not yet been defined.
This behaviour is shown in Figure~\ref{ch3:fig:valley-filling-procedure}, where several iterations numbered $n$ are shown, and for each subsequent iteration, some amount of prescheduled power is reallocated to different time-slots.

\begin{figure}\centering
\subfloat[]{
\begin{tikzpicture}[node distance=2cm, shorten >= 1pt, >=stealth', auto, scale=1, transform shape]
	\draw [<-] (0, 3) to node[pos=0.5,rotate=90,anchor=south]{power} (0, 0);
	\draw [->] (0, 0) to node[pos=0.5,anchor=north]{time} (3, 0);
	\draw (2.25, 0.3) node [fill=green!10,minimum width=0.5cm,minimum height=2.2cm,anchor=south](rect_add) {};
	\draw [very thick]
	(0.0, 2.0) -- (0.5, 2.0) --
	(0.5, 1.3) -- (1.0, 1.3) --
	(1.0, 1.1) -- (1.5, 1.1) --
	(1.5, 0.6) -- (2.0, 0.6) --
	(2.0, 0.3) -- (2.5, 0.3) --
	(2.5, 1.4) -- (3.0, 1.4);
	\draw [thick, green]
	(0.0, 2.0) -- (0.5, 2.0) --
	(0.5, 1.3) -- (1.0, 1.3) --
	(1.0, 1.1) -- (1.5, 1.1) --
	(1.5, 0.6) -- (2.0, 0.6) --
	(2.0, 2.5) -- (2.5, 2.5) --
	(2.5, 1.4) -- (3.0, 1.4);
\end{tikzpicture}
\label{ch3:subfig:valley-filling-1}
}
\hspace{10mm}
\subfloat[]{
\begin{tikzpicture}[node distance=2cm, shorten >= 1pt, >=stealth', auto, scale=1, transform shape]
	\draw [<-] (0, 3) to node[pos=0.5,rotate=90,anchor=south]{power} (0, 0);
	\draw [->] (0, 0) to node[pos=0.5,anchor=north]{time} (3, 0);
	\draw (2.25, 2.0) node [fill=red!10,minimum width=0.5cm,minimum height=0.5cm,anchor=south](rect_sub) {};
	\draw (1.75, 0.6) node [fill=green!10,minimum width=0.5cm,minimum height=0.5cm,anchor=south](rect_add) {};
	\draw [very thick]
	(0.0, 2.0) -- (0.5, 2.0) --
	(0.5, 1.3) -- (1.0, 1.3) --
	(1.0, 1.1) -- (1.5, 1.1) --
	(1.5, 0.6) -- (2.0, 0.6) --
	(2.0, 0.3) -- (2.5, 0.3) --
	(2.5, 1.4) -- (3.0, 1.4);
	\draw [thick, green]
	(0.0, 2.0) -- (0.5, 2.0) --
	(0.5, 1.3) -- (1.0, 1.3) --
	(1.0, 1.1) -- (1.5, 1.1) --
	(1.5, 1.1) -- (2.0, 1.1) --
	(2.0, 2.0) -- (2.5, 2.0) --
	(2.5, 1.4) -- (3.0, 1.4);
	\draw [->, bend right] (rect_sub.180) to (rect_add.90);
\end{tikzpicture}
\label{ch3:subfig:valley-filling-2}
}
\hspace{10mm}
\subfloat[]{
\begin{tikzpicture}[node distance=2cm, shorten >= 1pt, >=stealth', auto, scale=1, transform shape]
	\draw [<-] (0, 3) to node[pos=0.5,rotate=90,anchor=south]{power} (0, 0);
	\draw [->] (0, 0) to node[pos=0.5,anchor=north]{time} (3, 0);
	\draw (2.25, 1.6) node [fill=red!10,minimum width=0.5cm,minimum height=0.4cm,anchor=south](rect_sub_1) {};
	\draw (1.75, 1.0) node [fill=green!10,minimum width=0.5cm,minimum height=0.4cm,anchor=south](rect_add_1) {};
	\draw (1.25, 1.1) node [fill=green!10,minimum width=0.5cm,minimum height=0.3cm,anchor=south](rect_add_2) {};
	\draw [fill=red!10,red!10] (1.5, 1.0) rectangle (2.0,1.1);
	\draw [very thick]
	(0.0, 2.0) -- (0.5, 2.0) --
	(0.5, 1.3) -- (1.0, 1.3) --
	(1.0, 1.1) -- (1.5, 1.1) --
	(1.5, 0.6) -- (2.0, 0.6) --
	(2.0, 0.3) -- (2.5, 0.3) --
	(2.5, 1.4) -- (3.0, 1.4);
	\draw [thick, green]
	(0.0, 2.0) -- (0.5, 2.0) --
	(0.5, 1.3) -- (1.0, 1.3) --
	(1.0, 1.4) -- (1.5, 1.4) --
	(1.5, 1.4) -- (2.0, 1.4) --
	(2.0, 1.6) -- (2.5, 1.6) --
	(2.5, 1.4) -- (3.0, 1.4);
	\draw [->, bend right] (rect_sub_1.180) to (rect_add_2.45);
\end{tikzpicture}
\label{ch3:subfig:valley-filling-3}
}
\vspace{1mm}
\subfloat[]{
\begin{tikzpicture}[node distance=2cm, shorten >= 1pt, >=stealth', auto, scale=1, transform shape]
	\draw [<-] (0, 3) to node[pos=0.5,rotate=90,anchor=south]{power} (0, 0);
	\draw [->] (0, 0) to node[pos=0.5,anchor=north]{time} (3, 0);
	\draw [very thick]
	(0.0, 2.0) -- (0.5, 2.0) --
	(0.5, 1.3) -- (1.0, 1.3) --
	(1.0, 1.1) -- (1.5, 1.1) --
	(1.5, 0.6) -- (2.0, 0.6) --
	(2.0, 0.3) -- (2.5, 0.3) --
	(2.5, 1.4) -- (3.0, 1.4);
	\draw [thick, green]
	(0.0, 2.0) -- (0.5, 2.0) --
	(0.5, 1.42) -- (1.0, 1.42) --
	(1.0, 1.42) -- (1.5, 1.42) --
	(1.5, 1.42) -- (2.0, 1.42) --
	(2.0, 1.42) -- (2.5, 1.42) --
	(2.5, 1.42) -- (3.0, 1.42);
	\draw (1.5, 3.3) node[anchor=south] {$\vdots$};
\end{tikzpicture}
\label{ch3:subfig:valley-filling-4}
}
\caption{Charging power (green line) allocation on top of base network load (black line) for valley-filling behaviour. Here $n=1$ for Fig. \ref{ch3:subfig:valley-filling-1}, $n=2$ for Fig. \ref{ch3:subfig:valley-filling-2}, $n=3$ for Fig. \ref{ch3:subfig:valley-filling-3}, and $n=N$ for Fig. \ref{ch3:subfig:valley-filling-4}.}
\label{ch3:fig:valley-filling-procedure}	
\end{figure}

\nomenclature[K]{$p_{\text{base},n}(t)$}{Base load at time $t$, where $p_{\text{base},n}(t) \in \mathbb{Z}^{\geq0}$}
\nomenclature[K]{$\textbf{p}_{\text{base},n}$}{Base load vector, where $p_{\text{base},n}(t) \in \textbf{p}_{\text{base},n}$}
\nomenclature[K]{$\hat{p}_{\text{base},n}(t)$}{Temporary demand at time $t$, i.e. the aggregate of all EV charge vector and the base load vector, where $\hat{p}_{\text{base},n}(t) \in \mathbb{Z}^{\geq0}$}
\nomenclature[K]{$\hat{\textbf{p}}_{\text{base},n}$}{Temporary demand vector, i.e. the aggregate of all EV charge vector and the base load vector, where $\hat{p}_{\text{base},n}(t) \in \hat{\textbf{p}}_{\text{base},n}$}

For every iteration in Figure~\ref{ch3:fig:valley-filling-procedure}, charging profiles are added onto a base network load, $\textbf{p}_{base,n}$, where $p_{base,n}(t) \in \textbf{p}_{base,n}$.
This base load is shown as the bold black line and does not change throughout EV scheduling.
For any iteration, the charging profile during iteration number $n$, for EV $u$, is defined as $\textbf{p}_n$ (where $p_{u,n}(t) \in \textbf{p}_n$).
During the first iteration however, i.e. Figure~\ref{ch3:subfig:valley-filling-1} where $n=1$, this charging profile is determined by assigning the maximum EV charging power to the time-slots of lowest load, until the total EV energy demand is met, i.e. at time-slot $\tau$ where $\tau = \text{argmin}(\textbf{p}_\text{base})$.
Since all EVs schedule their profiles based upon the same knowledge of $\textbf{p}_{base,n}$, the aggregated charging power is likely to generate a new spike.
This spike is seen on an updated or temporary demand profile, $\hat{\textbf{p}}_{\text{base}, n}$, where $\hat{p}_{\text{base}, n}(t) \in \hat{\textbf{p}}_{\text{base}, n}$ is defined as:

\begin{equation}
	\hat{p}^\text{base}_{n}(t) := p^\text{base}_{n}(t) + \sum_{u=1}^{U} p_{u,n}(t) \forall t, n
	\label{ch3:equ:updated-demand-profile}
\end{equation}

\nomenclature[K]{$\hat{p}_{\text{EV}, u, n}(t)$}{Temporary charging power during iteration number $n$, for EV $u$, at time $t$, where $\hat{p}_{\text{EV}, u, n}(t) \in \textbf{Z}^{\geq0}$}
\nomenclature[K]{$\hat{\textbf{p}}_{\text{EV}, u, n}$}{Temporary charging vector during iteration number $n$, for EV $u$, where $\hat{p}_{\text{EV}, u, n}(t) \in \hat{\textbf{p}}_{\text{EV}, u, n}$}

For the next iteration $n+1$, i.e. Figure~\ref{ch3:subfig:valley-filling-2} where $n=2$, a proportion of the previously scheduled power vector $\textbf{p}_{\text{EV},n-1}$ is ``undone''.
Subsequently, the spike in the resulting $\hat{\textbf{p}}_{\text{base}, n}$ is reduced, yet the energy that has been undone needs to be reallocated.
The amount by which $\textbf{p}_{\text{EV},n-1}$ is reduced is determined by the ``\textit{undoing}'' parameter $\alpha$, where $\alpha \in [0, 1)$.
A new reduced or temporary charging vector $\hat{p}_{\text{EV}, u, n}(t)$ is therefore defined as:

\begin{equation}
	\hat{p}^\text{EV}_{u,n}(t) := p^\text{EV}_{u,n-1}(t)(1-\alpha)
	\label{ch3:equ:temporary-charging-power}
\end{equation}

Using this temporary charging power, the regained or temporary energy demand, $\hat{E}_{u,n}$, that needs to be reallocated, can also be defined:

\begin{equation}
	\hat{E}_{u,n} := E_u - \sum_{\tau=1}^{T_\text{sch}} \hat{p}_{\text{EV},u,n}(\tau)\Delta t \forall u \text{ and } n > 1
	\label{ch3:equ:temporary-charging-demand}
\end{equation}

To include the first iteration of the algorithm, Equation~\ref{ch3:equ:temporary-charging-demand} needs to be expanded to redefine $\hat{E}_{u,n}$ for all possible algorithm iterations $n$:

\begin{equation}
	\hat{E}_{u,n} :=
	\begin{cases}
		E_u &\text{if } n=1\\
		E_u - \sum_{\tau=1}^{T_\text{sch}} \hat{p}_{\text{EV},u,n}(\tau)\Delta t &\text{otherwise}
	\end{cases}
	\forall u \text{ and } \forall n
	\label{ch3:equ:temporary-charging-demand-expanded}
\end{equation}

Following the similar procedure as for the first iteration, $\hat{E}_{u,n}$ needs to be allocated to different time-slots, where the rule of performing the power allocation is defined as:

\begin{equation}
\begin{split}
	p_{u,k}(n) :=
	\begin{cases}
		\frac{\hat{E}_u(n)}{\kappa}\beta &\text{if } \frac{\hat{E}_u(n)}{\kappa}\beta \leq P_{max,u}\\
		P_{max,i} &\text{otherwise}
 	\end{cases}
 	\forall u  \\\text{ where } k = \text{argmin}_k(\textbf{p}_{base}(n)) \text{ and } n > 1
\end{split}
\label{ch3:equ:valley-filling-equation}
\end{equation}

\nomenclature[K]{$\beta$}{Allocation parameter to assign a portion of the temporary energy demand, $\hat{E}_{u,n}$, where $\beta \in (0, 1]$}

Here, a maximum ``allocation'' parameter, $\beta$, where $\beta \in (0, 1]$ limits the power that may be allocated to any successive time-slot, $\tau$.
To not exceed the EV's maximum charging power, any value in the charging vector, $\textbf{p}_{\text{EV},u,n}$, is capped to $P_{\text{max},u}$.
If $\beta$ is chosen as one, then the undone energy is allocated as quickly as possible.
For smaller values of $\beta$ on the other hand, the undone charge is reallocated in smaller portions.
Since EV scheduling takes place over a finite scheduling horizon, $T_\text{sch}$, a constraint was added in Equation~\ref{ch3:equ:valley-filling-equation}.
This was done to assure that the temporary energy demand equates after some charging power was assigned to every time-slot of $\textbf{p}_{\text{EV},u,n}$.

In any following algorithm iteration, i.e. $n>2$ as shown in Figure~\ref{ch3:subfig:valley-filling-3}, each EV's charging profile is adjusted and spread further over the base load, $\textbf{p}_\text{base}$.
In the end, i.e. when $n=N$, the ideal EV charging profiles add to the base load in such a way, that the resulting network load has an optimally filled valley.
This valley filling behaviour is achieved with the ``undoing'' and ``allocation'' of EV charging power from one algorithm iteration to the next.
Regardless of the final network load's shape, the algorithm terminates when the final iteration is reached, i.e. $n=N$.
Rate of convergence of the algorithm differs based upon the choice of $\alpha$ and $\beta$ values.
However, convergence is in fact guaranteed when selecting values of $\alpha < 1$ and $\beta < 1$, since the algorithm satisfies the D'Alembert Criterion in those cases.

To summarise this section, the complete EV scheduling algorithm was developed by: 
\begin{enumerate*}
	\item defining the message exchange and synchronisation mechanism, which is shown in Figure~\ref{ch3:fig:agent-synchronisation};
	\item formulating the initial and successive ``undoing'' of charging power, as shown in Equation~\ref{ch3:equ:temporary-charging-power}; and
	\item defining the iterative update and ``allocation'' of the temporary energy demand, as defined in Equation~\ref{ch3:equ:temporary-charging-demand-expanded}.
\end{enumerate*}
For clarity, the this smart charging algorithm's pseudocode, performing the complete valley filling procedure, has been included in Algorithm \ref{ch3:alg:valley-filling}.

\begin{algorithm}
 \SetKwFunction{Query}{query}
 \SetKwFunction{Update}{update}
 \SetKwFunction{Break}{break}
 \SetKwFunction{Sync}{synchronising}
 \KwData{$\textbf{p}_{\text{base},n}$, $E_u$, $P_{\text{max},u}$, $P_{\text{min},u}$, $\Delta t$, $T_\text{sch}$}
 \KwResult{$\textbf{p}_{\text{EV},u,n}$}
 \For{$n\leftarrow 1$ \KwTo $N$}{
  \tcp{Query for base load}\label{cmt}
  $\textbf{p}_{\text{base},n} \leftarrow$ \Query{}\;
  \tcp{Forward and undo previous schedule}\label{cmt}
  \eIf{$n>1$}{
    $\textbf{p}_{\text{EV},u,n} \leftarrow \textbf{p}_{\text{EV},u,n-1}\alpha$\;
   }{
    $\textbf{p}_{\text{EV},u,n} \leftarrow [0, 0, \dots, 0]$\;
  }
  \tcp{Determine unallocated energy}\label{cmt}
  $\hat{E}_{u,n} = E_u - \sum_{\tau=1}^{T_\text{sch}} p_{\text{EV},u,n}(\tau)\Delta t$\;
  \tcp{Fill valley}\label{cmt}
  \For{$\tau\leftarrow \text{argmin}(\textbf{p}_{\text{base},n})$ \KwTo $\text{argmax}(\textbf{p}_{\text{base},n})$}{
   \eIf{$p_{\text{EV},u,n}(\tau) + \frac{\hat{E}_{u,n}}{\Delta t}\beta \leq P_{\text{max},u}$}{
    $p_{\text{EV},u,n}(\tau) \leftarrow p_{\text{EV},u,n}(\tau) + \frac{\hat{E}_{u,n}}{\Delta t}\beta$\;
   }{
    $p_{\text{EV},u,n}(\tau) \leftarrow P_{\text{max},u}$\;
   }
   $\hat{E}_{u,n} = E_u - \sum_{\tau=1}^{T_\text{sch}} p_{\text{EV},u,n}(\tau) \Delta t$\;
   \tcp{Once EV profile is found, send update}\label{cmt}
   \If{$\hat{E}_{u,n} = 0$}{
    \Update{$\textbf{p}_{\text{EV},u,n}$}\;
   	\Break{}\;
   }
  }
  \Sync{}\;
 }
 \caption{Robust valley filling algorithm for a single EV in}
 \label{ch3:alg:valley-filling}
\end{algorithm}













