\section{Distributed Systems}
\label{ch3:sec:distributed-systems}

As discussed in the literature review in Chapter \ref{ch-literature}, several mechanisms exist to decentralise control of DERs.
Due to their reactivity, pro-activeness and social ability, the Multi Agent System (MAS) distinguished itself from traditional software and hardware systems, which is also why it was also chosen for the coordination of smart EV charging.
Several agent package implementations exist, each following different interaction paradigms.
These paradigms include ``Belief, Desire and Intention'' (BDI), neutral behaviour or other specialised functionality \cite{Luck2004}.
From this catalogue of MAS paradigms, the Java Agent Development Framework (JADE) was chosen, since it natively implements the Foundation for Intelligent Physical Agent (FIPA) specification \cite{JADE-website, FIPA-agent-specs}.
Furthermore, JADE is an application independent package that has become quite popular, as seen by the increasing number of publications \cite{Karfopoulos2013, Eddy2011, Kuo2013, Mocci2014, Li2017}.

In this work, multiple virtual trading agents are used to negotiate their corresponding EV's charging profile with other trading agents.
Tying virtual agents to a physical entity is not new \cite{Dimeas2005, Nguyen2011, Nagata2011, Nagata2012}, yet in previous work these agents directly controlled the entity whilst the agents in the presented work negotiate schedules that will be applied after schedule ratification.
So called ``broker'' agents, following the Brokering Interaction Protocol (BIP), are implemented to facilitate this trading process.
The brokers communicate with each other by following the Contact-Net Protocol (CNP).
In addidion, all FIPA protocols are based on the FIPA Agent Communication Language (ACL) that is required to communicate over a shared telecommunications infrastructure, and possibly include different agent packages, too.
High level explanations of all protocols that were used in this implementation of FIPA agents are included in Appendix \ref{appx-b:multi-agent-systems}.
In this work, each broker is linked to a single EV and negotiates its charging profile over the aforementioned scheduling horizon, $T_\text{sch}$.
This link is shown in Figure \ref{ch3:fig:agent-network}.

\begin{figure}\centering
\tikzstyle{box} = [%
	draw,%
	rectangle,%
	%fill=green!20,%
	minimum height=2em,%
	minimum width=2em,%
]
\begin{tikzpicture}[node distance=2cm, shorten >= 1pt, >=stealth', auto, scale=1, transform shape]
	\pgfmathsetmacro\N{4}

	% Draw supplyer
    \draw
    (0, 0)
    node [box, fill=black!10](supplier) {Supplier}
    node [box, below left of=supplier](buyer_s) {Buyer}
    node [box, below right of=supplier](seller_s) {Seller};
    
    \draw (5, 0) node(bracket) {};
    
    \draw
    node [box, below of=buyer_s, yshift=-10mm](buyer_1) {Buyer}
    node [box, below left of=buyer_1, fill=black!10](load_1) {Load}
    node [box, above left of=load_1](seller_1) {Seller}
    node [box, below of=seller_s, yshift=-10mm](seller_2) {Seller}
    node [box, below right of=seller_2, fill=black!10](load_2) {Load}
    node [box, above right of=load_2](buyer_2) {Buyer};
    
    \draw [-] (supplier) to (buyer_s);
    \draw [-] (supplier) to (seller_s);
    \draw [-] (load_1) to (buyer_1);
    \draw [-] (load_1) to (seller_1);
    \draw [-] (load_2) to (buyer_2);
    \draw [-] (load_2) to (seller_2);
    
    \draw [->] (seller_s) to node[pos=0.7, align=center, left, xshift=-5mm]{buys\\from} (buyer_1);
    \draw [->] (seller_s) to node[pos=0.7, align=center, right, xshift=5mm]{buys\\from} (buyer_2);
    \draw [->] (seller_1) to node[pos=0.3, align=center, left, xshift=-5mm]{sells\\to} (buyer_s);
    \draw [->] (seller_2) to node[pos=0.3, align=center, right, xshift=5mm]{sells\\to} (buyer_s);
    
	
    \draw [decorate,decoration={brace,amplitude=5pt,raise=0pt},yshift=0pt](supplier.90-|bracket.0) -- (supplier.270-|bracket.0) node [black,midway,xshift=2mm,align=left] {physical entity};
    \draw [decorate,decoration={brace,amplitude=5pt,raise=0pt},yshift=0pt](seller_s.90-|bracket.0) -- (seller_s.270-|bracket.0) node [black,midway,xshift=2mm,align=left] {CNP responding\\broker agents};
    \draw [decorate,decoration={brace,amplitude=5pt,raise=0pt},yshift=0pt](seller_2.90-|bracket.0) -- (seller_2.270-|bracket.0) node [black,midway,xshift=2mm,align=left] {CNP initiating\\broker agents};
    \draw [decorate,decoration={brace,amplitude=5pt,raise=0pt},yshift=0pt](load_2.90-|bracket.0) -- (load_2.270-|bracket.0) node [black,midway,xshift=2mm,align=left] {physical entities};
    
\end{tikzpicture}
\caption{A simplified MAS structure containing virtual seller and buyer agents (white), that negotiate power/charging profiles for physical entities (grey).}
\label{ch3:fig:agent-network}
\end{figure}

In the example in Figure \ref{ch3:fig:agent-network}, a single supplier with two brokers (one for selling and one for buying, i.e. ``undoing'', energy) acts as the energy source, and two loads represent the charging EVs that dispatch their two brokers (i.e. to ``allocate'' and ``undo'' charging demand).
With this kind of system architecture, charging spikes are designed to be mitigated, since all loads and their corresponding brokers execute the previously defined smart charging algorithm.

Since the architecture of MAS has now been defined, the means of implementing and desynchronising them is explained.
Subsequently, all case studies and metrics that are used to assess the MAS performance are defined.

\subsection{MAS Implementation}
\label{ch3:subsec:implementation}

The MAS was implemented in Java and ran on a parallel compute cluster (i.e. the \textit{HTCondor} cluster at the former ``\textit{School of Systems Engineering}'').
In Figure \ref{ch3:fig:agent-implementation}, the complete structure and communications procedure is explained.

\begin{figure}\centering
\tikzstyle{box} = [%
	draw,%
	rectangle,%
	%fill=green!20,%
	minimum height=2em,%
	minimum width=2em,%
]
\begin{tikzpicture}[node distance=2cm, shorten >= 1pt, >=stealth', auto, scale=1, transform shape]
	\draw (0, 0) node [box](biffo) {lab PC};
	\draw (5, 0) node [box, align=center](master) {\textit{HTCondor}\\submit node};
	\draw (10, 0) node [box](data) {data store};
	
	\draw (10, -5) node [box](supplier) {supplier};
	\draw node[above right of=supplier, yshift=-7.5mm] {\textit{worker$_5$}};
	
	\draw [->, line width=0.2mm, dotted] (biffo) to node[]{launch} (master);
	\draw [->, line width=0.2mm] (data) to node[pos=0.25, align=left, align=right, left]{network\\load} (supplier);
	
	\foreach[count=\i] \x in {0,3,...,10}
	{
		\draw (\x, -6.5) node [box](ev_\i) {EV};
		\draw node[below left of=ev_\i, yshift=7.5mm, xshift=5mm] {\textit{worker$_\i$}};
		\draw [<->, line width=0.4mm, bend right] (supplier) to (ev_\i);
		\draw [->, line width=0.2mm, dotted] (master) to (ev_\i);
	}
	
%	\draw [line, dashed] node (0, -2) -- node (10, -2);
	

%	\foreach[count=\i, evaluate=\i as \l using int(\i-1)] \t in {1,4,...,10}
%	{
%		\draw (\t,0.1) -- (\t,-0.1);
%		\ifthenelse{\l=0}
%		{\draw (\t,-0.1) node[anchor=north] {$t$}}
%		{
%			\ifthenelse{\l=1}
%			{\draw (\t,-0.1) node[anchor=north] {$t+\Delta t$}}
%			{\draw (\t,-0.1) node[anchor=north] {$t+\l\Delta t$}};
%		};
%	}

%	\pgfmathsetmacro\N{4}
%
%	% Draw supplyer
%    \draw
%    (0, 0)
%    node [box, fill=black!10](supplier) {Supplier}
%    node [box, below left of=supplier](buyer_s) {Buyer}
%    node [box, below right of=supplier](seller_s) {Seller};
%    
%    \draw (5, 0) node(bracket) {};
%    
%    \draw
%    node [box, below of=buyer_s, yshift=-10mm](buyer_1) {Buyer}
%    node [box, below left of=buyer_1, fill=black!10](load_1) {Load}
%    node [box, above left of=load_1](seller_1) {Seller}
%    node [box, below of=seller_s, yshift=-10mm](seller_2) {Seller}
%    node [box, below right of=seller_2, fill=black!10](load_2) {Load}
%    node [box, above right of=load_2](buyer_2) {Buyer};
%    
%    \draw [-] (supplier) to (buyer_s);
%    \draw [-] (supplier) to (seller_s);
%    \draw [-] (load_1) to (buyer_1);
%    \draw [-] (load_1) to (seller_1);
%    \draw [-] (load_2) to (buyer_2);
%    \draw [-] (load_2) to (seller_2);
%    
%    \draw [->] (seller_s) to node[pos=0.7, align=center, left, xshift=-5mm]{buys\\from} (buyer_1);
%    \draw [->] (seller_s) to node[pos=0.7, align=center, right, xshift=5mm]{buys\\from} (buyer_2);
%    \draw [->] (seller_1) to node[pos=0.3, align=center, left, xshift=-5mm]{sells\\to} (buyer_s);
%    \draw [->] (seller_2) to node[pos=0.3, align=center, right, xshift=5mm]{sells\\to} (buyer_s);
%    
%	
%    \draw [decorate,decoration={brace,amplitude=5pt,raise=0pt},yshift=0pt](supplier.90-|bracket.0) -- (supplier.270-|bracket.0) node [black,midway,xshift=2mm,align=left] {physical entity};
%    \draw [decorate,decoration={brace,amplitude=5pt,raise=0pt},yshift=0pt](seller_s.90-|bracket.0) -- (seller_s.270-|bracket.0) node [black,midway,xshift=2mm,align=left] {CNP responding\\broker agents};
%    \draw [decorate,decoration={brace,amplitude=5pt,raise=0pt},yshift=0pt](seller_2.90-|bracket.0) -- (seller_2.270-|bracket.0) node [black,midway,xshift=2mm,align=left] {CNP initiating\\broker agents};
%    \draw [decorate,decoration={brace,amplitude=5pt,raise=0pt},yshift=0pt](load_2.90-|bracket.0) -- (load_2.270-|bracket.0) node [black,midway,xshift=2mm,align=left] {physical entities};
    
\end{tikzpicture}
\caption{The implementation of FIPA on \textit{HTCondor}.}
\label{ch3:fig:agent-implementation}
\end{figure}

Here, the laboratory workstation uploads the simulation task to a submit node.
This node starts a certain number of workers, which in turn run the Java agent software to communicate with each other and to schedule the charging of their simulated EVs.
One of the workers is chosen to be the supplier, that obtains the network load data from a remote data store and propagates this information in a synchronised manner.
In the diagram in Figure \ref{ch3:fig:agent-implementation}, all bold arrows indicate the inter-agent communication, the solid arrow indicates the loading of simulation data, and the dotted arrows indicate the software launches.

\subsection{MAS Desynchronisation}
\label{ch3:subsec:desynchronisation}

Operating the agents in a desynchronised manner can be achieved in two distinct ways.
Either, any idle state that is interrupted by the synchronising signal (which is shown in Figure \ref{ch3:fig:agent-synchronisation}) is removed, or agents are launched with a random jitter and operate in an execution loop of a fixed time.
Although the first approach would result in a quick simulation execution, only the second way of desynchronising agents is compliant with the compute cluster's terms of usage.
Therefore, the latter way was implemented uses and each agent was equipped with its individual loop timer to regulate its execution.

The desynchronised agents are launched in succession and immediately begun their scheduling task.
To simulate communication irregularities, a jitter was added to the delay between successive agent launches.
The resulting MAS then consisted of a completely desynchronised collection of agents, i.e. none of the agent's loop execution is aligned or dependent on any other agent's loop execution.
This successful desynchronisation can be observed in Figure \ref{ch3:fig:agent-desynchronisation}, where the supplier agent never receives more than one message at once.

\begin{figure}\centering
\begin{tikzpicture}[node distance=2cm, shorten >= 1pt, >=stealth', auto, scale=1, transform shape]

	\draw (0, 1) node [draw, rectangle, fill=black!10](supplier) {S};
	\draw [thick] (-0.5, 0.25) to node[pos=1, right]{START} (10.50, 0.25);
	\draw (0, -11.00) node [](s_end) {};
	\draw (9.50, 0.25) node [](legend) {};
	\draw [dotted] (supplier) -- (s_end-|supplier); 
	\draw (2.00, 1) node [draw, rectangle, fill=black!10](ev0) {EV$_0$};
	\draw [dotted] (ev0) -- (s_end-|ev0); 
	\draw (2.00,-0.03) node [draw, circle, inner sep=0pt, minimum size=1mm, fill=white](ev0y0) {};
	\draw (0,-0.03) node [draw, circle, inner sep=0pt, minimum size=1mm, fill=black](ss0y0) {};
	\draw [->] (ev0y0) to (ss0y0);
	\draw (2.00,-2.01) node [draw, circle, inner sep=0pt, minimum size=1mm, fill=white](ev0y1) {};
	\draw (0,-2.01) node [draw, circle, inner sep=0pt, minimum size=1mm, fill=black](ss0y1) {};
	\draw [->] (ev0y1) to (ss0y1);
	\draw (2.00,-3.99) node [draw, circle, inner sep=0pt, minimum size=1mm, fill=white](ev0y2) {};
	\draw (0,-3.99) node [draw, circle, inner sep=0pt, minimum size=1mm, fill=black](ss0y2) {};
	\draw [->] (ev0y2) to (ss0y2);
	\draw (2.00,-5.96) node [draw, circle, inner sep=0pt, minimum size=1mm, fill=white](ev0y3) {};
	\draw (0,-5.96) node [draw, circle, inner sep=0pt, minimum size=1mm, fill=black](ss0y3) {};
	\draw [->] (ev0y3) to (ss0y3);
	\draw (2.00,-7.94) node [draw, circle, inner sep=0pt, minimum size=1mm, fill=white](ev0y4) {};
	\draw (0,-7.94) node [draw, circle, inner sep=0pt, minimum size=1mm, fill=black](ss0y4) {};
	\draw [->] (ev0y4) to (ss0y4);
	\draw (2.00,-9.92) node [draw, circle, inner sep=0pt, minimum size=1mm, fill=white](ev0y5) {};
	\draw (0,-9.92) node [draw, circle, inner sep=0pt, minimum size=1mm, fill=black](ss0y5) {};
	\draw [->] (ev0y5) to (ss0y5);
	\draw (3.80, 1) node [draw, rectangle, fill=black!10](ev1) {EV$_1$};
	\draw [dotted] (ev1) -- (s_end-|ev1); 
	\draw (3.80,-0.16) node [draw, circle, inner sep=0pt, minimum size=1mm, fill=white](ev1y0) {};
	\draw (0,-0.16) node [draw, circle, inner sep=0pt, minimum size=1mm, fill=black](ss1y0) {};
	\draw [->] (ev1y0) to (ss1y0);
	\draw (3.80,-2.24) node [draw, circle, inner sep=0pt, minimum size=1mm, fill=white](ev1y1) {};
	\draw (0,-2.24) node [draw, circle, inner sep=0pt, minimum size=1mm, fill=black](ss1y1) {};
	\draw [->] (ev1y1) to (ss1y1);
	\draw (3.80,-4.32) node [draw, circle, inner sep=0pt, minimum size=1mm, fill=white](ev1y2) {};
	\draw (0,-4.32) node [draw, circle, inner sep=0pt, minimum size=1mm, fill=black](ss1y2) {};
	\draw [->] (ev1y2) to (ss1y2);
	\draw (3.80,-6.40) node [draw, circle, inner sep=0pt, minimum size=1mm, fill=white](ev1y3) {};
	\draw (0,-6.40) node [draw, circle, inner sep=0pt, minimum size=1mm, fill=black](ss1y3) {};
	\draw [->] (ev1y3) to (ss1y3);
	\draw (3.80,-8.48) node [draw, circle, inner sep=0pt, minimum size=1mm, fill=white](ev1y4) {};
	\draw (0,-8.48) node [draw, circle, inner sep=0pt, minimum size=1mm, fill=black](ss1y4) {};
	\draw [->] (ev1y4) to (ss1y4);
	\draw (3.80,-10.56) node [draw, circle, inner sep=0pt, minimum size=1mm, fill=white](ev1y5) {};
	\draw (0,-10.56) node [draw, circle, inner sep=0pt, minimum size=1mm, fill=black](ss1y5) {};
	\draw [->] (ev1y5) to (ss1y5);
	\draw (5.60, 1) node [draw, rectangle, fill=black!10](ev2) {EV$_2$};
	\draw [dotted] (ev2) -- (s_end-|ev2); 
	\draw (5.60,-0.74) node [draw, circle, inner sep=0pt, minimum size=1mm, fill=white](ev2y0) {};
	\draw (0,-0.74) node [draw, circle, inner sep=0pt, minimum size=1mm, fill=black](ss2y0) {};
	\draw [->] (ev2y0) to (ss2y0);
	\draw (5.60,-2.63) node [draw, circle, inner sep=0pt, minimum size=1mm, fill=white](ev2y1) {};
	\draw (0,-2.63) node [draw, circle, inner sep=0pt, minimum size=1mm, fill=black](ss2y1) {};
	\draw [->] (ev2y1) to (ss2y1);
	\draw (5.60,-4.52) node [draw, circle, inner sep=0pt, minimum size=1mm, fill=white](ev2y2) {};
	\draw (0,-4.52) node [draw, circle, inner sep=0pt, minimum size=1mm, fill=black](ss2y2) {};
	\draw [->] (ev2y2) to (ss2y2);
	\draw (5.60,-6.41) node [draw, circle, inner sep=0pt, minimum size=1mm, fill=white](ev2y3) {};
	\draw (0,-6.41) node [draw, circle, inner sep=0pt, minimum size=1mm, fill=black](ss2y3) {};
	\draw [->] (ev2y3) to (ss2y3);
	\draw (5.60,-8.30) node [draw, circle, inner sep=0pt, minimum size=1mm, fill=white](ev2y4) {};
	\draw (0,-8.30) node [draw, circle, inner sep=0pt, minimum size=1mm, fill=black](ss2y4) {};
	\draw [->] (ev2y4) to (ss2y4);
	\draw (5.60,-10.19) node [draw, circle, inner sep=0pt, minimum size=1mm, fill=white](ev2y5) {};
	\draw (0,-10.19) node [draw, circle, inner sep=0pt, minimum size=1mm, fill=black](ss2y5) {};
	\draw [->] (ev2y5) to (ss2y5);
	\draw (7.40, 1) node [draw, rectangle, fill=black!10](ev3) {EV$_3$};
	\draw [dotted] (ev3) -- (s_end-|ev3); 
	\draw (7.40,-1.14) node [draw, circle, inner sep=0pt, minimum size=1mm, fill=white](ev3y0) {};
	\draw (0,-1.14) node [draw, circle, inner sep=0pt, minimum size=1mm, fill=black](ss3y0) {};
	\draw [->] (ev3y0) to (ss3y0);
	\draw (7.40,-3.20) node [draw, circle, inner sep=0pt, minimum size=1mm, fill=white](ev3y1) {};
	\draw (0,-3.20) node [draw, circle, inner sep=0pt, minimum size=1mm, fill=black](ss3y1) {};
	\draw [->] (ev3y1) to (ss3y1);
	\draw (7.40,-5.27) node [draw, circle, inner sep=0pt, minimum size=1mm, fill=white](ev3y2) {};
	\draw (0,-5.27) node [draw, circle, inner sep=0pt, minimum size=1mm, fill=black](ss3y2) {};
	\draw [->] (ev3y2) to (ss3y2);
	\draw (7.40,-7.33) node [draw, circle, inner sep=0pt, minimum size=1mm, fill=white](ev3y3) {};
	\draw (0,-7.33) node [draw, circle, inner sep=0pt, minimum size=1mm, fill=black](ss3y3) {};
	\draw [->] (ev3y3) to (ss3y3);
	\draw (7.40,-9.40) node [draw, circle, inner sep=0pt, minimum size=1mm, fill=white](ev3y4) {};
	\draw (0,-9.40) node [draw, circle, inner sep=0pt, minimum size=1mm, fill=black](ss3y4) {};
	\draw [->] (ev3y4) to (ss3y4);
	\draw (9.20, 1) node [draw, rectangle, fill=black!10](ev4) {EV$_4$};
	\draw [dotted] (ev4) -- (s_end-|ev4); 
	\draw (9.20,-1.44) node [draw, circle, inner sep=0pt, minimum size=1mm, fill=white](ev4y0) {};
	\draw (0,-1.44) node [draw, circle, inner sep=0pt, minimum size=1mm, fill=black](ss4y0) {};
	\draw [->] (ev4y0) to (ss4y0);
	\draw [decorate, decoration={brace,amplitude=5pt,raise=1mm}, yshift=0mm, color=black!50] (legend) -- (legend|-ev4y0) node[black, midway, xshift=2mm, align=left, color=black!50]{start delay};
	\draw (9.20,-3.45) node [draw, circle, inner sep=0pt, minimum size=1mm, fill=white](ev4y1) {};
	\draw (0,-3.45) node [draw, circle, inner sep=0pt, minimum size=1mm, fill=black](ss4y1) {};
	\draw [->] (ev4y1) to (ss4y1);
	\draw [decorate, decoration={brace,amplitude=5pt,raise=1mm}, yshift=0mm, color=black!50] (legend|-ev4y0) -- (legend|-ev4y1) node[black, midway, xshift=2mm, align=left, color=black!50]{loop jitter};
	\draw (9.20,-5.47) node [draw, circle, inner sep=0pt, minimum size=1mm, fill=white](ev4y2) {};
	\draw (0,-5.47) node [draw, circle, inner sep=0pt, minimum size=1mm, fill=black](ss4y2) {};
	\draw [->] (ev4y2) to (ss4y2);
	\draw [decorate, decoration={brace,amplitude=5pt,raise=1mm}, yshift=0mm, color=black!50] (legend|-ev4y1) -- (legend|-ev4y2) node[black, midway, xshift=2mm, align=left, color=black!50]{loop jitter};
	\draw (9.20,-7.48) node [draw, circle, inner sep=0pt, minimum size=1mm, fill=white](ev4y3) {};
	\draw (0,-7.48) node [draw, circle, inner sep=0pt, minimum size=1mm, fill=black](ss4y3) {};
	\draw [->] (ev4y3) to (ss4y3);
	\draw [decorate, decoration={brace,amplitude=5pt,raise=1mm}, yshift=0mm, color=black!50] (legend|-ev4y2) -- (legend|-ev4y3) node[black, midway, xshift=2mm, align=left, color=black!50]{loop jitter};
	\draw (9.20,-9.49) node [draw, circle, inner sep=0pt, minimum size=1mm, fill=white](ev4y4) {};
	\draw (0,-9.49) node [draw, circle, inner sep=0pt, minimum size=1mm, fill=black](ss4y4) {};
	\draw [->] (ev4y4) to (ss4y4);
	\draw [decorate, decoration={brace,amplitude=5pt,raise=1mm}, yshift=0mm, color=black!50] (legend|-ev4y3) -- (legend|-ev4y4) node[black, midway, xshift=2mm, align=left, color=black!50]{loop jitter};

\end{tikzpicture}
\caption{Agent desynchronisation when running through algorithm iterations in their respective execution loop.}
\label{ch3:fig:agent-desynchronisation}
\end{figure}


\subsection{Cases and Performance Metrics}
\label{ch3:subsec:cases-and-metrics}

\subsection{Synchronised vs Desynchronised Implementation}

Whilst the synchronised and desynchronised implementation of the algorithm do not differ in the scheduling method, their updating process distinguishes them.
For the synchronised implementation, the algorithm performs the network's demand update (i.e. Equ. \ref{ch3:equ:updated-demand-profile}) after all EVs have sent their updated charging profiles.
However, in the desynchronised implementation of updates are sent immediately and without regarding the compute state of other agents.
To achieve this, all physical agents would dispatch their brokers simultaneously, which could be accomplished in two ways:

\begin{enumerate}
	\item All agents were launched at the same time and a long enough waiting delay was introduced after the algorithm rescheduled. Once the waiting period has passed, all agents broadcast their new charging profile and acquire the new net demand after another waiting preset delay. Despite the additional time requirements added to the algorithm, all agents would have synchronised network information and schedule in a synchronised manner.
	\item Alternatively, a synchronisation ping was sent from the substation once all agents' charging profiles were received. Whilst this method does not cause additional system delays, it adds additional strain onto the communications infrastructure. Due to the a sufficiently equipped communications network, this approach was chosen here.
\end{enumerate}

In contrast, for the asynchronous implementation, the agent update is broadcasted as soon as an EV has found its updated charging profile. No synchronisation mechanism is required and only the agent's solving speed limits the system's performance. The pseudo random delay between agent's solving time causes each agent to obtain an already modified load profile when querying the substations. By introducing this distorted network load to all EVs, the algorithm's initialisation and all consecutive iterations schedule the charging load differently.









