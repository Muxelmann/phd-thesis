\section{Distributed Systems}
\label{ch3:sec:distributed-systems}

As it was discussed in the literature review in Section \ref{ch-review}, several mechanisms exist to decentralise control of DERs.
Du to their reactivity, pro-activeness and social ability, a Multi Agent System (MAS) distinguishes itself from traditional software and hardware systems, which is also why it was chosen for the coordination of EV charging.
Several agent packages implementations exist and each may follow a different communication paradigm.
These paradigms include ``Belief, Desire and Intention'' (BDI), neutral behaviour or other specialised functionality \cite{Luck2004}.
From this comprehensive catalogue, the Java Agent Development Framework (JADE) was chosen since it natively implements FIPA specification \cite{JADE-website, FIPA-agent-specs}.
Furthermore, JADE is an application independent package that has become quite popular, as seen by the number of publications using JADE \cite{Karfopoulos2013, Eddy2011, Kuo2013, Mocci2014, Li2017}.

In this work, multiple virtual trading agents are designed to negotiate a corresponding EV's charging profile with other trading agents.
Tying virtual agents to a physical entity is not new \cite{Nagata2002, Dimeas2005, Nguyen2011, Nagata2012}, yet in previous work these agents directly controlled the entity whilst the agents in the presented work negotiate schedules that will be applied after ratification.
So called ``broker'' agents are implemented to facilitate this trading process, and the implement the Contact-Net Protocol (CNP), which in turn is based on the FIPA Agent Communication Language (ACL) that is required to communicate over a shared telecommunications infrastructure.
The broker is linked to one EV and negotiates the charging profile over a scheduling horizon, $H$.
This link is shown in Figure \ref{ch3:fig:agent-network}.

\begin{figure}\centering
\tikzstyle{box} = [%
	draw,%
	rectangle,%
	%fill=green!20,%
	minimum height=2em,%
	minimum width=2em,%
]
\begin{tikzpicture}[node distance=2cm, shorten >= 1pt, >=stealth', auto, scale=1, transform shape]
	\pgfmathsetmacro\N{4}

	% Draw supplyer
    \draw
    (0, 0)
    node [box, fill=black!10](supplier) {Supplier}
    node [box, below left of=supplier](buyer_s) {Buyer}
    node [box, below right of=supplier](seller_s) {Seller};
    
    \draw (5, 0) node(bracket) {};
    
    \draw
    node [box, below of=buyer_s, yshift=-10mm](buyer_1) {Buyer}
    node [box, below left of=buyer_1, fill=black!10](load_1) {Load}
    node [box, above left of=load_1](seller_1) {Seller}
    node [box, below of=seller_s, yshift=-10mm](seller_2) {Seller}
    node [box, below right of=seller_2, fill=black!10](load_2) {Load}
    node [box, above right of=load_2](buyer_2) {Buyer};
    
    \draw [-] (supplier) to (buyer_s);
    \draw [-] (supplier) to (seller_s);
    \draw [-] (load_1) to (buyer_1);
    \draw [-] (load_1) to (seller_1);
    \draw [-] (load_2) to (buyer_2);
    \draw [-] (load_2) to (seller_2);
    
    \draw [->] (seller_s) to node[pos=0.7, align=center, left, xshift=-5mm]{buys\\from} (buyer_1);
    \draw [->] (seller_s) to node[pos=0.7, align=center, right, xshift=5mm]{buys\\from} (buyer_2);
    \draw [->] (seller_1) to node[pos=0.3, align=center, left, xshift=-5mm]{sells\\to} (buyer_s);
    \draw [->] (seller_2) to node[pos=0.3, align=center, right, xshift=5mm]{sells\\to} (buyer_s);
    
	
    \draw [decorate,decoration={brace,amplitude=5pt,raise=0pt},yshift=0pt](supplier.90-|bracket.0) -- (supplier.270-|bracket.0) node [black,midway,xshift=2mm,align=left] {physical entity};
    \draw [decorate,decoration={brace,amplitude=5pt,raise=0pt},yshift=0pt](seller_s.90-|bracket.0) -- (seller_s.270-|bracket.0) node [black,midway,xshift=2mm,align=left] {CNP responding\\broker agents};
    \draw [decorate,decoration={brace,amplitude=5pt,raise=0pt},yshift=0pt](seller_2.90-|bracket.0) -- (seller_2.270-|bracket.0) node [black,midway,xshift=2mm,align=left] {CNP initiating\\broker agents};
    \draw [decorate,decoration={brace,amplitude=5pt,raise=0pt},yshift=0pt](load_2.90-|bracket.0) -- (load_2.270-|bracket.0) node [black,midway,xshift=2mm,align=left] {physical entities};
    
\end{tikzpicture}
\caption{A simplified MAS structure containing virtual seller and buyer agents (white), that negotiate power/charging profiles for physical entities (grey).}
\label{ch3:fig:agent-network}
\end{figure}

The dominant goal of the negotiation procedure is to prevent charging spikes when multiple EVs connect and begin their charging.
