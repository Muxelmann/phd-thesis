\section{Problem statement and research objectives}
\label{ch-introduction:sec:problem-statement}

With non-uniform increasing demand across LV networks, DNOs need to identify and develop BESS control algorithms in order to support their network operation; instead of reinforcing network assets.
Whilst the core research which is presented in literature (in Chapter \ref{ch-literature}) consists of identifying network management methods and their performance weighted against computational resources and robustness, the research presented here limits its scope on technical issues regarding impact on network operation, deployability within the \textit{NTVV} project and conformity to all system requirements.
Therefore, the aim of this thesis is to identify key parameters that indicate the network's performance, to develop a control strategy that improves peak-reduction performance whilst adhering to traditional BESS control approaches, and to assess the performance of collaborating distributed energy resources with particular emphasis on desynchronised and communication-less operation.

In accordance to the identified key features that motivate the conducted research (introduced in Section \ref{ch-introduction:subsec:motivation}), a set of objectives is presented in order to achieve the aim of contributing to the existing field:

\begin{enumerate}
	\item Develop a control mechanism for a single BESS to further improve three-phase network operation without deviating from its predetermined BESS schedule.
	\item Extend the control mechanism to reducing sub-half-hourly load peaks by dynamically adjusting the underlying half-hourly BESS schedules.
	\item Develop and compare operation of a charge scheduling algorithm for multiple BESSs in a synchronised and desynchronised communications environment.
	\item Develop a control strategy for distributed BESSs by individually assigning control parameters to an extended charge-discharge-algorithm.
\end{enumerate}

