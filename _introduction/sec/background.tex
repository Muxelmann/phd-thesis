\section{Background and motivation}
\label{ch-introduction:sec:background}

Today's society and economy are highly dependent on the continuous availability of energy, or more specifically: electric energy.
In the UK, demand for electricity has increased over the past decades, and this trend is expected to continue into the future \cite{HMGovernment2009}.
This demand increase is only accelerated since a major focus of UK energy policies has been put on transitioning towards a low carbon economy \cite{RoyalAcademyofEngineering2010}.
Particularly the decarbonisation of heat and transport sectors are two areas of significant strategic focus and Low Carbon Technology (LCT) such as Photo-Voltaic (PV) installations, Electric Vehicles (EVs) and heat pumps are expected to contribute significantly to this transition.

\nomenclature[G]{FES}{Future Energy Scenarios}

As adaptation of these LCTs increases and they start to penetrate power distribution networks, stress on these networks will also continue to increase, which is expected to result in operating issues that can eventually lead to additional service disruptions.
Furthermore, the uptake of LCTs is not expected to progress evenly throughout the entire power network, and instead clusters of early adopters are predicted to form, leading to certain Low-Voltage (LV) networks exceeding their operational constraints, even at relatively low national rate of LCT adaption \cite{Poghosyan2014}.
The scale of this energy transition becomes becomes particularly apparent when referring to the UK's 2017 Future Energy Scenarios (FES) that compare the predicted future load scenarios.

\input{_introduction/fig/electricity-demand-residential}

Figure \ref{ch-introduction:fig:electricity-demand-residential} shows the predicted increase in annual residential electricity demand, where the market driven scenario (i.e. ``Consumer Power'') shows the largest increase until 2050.
When putting more emphasis on environmental concerns, the ``Two Degrees`` scenario emerges - this scenario refers to the internation of limiting global temperature increase to two degrees (previous FES reports referred to this scenario as ``Gone Green``).
According to the FES 2017, those two scenarios enable a continuing growth of the UK economy, but put very different emphasis on decarbonisation targets.
Although the rise in demand for electricity is projected to differ by more than 30TWh in 2050 if the ``Two Degrees`` scenario is achieved, the aforementioned uptake of LCTs like Electric Vehicles (EVs) is expected to put a noticeable stain onto the power network in both scenarios.

\begin{figure}\centering
	\includegraphics{_introduction/fig/electricity-demand-ev}
	\caption{Rise in energy demand due to the uptake of electric vehicles as predicted by FES 2017 \cite{FES2017}}
	\label{ch-introduction:fig:electricity-demand-ev}
\end{figure}

Focusing on the electrification of personal transport by introducing EVs into the electricity demand, Figure \ref{ch-introduction:fig:electricity-demand-ev} shows that both the ``Consumer Power'' as well as the ``Two Degrees'' scenario will add more than 30TWh of annual energy demand by 2045.
Since most of these EVs are expected to charge at home and (at least initially) at similar times, aggregating effects are feared to not only exhaust national energy supply capabilities, but also power distribution capacities.

\begin{figure}\centering
	\includegraphics{_introduction/fig/electricity-peak-demand}
	\caption{Change in annual peak power demand as predicted by FES2017 \cite{FES2017}}
	\label{ch-introduction:fig:electricity-peak-demand}
\end{figure}

\nomenclature[G]{CREST}{Centre for Renewable Energy Systems Technology}
\nomenclature[G]{ENW}{Electricity North West}

Figure \ref{ch-introduction:fig:electricity-peak-demand} supports this worry since uncontrolled proliferation of LCTs and EVs, as shown in the ``Consumer Power'' scenario, will add more than 20GW of demand onto today's peak power levels.
Such a rise in peak demand would exceed today's supply capacities, especially since the new energy mix and decommissioning of fossil fuelled power plants have resulted in this year's winter capacity margin to only lie between 3.7GW and 4.9GW \cite{NationalGrid2017a}.
Although FES 2017 expects the ``Two Degrees'' scenario to also show an increase in peak power, better coordination and control is expected to mitigate half of the 20GW peak increase.
From this point of view, today's narrow capacity margins could already cater for nearly half of the increase in peak demand that is expected to occur in 33 years, i.e. 2050.
Whilst UK power transmission networks are being upgraded and will therefore be capable of handling this increased power demand, the increasing stress on the distribution network still remains.
This is due to the fact that loads in the residential and commercial sectors are typically situated at the network edge, i.e. in the LV distribution network.

\subsection{Topology and challenges of the UK electricity network}
\label{ch-introduction:subsec:topology-of-lv-network}

The UK electricity network in today's form has grown over the past century and is based on an interconnected high-voltage gird.
Its largest part is also known as the transmission network, which connects remote power stations to distribution networks.
Those distribution networks supply electricity to all loads across the mainland of the UK, including industrial, urban and rural customers\footnote{Some small and remote UK islands like the Shetland islands are not connected to this national grid and have their separate electricity infrastructure. Therefore they are not considered as part of this thesis since the study of this kind of network lies outside the research scope.}.

The entire structure of the electricity network is a three-phase Alternating Current (AC) system since this allows easy voltage level conversion with the use of transformers, i.e. without the need of power electronics.
In the UK, the highest voltage level for generation and transmission is 400kV.
Such a high voltage requires a relatively small current to transmit the generated bulk power, which in turn reduces conduction losses and maximises the efficiency of the high-voltage network.
Regional supply points step-down this high voltage to 132kV\footnote{In some cases regional supply points provide 127kV instead of 132kV.} to deliver power to Distribution Network Operators (DNOs).
From the primary level of the distribution network and onwards, this so called medium-voltage is stepped down to 33kV, then 11kV and finally 400V phase to phase (P2P), in order to cater for heavy industry, medium clients and household sized customers, respectively.
In the UK, all households are connected to one of the three phases of the distribution network and therefore are supplied at a nominal voltage of 230V (phase to neutral - P2N).
To achieve a balanced network, each customer's phase allocation is chosen at random.

\nomenclature[G]{DER}{Distributed Energy Resource}
\nomenclature[G]{ADMD}{After Diversity Maximum Demand}
\nomenclature[G]{OLTC}{On-Line Tap-Changer}

This LV part of the electricity network is its weakest part, since its assets were designed to cater for small powers between 315kVA to 500kVA \cite{EDS08-0115}.
Despite this capacity limitation, DNOs aim to maintain distribution level voltages within their statutory operating bands, i.e. 230V +10\% -6\% for the LV network as specified by the Electricity Supply Quality and Continuity Regulation (ESQRC) \cite{HealthandSafetyExecutive2002} and Engineering Recommendation G59 \cite{EnergyNetworksAssociation2013}.
Primary substations in the UK are equipped with regulation equipment, e.g. On-Line Tap-Changers (OLTC), to increase or decrease the voltage on the secondary transformer side depending on the current level of demand, but secondary transformers do not have such regulating equipment and instead apply a constant voltage conversion ratio which is set according to the network's typical demand.

A project based on the findings from \textit{Electricity North West} (ENW) in \cite{ElectricityNorthWestLtd2014} emphasises the issues that result from residential increase in demand for electricity, of e.g. voltage deviation due to an uptake of LCTs.
More specifically, in the ENW lead project 200 LV networks in the UK were monitored to assess the capacity for LCT adoption.
Initial findings showed that even in today's distribution networks, 15\% of all monitored substations experienced reverse power flow, 4.5\% substations reported high voltages (i.e. above 253V), and only 2\% substations reported occasional low voltages (i.e. below 216V).
With the voltage drop assumption however, customers in these network were still believed to be operating at BSEN50160 compliant voltages.
Models to assess the LCT headroom indicated that the first issue is always voltage deviation and thermal or capacity limits are of second concern.
This finding is based on the fact that, LV networks are highly resistive and rarely managed in an active manner; hence as the number of PV installations is expected to grow, the voltage deviation magnitude and frequency is also going to increase \cite{Woyte2006}.
Since the expected voltage deviation was caused despite a relatively low rate of LCT adaptation in the residential environment, strict regulation has been put in place to assure continuous operation without violating any operating constraints.
Otherwise additional voltage deviation, unbalanced network operation or potential asset overloads could be the result.

Traditional network planning approaches that have been used by DNOs to expand and install network assets were designed to circumvent such constraint violations in order to follow the aforementioned standards and regulations.
The most common approach follows the commonly used practice of aggregating a large number of customers and designing the power delivery network to cater for their largest probable demand, i.e. the After Diversity Maximum Demand (ADMD) method \cite{Richardson2010a}.
This ADMD method has remained the same for many years and uses historical load analysis and standard growth assumptions that are both no longer valid in this unprecedented LCT uptake scenario \cite{Yunusov2016}.
To make things worse, LV networks in the UK are generally unmonitored once installed.
Distribution Network Operators (DNOs) have become aware of this issue and are developing updated planning strategies involving ``smart'' and ``flexible'' electricity grids \cite{Fang2012}.
However, in situ equipment that will become subject to the same adaptation of LCT needs to be managed actively via innovation in the use of existing and new technologies; otherwise both frequency of service disruptions and customer minutes lost will increase alongside the proliferation of LCTs \cite{Ault2008a}.
One such innovative technology, which is the main focus of the presented thesis, is the installation and management of battery storage \cite{Chen2009}.
The following section is going to introduce the favoured solutions and explain why battery energy storage was chosen as part of the research that was carried out in this PhD.


\subsection{Solutions to mitigate impact of LCT}
\label{ch-introduction:subsec:solutions-to-mitigate-impact-of-lct}

The two most favoured solutions that allow DNOs to support LV network operation are:
\begin{enumerate*}
	\item the reinforcement of in situ network assets or
	\item the deployment of network support equipment.
\end{enumerate*}
Whilst network reinforcement would certainly address immediate issues of current network capacity constraints, this approach is also the more expensive and disruptive option.
More specifically, customers will need to deal with outages during periods of asset upgrades (e.g. transformer upgrade and line re-conductoring after secondary transformers' tap settings have been adjusted).
Therefore, alternatives to defer or avoid network reinforcements have been sought and assessed \cite{Harrison2007, Zangs2016a, VanderKlauw2016d, Greenwood2017}.
Most promising alternatives are to install flexible and controllable Distributed Energy Resources (DERs), or more specifically: Battery Energy Storage Solutions (BESS) \cite{Wade2010}.
BESS has not only seen significant advancements in technology, but also received increasing attention in both academic studies and industry trials \cite{Palizban2016}.

Installing BESS on a strategic location in the LV network brings several advantages to DNOs' control over the network's performance.
Roles for energy storage in general are extensively reviewed in Section \ref{ch-literature:sec:role-of-energy-storage-a-survey}.
A few examples of potential benefits from BESS do however include the regulation of voltages in order to to operate within statutory voltage bands \cite{Yang2014}, shaving peak loads to relieve stress from the installed network assets \cite{Bennett2015}, and reducing phase unbalance to increase network efficiency \cite{Wang2015b} .
Whilst the questions regarding locating and scaling of BESS have mostly been addressed, BESS control can be split into two complementing yet unmarried approaches:

\begin{enumerate}
	\item ``off-line'' control, using load forecasts and BESS schedules \cite{Cecati2011, Chaouachi2013, Palma-Behnke2013, Khodaei2014}, and
	\item ``on-line'' control, using Set-Points Control (SPC), Model Predictive Control (MPC) or similar dynamic control methods \cite{Salinas2013, Huang2013, Huang2014a, Sun2014a}.
\end{enumerate}

Furthermore, with the anticipated uptake of household BESS, mechanisms to control several storage systems also need to be considered.
For instance, several industry leaders propose to store solar energy in order to support charging of EVs \cite{Baumann2017}.
Without rooftop PV installations, batteries need work in a cooperative manner to not impose additional strain onto the network.
The full review of storage control strategies to achieve both off-line and on-line, as well as centralised/individual and distributed battery control is presented in Section \ref{ch-literature:sec:control-of-energy-storage}.


\subsection{Smart control}
\label{ch-introduction:subsec:smart-control}

\nomenclature[G]{IoT}{Internet of Things}

As already mentioned in Section \ref{ch-introduction:subsec:solutions-to-mitigate-impact-of-lct}, off-line and on-line control strategies exist to manage BESS.
This traditional control often dealt with the dispatch of a single energy entity, but due to the distributed nature of the expected LCT uptake, methods to result in cooperative behaviour needed to be proposed.
With the penetration of smart meters and communication-enabled devices in the so called ``Internet of Things`` (IoT), power systems have the potential of becoming interlinked networks of smart devices, too.
So called ``smart control'' mechanisms complement the traditional off-line and on-line control strategies and are of great research interest to enable the uptake of distributed LCTs.

For example, ``smart charging'' is a key term that summarises EV charging mechanisms where the limited network capacity causes multiple EVs to share the available resource amongst themselves \cite{Sortomme2011, Vaya2012, Garcia-Villalobos2014}.
Intelligently limiting the EVs' maximum charging rates prevents them from adding unnecessary stress onto the network at the cost of longer charging times.
A similar key term is the so called ``smart grid'' where distributed resources communicate and cooperate in order to e.g. shed load using Demand Side Response (DSR) or maintain micrigrid operation in fault situations \cite{Samadi2012, Liu2014, Liang2014}.
The fundamental problem that is addressed in the research field of smart control is to find an optimal balance between network benefits and customer cost.
Also, the fundamental requirement for the successful realisation of smart control is the reliable exchange of information amongst the partaking smart entities.

Hence, smart control does not only require robust control mechanisms, but also a robust communications infrastructure.
Literature, which is reviewed in Chapter \ref{ch-literature}, shows that all control mechanisms dealing with the coordination of distributed energy resources either explicitly or implicitly assume a robust communication infrastructure.
For instance, this requirement is assumed when messages are received without delay and immediately after dispatched or when a single control instruction result in the synchronised reaction of all entities.
In reality however, the strength of the communication link may vary with e.g. weather or current network traffic, and fixed message delays and exact device synchronisation can no longer be guaranteed.
Therefore, not only smart control algorithms, but also their sensitivity to the strength of the underlying communications infrastructure is of interest.

\subsection{Challenges to control BESS}
\label{ch-introduction:subsec:motivation}

From the extensive catalogue of possible roles for energy storage in the electricity grid that was presented in Section \ref{ch-literature:sec:role-of-energy-storage-a-survey}, the focus of the research in this thesis is put on aiding DNOs to manage and operate their power distribution networks.
More specifically, battery energy storage is the main focus since is has the potential to defer or even mitigate costly network reinforcements.
Modern battery technology allows the storage of electrical energy in ever-decreasing form factory, whilst power electronics technology becomes more efficient at integrating batteries into power networks.

As shown in the literature review in Chapter \ref{ch-literature}, methods of controlling BESS to optimise power flow have been of great research interest.
However, the impact on particular key parameters of the three-phase networks still need to be investigated.
Subsequently, the challenge of applying real-time corrections to BESS schedules in order to decrease peak demand whilst obeying to technical and operational constraints is also a remaining research question.
Also, since the expected uptake of distributed LCTs through proliferation of household storage solutions (e.g. to counteract the impact of EVs) requires sophisticated coordination mechanisms, two additional research challenges have been identified.
The first challenge focuses on improving cooperating device behaviour despite communication disturbances (i.e. through message desynchronisation), and the second builds upon the findings from key network improvements to construct a functioning BESS control mechanism despite the absence of a telecommunications infrastructure.





