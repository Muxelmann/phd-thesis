\section{Overview}
\label{ch-introduction:sec:overview}

Today's society and its lifestyle are highly dependent on the continuous availability of electricity.
With the ongoing increase in demand for energy, power delivery networks that were constructed several decades ago are operated dangerously close to their capacity limits.
Sociopolitical incentives to accelerate the uptake of Low-Carbon Technologies (LTCs) have commenced; e.g. to reduce the national green house gas emissions and reduce the dependency on oil, coal and gas.
LTCs are twofold in nature.
On the one hand Distributed Energy Resources (DERs), like wind or Photo Voltaic (PV) systems, can provide additional electrical energy, whilst the electrification of consumer appliances and transportation impose additional loads.
For example, in 2012, due to government investment incentives like feed in tariffs or Renewable Obligation Certificates (ROCs), some European country's annual PV capacity increased by an astonishing 33GW \cite{Hockenos2013}.
Furthermore, in just the UK, an uptake of 9.7 million EVs and Plug-in Hybrid Electric Vehicles (PHEV) is expected by 2040; requiring an additional 24TWh of energy \cite{DBER2008, FES2015}.
Since this added generation and energy demand need not align in time, highly volatile demand fluctuations of large amplitude are the result.
In fact, uncoordinated EV charging could impose an additional 6.5GW of demand onto the already stressed UK power grid \cite{FES2016}.
To address the issue at its root cause, energy storage has been proposed \cite{Manz2012}.

\subsection{Electrical Energy Storage}

Electrical energy storage systems have been discussed for a long time, and their important role in future energy systems has already been identified in the 70s by Kalhammer \cite{Kalhammer1979}.
Phasing out oil and gas powered plants to replace them with sustainable energy sources, and enabling the uptake of future transportation (e.g. EVs) was already envisioned.
Back then, power levelling functions were seen as the major features that could be provided by energy storage systems to the electricity grid.
Levelling power demand was envisioned to mitigate network losses, since they are linked to the actions of ramping up and powering down energy production, which is required to match energy demand.
With the aforementioned increasing demand for electricity, the power fluctuations increase accordingly.
This in turn increased the demand for additional energy storage capacity, which lead to a significant uptake in installed energy storage capacity.
According to the EPRI, 127GW of grid scale energy storage have been installed \cite{Rehman2015}, where 99\% of the world's bulk energy storage capacity consisted of pumped hydro-electric energy storage \cite{TheEconomist2012a}.
In combination with the latest reviews and datasets \cite{Barbour2016, Barbour2015}, Kalhammer's prediction on the demand for energy storage was therefore proven correct; capacity of global pumped hydro has tripled since his publication.
Due to its large geographical size, bulk energy storage systems are not suitable for supporting Low-Voltage (LV) distribution network operation; i.e. addressing the issue at its root cause.

Alternative energy storage solutions had to be smaller in size, safe for deployment in urban and rural environments, and affordable to be viable for Distribution Network Operators (DNOs).
One of the first comprehensive reviews of such energy storage technologies was published by McLaron and Cairns \cite{McLarnon1989}, yet technology advancements allowed small scale energy storage to fulfil additional functions \cite{Ibrahim2008, Chen2009, Hadjipaschalis2009, Luo2015}.
In fact, the functions provided by those energy storage systems has become as diverse as the underlying technologies themselves.
Technologies included smaller pumped hydro-electric energy storage, Thermal Energy Storage (TES), Flywheel Energy Storage (FES), Compressed Air Energy Storage (CAES), Fuel Cell driven Energy Storage (FCES), and chemical, i.e. Battery Energy Storage (BES).