\section{Thesis structure}
\label{ch-introduction:sec:thesis-structure}

The structure of this thesis is organised as follows:

\begin{itemize}
	\item
	\textbf{Chapter~\ref{ch-literature}} carries out an extensive review of the literature surrounding the field in order to support the problem statement and proposed contribution.
	\item
	\textbf{Chapter~\ref{ch1}} develops a BESS scheduling mechanism and identifies key network parameters that are used in their corresponding cost functions to improve network operation.
	Then, this chapter address \ref{objective-1} by presenting a method that assigns a BESS schedule to the three-phase power distribution network whilst minimising the aforementioned cost functions; therefore improving network operation.
	Results are compared against a ``baseline'' and a ``normal'' (or traditional) operation case by assessing them on a temporal and probabilistic level.
	\item
	\textbf{Chapter~\ref{ch2}} then extends the work in Chapter~\ref{ch1} by presenting a dynamic control method that adjusts a half-hourly BESS schedule at sub-half-hourly temporal resolution in order to reduce both volatile and the daily load peak.
	This is achieved by combining two PID compensated control loops with a MPC and BESS schedule.
	Therefore, this chapter addresses \ref{objective-2}.
	\item
	\textbf{Chapter~\ref{ch3}} addresses \ref{objective-3} by presenting a cooperative battery charging algorithm that is deployed on a Multi-Agent System and assessed in both a synchronised and desynchronised communications environment.
	In this chapter, both algorithm convergence and algorithm performance is compared between its implementation in the synchronised and desynchronised scenario.
	\item
	\textbf{Chapter~\ref{ch4}} develops a stochastic EV demand model that is based on real vehicle mobility behaviour, and it will develop a control algorithm for distributed BESS to mitigate the negative impact from those EV's demand.
	This chapter address \ref{objective-4}, the final research objective, by extending the Additive-Increase Multiplicative-Decrease algorithm to enable cooperating BESS operation under the absence of a shared communications infrastructure.
	\item
	\textbf{Chapter~\ref{ch-conclusion}} presents a detailed conclusion that relates all findings back to the initial problem statement and the overarching aim of the presented PhD thesis.
	Also, this chapter highlights potential future work based on the findings from the conducted research.
\end{itemize}