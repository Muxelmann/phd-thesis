\chapter{Introduction}
\label{ch-introduction}

\section{Overview}
\label{ch-introduction:sec:overview}



\section{Background and Research Motivation}
\label{ch-introduction:sec:background-and-research-motivation}

National power grids have become extremely large and complex systems. They have been designed to deliver electricity from remote power plants to a wide variety of energy consumers. Since electric energy has to be transmitted over long distances of several kilometres before it reaches the end consumer, different voltage levels are used in order to minimise the effect of current losses. The grid itself has thus been split into four networks of different voltage levels: the transmission network, operating at above 220kV; the High-Voltage (HV) network, operating at 110kV; the Medium-Voltage (MV) network, operating at 1-50kV; and the Low-Voltage (LV) network, that operates at 230V (or 400V phase voltage). Here, each voltage level provides different functions to assure reliability of the entire electricity grid.

Transmission and HV networks are usually loop networks to assure system capacity and redundancy by allowing energy flow through alternate path. Although some consumers (e.g. railway and large production plants) may be directly connected to the HV network, all household demand is fed into the MV and LV network. Also, LV and MV networks are usually radial distribution networks, which are connected at a single point to the HV network. In the UK, the HV-MV link is referred to as the primary substation, which also changes the power transmission from thin overhead to thick underground cables.


\section{Research Aim}
\label{ch-introduction:sec:research-aim}

